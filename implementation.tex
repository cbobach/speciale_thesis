The main idea of this chapter is to introduce the different stages during the implementation. Here I will introduce the different problems and the solutions I have chosen to overcome those problems.

\todo{Introduce different sections}

%%%%%%%%%%%%%%%%% CIRCUIT IMPLEMENTATION %%%%%%%%%%%%%%%%%%%
%%%%%%%%%%%%%%%%%%%%%%%%%%%%%%%%%%%%%%%%%%%%%%%%%%%%%%%%%%%%
\section{Shuffle algorithm implementation}
\todo{Describe the implementation of the algorithms}
\todo{Describe problems}
\todo{Describe solutions}

\todo{Describe gadgets of the two shuffle protocols}

Because the $DUPLO$ protocol was chosen as the MPC protocol to use in the implementation the first hurdel was to make the circuits that should handle the shuffling of cards. Since circuits become rather complex when trying to implement functions it was important to have a compiler that could translate the algorithms into the desired circuit representation. Luckly there was a compiler for generating $DUPLO$-circuits that could be used. Therefor before staring to implemt the shuffle algorithms I read up on the documentation for the compiler\footnote{Which can be found at github: \url{https://github.com/AarhusCrypto/DUPLO/tree/master/frigate}}. The documentation is from the first version of the $Frigate$ compiler which was extended to suit the $DUPLO$ format. The documentation was not well specified and it was a long time since I last had worked with circuits in such a way. This resulted in some hard earned experience through trial and error on small exapmles.

The most important and different things to take into account here is the possibility of wirre acces and only one level functions. First of all it is possible to specify which and how many wires should represent a value like $y=x\{index:size\}$. This is the way bit inputs is translated into higher level representations. When it comes to functions there are some restrictions. First of all only one level of functions is allowed. Second of all only assignments in the main method is allowed through function calls. Otherwise the programing language used resembles $C$.

\bigskip

In the implementation of the two shuffle algorithms there are five different functions used. For the $Conditional~swap$ algorithm the first function used is the $xorSeed$ which handles the $XOR$ of the $seeds$ recieved from the two parties. Which is straightforward. The next function used is the $initDeck$ function. This functions inicilises the sorted deck wich is to be shuffled. This is done by a for-loop inserting the right values on the right wires. Here $6$ bits is used for the representation of cards. It is importand to notice that the variable holding the start position of every card needs its own representation with at least $6$ bits to hold the correct value. If only $6$ bits were used only vires up to opsition $63$ could be assigned t value. Therefor $9$ bit is used for this variable. The $9$ bits comes from the fact that $\lceil \log_2(52\cdot 6)\rceil=9$ bits. The last function used in the $Conditional~swap$ algorithm is the $shuffleDeck$ function. This is the function handling the actual shuffling. This is implemented using two for-loops one for the layers of the network and another for the swap gates in each layer.

\bigskip

In the case for the $Fisher\text{-}Yates$ the things stack up a bit differently. The first function in this case is the $correctSeed$ function which takes the $seeds$ from the two parties and correct them as descussed earlier. First a card representation of $6$ bits is added from each of the parties. This can not be represented in $6$ bits therefor a bigger representation is used. The idea is that after the addition a modulo reduction will be use to ensure that the new seed value is inside the right intervall. But because the modulo reduction implemented in $Frigate$ only supports a divisor of power of $2$ modulo can not be used. Because the seed values calculated uses devisor different from these. I used some time to figure out that this was only supported since it is not statet anyware in the documentation. After experiencing this I tried to find some articles on how to implement a circuit for the modulo reduction. But during my search I found out that this is not a trivial task to do. Therefor another solution was chosen. Since the input $seeds$ to the functionality is assumed to be in the right intervalls the solution was to subtract the boundray of the interval if the sum exeeded this. This was done by introducing an $if$ statment. It is nothworthy to mention that all values have had an unsigned representation until now. Since the comparison of to values is needed a signed representation is introduced since the documentation stats that the comparison of two values only works on signed representations. This implemetations now ensures that the randomness used for the $shuffleDeck$ function has the right form. But only if the original inputs are inside the right intervalls.

The second function is the same as in the case for the $Conditional~swap$ algorithm which is the $initDeck$ function. The last function called is the $shuffleDeck$ function which is different from the one from the $Conditional~swap$ algorithm. This function constist of an outer-loop that runs through the cards. Because circuits are static as disscused earlier it is not possible to assign a wire value based on a variable input. Therefor layers of conditional swaps are generated. This is represented by the inner-loop. In this way a gadget for each is constructed such that the index specified by the outer-loop can be swapped with any other value in the rest of the card deck.


%%%%%%%%%%%%%%%%%%% CIRCUIT GENERATION %%%%%%%%%%%%%%%%%%%%%
%%%%%%%%%%%%%%%%%%%%%%%%%%%%%%%%%%%%%%%%%%%%%%%%%%%%%%%%%%%%
\subsection{Circuit generation}
In this section I will introduce how the circuit compiler worked. Which problems I encountered and how some of them were fixed. 

First of all the compiler needs to be installed which require some special versions of some libarys which are not the latest. This requires that some specific setup for the compiler is done. But folowing the instructions in the installation guide and some internet seach made the compiler run.

Then when a program has been written in the $wir$ format specified by the documentation one compiles it to th $DUPLO$-format using the following command:

\begin{center}
\begin{verbatim}
./built/release/Frigate path/to/file.wir -dp
\end{verbatim}
\end{center}

This generates some different files where the one with the extention $.wir.GC\_duplo$ is the one that $DOPLO$ can use as input.

\bigskip

The $DOPLO$ framework has a function that allowes for evaluation of these inputfiles. This gives the posibility to specify the inputs from the two parties and get the result. This possibilty allowed me to study the implementations of the algorithms to see if thay did as I expected. First I tried to implement the $Fisher\text{-}Yates$ algorithm and as described earlier this revealed some problems. Since I did not have earlier experience working with the $Frigate$ compiler I was not sure if the problem was in my implementation or in the compiler. At first the focus was on my implementation but it was later discovered that it was the version of the $Frigate$ compiler used to generate $DUPLO$-circuits. First I expected it was in my implemetation. Why I starte to break the implementaion down into smaller modules that could be tested. So I created a framework that allowed me to test the different modules one by one. After this it was clear that something were off when using the modulo reduction. After different attempts to get it to work without any luck the focus shifted from being on the implemetaion to be on the compiler. After breaking the implementation down and testing only the modulo operator $\%$ which is in the documentation for the compiler it could not be the case that I had made any error. Therefor I started to look into the compiler. Since the compiler I used was an extention of an existing one the idea was that a bug could have been introduced when adding the new fatures. Therefor the old version of $Frigate$ was installed to test if this implemetation had the same bug. The problem was that the old version did not support any circuit output format that $DUPLO$ could read. Since $DUPLO$ supports what is known as composed circuit format files and $bristol$ formatted circuit files. I wrote a parser that took the output from the old version of $Frigate$ and translated that to $bristol$ format. This gave me two versions of the same program that I could run against each other and compare their results. This showed that there was a difference in the results produced. When I looked deeper into the problem it came clear that not all gate types was implemeted an the modulo operator triggered one of these gates.

This resulted in a fix of the new version of the $Frigate$ compiler and a complete change in the representation format of the circuits. Now all gates in the $DUPLO$ circuit format has two input and one putput wire. The representation of gates chaged from a more human readable type like $XOR$ to a truth table frendly type like $0110$ for $XOR$. And now all $16$ gate types from the truth table figure are implemented. The representation of the two constant wires $\textbf{0}$ and $\textbf{1}$ is handled as special cases since all gates now has two input wires. For the case of $\textbf{0}$ it is handled as an $XOR$ gate with two input wires with the same value. For the case of $\textbf{1}$ it was handled as the $NXOR$ of two wires with the same input value.

In this way I have contributed to the $DUPLO$ project and helped secure a stronger research product.

On the other side the compiler has not been updated to support modulo with a deviser that is not the power of $2$. As mentioned earlier the small amount of reserch I did into that area when I noticed the problem seems to indicate that this is not trivilay fixed. Therefor this is leaft unfixed.

%From chapter \ref{ch:shuffle} we know the different algorithms as algorithm \ref{fisher_yates_alg} and \ref{con_swap_alg}. These were implemented in a $wir$ format which resembles the $C$ programming style. To compile these $wir$ files to circuits I used the $Frigate$ compiler. I used a modified version of the original $Frigate$ compiler which was modified to be able to output $duplo$ circuit format. Both the compiler and the $wir$ language was not well documented and therefor a lot of testing and small corrections was required. During this time I got to know a lot about circuit generation and their representation. The circuits generated by the $Frigate$ compiler was tested with the evaluator of the $duplo$ MPC protocol. This gave me the possibility to evaluate the circuits implemented and see if they did what was expected of them. The $Conditional Swap$ algorithm was rather straight forward to implement when I understood the $wir$ syntax. When parsing it through the evaluator with different $seeds$ its seemed to produce correct results.

%It was a complete other result when it came to the $Fisher\text{-}Yates$ algorithm. As described in chapter \ref{ch:shuffle} the algorithm goes through different rounds. I splitted the algorithm in to different stages when the result was not like expected. Such that I could investigate where the error was. The first stage was the $XOR$ stage where I did not find any problems. Then there was the second stag which was the stage where the $seeds$ is modulated out. This stage gave some strange results. The problem seemed to be the modulo operator $\%$. After many tests and different approaches where I could not get it to work I tried to use the old version of $Frigate$ to exclude that error could have been introduced in the modification process. But the output file from the old version of the compiler could not be read by the evaluator of $duplo$. Therefor I wrote a parser from the $Frigate$ circuit format to the $duplo$ format whiteout any optimizations. Just as straight forward and stupid as it could be. Then after testing the old compiler and the new parser on the test cases made that far I had a strong believe that the parser was correct. When using the old compiler and the parser on the modulo stage circuit it revealed that the old and new compiler did not produce the same result. This gave me the confirmation that something could be wrong with the new modified version of the circuit compiler. After further test it was clear that some of the gates the old compiler produced was not handled by the new compiler and therefor introduced errors in the circuit. This resulted in a correction of the new compiler which now adds support of all 16 different gates and $duplo$ was also updated to handle all 16 gate types. All in all the detection of this error resulted in a more thorough tested circuit compiler and a protocol that now handles all gate types. But it is still the case that the old compiler does not handle modulo for all cases. It handles the cases for powers of 2 correct but when doing modulo anything else is not sure to be correct. Therefor will the $Fisher\text{-}Yates$ algorithm not work correctly which is a shame but as stated earlier it is not secure in this setting.

\todo{Technical comparison of the circuits, number of gates, XOR gates, NON-XOR gates (table)}



%%%%%%%%%%%%%%%% 2PC POKER IMPLEMENTATION %%%%%%%%%%%%%%%%%%
%%%%%%%%%%%%%%%%%%%%%%%%%%%%%%%%%%%%%%%%%%%%%%%%%%%%%%%%%%%%
\section{Poker implementation}
\todo{Describe the roles of the two parties}
\todo{Describe the setting}
\todo{Describe the different stages in the protocol and the sequense of the stages}

\todo{why can only one hand be played at a time, broken bristol compiler ehn frigate v2 was fixed}

\todo{Describe the more interresting setting, why is it interresting and why is it not implemented}
