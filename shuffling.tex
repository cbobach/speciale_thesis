In this chapter I will introduce the different shuffling algorithms studied during this project. I will introduce the ideas behind each algorithm studied and what makes it special. I will introduce the idea behind why these were chosen. I will explain how they were optimized to fit better to the specific needs for an application as a poker game. Lastly I will compare the algorithms to see the different benefits. 

All the permutation algorithms studied are with the purpose of shuffling card decks. It is important to chose an algorithm that ensures that the correct amount of permutations is reached.

The first algorithm studied is the Fisher-Yates algorithm introduced in \citeB{fisher_yates}. It may also be known as Knuth shuffle and was introduced to computer science by R. Durstenfeld in \citeA{durstenfeld} as algorithm 235. This algorithm uses an in place permutation approach and gives a perfect uniform random permutation.

The second algorithm proposed uses ideas from shuffling networks which is introduced in \citeA{psi} and the well known $bubble\text{-}sort$ algorithm. The idea is simple and use conditional swaps which swaps two inputs based on some condition. Tjis also yields a perfect uniform permutation.
An other type of shuffling networks called Bitonic shuffle network will be introduced and discussed. But no implementation of such a shuffle network was done.

All these shuffle algorithms is optimized to fit to the poker game introduced in the section on poker in chapter \ref{ch:intro}. This is done such that it only shuffles the first cards that are used during a game and not the whole deck.

%%%%%%%%%%%%%%%%%%%%%%% FISHER YATES %%%%%%%%%%%%%%%%%%%%%%%
%%%%%%%%%%%%%%%%%%%%%%%%%%%%%%%%%%%%%%%%%%%%%%%%%%%%%%%%%%%%
\section{Fisher-Yates}
\begin{algorithm}
\caption{\textbf{\textit{Fisher-Yates}} \newline
    $deck$ is initialised to hold $n$ cards $c$. \newline
    $seeds$ is initialised to hold $n$ random $r$ values where $r_i\in[i,n]$ for $i\in [1,n]$.
}
\label{fisher_yates_alg}

\begin{algorithmic}[1]
\Function{Swap}{card1, card2}
\State $tmp = card1$
\State $card1 = card2$
\State $card2 = tmp$
\EndFunction
\State
\Function{Shuffle}{deck, seeds}
\For{i=1 to n}
\State $r = seeds[i]$
\State \Call{Swap}{$deck[i],~deck[r]$}
\EndFor
\EndFunction
\end{algorithmic}
\end{algorithm}

$Fisher\text{-}Yates$ is a well known in place permutation algorithm that given two arrays as input one of some values that should be shuffled and another such that the first array is shuffle accordingly to the values of the second array. These swap values of the second array indicate where each of the original values should go in the swapp. When the algorithm runs through the first array which is supposed to be permuted it swaps the value at an given index whit the value specified by the swap value of the second array. Think of the input to be shuffled as a card deck then you take the top card of the deck and swap it with another card at a position predefined by the swap value.

This implies that the algorithm takes two inputs of the same size where the one is holding the values to be permuted, denoted $deck$ with $n$ $c_i$ values, for $i=0,\dots,n$. The other holding the values for which the different $c_i$ in the $deck$ is to be swapped, denoted the $seeds$ with $n$ $r_i$ values. If the swap values $r_i$ from the $seeds$ are not given in the correct interval the probability for the different permutations is not equally likely. Therefore it is important that the $r_i$ values are chosen accordingly to the algorithm. The algorithm states that $r_i$ is chosen from an interval starting with its own index $i$ to the size of the $deck$ which is $n$. This gives exactly the number of permutations required as the first card of $deck$ denoted $c_1$ has exactly $n$ possible places to go. $c_2$ has one less possible places to go an so forth until the algorithm reaches the last card $c_n$ which has no other place to go. Since $r_i\in[i,n]$ we have $n!$ because $i$ runs from $1$ to $n$ which should be the case as described in chapter \ref{ch:intro}.

If the values contained in $seeds$ is not chosen for the right interval but instead chosen on all $r_i$ from $0$ to $n$ we would end up having a skew on probability of the different permutations. As $r_i$ in this case has $n$ possible places to go yields $n^n$ distinct possible sequences of swaps. This introduces an error into the algorithm as there are only $n!$ and $n^n$ cannot be divisible by $n!$ for $n>2$. Resulting in a non uniform probability for the different permutations. The same is the problem if $r_i$ is not chosen from $[i,n]$ but instead $]i,n]$ such that the own index is not in the interval. By introducing this error to the algorithm the empty shuffle is not possible. In other words it is not possible to get the same output as the input. Which does not give the desired uniform distribution of permutations.

\bigskip

As described in the section on poker in chapter \ref{ch:intro} we only need a permutation of the first 20 cards. Which means that we only need the $\frac{52!}{32!}$ specific permutations out of the total of $52!$ different permutations. Doing a $m$ out of $n$ permutation using the $Fisher\text{-}Yates$ algorithm is straight forward. Instead of running through $n$ swaps indicated by the size of $seeds$ it is enough to run through $m$ swaps. Resulting in the input $seeds$ only need to have size $20$ and therefore a for-loop running fewer rounds. Those giving us a full permutation on the first $m$ indexes of $deck$.

\begin{figure}
\label{fisher_yates_fig}
\centering
\scalebox{1.5}{%LaTeX with PSTricks extensions
%%Creator: inkscape 0.91
%%Please note this file requires PSTricks extensions
\psset{xunit=.5pt,yunit=.5pt,runit=.5pt}
\begin{pspicture}(277.51337787,256.36721033)
{
\newrgbcolor{curcolor}{0 0 0}
\pscustom[linestyle=none,fillstyle=solid,fillcolor=curcolor]
{
\newpath
\moveto(8.54929742,246.17650195)
\curveto(9.07812013,246.17650195)(9.48103266,246.27723009)(9.75803503,246.47868635)
\curveto(10.0350374,246.68014262)(10.17353859,246.98232703)(10.17353859,247.38523956)
\curveto(10.17353859,247.62866589)(10.12317452,247.83431916)(10.02244639,248.00219938)
\curveto(9.92171825,248.17847362)(9.78741407,248.32956582)(9.61953385,248.45547599)
\curveto(9.45165363,248.58978017)(9.26278837,248.70309932)(9.05293809,248.79543344)
\curveto(8.84308781,248.89616157)(8.62904053,248.9884957)(8.41079624,249.07243581)
\curveto(8.1589759,249.17316394)(7.91554958,249.28648309)(7.68051727,249.41239326)
\curveto(7.45387896,249.53830343)(7.24822569,249.68519863)(7.06355744,249.85307885)
\curveto(6.88728321,250.02095907)(6.74458502,250.21821834)(6.63546287,250.44485664)
\curveto(6.53473474,250.67149494)(6.48437067,250.9401033)(6.48437067,251.25068171)
\curveto(6.48437067,251.93059662)(6.69841796,252.45941932)(7.12651253,252.83714982)
\curveto(7.5546071,253.22327434)(8.15058189,253.41633659)(8.91443691,253.41633659)
\curveto(9.12428719,253.41633659)(9.32994046,253.39954857)(9.53139673,253.36597253)
\curveto(9.74124701,253.34079049)(9.93430927,253.30721445)(10.1105835,253.26524439)
\curveto(10.28685774,253.22327434)(10.44214694,253.17291027)(10.57645112,253.11415219)
\curveto(10.71914931,253.05539411)(10.83246846,252.99663604)(10.91640858,252.93787796)
\lineto(10.58904214,252.0690978)
\curveto(10.42116192,252.16982593)(10.19032661,252.27055407)(9.89653622,252.3712822)
\curveto(9.60274583,252.47201034)(9.27537939,252.5223744)(8.91443691,252.5223744)
\curveto(8.53670641,252.5223744)(8.20933997,252.42584328)(7.9323376,252.23278102)
\curveto(7.65533523,252.04811277)(7.51683405,251.7669134)(7.51683405,251.38918289)
\curveto(7.51683405,251.1709386)(7.5546071,250.98627036)(7.6301532,250.83517816)
\curveto(7.71409331,250.68408596)(7.82321546,250.54978178)(7.95751964,250.43226562)
\curveto(8.10021783,250.32314347)(8.26390104,250.22241534)(8.44856929,250.13008122)
\curveto(8.63323754,250.03774709)(8.8346938,249.94541297)(9.05293809,249.85307885)
\curveto(9.37191052,249.71877467)(9.6615039,249.58447049)(9.92171825,249.45016631)
\curveto(10.19032661,249.31586213)(10.41696491,249.15637592)(10.60163316,248.97170767)
\curveto(10.79469541,248.78703943)(10.94159061,248.56879514)(11.04231874,248.3169748)
\curveto(11.15144089,248.06515447)(11.20600196,247.75877306)(11.20600196,247.39783058)
\curveto(11.20600196,246.71791567)(10.97516665,246.19328997)(10.51349604,245.82395348)
\curveto(10.06021944,245.463011)(9.40548656,245.28253976)(8.54929742,245.28253976)
\curveto(8.27229505,245.28253976)(8.01208071,245.30352479)(7.76865438,245.34549485)
\curveto(7.52522806,245.37907089)(7.30698377,245.42523795)(7.11392151,245.48399603)
\curveto(6.92925327,245.54275411)(6.76557005,245.60151219)(6.62287186,245.66027026)
\curveto(6.48017367,245.71902834)(6.37105152,245.77358942)(6.29550542,245.82395348)
\lineto(6.61028084,246.68014262)
\curveto(6.77816106,246.5878085)(7.02578439,246.47868635)(7.35315083,246.35277619)
\curveto(7.68051727,246.23526003)(8.0792328,246.17650195)(8.54929742,246.17650195)
\closepath
}
}
{
\newrgbcolor{curcolor}{0 0 0}
\pscustom[linestyle=none,fillstyle=solid,fillcolor=curcolor]
{
\newpath
\moveto(15.15958123,251.42695594)
\curveto(15.94861828,251.42695594)(16.55718409,251.17933262)(16.98527866,250.68408596)
\curveto(17.41337323,250.19723331)(17.62742051,249.45436332)(17.62742051,248.45547599)
\lineto(17.62742051,248.0903365)
\lineto(13.44720295,248.0903365)
\curveto(13.489173,247.4859677)(13.68643227,247.02429708)(14.03898073,246.70532466)
\curveto(14.39992322,246.39474624)(14.90356389,246.23945704)(15.54990275,246.23945704)
\curveto(15.91923924,246.23945704)(16.23401466,246.26883608)(16.494229,246.32759415)
\curveto(16.75444335,246.38635223)(16.95170261,246.44930732)(17.08600679,246.5164594)
\lineto(17.22450798,245.63508823)
\curveto(17.09859781,245.56793614)(16.87195951,245.49658705)(16.54459307,245.42104095)
\curveto(16.21722664,245.34549485)(15.84789014,245.3077218)(15.4365836,245.3077218)
\curveto(14.93294293,245.3077218)(14.48806033,245.3832679)(14.10193582,245.5343601)
\curveto(13.72420532,245.69384631)(13.4094299,245.90789359)(13.15760956,246.17650195)
\curveto(12.90578923,246.44511031)(12.71692397,246.76408273)(12.59101381,247.13341923)
\curveto(12.46510364,247.51114973)(12.40214855,247.91825927)(12.40214855,248.35474785)
\curveto(12.40214855,248.87517655)(12.48189166,249.32845315)(12.64137787,249.71457766)
\curveto(12.80086409,250.10070218)(13.01071437,250.4196746)(13.27092871,250.67149494)
\curveto(13.53114306,250.92331527)(13.82493345,251.11218053)(14.15229989,251.23809069)
\curveto(14.47966632,251.36400086)(14.81542677,251.42695594)(15.15958123,251.42695594)
\closepath
\moveto(16.56977511,248.93393462)
\curveto(16.56977511,249.42918128)(16.43966793,249.8195028)(16.17945359,250.10489918)
\curveto(15.91923924,250.39868958)(15.57508478,250.54558477)(15.14699021,250.54558477)
\curveto(14.90356389,250.54558477)(14.68112259,250.49941771)(14.47966632,250.40708359)
\curveto(14.28660406,250.31474946)(14.11872384,250.1930363)(13.97602565,250.0419441)
\curveto(13.83332746,249.8908519)(13.72000831,249.71877467)(13.6360682,249.52571241)
\curveto(13.55212809,249.33265016)(13.49756701,249.13539089)(13.47238498,248.93393462)
\lineto(16.56977511,248.93393462)
\closepath
}
}
{
\newrgbcolor{curcolor}{0 0 0}
\pscustom[linestyle=none,fillstyle=solid,fillcolor=curcolor]
{
\newpath
\moveto(21.45508961,251.42695594)
\curveto(22.24412666,251.42695594)(22.85269247,251.17933262)(23.28078704,250.68408596)
\curveto(23.70888161,250.19723331)(23.9229289,249.45436332)(23.9229289,248.45547599)
\lineto(23.9229289,248.0903365)
\lineto(19.74271133,248.0903365)
\curveto(19.78468139,247.4859677)(19.98194065,247.02429708)(20.33448912,246.70532466)
\curveto(20.6954316,246.39474624)(21.19907227,246.23945704)(21.84541113,246.23945704)
\curveto(22.21474762,246.23945704)(22.52952304,246.26883608)(22.78973739,246.32759415)
\curveto(23.04995174,246.38635223)(23.247211,246.44930732)(23.38151518,246.5164594)
\lineto(23.52001636,245.63508823)
\curveto(23.39410619,245.56793614)(23.16746789,245.49658705)(22.84010146,245.42104095)
\curveto(22.51273502,245.34549485)(22.14339853,245.3077218)(21.73209198,245.3077218)
\curveto(21.22845131,245.3077218)(20.78356872,245.3832679)(20.3974442,245.5343601)
\curveto(20.0197137,245.69384631)(19.70493828,245.90789359)(19.45311795,246.17650195)
\curveto(19.20129761,246.44511031)(19.01243236,246.76408273)(18.88652219,247.13341923)
\curveto(18.76061202,247.51114973)(18.69765694,247.91825927)(18.69765694,248.35474785)
\curveto(18.69765694,248.87517655)(18.77740005,249.32845315)(18.93688626,249.71457766)
\curveto(19.09637247,250.10070218)(19.30622275,250.4196746)(19.5664371,250.67149494)
\curveto(19.82665144,250.92331527)(20.12044183,251.11218053)(20.44780827,251.23809069)
\curveto(20.77517471,251.36400086)(21.11093515,251.42695594)(21.45508961,251.42695594)
\closepath
\moveto(22.86528349,248.93393462)
\curveto(22.86528349,249.42918128)(22.73517632,249.8195028)(22.47496197,250.10489918)
\curveto(22.21474762,250.39868958)(21.87059317,250.54558477)(21.4424986,250.54558477)
\curveto(21.19907227,250.54558477)(20.97663097,250.49941771)(20.77517471,250.40708359)
\curveto(20.58211245,250.31474946)(20.41423223,250.1930363)(20.27153404,250.0419441)
\curveto(20.12883585,249.8908519)(20.01551669,249.71877467)(19.93157658,249.52571241)
\curveto(19.84763647,249.33265016)(19.7930754,249.13539089)(19.76789336,248.93393462)
\lineto(22.86528349,248.93393462)
\closepath
}
}
{
\newrgbcolor{curcolor}{0 0 0}
\pscustom[linestyle=none,fillstyle=solid,fillcolor=curcolor]
{
\newpath
\moveto(28.8460155,250.05453512)
\curveto(28.72849935,250.16365726)(28.55642212,250.2643854)(28.32978381,250.35671952)
\curveto(28.10314551,250.45744765)(27.8681132,250.50781172)(27.62468688,250.50781172)
\curveto(27.34768451,250.50781172)(27.10845519,250.45325065)(26.90699892,250.3441285)
\curveto(26.71393666,250.23500636)(26.55445045,250.08391416)(26.42854028,249.8908519)
\curveto(26.30263011,249.70618365)(26.21029599,249.47954535)(26.15153791,249.21093699)
\curveto(26.09277984,248.95072265)(26.0634008,248.66952327)(26.0634008,248.36733887)
\curveto(26.0634008,247.68742396)(26.22288701,247.16279827)(26.54185943,246.79346177)
\curveto(26.86083186,246.42412528)(27.27633541,246.23945704)(27.78837009,246.23945704)
\curveto(28.04858444,246.23945704)(28.26682873,246.25204805)(28.44310297,246.27723009)
\curveto(28.62777121,246.30241212)(28.76207539,246.32759415)(28.8460155,246.35277619)
\lineto(28.8460155,250.05453512)
\closepath
\moveto(28.8460155,253.99552337)
\lineto(29.89106989,254.1717976)
\lineto(29.89106989,245.6099062)
\curveto(29.66443159,245.54275411)(29.37483821,245.47560202)(29.02228974,245.40844993)
\curveto(28.66974127,245.34129784)(28.25843472,245.3077218)(27.78837009,245.3077218)
\curveto(27.36866953,245.3077218)(26.98674203,245.37907089)(26.64258757,245.52176908)
\curveto(26.29843311,245.66446727)(26.00464272,245.86592354)(25.76121639,246.12613789)
\curveto(25.51779007,246.39474624)(25.32892482,246.71791567)(25.19462064,247.09564618)
\curveto(25.06031646,247.47337668)(24.99316437,247.89727424)(24.99316437,248.36733887)
\curveto(24.99316437,248.82061547)(25.04772544,249.23192202)(25.15684759,249.60125851)
\curveto(25.27436375,249.97898902)(25.44224397,250.30215845)(25.66048826,250.5707668)
\curveto(25.87873255,250.83937516)(26.1389469,251.04922544)(26.4411313,251.20031764)
\curveto(26.75170971,251.35140984)(27.10425818,251.42695594)(27.49877671,251.42695594)
\curveto(27.80935512,251.42695594)(28.08216048,251.38918289)(28.3171928,251.31363679)
\curveto(28.55222511,251.23809069)(28.72849935,251.15834759)(28.8460155,251.07440748)
\lineto(28.8460155,253.99552337)
\closepath
}
}
{
\newrgbcolor{curcolor}{0 0 0}
\pscustom[linestyle=none,fillstyle=solid,fillcolor=curcolor]
{
\newpath
\moveto(35.20447992,246.90678092)
\curveto(35.20447992,247.1166312)(35.11634281,247.28870843)(34.94006857,247.42301261)
\curveto(34.77218835,247.55731679)(34.55814106,247.67483295)(34.29792672,247.77556108)
\curveto(34.04610638,247.87628922)(33.76910401,247.97282034)(33.46691961,248.06515447)
\curveto(33.16473521,248.1658826)(32.88353583,248.28759576)(32.62332149,248.43029395)
\curveto(32.37150115,248.57299214)(32.15745387,248.74926638)(31.98117963,248.95911666)
\curveto(31.81329941,249.16896694)(31.7293593,249.4417723)(31.7293593,249.77753275)
\curveto(31.7293593,250.24759737)(31.91822455,250.63791889)(32.29595505,250.94849731)
\curveto(32.68207956,251.26746973)(33.28225136,251.42695594)(34.09647045,251.42695594)
\curveto(34.41544287,251.42695594)(34.74280931,251.40177391)(35.07856976,251.35140984)
\curveto(35.42272421,251.30943979)(35.71651461,251.25068171)(35.95994093,251.17513561)
\lineto(35.77107568,250.24340037)
\curveto(35.70392359,250.27697641)(35.61158947,250.31055246)(35.49407331,250.3441285)
\curveto(35.37655715,250.38609856)(35.24225297,250.4196746)(35.09116077,250.44485664)
\curveto(34.94006857,250.47843268)(34.77638535,250.50361472)(34.60011112,250.52040274)
\curveto(34.4322309,250.53719076)(34.26854768,250.54558477)(34.10906147,250.54558477)
\curveto(33.20250826,250.54558477)(32.74923165,250.29796144)(32.74923165,249.80271478)
\curveto(32.74923165,249.62644055)(32.83317177,249.47534835)(33.00105199,249.34943818)
\curveto(33.17732622,249.23192202)(33.39557052,249.12279988)(33.65578486,249.02207174)
\curveto(33.91599921,248.92134361)(34.19719858,248.81641847)(34.49938299,248.70729632)
\curveto(34.80156739,248.60656819)(35.08276676,248.48065802)(35.34298111,248.32956582)
\curveto(35.60319546,248.17847362)(35.81724274,247.99380537)(35.98512296,247.77556108)
\curveto(36.1613972,247.5657108)(36.24953432,247.29710244)(36.24953432,246.96973601)
\curveto(36.24953432,246.4409133)(36.04388104,246.02960676)(35.63257449,245.73581637)
\curveto(35.22126795,245.45041999)(34.57073208,245.3077218)(33.6809669,245.3077218)
\curveto(33.27805436,245.3077218)(32.90871787,245.34129784)(32.57295742,245.40844993)
\curveto(32.23719697,245.47560202)(31.91822455,245.57633015)(31.61604015,245.71063433)
\lineto(31.81749641,246.65496059)
\curveto(32.1112868,246.52065641)(32.41766821,246.40733726)(32.73664064,246.31500314)
\curveto(33.06400707,246.23106302)(33.41235854,246.18909297)(33.78169503,246.18909297)
\curveto(34.73021829,246.18909297)(35.20447992,246.42832229)(35.20447992,246.90678092)
\closepath
}
}
{
\newrgbcolor{curcolor}{0 0 0}
\pscustom[linestyle=none,fillstyle=solid,fillcolor=curcolor]
{
\newpath
\moveto(41.16003181,246.25204805)
\curveto(41.16003181,246.00022772)(41.0760917,245.77778642)(40.90821147,245.58472416)
\curveto(40.74033125,245.39166191)(40.51788995,245.29513078)(40.24088759,245.29513078)
\curveto(39.95549121,245.29513078)(39.7288529,245.39166191)(39.56097268,245.58472416)
\curveto(39.39309246,245.77778642)(39.30915234,246.00022772)(39.30915234,246.25204805)
\curveto(39.30915234,246.5122624)(39.39309246,246.7389007)(39.56097268,246.93196296)
\curveto(39.7288529,247.12502522)(39.95549121,247.22155634)(40.24088759,247.22155634)
\curveto(40.51788995,247.22155634)(40.74033125,247.12502522)(40.90821147,246.93196296)
\curveto(41.0760917,246.7389007)(41.16003181,246.5122624)(41.16003181,246.25204805)
\closepath
\moveto(41.16003181,250.31894647)
\curveto(41.16003181,250.06712613)(41.0760917,249.84468484)(40.90821147,249.65162258)
\curveto(40.74033125,249.45856032)(40.51788995,249.36202919)(40.24088759,249.36202919)
\curveto(39.95549121,249.36202919)(39.7288529,249.45856032)(39.56097268,249.65162258)
\curveto(39.39309246,249.84468484)(39.30915234,250.06712613)(39.30915234,250.31894647)
\curveto(39.30915234,250.57916082)(39.39309246,250.80579912)(39.56097268,250.99886137)
\curveto(39.7288529,251.19192363)(39.95549121,251.28845476)(40.24088759,251.28845476)
\curveto(40.51788995,251.28845476)(40.74033125,251.19192363)(40.90821147,250.99886137)
\curveto(41.0760917,250.80579912)(41.16003181,250.57916082)(41.16003181,250.31894647)
\closepath
}
}
{
\newrgbcolor{curcolor}{0 0 0}
\pscustom[linestyle=none,fillstyle=solid,fillcolor=curcolor]
{
\newpath
\moveto(17.69037607,229.18181275)
\curveto(17.69037607,228.45992779)(17.59804195,227.84716497)(17.41337371,227.3435243)
\curveto(17.23709947,226.83988363)(16.98947614,226.42857708)(16.67050372,226.10960466)
\curveto(16.3599253,225.79902625)(15.98639181,225.57238794)(15.54990322,225.42968975)
\curveto(15.12180865,225.28699156)(14.65594103,225.21564247)(14.15230036,225.21564247)
\curveto(13.63187167,225.21564247)(13.10724597,225.27859755)(12.57842327,225.40450772)
\lineto(12.57842327,232.95911778)
\curveto(13.10724597,233.08502795)(13.63187167,233.14798303)(14.15230036,233.14798303)
\curveto(14.65594103,233.14798303)(15.12180865,233.07663394)(15.54990322,232.93393575)
\curveto(15.98639181,232.79123756)(16.3599253,232.56040225)(16.67050372,232.24142983)
\curveto(16.98947614,231.9224574)(17.23709947,231.51115085)(17.41337371,231.00751018)
\curveto(17.59804195,230.50386951)(17.69037607,229.8953037)(17.69037607,229.18181275)
\closepath
\moveto(13.61088664,226.14737771)
\curveto(13.82073692,226.12219568)(14.03478421,226.10960466)(14.2530285,226.10960466)
\curveto(14.630759,226.10960466)(14.96651945,226.16416573)(15.26030984,226.27328788)
\curveto(15.55410023,226.39080403)(15.80172356,226.57127527)(16.00317983,226.8147016)
\curveto(16.2046361,227.06652193)(16.3599253,227.38549436)(16.46904745,227.77161887)
\curveto(16.57816959,228.1661374)(16.63273067,228.63620202)(16.63273067,229.18181275)
\curveto(16.63273067,230.23945816)(16.42707739,231.01590419)(16.01577084,231.51115085)
\curveto(15.61285831,232.00639751)(15.01268651,232.25402084)(14.21525545,232.25402084)
\curveto(14.1061333,232.25402084)(13.99701116,232.24982384)(13.88788901,232.24142983)
\curveto(13.78716088,232.24142983)(13.69482675,232.23303581)(13.61088664,232.21624779)
\lineto(13.61088664,226.14737771)
\closepath
}
}
{
\newrgbcolor{curcolor}{0 0 0}
\pscustom[linestyle=none,fillstyle=solid,fillcolor=curcolor]
{
\newpath
\moveto(21.45509009,231.25933052)
\curveto(22.24412714,231.25933052)(22.85269295,231.01170719)(23.28078752,230.51646053)
\curveto(23.70888209,230.02960788)(23.92292938,229.28673789)(23.92292938,228.28785056)
\lineto(23.92292938,227.92271107)
\lineto(19.74271181,227.92271107)
\curveto(19.78468186,227.31834227)(19.98194113,226.85667165)(20.3344896,226.53769923)
\curveto(20.69543208,226.22712082)(21.19907275,226.07183161)(21.84541161,226.07183161)
\curveto(22.2147481,226.07183161)(22.52952352,226.10121065)(22.78973787,226.15996873)
\curveto(23.04995221,226.2187268)(23.24721148,226.28168189)(23.38151565,226.34883398)
\lineto(23.52001684,225.4674628)
\curveto(23.39410667,225.40031071)(23.16746837,225.32896162)(22.84010193,225.25341552)
\curveto(22.5127355,225.17786942)(22.14339901,225.14009637)(21.73209246,225.14009637)
\curveto(21.22845179,225.14009637)(20.78356919,225.21564247)(20.39744468,225.36673467)
\curveto(20.01971418,225.52622088)(19.70493876,225.74026817)(19.45311842,226.00887652)
\curveto(19.20129809,226.27748488)(19.01243284,226.59645731)(18.88652267,226.9657938)
\curveto(18.7606125,227.3435243)(18.69765742,227.75063384)(18.69765742,228.18712243)
\curveto(18.69765742,228.70755112)(18.77740052,229.16082772)(18.93688674,229.54695224)
\curveto(19.09637295,229.93307675)(19.30622323,230.25204918)(19.56643757,230.50386951)
\curveto(19.82665192,230.75568985)(20.12044231,230.9445551)(20.44780875,231.07046527)
\curveto(20.77517518,231.19637543)(21.11093563,231.25933052)(21.45509009,231.25933052)
\closepath
\moveto(22.86528397,228.7663092)
\curveto(22.86528397,229.26155586)(22.73517679,229.65187738)(22.47496245,229.93727376)
\curveto(22.2147481,230.23106415)(21.87059364,230.37795934)(21.44249907,230.37795934)
\curveto(21.19907275,230.37795934)(20.97663145,230.33179228)(20.77517518,230.23945816)
\curveto(20.58211293,230.14712404)(20.4142327,230.02541087)(20.27153451,229.87431867)
\curveto(20.12883632,229.72322647)(20.01551717,229.55114924)(19.93157706,229.35808699)
\curveto(19.84763695,229.16502473)(19.79307588,228.96776547)(19.76789384,228.7663092)
\lineto(22.86528397,228.7663092)
\closepath
}
}
{
\newrgbcolor{curcolor}{0 0 0}
\pscustom[linestyle=none,fillstyle=solid,fillcolor=curcolor]
{
\newpath
\moveto(25.10648495,228.18712243)
\curveto(25.10648495,228.71594513)(25.19042506,229.17341874)(25.35830529,229.55954325)
\curveto(25.52618551,229.94566777)(25.75702082,230.26464019)(26.05081121,230.51646053)
\curveto(26.3446016,230.76828086)(26.68455905,230.95294911)(27.07068357,231.07046527)
\curveto(27.46520209,231.19637543)(27.88490265,231.25933052)(28.32978525,231.25933052)
\curveto(28.61518163,231.25933052)(28.896381,231.23834549)(29.17338337,231.19637543)
\curveto(29.45877975,231.16279939)(29.76096415,231.0956473)(30.07993658,230.99491917)
\lineto(29.84070726,230.10095697)
\curveto(29.56370489,230.20168511)(29.30768755,230.26464019)(29.07265523,230.28982223)
\curveto(28.84601693,230.32339827)(28.61518163,230.34018629)(28.38014931,230.34018629)
\curveto(28.07796491,230.34018629)(27.79256853,230.29821624)(27.52396017,230.21427613)
\curveto(27.25535181,230.13873003)(27.0203195,230.01281986)(26.81886323,229.83654562)
\curveto(26.62580098,229.6686654)(26.47051177,229.4462241)(26.35299561,229.16922173)
\curveto(26.23547946,228.90061338)(26.17672138,228.57324694)(26.17672138,228.18712243)
\curveto(26.17672138,227.81778593)(26.23128245,227.49881351)(26.3404046,227.23020515)
\curveto(26.44952674,226.9699908)(26.60061894,226.75174651)(26.7936812,226.57547228)
\curveto(26.99513747,226.40759206)(27.23436679,226.28168189)(27.51136916,226.19774178)
\curveto(27.78837152,226.11380166)(28.09475293,226.07183161)(28.43051338,226.07183161)
\curveto(28.69912174,226.07183161)(28.95513908,226.08442263)(29.1985654,226.10960466)
\curveto(29.45038574,226.1431807)(29.7231911,226.21033279)(30.01698149,226.31106093)
\lineto(30.16807369,225.44228077)
\curveto(29.8742833,225.33315862)(29.57629591,225.25761252)(29.2741115,225.21564247)
\curveto(28.9719271,225.1652784)(28.64456066,225.14009637)(28.29201219,225.14009637)
\curveto(27.82194757,225.14009637)(27.38965599,225.20305145)(26.99513747,225.32896162)
\curveto(26.60901295,225.4632658)(26.27325251,225.65632806)(25.98785613,225.90814839)
\curveto(25.71085376,226.15996873)(25.49260947,226.47474415)(25.33312325,226.85247465)
\curveto(25.18203105,227.23859916)(25.10648495,227.68348176)(25.10648495,228.18712243)
\closepath
}
}
{
\newrgbcolor{curcolor}{0 0 0}
\pscustom[linestyle=none,fillstyle=solid,fillcolor=curcolor]
{
\newpath
\moveto(33.78169455,228.48930683)
\curveto(34.00833285,228.3214266)(34.2643502,228.10737932)(34.54974658,227.84716497)
\curveto(34.83514296,227.59534464)(35.11634233,227.32253927)(35.3933447,227.02874888)
\curveto(35.67874108,226.73495849)(35.94734944,226.43277409)(36.19916977,226.12219568)
\curveto(36.45099011,225.82001127)(36.65244638,225.5388119)(36.80353858,225.27859755)
\lineto(35.56961893,225.27859755)
\curveto(35.41013272,225.5388119)(35.21287346,225.80322325)(34.97784115,226.07183161)
\curveto(34.74280883,226.34883398)(34.4909885,226.61324533)(34.22238014,226.86506567)
\curveto(33.96216579,227.116886)(33.69775444,227.34772131)(33.42914608,227.55757159)
\curveto(33.16893174,227.76742187)(32.93389942,227.9436961)(32.72404914,228.08639429)
\lineto(32.72404914,225.27859755)
\lineto(31.67899475,225.27859755)
\lineto(31.67899475,233.82789794)
\lineto(32.72404914,234.00417217)
\lineto(32.72404914,228.67817208)
\curveto(33.18571976,229.08108462)(33.64739037,229.47980015)(34.10906099,229.87431867)
\curveto(34.5707316,230.27723121)(34.98623516,230.69273476)(35.35557165,231.12082933)
\lineto(36.57690028,231.12082933)
\curveto(36.21595779,230.69273476)(35.77946921,230.25204918)(35.26743453,229.79877257)
\curveto(34.76379386,229.34549597)(34.2685472,228.90900739)(33.78169455,228.48930683)
\closepath
}
}
{
\newrgbcolor{curcolor}{0 0 0}
\pscustom[linestyle=none,fillstyle=solid,fillcolor=curcolor]
{
\newpath
\moveto(41.16003133,226.08442263)
\curveto(41.16003133,225.83260229)(41.07609122,225.61016099)(40.908211,225.41709874)
\curveto(40.74033077,225.22403648)(40.51788948,225.12750535)(40.24088711,225.12750535)
\curveto(39.95549073,225.12750535)(39.72885243,225.22403648)(39.5609722,225.41709874)
\curveto(39.39309198,225.61016099)(39.30915187,225.83260229)(39.30915187,226.08442263)
\curveto(39.30915187,226.34463697)(39.39309198,226.57127527)(39.5609722,226.76433753)
\curveto(39.72885243,226.95739979)(39.95549073,227.05393092)(40.24088711,227.05393092)
\curveto(40.51788948,227.05393092)(40.74033077,226.95739979)(40.908211,226.76433753)
\curveto(41.07609122,226.57127527)(41.16003133,226.34463697)(41.16003133,226.08442263)
\closepath
\moveto(41.16003133,230.15132104)
\curveto(41.16003133,229.89950071)(41.07609122,229.67705941)(40.908211,229.48399715)
\curveto(40.74033077,229.2909349)(40.51788948,229.19440377)(40.24088711,229.19440377)
\curveto(39.95549073,229.19440377)(39.72885243,229.2909349)(39.5609722,229.48399715)
\curveto(39.39309198,229.67705941)(39.30915187,229.89950071)(39.30915187,230.15132104)
\curveto(39.30915187,230.41153539)(39.39309198,230.63817369)(39.5609722,230.83123595)
\curveto(39.72885243,231.0242982)(39.95549073,231.12082933)(40.24088711,231.12082933)
\curveto(40.51788948,231.12082933)(40.74033077,231.0242982)(40.908211,230.83123595)
\curveto(41.07609122,230.63817369)(41.16003133,230.41153539)(41.16003133,230.15132104)
\closepath
}
}
{
\newrgbcolor{curcolor}{0 0 0}
\pscustom[linestyle=none,fillstyle=solid,fillcolor=curcolor]
{
\newpath
\moveto(62.57650865,252.00614354)
\curveto(63.00460322,252.17402376)(63.42010677,252.38387404)(63.82301931,252.63569437)
\curveto(64.22593184,252.89590872)(64.59946534,253.22327516)(64.9436198,253.61779368)
\lineto(65.67389877,253.61779368)
\lineto(65.67389877,246.70532548)
\lineto(67.14704773,246.70532548)
\lineto(67.14704773,245.8239543)
\lineto(62.95423915,245.8239543)
\lineto(62.95423915,246.70532548)
\lineto(64.6414354,246.70532548)
\lineto(64.6414354,252.16982675)
\curveto(64.54910127,252.08588664)(64.43578212,251.99774953)(64.30147794,251.9054154)
\curveto(64.17556778,251.82147529)(64.03286959,251.73753518)(63.87338337,251.65359507)
\curveto(63.72229117,251.56965496)(63.56280496,251.48991185)(63.39492474,251.41436575)
\curveto(63.22704451,251.33881965)(63.0633613,251.27586456)(62.90387508,251.2255005)
\lineto(62.57650865,252.00614354)
\closepath
}
}
{
\newrgbcolor{curcolor}{0 0 0}
\pscustom[linestyle=none,fillstyle=solid,fillcolor=curcolor]
{
\newpath
\moveto(76.56512732,250.60854068)
\curveto(77.68992482,250.56657062)(78.50834091,250.31894729)(79.02037559,249.86567069)
\curveto(79.53241028,249.41239408)(79.78842762,248.80382827)(79.78842762,248.03997325)
\curveto(79.78842762,247.6958188)(79.73386654,247.37684637)(79.6247444,247.08305598)
\curveto(79.51562225,246.78926559)(79.34354502,246.53744525)(79.10851271,246.32759497)
\curveto(78.88187441,246.11774469)(78.58808402,245.95406148)(78.22714154,245.83654532)
\curveto(77.87459307,245.71902916)(77.45489251,245.66027108)(76.96803986,245.66027108)
\curveto(76.76658359,245.66027108)(76.56512732,245.67705911)(76.36367106,245.71063515)
\curveto(76.1706088,245.73581719)(75.98594055,245.76939323)(75.80966632,245.81136329)
\curveto(75.63339208,245.85333334)(75.47810288,245.8953034)(75.3437987,245.93727345)
\curveto(75.20949452,245.98763752)(75.11296339,246.02960758)(75.05420531,246.06318362)
\lineto(75.25566158,246.95714581)
\curveto(75.38996576,246.88999372)(75.59561903,246.81025062)(75.8726214,246.71791649)
\curveto(76.15801778,246.62558237)(76.51476326,246.57941531)(76.94285783,246.57941531)
\curveto(77.27861827,246.57941531)(77.55981765,246.61718836)(77.78645595,246.69273446)
\curveto(78.01309425,246.76828056)(78.19356549,246.86900869)(78.32786967,246.99491886)
\curveto(78.47056786,247.12922304)(78.571296,247.28031524)(78.63005407,247.44819547)
\curveto(78.69720616,247.61607569)(78.73078221,247.79234992)(78.73078221,247.97701817)
\curveto(78.73078221,248.26241455)(78.68041814,248.51423489)(78.57969001,248.73247918)
\curveto(78.48735588,248.95911748)(78.31947566,249.14798273)(78.07604934,249.29907493)
\curveto(77.84101702,249.45016713)(77.51365059,249.56348628)(77.09395003,249.63903238)
\curveto(76.68264348,249.7229725)(76.16221479,249.76494255)(75.53266395,249.76494255)
\curveto(75.58302802,250.13427904)(75.62080107,250.4784335)(75.6459831,250.79740593)
\curveto(75.67955914,251.12477236)(75.70474118,251.43954778)(75.7215292,251.74173218)
\curveto(75.74671123,252.0523106)(75.76349926,252.35869201)(75.77189327,252.66087641)
\curveto(75.78868129,252.96306081)(75.80546931,253.28203324)(75.82225733,253.61779368)
\lineto(79.57438033,253.61779368)
\lineto(79.57438033,252.73642251)
\lineto(76.72881054,252.73642251)
\curveto(76.72041653,252.61890635)(76.70782551,252.46361715)(76.69103749,252.27055489)
\curveto(76.68264348,252.08588664)(76.67005246,251.88862738)(76.65326444,251.6787771)
\curveto(76.63647642,251.46892682)(76.6196884,251.26747055)(76.60290037,251.0744083)
\curveto(76.58611235,250.88134604)(76.57352134,250.72605683)(76.56512732,250.60854068)
\closepath
}
}
{
\newrgbcolor{curcolor}{0 0 0}
\pscustom[linestyle=none,fillstyle=solid,fillcolor=curcolor]
{
\newpath
\moveto(81.4630338,252.00614354)
\curveto(81.89112837,252.17402376)(82.30663192,252.38387404)(82.70954446,252.63569437)
\curveto(83.112457,252.89590872)(83.4859905,253.22327516)(83.83014495,253.61779368)
\lineto(84.56042393,253.61779368)
\lineto(84.56042393,246.70532548)
\lineto(86.03357289,246.70532548)
\lineto(86.03357289,245.8239543)
\lineto(81.8407643,245.8239543)
\lineto(81.8407643,246.70532548)
\lineto(83.52796055,246.70532548)
\lineto(83.52796055,252.16982675)
\curveto(83.43562643,252.08588664)(83.32230728,251.99774953)(83.1880031,251.9054154)
\curveto(83.06209293,251.82147529)(82.91939474,251.73753518)(82.75990853,251.65359507)
\curveto(82.60881633,251.56965496)(82.44933011,251.48991185)(82.28144989,251.41436575)
\curveto(82.11356967,251.33881965)(81.94988645,251.27586456)(81.79040024,251.2255005)
\lineto(81.4630338,252.00614354)
\closepath
}
}
{
\newrgbcolor{curcolor}{0 0 0}
\pscustom[linestyle=none,fillstyle=solid,fillcolor=curcolor]
{
\newpath
\moveto(94.05404866,252.00614354)
\curveto(94.48214323,252.17402376)(94.89764679,252.38387404)(95.30055932,252.63569437)
\curveto(95.70347186,252.89590872)(96.07700536,253.22327516)(96.42115982,253.61779368)
\lineto(97.15143879,253.61779368)
\lineto(97.15143879,246.70532548)
\lineto(98.62458775,246.70532548)
\lineto(98.62458775,245.8239543)
\lineto(94.43177917,245.8239543)
\lineto(94.43177917,246.70532548)
\lineto(96.11897541,246.70532548)
\lineto(96.11897541,252.16982675)
\curveto(96.02664129,252.08588664)(95.91332214,251.99774953)(95.77901796,251.9054154)
\curveto(95.65310779,251.82147529)(95.5104096,251.73753518)(95.35092339,251.65359507)
\curveto(95.19983119,251.56965496)(95.04034498,251.48991185)(94.87246475,251.41436575)
\curveto(94.70458453,251.33881965)(94.54090131,251.27586456)(94.3814151,251.2255005)
\lineto(94.05404866,252.00614354)
\closepath
}
}
{
\newrgbcolor{curcolor}{0 0 0}
\pscustom[linestyle=none,fillstyle=solid,fillcolor=curcolor]
{
\newpath
\moveto(99.85850835,248.51843189)
\curveto(100.00120654,248.85419234)(100.19426879,249.24031685)(100.43769512,249.67680543)
\curveto(100.68951545,250.11329402)(100.97071483,250.56237361)(101.28129324,251.02404423)
\curveto(101.59187166,251.49410885)(101.9234351,251.95158246)(102.27598357,252.39646506)
\curveto(102.62853204,252.84974166)(102.98527751,253.2568512)(103.34621999,253.61779368)
\lineto(104.35350133,253.61779368)
\lineto(104.35350133,248.66952409)
\lineto(105.27264556,248.66952409)
\lineto(105.27264556,247.81333495)
\lineto(104.35350133,247.81333495)
\lineto(104.35350133,245.8239543)
\lineto(103.34621999,245.8239543)
\lineto(103.34621999,247.81333495)
\lineto(99.85850835,247.81333495)
\lineto(99.85850835,248.51843189)
\closepath
\moveto(103.34621999,252.38387404)
\curveto(103.11958169,252.14044772)(102.88874638,251.87183936)(102.65371407,251.57804897)
\curveto(102.42707577,251.28425858)(102.20463447,250.97368016)(101.98639018,250.64631373)
\curveto(101.76814589,250.3273413)(101.56249262,249.99997486)(101.36943036,249.66421442)
\curveto(101.18476211,249.32845397)(101.01688189,248.99689053)(100.86578969,248.66952409)
\lineto(103.34621999,248.66952409)
\lineto(103.34621999,252.38387404)
\closepath
}
}
{
\newrgbcolor{curcolor}{0 0 0}
\pscustom[linestyle=none,fillstyle=solid,fillcolor=curcolor]
{
\newpath
\moveto(117.28447746,251.62841303)
\curveto(117.28447746,251.35980468)(117.22991639,251.09959033)(117.12079425,250.84776999)
\curveto(117.02006611,250.59594966)(116.88156493,250.34832633)(116.70529069,250.1049)
\curveto(116.53741047,249.86147368)(116.34434821,249.62224436)(116.12610392,249.38721205)
\curveto(115.90785963,249.15217974)(115.68541833,248.92134443)(115.45878003,248.69470613)
\curveto(115.33286986,248.56879596)(115.18597467,248.41770376)(115.01809445,248.24142952)
\curveto(114.85021422,248.06515529)(114.69072801,247.88468405)(114.53963581,247.7000158)
\curveto(114.38854361,247.51534756)(114.26263344,247.33487632)(114.16190531,247.15860208)
\curveto(114.06117717,246.98232785)(114.0108131,246.83123564)(114.0108131,246.70532548)
\lineto(117.59925288,246.70532548)
\lineto(117.59925288,245.8239543)
\lineto(112.87762159,245.8239543)
\curveto(112.86922758,245.86592436)(112.86503058,245.90789441)(112.86503058,245.94986447)
\lineto(112.86503058,246.08836565)
\curveto(112.86503058,246.44091412)(112.92378866,246.76828056)(113.04130481,247.07046496)
\curveto(113.15882097,247.37264937)(113.30991317,247.65804575)(113.49458142,247.9266541)
\curveto(113.67924966,248.19526246)(113.88490294,248.4470828)(114.11154124,248.68211511)
\curveto(114.34657355,248.92554143)(114.57740886,249.16057375)(114.80404716,249.38721205)
\curveto(114.98871541,249.57188029)(115.16498964,249.75235153)(115.33286986,249.92862577)
\curveto(115.5091441,250.1049)(115.6602363,250.28117424)(115.78614647,250.45744847)
\curveto(115.92045065,250.63372271)(116.02537579,250.81419395)(116.10092189,250.99886219)
\curveto(116.184862,251.19192445)(116.22683206,251.38918371)(116.22683206,251.59063998)
\curveto(116.22683206,251.81727828)(116.189059,252.01034054)(116.1135129,252.16982675)
\curveto(116.04636081,252.32931297)(115.94982969,252.45942014)(115.82391952,252.56014827)
\curveto(115.70640336,252.66927042)(115.56790218,252.74901353)(115.40841597,252.79937759)
\curveto(115.24892975,252.84974166)(115.07685252,252.87492369)(114.89218428,252.87492369)
\curveto(114.67393999,252.87492369)(114.47248372,252.84554465)(114.28781547,252.78678658)
\curveto(114.11154124,252.7280285)(113.95205503,252.6566794)(113.80935684,252.57273929)
\curveto(113.67505266,252.48879918)(113.5575365,252.40485907)(113.45680837,252.32091896)
\curveto(113.35608023,252.24537286)(113.28053413,252.18241777)(113.23017006,252.1320537)
\lineto(112.71393838,252.86233268)
\curveto(112.78109047,252.93787878)(112.8818186,253.0302129)(113.01612278,253.13933505)
\curveto(113.15042696,253.24845719)(113.30991317,253.34918533)(113.49458142,253.44151945)
\curveto(113.68764367,253.54224758)(113.90169096,253.62618769)(114.13672327,253.69333978)
\curveto(114.37175558,253.76049187)(114.62357592,253.79406792)(114.89218428,253.79406792)
\curveto(115.70640336,253.79406792)(116.30657516,253.60520267)(116.69269968,253.22747216)
\curveto(117.0872182,252.85813567)(117.28447746,252.32511596)(117.28447746,251.62841303)
\closepath
}
}
{
\newrgbcolor{curcolor}{0 0 0}
\pscustom[linestyle=none,fillstyle=solid,fillcolor=curcolor]
{
\newpath
\moveto(122.16979292,249.86567069)
\curveto(122.16979292,249.6474264)(122.10264083,249.45856114)(121.96833666,249.29907493)
\curveto(121.84242649,249.13958872)(121.67454626,249.05984561)(121.46469598,249.05984561)
\curveto(121.24645169,249.05984561)(121.07017746,249.13958872)(120.93587328,249.29907493)
\curveto(120.8015691,249.45856114)(120.73441701,249.6474264)(120.73441701,249.86567069)
\curveto(120.73441701,250.08391498)(120.8015691,250.27697723)(120.93587328,250.44485746)
\curveto(121.07017746,250.61273768)(121.24645169,250.69667779)(121.46469598,250.69667779)
\curveto(121.67454626,250.69667779)(121.84242649,250.61273768)(121.96833666,250.44485746)
\curveto(122.10264083,250.27697723)(122.16979292,250.08391498)(122.16979292,249.86567069)
\closepath
\moveto(118.87094653,249.7271695)
\curveto(118.87094653,251.03663525)(119.09338783,252.03971958)(119.53827042,252.73642251)
\curveto(119.99154702,253.44151945)(120.62529487,253.79406792)(121.43951395,253.79406792)
\curveto(122.26212705,253.79406792)(122.89587489,253.44151945)(123.34075748,252.73642251)
\curveto(123.78564008,252.03971958)(124.00808137,251.03663525)(124.00808137,249.7271695)
\curveto(124.00808137,248.41770376)(123.78564008,247.41042242)(123.34075748,246.70532548)
\curveto(122.89587489,246.00862255)(122.26212705,245.66027108)(121.43951395,245.66027108)
\curveto(120.62529487,245.66027108)(119.99154702,246.00862255)(119.53827042,246.70532548)
\curveto(119.09338783,247.41042242)(118.87094653,248.41770376)(118.87094653,249.7271695)
\closepath
\moveto(122.95043596,249.7271695)
\curveto(122.95043596,250.15526407)(122.92525393,250.55817661)(122.87488986,250.93590711)
\curveto(122.8245258,251.32203163)(122.74058568,251.65779207)(122.62306953,251.94318845)
\curveto(122.50555337,252.22858483)(122.35026416,252.45522313)(122.15720191,252.62310336)
\curveto(121.96413965,252.79098358)(121.72491033,252.87492369)(121.43951395,252.87492369)
\curveto(121.15411757,252.87492369)(120.91488825,252.79098358)(120.721826,252.62310336)
\curveto(120.52876374,252.45522313)(120.37347453,252.22858483)(120.25595838,251.94318845)
\curveto(120.13844222,251.65779207)(120.05450211,251.32203163)(120.00413804,250.93590711)
\curveto(119.95377397,250.55817661)(119.92859194,250.15526407)(119.92859194,249.7271695)
\curveto(119.92859194,249.29907493)(119.95377397,248.89196539)(120.00413804,248.50584087)
\curveto(120.05450211,248.12811037)(120.13844222,247.79654693)(120.25595838,247.51115055)
\curveto(120.37347453,247.22575417)(120.52876374,246.99911587)(120.721826,246.83123564)
\curveto(120.91488825,246.66335542)(121.15411757,246.57941531)(121.43951395,246.57941531)
\curveto(121.72491033,246.57941531)(121.96413965,246.66335542)(122.15720191,246.83123564)
\curveto(122.35026416,246.99911587)(122.50555337,247.22575417)(122.62306953,247.51115055)
\curveto(122.74058568,247.79654693)(122.8245258,248.12811037)(122.87488986,248.50584087)
\curveto(122.92525393,248.89196539)(122.95043596,249.29907493)(122.95043596,249.7271695)
\closepath
}
}
{
\newrgbcolor{curcolor}{0 0 0}
\pscustom[linestyle=none,fillstyle=solid,fillcolor=curcolor]
{
\newpath
\moveto(131.82709706,252.00614354)
\curveto(132.25519163,252.17402376)(132.67069519,252.38387404)(133.07360772,252.63569437)
\curveto(133.47652026,252.89590872)(133.85005376,253.22327516)(134.19420822,253.61779368)
\lineto(134.92448719,253.61779368)
\lineto(134.92448719,246.70532548)
\lineto(136.39763615,246.70532548)
\lineto(136.39763615,245.8239543)
\lineto(132.20482757,245.8239543)
\lineto(132.20482757,246.70532548)
\lineto(133.89202381,246.70532548)
\lineto(133.89202381,252.16982675)
\curveto(133.79968969,252.08588664)(133.68637054,251.99774953)(133.55206636,251.9054154)
\curveto(133.42615619,251.82147529)(133.283458,251.73753518)(133.12397179,251.65359507)
\curveto(132.97287959,251.56965496)(132.81339338,251.48991185)(132.64551315,251.41436575)
\curveto(132.47763293,251.33881965)(132.31394971,251.27586456)(132.1544635,251.2255005)
\lineto(131.82709706,252.00614354)
\closepath
}
}
{
\newrgbcolor{curcolor}{0 0 0}
\pscustom[linestyle=none,fillstyle=solid,fillcolor=curcolor]
{
\newpath
\moveto(141.05631331,249.86567069)
\curveto(141.05631331,249.6474264)(140.98916122,249.45856114)(140.85485704,249.29907493)
\curveto(140.72894687,249.13958872)(140.56106665,249.05984561)(140.35121637,249.05984561)
\curveto(140.13297208,249.05984561)(139.95669785,249.13958872)(139.82239367,249.29907493)
\curveto(139.68808949,249.45856114)(139.6209374,249.6474264)(139.6209374,249.86567069)
\curveto(139.6209374,250.08391498)(139.68808949,250.27697723)(139.82239367,250.44485746)
\curveto(139.95669785,250.61273768)(140.13297208,250.69667779)(140.35121637,250.69667779)
\curveto(140.56106665,250.69667779)(140.72894687,250.61273768)(140.85485704,250.44485746)
\curveto(140.98916122,250.27697723)(141.05631331,250.08391498)(141.05631331,249.86567069)
\closepath
\moveto(137.75746692,249.7271695)
\curveto(137.75746692,251.03663525)(137.97990821,252.03971958)(138.42479081,252.73642251)
\curveto(138.87806741,253.44151945)(139.51181525,253.79406792)(140.32603434,253.79406792)
\curveto(141.14864743,253.79406792)(141.78239528,253.44151945)(142.22727787,252.73642251)
\curveto(142.67216046,252.03971958)(142.89460176,251.03663525)(142.89460176,249.7271695)
\curveto(142.89460176,248.41770376)(142.67216046,247.41042242)(142.22727787,246.70532548)
\curveto(141.78239528,246.00862255)(141.14864743,245.66027108)(140.32603434,245.66027108)
\curveto(139.51181525,245.66027108)(138.87806741,246.00862255)(138.42479081,246.70532548)
\curveto(137.97990821,247.41042242)(137.75746692,248.41770376)(137.75746692,249.7271695)
\closepath
\moveto(141.83695635,249.7271695)
\curveto(141.83695635,250.15526407)(141.81177432,250.55817661)(141.76141025,250.93590711)
\curveto(141.71104618,251.32203163)(141.62710607,251.65779207)(141.50958991,251.94318845)
\curveto(141.39207376,252.22858483)(141.23678455,252.45522313)(141.04372229,252.62310336)
\curveto(140.85066004,252.79098358)(140.61143072,252.87492369)(140.32603434,252.87492369)
\curveto(140.04063796,252.87492369)(139.80140864,252.79098358)(139.60834638,252.62310336)
\curveto(139.41528412,252.45522313)(139.25999492,252.22858483)(139.14247876,251.94318845)
\curveto(139.0249626,251.65779207)(138.94102249,251.32203163)(138.89065843,250.93590711)
\curveto(138.84029436,250.55817661)(138.81511232,250.15526407)(138.81511232,249.7271695)
\curveto(138.81511232,249.29907493)(138.84029436,248.89196539)(138.89065843,248.50584087)
\curveto(138.94102249,248.12811037)(139.0249626,247.79654693)(139.14247876,247.51115055)
\curveto(139.25999492,247.22575417)(139.41528412,246.99911587)(139.60834638,246.83123564)
\curveto(139.80140864,246.66335542)(140.04063796,246.57941531)(140.32603434,246.57941531)
\curveto(140.61143072,246.57941531)(140.85066004,246.66335542)(141.04372229,246.83123564)
\curveto(141.23678455,246.99911587)(141.39207376,247.22575417)(141.50958991,247.51115055)
\curveto(141.62710607,247.79654693)(141.71104618,248.12811037)(141.76141025,248.50584087)
\curveto(141.81177432,248.89196539)(141.83695635,249.29907493)(141.83695635,249.7271695)
\closepath
}
}
{
\newrgbcolor{curcolor}{0 0 0}
\pscustom[linestyle=none,fillstyle=solid,fillcolor=curcolor]
{
\newpath
\moveto(152.43859438,246.57941531)
\curveto(153.10172126,246.57941531)(153.57178589,246.70952248)(153.84878825,246.96973683)
\curveto(154.13418463,247.23834519)(154.27688283,247.59509066)(154.27688283,248.03997325)
\curveto(154.27688283,248.32536963)(154.21812475,248.56459895)(154.10060859,248.75766121)
\curveto(153.98309243,248.95072347)(153.82780323,249.10601267)(153.63474097,249.22352883)
\curveto(153.44167871,249.34104499)(153.21923742,249.4249851)(152.96741708,249.47534917)
\curveto(152.71559675,249.52571323)(152.45118539,249.55089527)(152.17418302,249.55089527)
\lineto(151.90977167,249.55089527)
\lineto(151.90977167,250.39449339)
\lineto(152.27491116,250.39449339)
\curveto(152.4595794,250.39449339)(152.64844466,250.41128141)(152.84150691,250.44485746)
\curveto(153.04296318,250.48682751)(153.22343442,250.5539796)(153.38292063,250.64631373)
\curveto(153.55080086,250.74704186)(153.68510504,250.88134604)(153.78583317,251.04922626)
\curveto(153.88656131,251.21710649)(153.93692537,251.43115377)(153.93692537,251.69136812)
\curveto(153.93692537,252.11946269)(153.80262119,252.42164709)(153.53401284,252.59792132)
\curveto(153.27379849,252.78258957)(152.96741708,252.87492369)(152.61486861,252.87492369)
\curveto(152.25392613,252.87492369)(151.94754472,252.82036262)(151.69572439,252.71124048)
\curveto(151.44390405,252.61051234)(151.23405377,252.5055872)(151.06617355,252.39646506)
\lineto(150.66326101,253.18969911)
\curveto(150.83953525,253.31560928)(151.1039466,253.44571645)(151.45649507,253.58002063)
\curveto(151.81743755,253.72271882)(152.21615308,253.79406792)(152.65264166,253.79406792)
\curveto(153.06394821,253.79406792)(153.41649668,253.74370385)(153.71028707,253.64297572)
\curveto(154.00407746,253.54224758)(154.24330678,253.39954939)(154.42797503,253.21488115)
\curveto(154.62103728,253.0302129)(154.76373547,252.81196861)(154.8560696,252.56014827)
\curveto(154.94840372,252.31672195)(154.99457078,252.04811359)(154.99457078,251.7543232)
\curveto(154.99457078,251.34301665)(154.88544864,250.99466519)(154.66720434,250.70926881)
\curveto(154.45735407,250.42387243)(154.1845487,250.20562814)(153.84878825,250.05453594)
\curveto(154.25170079,249.93701978)(154.60005226,249.70618447)(154.89384265,249.36203001)
\curveto(155.18763304,249.02626957)(155.33452823,248.57718997)(155.33452823,248.01479122)
\curveto(155.33452823,247.67903077)(155.27577016,247.36425535)(155.158254,247.07046496)
\curveto(155.04913185,246.78506858)(154.87705462,246.53744525)(154.64202231,246.32759497)
\curveto(154.41538401,246.11774469)(154.11739661,245.95406148)(153.74806012,245.83654532)
\curveto(153.38711764,245.71902916)(152.95482606,245.66027108)(152.45118539,245.66027108)
\curveto(152.25812314,245.66027108)(152.05666687,245.67705911)(151.84681659,245.71063515)
\curveto(151.64536032,245.73581719)(151.45649507,245.77359024)(151.28022083,245.8239543)
\curveto(151.1039466,245.86592436)(150.94446039,245.90789441)(150.8017622,245.94986447)
\curveto(150.66745802,246.00022854)(150.57092689,246.03800159)(150.51216881,246.06318362)
\lineto(150.71362508,246.95714581)
\curveto(150.84792926,246.88999372)(151.06197654,246.81025062)(151.35576693,246.71791649)
\curveto(151.64955733,246.62558237)(152.01049981,246.57941531)(152.43859438,246.57941531)
\closepath
}
}
{
\newrgbcolor{curcolor}{0 0 0}
\pscustom[linestyle=none,fillstyle=solid,fillcolor=curcolor]
{
\newpath
\moveto(158.01641576,245.8239543)
\curveto(158.05838582,246.4199291)(158.16331095,247.04947993)(158.33119118,247.71260682)
\curveto(158.50746541,248.38412771)(158.71731569,249.03046657)(158.96074202,249.6516234)
\curveto(159.20416834,250.28117424)(159.4727767,250.86036101)(159.76656709,251.38918371)
\curveto(160.06035748,251.92640043)(160.34995087,252.36708602)(160.63534725,252.71124048)
\lineto(156.85804222,252.71124048)
\lineto(156.85804222,253.61779368)
\lineto(161.79372079,253.61779368)
\lineto(161.79372079,252.74901353)
\curveto(161.54190045,252.45522313)(161.26909509,252.06070461)(160.9753047,251.56545795)
\curveto(160.68151431,251.07021129)(160.40031493,250.51200955)(160.13170658,249.89085272)
\curveto(159.87149223,249.2780899)(159.64485393,248.61916003)(159.45179167,247.91406309)
\curveto(159.25872941,247.21736016)(159.13701625,246.52065723)(159.08665218,245.8239543)
\lineto(158.01641576,245.8239543)
\closepath
}
}
{
\newrgbcolor{curcolor}{0 0 0}
\pscustom[linestyle=none,fillstyle=solid,fillcolor=curcolor]
{
\newpath
\moveto(180.57952067,250.49522152)
\curveto(180.57952067,248.94232946)(180.20179017,247.7713649)(179.44632916,246.98232785)
\curveto(178.69086816,246.20168481)(177.55347964,245.80716628)(176.03416362,245.79877227)
\lineto(175.99639057,246.68014344)
\curveto(176.93651982,246.68014344)(177.69198083,246.86481169)(178.26277359,247.23414818)
\curveto(178.84196036,247.61187868)(179.22388787,248.25821754)(179.40855611,249.17316476)
\curveto(179.20709984,249.08083064)(178.98465855,249.00528454)(178.74123222,248.94652646)
\curveto(178.4978059,248.89616239)(178.24178856,248.87098036)(177.9731802,248.87098036)
\curveto(177.52829761,248.87098036)(177.15476411,248.93393544)(176.85257971,249.05984561)
\curveto(176.5503953,249.19414979)(176.30696898,249.37042403)(176.12230073,249.58866832)
\curveto(175.93763249,249.81530662)(175.80332831,250.07132396)(175.7193882,250.35672034)
\curveto(175.63544809,250.64211672)(175.59347803,250.94430112)(175.59347803,251.26327355)
\curveto(175.59347803,251.54866993)(175.63964509,251.83826331)(175.73197921,252.1320537)
\curveto(175.82431334,252.43423811)(175.96701153,252.70704347)(176.16007378,252.95046979)
\curveto(176.35313604,253.19389612)(176.60075937,253.39535239)(176.90294377,253.5548386)
\curveto(177.20512818,253.71432481)(177.56607066,253.79406792)(177.98577122,253.79406792)
\curveto(178.84196036,253.79406792)(179.48829922,253.50027753)(179.9247878,252.91269674)
\curveto(180.36127638,252.32511596)(180.57952067,251.51929089)(180.57952067,250.49522152)
\closepath
\moveto(178.08649935,249.7271695)
\curveto(178.35510771,249.7271695)(178.60273104,249.75235153)(178.82936934,249.8027156)
\curveto(179.06440165,249.85307967)(179.29103995,249.92442876)(179.50928425,250.01676289)
\curveto(179.51767826,250.100703)(179.52187526,250.1804461)(179.52187526,250.25599221)
\lineto(179.52187526,250.49522152)
\curveto(179.52187526,250.82258796)(179.49669323,251.13316637)(179.44632916,251.42695677)
\curveto(179.40435911,251.72074716)(179.324616,251.9767645)(179.20709984,252.19500879)
\curveto(179.0979777,252.41325308)(178.94268849,252.58533031)(178.74123222,252.71124048)
\curveto(178.54816997,252.84554465)(178.30054664,252.91269674)(177.99836223,252.91269674)
\curveto(177.7465419,252.91269674)(177.53669162,252.85813567)(177.36881139,252.74901353)
\curveto(177.20093117,252.64828539)(177.06242999,252.51817822)(176.95330784,252.35869201)
\curveto(176.8441857,252.2075998)(176.76444259,252.03552258)(176.71407852,251.84246032)
\curveto(176.67210847,251.64939806)(176.65112344,251.46472982)(176.65112344,251.28845558)
\curveto(176.65112344,250.7764209)(176.76444259,250.38609938)(176.99108089,250.11749102)
\curveto(177.2261132,249.85727667)(177.59125269,249.7271695)(178.08649935,249.7271695)
\closepath
}
}
{
\newrgbcolor{curcolor}{0 0 0}
\pscustom[linestyle=none,fillstyle=solid,fillcolor=curcolor]
{
\newpath
\moveto(190.21164278,246.57941531)
\curveto(190.87476966,246.57941531)(191.34483429,246.70952248)(191.62183666,246.96973683)
\curveto(191.90723304,247.23834519)(192.04993123,247.59509066)(192.04993123,248.03997325)
\curveto(192.04993123,248.32536963)(191.99117315,248.56459895)(191.87365699,248.75766121)
\curveto(191.75614083,248.95072347)(191.60085163,249.10601267)(191.40778937,249.22352883)
\curveto(191.21472711,249.34104499)(190.99228582,249.4249851)(190.74046548,249.47534917)
\curveto(190.48864515,249.52571323)(190.22423379,249.55089527)(189.94723143,249.55089527)
\lineto(189.68282007,249.55089527)
\lineto(189.68282007,250.39449339)
\lineto(190.04795956,250.39449339)
\curveto(190.23262781,250.39449339)(190.42149306,250.41128141)(190.61455531,250.44485746)
\curveto(190.81601158,250.48682751)(190.99648282,250.5539796)(191.15596904,250.64631373)
\curveto(191.32384926,250.74704186)(191.45815344,250.88134604)(191.55888157,251.04922626)
\curveto(191.65960971,251.21710649)(191.70997377,251.43115377)(191.70997377,251.69136812)
\curveto(191.70997377,252.11946269)(191.57566959,252.42164709)(191.30706124,252.59792132)
\curveto(191.04684689,252.78258957)(190.74046548,252.87492369)(190.38791701,252.87492369)
\curveto(190.02697453,252.87492369)(189.72059312,252.82036262)(189.46877279,252.71124048)
\curveto(189.21695245,252.61051234)(189.00710217,252.5055872)(188.83922195,252.39646506)
\lineto(188.43630941,253.18969911)
\curveto(188.61258365,253.31560928)(188.876995,253.44571645)(189.22954347,253.58002063)
\curveto(189.59048595,253.72271882)(189.98920148,253.79406792)(190.42569006,253.79406792)
\curveto(190.83699661,253.79406792)(191.18954508,253.74370385)(191.48333547,253.64297572)
\curveto(191.77712586,253.54224758)(192.01635518,253.39954939)(192.20102343,253.21488115)
\curveto(192.39408568,253.0302129)(192.53678387,252.81196861)(192.629118,252.56014827)
\curveto(192.72145212,252.31672195)(192.76761918,252.04811359)(192.76761918,251.7543232)
\curveto(192.76761918,251.34301665)(192.65849704,250.99466519)(192.44025275,250.70926881)
\curveto(192.23040247,250.42387243)(191.9575971,250.20562814)(191.62183666,250.05453594)
\curveto(192.02474919,249.93701978)(192.37310066,249.70618447)(192.66689105,249.36203001)
\curveto(192.96068144,249.02626957)(193.10757663,248.57718997)(193.10757663,248.01479122)
\curveto(193.10757663,247.67903077)(193.04881856,247.36425535)(192.9313024,247.07046496)
\curveto(192.82218025,246.78506858)(192.65010303,246.53744525)(192.41507071,246.32759497)
\curveto(192.18843241,246.11774469)(191.89044501,245.95406148)(191.52110852,245.83654532)
\curveto(191.16016604,245.71902916)(190.72787447,245.66027108)(190.22423379,245.66027108)
\curveto(190.03117154,245.66027108)(189.82971527,245.67705911)(189.61986499,245.71063515)
\curveto(189.41840872,245.73581719)(189.22954347,245.77359024)(189.05326923,245.8239543)
\curveto(188.876995,245.86592436)(188.71750879,245.90789441)(188.5748106,245.94986447)
\curveto(188.44050642,246.00022854)(188.34397529,246.03800159)(188.28521721,246.06318362)
\lineto(188.48667348,246.95714581)
\curveto(188.62097766,246.88999372)(188.83502494,246.81025062)(189.12881534,246.71791649)
\curveto(189.42260573,246.62558237)(189.78354821,246.57941531)(190.21164278,246.57941531)
\closepath
}
}
{
\newrgbcolor{curcolor}{0 0 0}
\pscustom[linestyle=none,fillstyle=solid,fillcolor=curcolor]
{
\newpath
\moveto(196.50715975,246.57941531)
\curveto(197.17028663,246.57941531)(197.64035125,246.70952248)(197.91735362,246.96973683)
\curveto(198.20275,247.23834519)(198.34544819,247.59509066)(198.34544819,248.03997325)
\curveto(198.34544819,248.32536963)(198.28669012,248.56459895)(198.16917396,248.75766121)
\curveto(198.0516578,248.95072347)(197.8963686,249.10601267)(197.70330634,249.22352883)
\curveto(197.51024408,249.34104499)(197.28780279,249.4249851)(197.03598245,249.47534917)
\curveto(196.78416211,249.52571323)(196.51975076,249.55089527)(196.24274839,249.55089527)
\lineto(195.97833704,249.55089527)
\lineto(195.97833704,250.39449339)
\lineto(196.34347653,250.39449339)
\curveto(196.52814477,250.39449339)(196.71701002,250.41128141)(196.91007228,250.44485746)
\curveto(197.11152855,250.48682751)(197.29199979,250.5539796)(197.451486,250.64631373)
\curveto(197.61936623,250.74704186)(197.75367041,250.88134604)(197.85439854,251.04922626)
\curveto(197.95512667,251.21710649)(198.00549074,251.43115377)(198.00549074,251.69136812)
\curveto(198.00549074,252.11946269)(197.87118656,252.42164709)(197.6025782,252.59792132)
\curveto(197.34236386,252.78258957)(197.03598245,252.87492369)(196.68343398,252.87492369)
\curveto(196.3224915,252.87492369)(196.01611009,252.82036262)(195.76428976,252.71124048)
\curveto(195.51246942,252.61051234)(195.30261914,252.5055872)(195.13473892,252.39646506)
\lineto(194.73182638,253.18969911)
\curveto(194.90810062,253.31560928)(195.17251197,253.44571645)(195.52506044,253.58002063)
\curveto(195.88600292,253.72271882)(196.28471845,253.79406792)(196.72120703,253.79406792)
\curveto(197.13251358,253.79406792)(197.48506205,253.74370385)(197.77885244,253.64297572)
\curveto(198.07264283,253.54224758)(198.31187215,253.39954939)(198.49654039,253.21488115)
\curveto(198.68960265,253.0302129)(198.83230084,252.81196861)(198.92463497,252.56014827)
\curveto(199.01696909,252.31672195)(199.06313615,252.04811359)(199.06313615,251.7543232)
\curveto(199.06313615,251.34301665)(198.954014,250.99466519)(198.73576971,250.70926881)
\curveto(198.52591943,250.42387243)(198.25311407,250.20562814)(197.91735362,250.05453594)
\curveto(198.32026616,249.93701978)(198.66861762,249.70618447)(198.96240802,249.36203001)
\curveto(199.25619841,249.02626957)(199.4030936,248.57718997)(199.4030936,248.01479122)
\curveto(199.4030936,247.67903077)(199.34433552,247.36425535)(199.22681937,247.07046496)
\curveto(199.11769722,246.78506858)(198.94561999,246.53744525)(198.71058768,246.32759497)
\curveto(198.48394938,246.11774469)(198.18596198,245.95406148)(197.81662549,245.83654532)
\curveto(197.45568301,245.71902916)(197.02339143,245.66027108)(196.51975076,245.66027108)
\curveto(196.3266885,245.66027108)(196.12523224,245.67705911)(195.91538196,245.71063515)
\curveto(195.71392569,245.73581719)(195.52506044,245.77359024)(195.3487862,245.8239543)
\curveto(195.17251197,245.86592436)(195.01302576,245.90789441)(194.87032757,245.94986447)
\curveto(194.73602339,246.00022854)(194.63949226,246.03800159)(194.58073418,246.06318362)
\lineto(194.78219045,246.95714581)
\curveto(194.91649463,246.88999372)(195.13054191,246.81025062)(195.4243323,246.71791649)
\curveto(195.71812269,246.62558237)(196.07906518,246.57941531)(196.50715975,246.57941531)
\closepath
}
}
{
\newrgbcolor{curcolor}{0 0 0}
\pscustom[linestyle=none,fillstyle=solid,fillcolor=curcolor]
{
\newpath
\moveto(209.09816316,246.57941531)
\curveto(209.76129005,246.57941531)(210.23135467,246.70952248)(210.50835704,246.96973683)
\curveto(210.79375342,247.23834519)(210.93645161,247.59509066)(210.93645161,248.03997325)
\curveto(210.93645161,248.32536963)(210.87769353,248.56459895)(210.76017738,248.75766121)
\curveto(210.64266122,248.95072347)(210.48737201,249.10601267)(210.29430976,249.22352883)
\curveto(210.1012475,249.34104499)(209.8788062,249.4249851)(209.62698587,249.47534917)
\curveto(209.37516553,249.52571323)(209.11075418,249.55089527)(208.83375181,249.55089527)
\lineto(208.56934046,249.55089527)
\lineto(208.56934046,250.39449339)
\lineto(208.93447995,250.39449339)
\curveto(209.11914819,250.39449339)(209.30801344,250.41128141)(209.5010757,250.44485746)
\curveto(209.70253197,250.48682751)(209.88300321,250.5539796)(210.04248942,250.64631373)
\curveto(210.21036964,250.74704186)(210.34467382,250.88134604)(210.44540196,251.04922626)
\curveto(210.54613009,251.21710649)(210.59649416,251.43115377)(210.59649416,251.69136812)
\curveto(210.59649416,252.11946269)(210.46218998,252.42164709)(210.19358162,252.59792132)
\curveto(209.93336728,252.78258957)(209.62698587,252.87492369)(209.2744374,252.87492369)
\curveto(208.91349492,252.87492369)(208.60711351,252.82036262)(208.35529317,252.71124048)
\curveto(208.10347284,252.61051234)(207.89362256,252.5055872)(207.72574234,252.39646506)
\lineto(207.3228298,253.18969911)
\curveto(207.49910403,253.31560928)(207.76351539,253.44571645)(208.11606386,253.58002063)
\curveto(208.47700634,253.72271882)(208.87572187,253.79406792)(209.31221045,253.79406792)
\curveto(209.723517,253.79406792)(210.07606547,253.74370385)(210.36985586,253.64297572)
\curveto(210.66364625,253.54224758)(210.90287557,253.39954939)(211.08754381,253.21488115)
\curveto(211.28060607,253.0302129)(211.42330426,252.81196861)(211.51563838,252.56014827)
\curveto(211.60797251,252.31672195)(211.65413957,252.04811359)(211.65413957,251.7543232)
\curveto(211.65413957,251.34301665)(211.54501742,250.99466519)(211.32677313,250.70926881)
\curveto(211.11692285,250.42387243)(210.84411749,250.20562814)(210.50835704,250.05453594)
\curveto(210.91126958,249.93701978)(211.25962104,249.70618447)(211.55341143,249.36203001)
\curveto(211.84720182,249.02626957)(211.99409702,248.57718997)(211.99409702,248.01479122)
\curveto(211.99409702,247.67903077)(211.93533894,247.36425535)(211.81782279,247.07046496)
\curveto(211.70870064,246.78506858)(211.53662341,246.53744525)(211.3015911,246.32759497)
\curveto(211.0749528,246.11774469)(210.7769654,245.95406148)(210.40762891,245.83654532)
\curveto(210.04668643,245.71902916)(209.61439485,245.66027108)(209.11075418,245.66027108)
\curveto(208.91769192,245.66027108)(208.71623565,245.67705911)(208.50638538,245.71063515)
\curveto(208.30492911,245.73581719)(208.11606386,245.77359024)(207.93978962,245.8239543)
\curveto(207.76351539,245.86592436)(207.60402917,245.90789441)(207.46133098,245.94986447)
\curveto(207.3270268,246.00022854)(207.23049568,246.03800159)(207.1717376,246.06318362)
\lineto(207.37319387,246.95714581)
\curveto(207.50749804,246.88999372)(207.72154533,246.81025062)(208.01533572,246.71791649)
\curveto(208.30912611,246.62558237)(208.67006859,246.57941531)(209.09816316,246.57941531)
\closepath
}
}
{
\newrgbcolor{curcolor}{0 0 0}
\pscustom[linestyle=none,fillstyle=solid,fillcolor=curcolor]
{
\newpath
\moveto(214.67599218,245.8239543)
\curveto(214.71796223,246.4199291)(214.82288737,247.04947993)(214.99076759,247.71260682)
\curveto(215.16704183,248.38412771)(215.37689211,249.03046657)(215.62031843,249.6516234)
\curveto(215.86374476,250.28117424)(216.13235311,250.86036101)(216.42614351,251.38918371)
\curveto(216.7199339,251.92640043)(217.00952728,252.36708602)(217.29492366,252.71124048)
\lineto(213.51761863,252.71124048)
\lineto(213.51761863,253.61779368)
\lineto(218.45329721,253.61779368)
\lineto(218.45329721,252.74901353)
\curveto(218.20147687,252.45522313)(217.92867151,252.06070461)(217.63488112,251.56545795)
\curveto(217.34109072,251.07021129)(217.05989135,250.51200955)(216.79128299,249.89085272)
\curveto(216.53106865,249.2780899)(216.30443034,248.61916003)(216.11136809,247.91406309)
\curveto(215.91830583,247.21736016)(215.79659267,246.52065723)(215.7462286,245.8239543)
\lineto(214.67599218,245.8239543)
\closepath
}
}
{
\newrgbcolor{curcolor}{0 0 0}
\pscustom[linewidth=0.4894965,linecolor=curcolor]
{
\newpath
\moveto(52.49916771,256.12246962)
\lineto(222.62575462,256.12246962)
\lineto(222.62575462,243.33186885)
\lineto(52.49916771,243.33186885)
\closepath
}
}
{
\newrgbcolor{curcolor}{0 0 0}
\pscustom[linewidth=0.51800942,linecolor=curcolor]
{
\newpath
\moveto(71.388229,243.38171933)
\lineto(71.388229,255.98111433)
}
}
{
\newrgbcolor{curcolor}{0 0 0}
\pscustom[linewidth=0.5154072,linecolor=curcolor]
{
\newpath
\moveto(90.268452,243.47603033)
\lineto(90.268452,255.94915633)
}
}
{
\newrgbcolor{curcolor}{0 0 0}
\pscustom[linewidth=0.51930571,linecolor=curcolor]
{
\newpath
\moveto(109.107742,243.47602933)
\lineto(109.107742,256.13855833)
}
}
{
\newrgbcolor{curcolor}{0 0 0}
\pscustom[linewidth=0.52573889,linecolor=curcolor]
{
\newpath
\moveto(128.036312,243.29745833)
\lineto(128.036312,256.27566033)
}
}
{
\newrgbcolor{curcolor}{0 0 0}
\pscustom[linewidth=0.51670998,linecolor=curcolor]
{
\newpath
\moveto(146.875592,243.47602933)
\lineto(146.875592,256.01228933)
}
}
{
\newrgbcolor{curcolor}{0 0 0}
\pscustom[linewidth=0.52059871,linecolor=curcolor]
{
\newpath
\moveto(165.625602,243.47603033)
\lineto(165.625602,256.20169433)
}
}
{
\newrgbcolor{curcolor}{0 0 0}
\pscustom[linewidth=0.51670998,linecolor=curcolor]
{
\newpath
\moveto(184.554172,243.38674433)
\lineto(184.554172,255.92300433)
}
}
{
\newrgbcolor{curcolor}{0 0 0}
\pscustom[linewidth=0.51670998,linecolor=curcolor]
{
\newpath
\moveto(203.661312,243.47603033)
\lineto(203.661312,256.01229033)
}
}
{
\newrgbcolor{curcolor}{0 0 0}
\pscustom[linestyle=none,fillstyle=solid,fillcolor=curcolor]
{
\newpath
\moveto(62.27863702,231.83480506)
\curveto(62.70861898,232.00194839)(63.1259544,232.21087754)(63.5306433,232.46159253)
\curveto(63.9353322,232.72066468)(64.31051253,233.04659416)(64.6561843,233.43938097)
\lineto(65.38968292,233.43938097)
\lineto(65.38968292,226.55725463)
\lineto(66.8693267,226.55725463)
\lineto(66.8693267,225.67975218)
\lineto(62.65803286,225.67975218)
\lineto(62.65803286,226.55725463)
\lineto(64.35266762,226.55725463)
\lineto(64.35266762,231.9977698)
\curveto(64.25992642,231.91419814)(64.14610766,231.8264479)(64.01121136,231.73451907)
\curveto(63.88474608,231.65094741)(63.74141877,231.56737575)(63.58122941,231.48380408)
\curveto(63.42947107,231.40023242)(63.26928172,231.32083934)(63.10066135,231.24562485)
\curveto(62.93204097,231.17041035)(62.76763611,231.10773161)(62.60744675,231.05758861)
\lineto(62.27863702,231.83480506)
\closepath
}
}
{
\newrgbcolor{curcolor}{0 0 0}
\pscustom[linestyle=none,fillstyle=solid,fillcolor=curcolor]
{
\newpath
\moveto(85.61148126,231.45873259)
\curveto(85.61148126,231.19130327)(85.55667964,230.93223112)(85.4470764,230.68151613)
\curveto(85.34590418,230.43080115)(85.20679237,230.18426474)(85.02974097,229.94190693)
\curveto(84.8611206,229.69954911)(84.66720717,229.46136987)(84.44800068,229.22736922)
\curveto(84.2287942,228.99336857)(84.0053722,228.7635465)(83.7777347,228.53790301)
\curveto(83.65126942,228.41254552)(83.50372659,228.26211653)(83.33510622,228.08661604)
\curveto(83.16648584,227.91111555)(83.00629649,227.73143647)(82.85453815,227.54757882)
\curveto(82.70277981,227.36372116)(82.57631453,227.18404209)(82.47514231,227.0085416)
\curveto(82.37397009,226.83304111)(82.32338397,226.68261212)(82.32338397,226.55725463)
\lineto(85.92764447,226.55725463)
\lineto(85.92764447,225.67975218)
\lineto(81.18519645,225.67975218)
\curveto(81.17676543,225.72153801)(81.17254992,225.76332384)(81.17254992,225.80510967)
\lineto(81.17254992,225.94300291)
\curveto(81.17254992,226.29400389)(81.23156705,226.61993337)(81.34960131,226.92079135)
\curveto(81.46763558,227.22164934)(81.61939391,227.50579299)(81.80487632,227.7732223)
\curveto(81.99035873,228.04065162)(82.19691869,228.29136661)(82.4245562,228.52536726)
\curveto(82.66062472,228.76772508)(82.89247773,229.00172573)(83.12011524,229.22736922)
\curveto(83.30559765,229.41122687)(83.48264904,229.59090595)(83.65126942,229.76640644)
\curveto(83.82832081,229.94190693)(83.98007915,230.11740742)(84.10654443,230.2929079)
\curveto(84.24144073,230.46840839)(84.34682846,230.64808747)(84.42270763,230.83194512)
\curveto(84.50701782,231.02415994)(84.54917291,231.22055335)(84.54917291,231.42112534)
\curveto(84.54917291,231.64676882)(84.51123332,231.83898365)(84.43535416,231.9977698)
\curveto(84.36790601,232.15655596)(84.27094929,232.28609204)(84.14448401,232.38637803)
\curveto(84.02644975,232.49502119)(83.88733794,232.57441427)(83.72714859,232.62455727)
\curveto(83.56695923,232.67470026)(83.39412335,232.69977176)(83.20864094,232.69977176)
\curveto(82.98943445,232.69977176)(82.78709,232.67052168)(82.60160759,232.61202152)
\curveto(82.4245562,232.55352135)(82.26436684,232.48248544)(82.12103952,232.39891378)
\curveto(81.98614323,232.31534212)(81.86810896,232.23177046)(81.76693674,232.14819879)
\curveto(81.66576451,232.0729843)(81.58988535,232.01030555)(81.53929923,231.96016256)
\lineto(81.02079158,232.68723601)
\curveto(81.08823973,232.76245051)(81.18941196,232.85437934)(81.32430826,232.9630225)
\curveto(81.45920456,233.07166566)(81.61939391,233.17195165)(81.80487632,233.26388048)
\curveto(81.99878975,233.36416647)(82.21378073,233.44773813)(82.44984925,233.51459546)
\curveto(82.68591778,233.58145279)(82.93884834,233.61488146)(83.20864094,233.61488146)
\curveto(84.02644975,233.61488146)(84.62926759,233.42684522)(85.01709445,233.05077274)
\curveto(85.41335233,232.68305743)(85.61148126,232.15237738)(85.61148126,231.45873259)
\closepath
}
}
{
\newrgbcolor{curcolor}{0 0 0}
\pscustom[linestyle=none,fillstyle=solid,fillcolor=curcolor]
{
\newpath
\moveto(101.95079454,226.43189713)
\curveto(102.61684502,226.43189713)(103.08898206,226.56143321)(103.36720568,226.82050536)
\curveto(103.65386032,227.08793468)(103.79718763,227.44311424)(103.79718763,227.88604405)
\curveto(103.79718763,228.1701877)(103.7381705,228.40836693)(103.62013624,228.60058176)
\curveto(103.50210198,228.79279658)(103.34612813,228.94740415)(103.1522147,229.06440448)
\curveto(102.95830127,229.1814048)(102.73487928,229.26497647)(102.48194872,229.31511946)
\curveto(102.22901816,229.36526246)(101.96344107,229.39033396)(101.68521745,229.39033396)
\lineto(101.41964036,229.39033396)
\lineto(101.41964036,230.23022916)
\lineto(101.78638967,230.23022916)
\curveto(101.97187209,230.23022916)(102.16157001,230.24694349)(102.35548344,230.28037216)
\curveto(102.55782789,230.32215799)(102.73909479,230.38901532)(102.89928414,230.48094414)
\curveto(103.06790452,230.58123014)(103.20280082,230.7149448)(103.30397304,230.88208812)
\curveto(103.40514526,231.04923144)(103.45573138,231.26233918)(103.45573138,231.52141133)
\curveto(103.45573138,231.94762681)(103.32083508,232.24848479)(103.05104248,232.42398528)
\curveto(102.7896809,232.60784293)(102.48194872,232.69977176)(102.12784593,232.69977176)
\curveto(101.76531213,232.69977176)(101.45757995,232.64545018)(101.20464938,232.53680702)
\curveto(100.95171882,232.43652103)(100.74094336,232.33205645)(100.57232298,232.22341329)
\lineto(100.16763409,233.01316549)
\curveto(100.34468548,233.13852299)(100.61026257,233.26805906)(100.96436535,233.40177372)
\curveto(101.32689916,233.54384555)(101.72737254,233.61488146)(102.16578552,233.61488146)
\curveto(102.57890543,233.61488146)(102.93300822,233.56473846)(103.22809387,233.46445247)
\curveto(103.52317953,233.36416647)(103.76346356,233.22209465)(103.94894597,233.03823699)
\curveto(104.1428594,232.85437934)(104.28618672,232.63709302)(104.37892792,232.38637803)
\curveto(104.47166913,232.14402021)(104.51803973,231.87659089)(104.51803973,231.58409008)
\curveto(104.51803973,231.17458894)(104.40843649,230.82776654)(104.18923,230.54362289)
\curveto(103.97845454,230.25947924)(103.70444643,230.04219292)(103.36720568,229.89176393)
\curveto(103.77189458,229.7747636)(104.12178185,229.54494153)(104.41686751,229.20229772)
\curveto(104.71195316,228.86801107)(104.85949599,228.42090268)(104.85949599,227.86097255)
\curveto(104.85949599,227.5266859)(104.80047886,227.21329217)(104.6824446,226.92079135)
\curveto(104.57284135,226.6366477)(104.40000547,226.3901113)(104.16393695,226.18118215)
\curveto(103.93629944,225.97225299)(103.63699828,225.80928825)(103.26603346,225.69228793)
\curveto(102.90349965,225.5752876)(102.46930219,225.51678744)(101.96344107,225.51678744)
\curveto(101.76952764,225.51678744)(101.56718319,225.53350177)(101.35640772,225.56693043)
\curveto(101.15406327,225.59200193)(100.96436535,225.62960918)(100.78731396,225.67975218)
\curveto(100.61026257,225.72153801)(100.45007321,225.76332384)(100.30674589,225.80510967)
\curveto(100.17184959,225.85525267)(100.07489288,225.89285992)(100.01587575,225.91793141)
\lineto(100.2182202,226.80796961)
\curveto(100.3531165,226.74111228)(100.56810747,226.6617192)(100.86319313,226.56979038)
\curveto(101.15827878,226.47786155)(101.52081259,226.43189713)(101.95079454,226.43189713)
\closepath
}
}
{
\newrgbcolor{curcolor}{0 0 0}
\pscustom[linestyle=none,fillstyle=solid,fillcolor=curcolor]
{
\newpath
\moveto(118.69480054,228.36240252)
\curveto(118.83812786,228.69668917)(119.03204129,229.08111881)(119.27654083,229.51569145)
\curveto(119.52947139,229.95026409)(119.81191052,230.39737248)(120.12385821,230.85701662)
\curveto(120.4358059,231.32501793)(120.76883114,231.78048348)(121.12293393,232.22341329)
\curveto(121.47703671,232.67470026)(121.83535501,233.08002282)(122.19788881,233.43938097)
\lineto(123.20961106,233.43938097)
\lineto(123.20961106,228.51283151)
\lineto(124.1328076,228.51283151)
\lineto(124.1328076,227.66040056)
\lineto(123.20961106,227.66040056)
\lineto(123.20961106,225.67975218)
\lineto(122.19788881,225.67975218)
\lineto(122.19788881,227.66040056)
\lineto(118.69480054,227.66040056)
\lineto(118.69480054,228.36240252)
\closepath
\moveto(122.19788881,232.21087754)
\curveto(121.97025131,231.96851972)(121.73839829,231.7010904)(121.50232977,231.40858959)
\curveto(121.27469226,231.11608877)(121.05127027,230.80687362)(120.83206378,230.48094414)
\curveto(120.6128573,230.16337183)(120.40629734,229.83744235)(120.21238391,229.5031557)
\curveto(120.0269015,229.16886905)(119.85828112,228.83876099)(119.70652279,228.51283151)
\lineto(122.19788881,228.51283151)
\lineto(122.19788881,232.21087754)
\closepath
}
}
{
\newrgbcolor{curcolor}{0 0 0}
\pscustom[linestyle=none,fillstyle=solid,fillcolor=curcolor]
{
\newpath
\moveto(139.56156703,230.4433369)
\curveto(140.69132353,230.40155106)(141.51334786,230.15501466)(142.02764,229.70372769)
\curveto(142.54193214,229.25244072)(142.79907821,228.64654617)(142.79907821,227.88604405)
\curveto(142.79907821,227.54340023)(142.74427658,227.22582792)(142.63467334,226.9333271)
\curveto(142.5250701,226.64082629)(142.35223422,226.3901113)(142.11616569,226.18118215)
\curveto(141.88852819,225.97225299)(141.59344253,225.80928825)(141.23090873,225.69228793)
\curveto(140.87680594,225.5752876)(140.45525501,225.51678744)(139.96625592,225.51678744)
\curveto(139.76391148,225.51678744)(139.56156703,225.53350177)(139.35922258,225.56693043)
\curveto(139.16530915,225.59200193)(138.97982674,225.6254306)(138.80277534,225.66721643)
\curveto(138.62572395,225.70900226)(138.46975011,225.75078809)(138.33485381,225.79257392)
\curveto(138.19995751,225.84271692)(138.10300079,225.88450275)(138.04398366,225.91793141)
\lineto(138.24632811,226.80796961)
\curveto(138.38122441,226.74111228)(138.58778437,226.6617192)(138.86600798,226.56979038)
\curveto(139.15266262,226.47786155)(139.51098091,226.43189713)(139.94096287,226.43189713)
\curveto(140.27820362,226.43189713)(140.56064274,226.46950438)(140.78828025,226.54471888)
\curveto(141.01591775,226.61993337)(141.19718465,226.72021937)(141.33208095,226.84557686)
\curveto(141.47540827,226.97929152)(141.5765805,227.12972051)(141.63559763,227.29686383)
\curveto(141.70304578,227.46400716)(141.73676985,227.63950765)(141.73676985,227.8233653)
\curveto(141.73676985,228.10750895)(141.68618374,228.35822394)(141.58501151,228.57551026)
\curveto(141.49227031,228.80115374)(141.32364993,228.98918998)(141.07915039,229.13961897)
\curveto(140.84308187,229.29004796)(140.51427214,229.40286971)(140.0927212,229.4780842)
\curveto(139.67960129,229.56165586)(139.15687813,229.6034417)(138.52455173,229.6034417)
\curveto(138.57513784,229.97115701)(138.61307742,230.31380082)(138.63837048,230.63137313)
\curveto(138.67209455,230.95730262)(138.69738761,231.27069635)(138.71424965,231.57155433)
\curveto(138.7395427,231.88076948)(138.75640474,232.18580604)(138.76483576,232.48666402)
\curveto(138.7816978,232.78752201)(138.79855984,233.10509432)(138.81542187,233.43938097)
\lineto(142.58408723,233.43938097)
\lineto(142.58408723,232.56187852)
\lineto(139.72597189,232.56187852)
\curveto(139.71754087,232.44487819)(139.70489434,232.29027062)(139.68803231,232.0980558)
\curveto(139.67960129,231.91419814)(139.66695476,231.71780474)(139.65009272,231.50887558)
\curveto(139.63323069,231.29994643)(139.61636865,231.09937444)(139.59950661,230.90715962)
\curveto(139.58264457,230.7149448)(139.56999805,230.56033722)(139.56156703,230.4433369)
\closepath
}
}
{
\newrgbcolor{curcolor}{0 0 0}
\pscustom[linestyle=none,fillstyle=solid,fillcolor=curcolor]
{
\newpath
\moveto(156.86202027,228.78861799)
\curveto(156.86202027,229.54076295)(156.96319249,230.20515766)(157.16553694,230.78180213)
\curveto(157.37631241,231.36680376)(157.67139806,231.85569798)(158.0507939,232.24848479)
\curveto(158.43862076,232.6412716)(158.9065423,232.94212958)(159.45455851,233.15105874)
\curveto(160.00257473,233.35998789)(160.6222546,233.46863105)(161.31359814,233.47698822)
\lineto(161.40212383,232.59948577)
\curveto(160.95527984,232.5911286)(160.54637544,232.5409856)(160.17541061,232.44905678)
\curveto(159.81287681,232.36548512)(159.48828259,232.22341329)(159.20162795,232.0228413)
\curveto(158.91497332,231.83062648)(158.67468928,231.5799115)(158.48077585,231.27069635)
\curveto(158.28686242,230.9614812)(158.1393196,230.58123014)(158.03814737,230.12994316)
\curveto(158.24049182,230.22187199)(158.45969831,230.29708649)(158.69576683,230.35558665)
\curveto(158.94026637,230.41408681)(159.19741244,230.4433369)(159.46720504,230.4433369)
\curveto(159.90561801,230.4433369)(160.27658284,230.37647957)(160.58009951,230.24276491)
\curveto(160.8920472,230.10905025)(161.14076225,229.92937118)(161.32624466,229.70372769)
\curveto(161.51172708,229.48644137)(161.64662338,229.2315478)(161.73093356,228.93904698)
\curveto(161.81524375,228.64654617)(161.85739884,228.34568819)(161.85739884,228.03647304)
\curveto(161.85739884,227.75232939)(161.81102824,227.45982857)(161.71828703,227.15897059)
\curveto(161.62554583,226.86646977)(161.48221851,226.59486187)(161.28830508,226.34414689)
\curveto(161.09439165,226.10178907)(160.8456766,225.90121708)(160.54215993,225.74243092)
\curveto(160.23864325,225.59200193)(159.87610945,225.51678744)(159.45455851,225.51678744)
\curveto(158.58616359,225.51678744)(157.93697515,225.80510967)(157.5069932,226.38175414)
\curveto(157.07701124,226.9583986)(156.86202027,227.76068655)(156.86202027,228.78861799)
\closepath
\moveto(159.35338629,229.59090595)
\curveto(159.08359369,229.59090595)(158.83487864,229.56583445)(158.60724114,229.51569145)
\curveto(158.38803465,229.46554845)(158.16461265,229.39033396)(157.93697515,229.29004796)
\curveto(157.92854413,229.2064763)(157.92432862,229.12290464)(157.92432862,229.03933298)
\lineto(157.92432862,228.78861799)
\curveto(157.92432862,228.46268851)(157.94540617,228.15347336)(157.98756126,227.86097255)
\curveto(158.03814737,227.5768289)(158.11824205,227.32193533)(158.22784529,227.09629184)
\curveto(158.34587956,226.87900552)(158.5018534,226.70350503)(158.69576683,226.56979038)
\curveto(158.88968026,226.44443288)(159.13839531,226.38175414)(159.44191199,226.38175414)
\curveto(159.69484255,226.38175414)(159.90561801,226.43189713)(160.07423839,226.53218313)
\curveto(160.24285876,226.64082629)(160.38197057,226.77454095)(160.49157381,226.9333271)
\curveto(160.60117706,227.09211326)(160.67705623,227.26761375)(160.71921132,227.45982857)
\curveto(160.76979743,227.66040056)(160.79509049,227.8484368)(160.79509049,228.02393729)
\curveto(160.79509049,228.53372443)(160.67705623,228.92233265)(160.4409877,229.18976197)
\curveto(160.2133502,229.45719129)(159.85081639,229.59090595)(159.35338629,229.59090595)
\closepath
}
}
{
\newrgbcolor{curcolor}{0 0 0}
\pscustom[linestyle=none,fillstyle=solid,fillcolor=curcolor]
{
\newpath
\moveto(177.10911454,225.67975218)
\curveto(177.15126963,226.27311098)(177.25665736,226.89989844)(177.42527774,227.56011457)
\curveto(177.60232913,228.22868786)(177.8131046,228.87218966)(178.05760414,229.49061995)
\curveto(178.30210368,230.11740742)(178.57189628,230.69405188)(178.86698194,231.22055335)
\curveto(179.16206759,231.75541198)(179.45293773,232.19416321)(179.73959237,232.53680702)
\lineto(175.94563396,232.53680702)
\lineto(175.94563396,233.43938097)
\lineto(180.90307295,233.43938097)
\lineto(180.90307295,232.57441427)
\curveto(180.65014239,232.28191345)(180.37613428,231.88912664)(180.08104863,231.39605384)
\curveto(179.78596297,230.90298104)(179.50352385,230.34722948)(179.23373125,229.72879919)
\curveto(178.97236967,229.11872606)(178.74473216,228.46268851)(178.55081873,227.76068655)
\curveto(178.3569053,227.06704176)(178.23465553,226.37339697)(178.18406942,225.67975218)
\lineto(177.10911454,225.67975218)
\closepath
}
}
{
\newrgbcolor{curcolor}{0 0 0}
\pscustom[linestyle=none,fillstyle=solid,fillcolor=curcolor]
{
\newpath
\moveto(199.79698872,227.69800781)
\curveto(199.79698872,227.04614885)(199.58621325,226.51964738)(199.16466232,226.1185034)
\curveto(198.7515424,225.71735943)(198.119216,225.51678744)(197.26768311,225.51678744)
\curveto(196.77868403,225.51678744)(196.37399513,225.57946618)(196.05361642,225.70482368)
\curveto(195.73323771,225.83853834)(195.47609164,226.00568166)(195.28217821,226.20625365)
\curveto(195.0966958,226.4151828)(194.9617995,226.64500487)(194.87748931,226.89571986)
\curveto(194.80161014,227.14643484)(194.76367056,227.39297124)(194.76367056,227.63532906)
\curveto(194.76367056,228.07825887)(194.88592033,228.47104568)(195.13041987,228.81368949)
\curveto(195.37491942,229.15633331)(195.66578956,229.43211979)(196.00303031,229.64104894)
\curveto(195.28639372,230.04219292)(194.92807543,230.65644463)(194.92807543,231.48380408)
\curveto(194.92807543,231.76794773)(194.98287705,232.03955563)(195.09248029,232.29862779)
\curveto(195.20208353,232.55769994)(195.35805738,232.78334342)(195.56040183,232.97555825)
\curveto(195.76274628,233.16777307)(196.00724582,233.32238064)(196.29390045,233.43938097)
\curveto(196.58898611,233.55638129)(196.91779584,233.61488146)(197.28032964,233.61488146)
\curveto(197.70188058,233.61488146)(198.06019887,233.55220271)(198.35528453,233.42684522)
\curveto(198.6588012,233.30148773)(198.90330074,233.13852299)(199.08878315,232.937951)
\curveto(199.27426556,232.74573618)(199.40916186,232.52844986)(199.49347205,232.28609204)
\curveto(199.57778224,232.05209138)(199.61993733,231.82226931)(199.61993733,231.59662583)
\curveto(199.61993733,231.15369602)(199.50611858,230.76926638)(199.27848107,230.4433369)
\curveto(199.05084357,230.12576458)(198.78948199,229.87087101)(198.49439633,229.67865619)
\curveto(199.36279126,229.26915505)(199.79698872,228.60893892)(199.79698872,227.69800781)
\closepath
\moveto(195.7753928,227.62279331)
\curveto(195.7753928,227.48907865)(195.80068586,227.34700683)(195.85127197,227.19657784)
\curveto(195.90185808,227.05450601)(195.98616827,226.92079135)(196.10420253,226.79543386)
\curveto(196.2222368,226.67007637)(196.37821064,226.56561179)(196.57212407,226.48204013)
\curveto(196.7660375,226.40682564)(197.00210602,226.36921839)(197.28032964,226.36921839)
\curveto(197.54169122,226.36921839)(197.76511322,226.40264705)(197.95059563,226.46950438)
\curveto(198.14450906,226.54471888)(198.3004829,226.64082629)(198.41851717,226.75782661)
\curveto(198.54498245,226.88318411)(198.63772365,227.02107735)(198.69674078,227.17150634)
\curveto(198.75575791,227.32193533)(198.78526648,227.47236432)(198.78526648,227.62279331)
\curveto(198.78526648,228.09915178)(198.6124306,228.46268851)(198.26675883,228.7134035)
\curveto(197.92108706,228.97247565)(197.44473451,229.16886905)(196.83770116,229.30258371)
\curveto(196.50046041,229.11872606)(196.23909883,228.88890399)(196.05361642,228.6131175)
\curveto(195.86813401,228.33733102)(195.7753928,228.00722296)(195.7753928,227.62279331)
\closepath
\moveto(198.60821509,231.59662583)
\curveto(198.60821509,231.70526899)(198.58292203,231.8264479)(198.53233592,231.96016256)
\curveto(198.48174981,232.10223438)(198.40165513,232.22759187)(198.29205188,232.33623503)
\curveto(198.19087966,232.45323536)(198.05598336,232.54934277)(197.88736299,232.62455727)
\curveto(197.71874261,232.70812893)(197.51639816,232.74991476)(197.28032964,232.74991476)
\curveto(197.0358301,232.74991476)(196.82927014,232.71230751)(196.66064977,232.63709302)
\curveto(196.50046041,232.56187852)(196.36556411,232.46577111)(196.25596087,232.34877078)
\curveto(196.14635763,232.24012762)(196.06626295,232.11477013)(196.01567684,231.97269831)
\curveto(195.97352174,231.83898365)(195.9524442,231.70526899)(195.9524442,231.57155433)
\curveto(195.9524442,231.22891052)(196.07469397,230.90715962)(196.31919351,230.60630164)
\curveto(196.57212407,230.31380082)(196.98524399,230.10069308)(197.55855326,229.96697842)
\curveto(197.87893197,230.15083608)(198.13186253,230.3681224)(198.31734494,230.61883739)
\curveto(198.51125837,230.86955237)(198.60821509,231.19548185)(198.60821509,231.59662583)
\closepath
}
}
{
\newrgbcolor{curcolor}{0 0 0}
\pscustom[linestyle=none,fillstyle=solid,fillcolor=curcolor]
{
\newpath
\moveto(218.74148294,230.33051515)
\curveto(218.74148294,228.78443941)(218.3620871,227.61861473)(217.60329542,226.83304111)
\curveto(216.84450374,226.05582466)(215.7021007,225.66303785)(214.17608632,225.65468068)
\lineto(214.13814673,226.53218313)
\curveto(215.08242083,226.53218313)(215.84121251,226.71604078)(216.41452178,227.08375609)
\curveto(216.99626207,227.45982857)(217.37987342,228.10333037)(217.56535583,229.01426148)
\curveto(217.36301139,228.92233265)(217.13958939,228.84711816)(216.89508985,228.78861799)
\curveto(216.65059031,228.738475)(216.39344424,228.7134035)(216.12365164,228.7134035)
\curveto(215.67680765,228.7134035)(215.30162731,228.77608224)(214.99811064,228.90143974)
\curveto(214.69459397,229.0351544)(214.45009443,229.21065489)(214.26461202,229.42794121)
\curveto(214.0791296,229.65358469)(213.9442333,229.90847826)(213.85992312,230.19262191)
\curveto(213.77561293,230.47676556)(213.73345784,230.77762354)(213.73345784,231.09519586)
\curveto(213.73345784,231.37933951)(213.77982844,231.66766174)(213.87256965,231.96016256)
\curveto(213.96531085,232.26102054)(214.10863817,232.53262844)(214.3025516,232.77498626)
\curveto(214.49646503,233.01734408)(214.74518008,233.21791606)(215.04869675,233.37670222)
\curveto(215.35221343,233.53548838)(215.71474723,233.61488146)(216.13629817,233.61488146)
\curveto(216.99626207,233.61488146)(217.64545051,233.32238064)(218.08386348,232.73737901)
\curveto(218.52227646,232.15237738)(218.74148294,231.35008943)(218.74148294,230.33051515)
\closepath
\moveto(216.23747039,229.56583445)
\curveto(216.50726299,229.56583445)(216.75597804,229.59090595)(216.98361554,229.64104894)
\curveto(217.21968407,229.69119194)(217.44732157,229.76222785)(217.66652806,229.85415668)
\curveto(217.67495908,229.93772834)(217.67917459,230.01712142)(217.67917459,230.09233592)
\lineto(217.67917459,230.33051515)
\curveto(217.67917459,230.65644463)(217.65388153,230.96565978)(217.60329542,231.2581606)
\curveto(217.56114033,231.55066141)(217.48104565,231.80555498)(217.36301139,232.0228413)
\curveto(217.25340814,232.24012762)(217.0974343,232.41144953)(216.89508985,232.53680702)
\curveto(216.70117642,232.67052168)(216.45246137,232.73737901)(216.14894469,232.73737901)
\curveto(215.89601413,232.73737901)(215.68523867,232.68305743)(215.51661829,232.57441427)
\curveto(215.34799792,232.47412828)(215.20888611,232.3445922)(215.09928287,232.18580604)
\curveto(214.98967962,232.03537705)(214.90958495,231.86405514)(214.85899883,231.67184032)
\curveto(214.81684374,231.4796255)(214.79576619,231.29576785)(214.79576619,231.12026736)
\curveto(214.79576619,230.61048022)(214.90958495,230.22187199)(215.13722245,229.95444267)
\curveto(215.37329097,229.69537052)(215.74004029,229.56583445)(216.23747039,229.56583445)
\closepath
}
}
{
\newrgbcolor{curcolor}{0 0 0}
\pscustom[linestyle=none,fillstyle=solid,fillcolor=curcolor]
{
\newpath
\moveto(236.15574727,226.48204013)
\curveto(236.15574727,226.23132515)(236.07143708,226.00986024)(235.90281671,225.81764542)
\curveto(235.73419634,225.6254306)(235.51077434,225.52932319)(235.23255072,225.52932319)
\curveto(234.94589609,225.52932319)(234.71825858,225.6254306)(234.54963821,225.81764542)
\curveto(234.38101783,226.00986024)(234.29670765,226.23132515)(234.29670765,226.48204013)
\curveto(234.29670765,226.74111228)(234.38101783,226.96675577)(234.54963821,227.15897059)
\curveto(234.71825858,227.35118541)(234.94589609,227.44729282)(235.23255072,227.44729282)
\curveto(235.51077434,227.44729282)(235.73419634,227.35118541)(235.90281671,227.15897059)
\curveto(236.07143708,226.96675577)(236.15574727,226.74111228)(236.15574727,226.48204013)
\closepath
}
}
{
\newrgbcolor{curcolor}{0 0 0}
\pscustom[linestyle=none,fillstyle=solid,fillcolor=curcolor]
{
\newpath
\moveto(242.47901991,226.48204013)
\curveto(242.47901991,226.23132515)(242.39470973,226.00986024)(242.22608935,225.81764542)
\curveto(242.05746898,225.6254306)(241.83404698,225.52932319)(241.55582337,225.52932319)
\curveto(241.26916873,225.52932319)(241.04153123,225.6254306)(240.87291085,225.81764542)
\curveto(240.70429048,226.00986024)(240.61998029,226.23132515)(240.61998029,226.48204013)
\curveto(240.61998029,226.74111228)(240.70429048,226.96675577)(240.87291085,227.15897059)
\curveto(241.04153123,227.35118541)(241.26916873,227.44729282)(241.55582337,227.44729282)
\curveto(241.83404698,227.44729282)(242.05746898,227.35118541)(242.22608935,227.15897059)
\curveto(242.39470973,226.96675577)(242.47901991,226.74111228)(242.47901991,226.48204013)
\closepath
}
}
{
\newrgbcolor{curcolor}{0 0 0}
\pscustom[linestyle=none,fillstyle=solid,fillcolor=curcolor]
{
\newpath
\moveto(248.80227723,226.48204013)
\curveto(248.80227723,226.23132515)(248.71796704,226.00986024)(248.54934667,225.81764542)
\curveto(248.3807263,225.6254306)(248.1573043,225.52932319)(247.87908068,225.52932319)
\curveto(247.59242605,225.52932319)(247.36478854,225.6254306)(247.19616817,225.81764542)
\curveto(247.02754779,226.00986024)(246.94323761,226.23132515)(246.94323761,226.48204013)
\curveto(246.94323761,226.74111228)(247.02754779,226.96675577)(247.19616817,227.15897059)
\curveto(247.36478854,227.35118541)(247.59242605,227.44729282)(247.87908068,227.44729282)
\curveto(248.1573043,227.44729282)(248.3807263,227.35118541)(248.54934667,227.15897059)
\curveto(248.71796704,226.96675577)(248.80227723,226.74111228)(248.80227723,226.48204013)
\closepath
}
}
{
\newrgbcolor{curcolor}{0 0 0}
\pscustom[linestyle=none,fillstyle=solid,fillcolor=curcolor]
{
\newpath
\moveto(266.02684364,230.4433369)
\curveto(267.15660014,230.40155106)(267.97862447,230.15501466)(268.49291661,229.70372769)
\curveto(269.00720875,229.25244072)(269.26435482,228.64654617)(269.26435482,227.88604405)
\curveto(269.26435482,227.54340023)(269.20955319,227.22582792)(269.09994995,226.9333271)
\curveto(268.99034671,226.64082629)(268.81751083,226.3901113)(268.5814423,226.18118215)
\curveto(268.3538048,225.97225299)(268.05871914,225.80928825)(267.69618534,225.69228793)
\curveto(267.34208255,225.5752876)(266.92053162,225.51678744)(266.43153253,225.51678744)
\curveto(266.22918809,225.51678744)(266.02684364,225.53350177)(265.82449919,225.56693043)
\curveto(265.63058576,225.59200193)(265.44510335,225.6254306)(265.26805195,225.66721643)
\curveto(265.09100056,225.70900226)(264.93502672,225.75078809)(264.80013042,225.79257392)
\curveto(264.66523412,225.84271692)(264.5682774,225.88450275)(264.50926027,225.91793141)
\lineto(264.71160472,226.80796961)
\curveto(264.84650102,226.74111228)(265.05306098,226.6617192)(265.33128459,226.56979038)
\curveto(265.61793923,226.47786155)(265.97625752,226.43189713)(266.40623948,226.43189713)
\curveto(266.74348023,226.43189713)(267.02591935,226.46950438)(267.25355686,226.54471888)
\curveto(267.48119436,226.61993337)(267.66246126,226.72021937)(267.79735756,226.84557686)
\curveto(267.94068488,226.97929152)(268.04185711,227.12972051)(268.10087424,227.29686383)
\curveto(268.16832239,227.46400716)(268.20204646,227.63950765)(268.20204646,227.8233653)
\curveto(268.20204646,228.10750895)(268.15146035,228.35822394)(268.05028812,228.57551026)
\curveto(267.95754692,228.80115374)(267.78892654,228.98918998)(267.544427,229.13961897)
\curveto(267.30835848,229.29004796)(266.97954875,229.40286971)(266.55799781,229.4780842)
\curveto(266.1448779,229.56165586)(265.62215474,229.6034417)(264.98982834,229.6034417)
\curveto(265.04041445,229.97115701)(265.07835403,230.31380082)(265.10364709,230.63137313)
\curveto(265.13737116,230.95730262)(265.16266422,231.27069635)(265.17952626,231.57155433)
\curveto(265.20481931,231.88076948)(265.22168135,232.18580604)(265.23011237,232.48666402)
\curveto(265.24697441,232.78752201)(265.26383644,233.10509432)(265.28069848,233.43938097)
\lineto(269.04936384,233.43938097)
\lineto(269.04936384,232.56187852)
\lineto(266.1912485,232.56187852)
\curveto(266.18281748,232.44487819)(266.17017095,232.29027062)(266.15330892,232.0980558)
\curveto(266.1448779,231.91419814)(266.13223137,231.71780474)(266.11536933,231.50887558)
\curveto(266.0985073,231.29994643)(266.08164526,231.09937444)(266.06478322,230.90715962)
\curveto(266.04792118,230.7149448)(266.03527466,230.56033722)(266.02684364,230.4433369)
\closepath
}
}
{
\newrgbcolor{curcolor}{0 0 0}
\pscustom[linestyle=none,fillstyle=solid,fillcolor=curcolor]
{
\newpath
\moveto(275.30940384,231.45873259)
\curveto(275.30940384,231.19130327)(275.25460222,230.93223112)(275.14499898,230.68151613)
\curveto(275.04382675,230.43080115)(274.90471494,230.18426474)(274.72766355,229.94190693)
\curveto(274.55904318,229.69954911)(274.36512975,229.46136987)(274.14592326,229.22736922)
\curveto(273.92671678,228.99336857)(273.70329478,228.7635465)(273.47565728,228.53790301)
\curveto(273.349192,228.41254552)(273.20164917,228.26211653)(273.03302879,228.08661604)
\curveto(272.86440842,227.91111555)(272.70421907,227.73143647)(272.55246073,227.54757882)
\curveto(272.40070239,227.36372116)(272.27423711,227.18404209)(272.17306489,227.0085416)
\curveto(272.07189266,226.83304111)(272.02130655,226.68261212)(272.02130655,226.55725463)
\lineto(275.62556704,226.55725463)
\lineto(275.62556704,225.67975218)
\lineto(270.88311903,225.67975218)
\curveto(270.87468801,225.72153801)(270.8704725,225.76332384)(270.8704725,225.80510967)
\lineto(270.8704725,225.94300291)
\curveto(270.8704725,226.29400389)(270.92948963,226.61993337)(271.04752389,226.92079135)
\curveto(271.16555815,227.22164934)(271.31731649,227.50579299)(271.5027989,227.7732223)
\curveto(271.68828131,228.04065162)(271.89484127,228.29136661)(272.12247878,228.52536726)
\curveto(272.3585473,228.76772508)(272.59040031,229.00172573)(272.81803782,229.22736922)
\curveto(273.00352023,229.41122687)(273.18057162,229.59090595)(273.349192,229.76640644)
\curveto(273.52624339,229.94190693)(273.67800172,230.11740742)(273.80446701,230.2929079)
\curveto(273.9393633,230.46840839)(274.04475104,230.64808747)(274.12063021,230.83194512)
\curveto(274.20494039,231.02415994)(274.24709549,231.22055335)(274.24709549,231.42112534)
\curveto(274.24709549,231.64676882)(274.2091559,231.83898365)(274.13327673,231.9977698)
\curveto(274.06582858,232.15655596)(273.96887187,232.28609204)(273.84240659,232.38637803)
\curveto(273.72437233,232.49502119)(273.58526052,232.57441427)(273.42507116,232.62455727)
\curveto(273.26488181,232.67470026)(273.09204593,232.69977176)(272.90656351,232.69977176)
\curveto(272.68735703,232.69977176)(272.48501258,232.67052168)(272.29953017,232.61202152)
\curveto(272.12247878,232.55352135)(271.96228942,232.48248544)(271.8189621,232.39891378)
\curveto(271.6840658,232.31534212)(271.56603154,232.23177046)(271.46485932,232.14819879)
\curveto(271.36368709,232.0729843)(271.28780792,232.01030555)(271.23722181,231.96016256)
\lineto(270.71871416,232.68723601)
\curveto(270.78616231,232.76245051)(270.88733454,232.85437934)(271.02223084,232.9630225)
\curveto(271.15712713,233.07166566)(271.31731649,233.17195165)(271.5027989,233.26388048)
\curveto(271.69671233,233.36416647)(271.91170331,233.44773813)(272.14777183,233.51459546)
\curveto(272.38384035,233.58145279)(272.63677092,233.61488146)(272.90656351,233.61488146)
\curveto(273.72437233,233.61488146)(274.32719016,233.42684522)(274.71501702,233.05077274)
\curveto(275.1112749,232.68305743)(275.30940384,232.15237738)(275.30940384,231.45873259)
\closepath
}
}
{
\newrgbcolor{curcolor}{0 0 0}
\pscustom[linewidth=0.56097835,linecolor=curcolor]
{
\newpath
\moveto(52.53491142,235.92539772)
\lineto(277.23287895,235.92539772)
\lineto(277.23287895,223.20627865)
\lineto(52.53491142,223.20627865)
\closepath
}
}
{
\newrgbcolor{curcolor}{0 0 0}
\pscustom[linewidth=0.51800942,linecolor=curcolor]
{
\newpath
\moveto(71.388229,223.22038733)
\lineto(71.388229,235.81978233)
}
}
{
\newrgbcolor{curcolor}{0 0 0}
\pscustom[linewidth=0.5154072,linecolor=curcolor]
{
\newpath
\moveto(90.268452,223.31469833)
\lineto(90.268452,235.78782433)
}
}
{
\newrgbcolor{curcolor}{0 0 0}
\pscustom[linewidth=0.51930571,linecolor=curcolor]
{
\newpath
\moveto(109.107742,223.31469733)
\lineto(109.107742,235.97722633)
}
}
{
\newrgbcolor{curcolor}{0 0 0}
\pscustom[linewidth=0.52573889,linecolor=curcolor]
{
\newpath
\moveto(128.036312,223.13612633)
\lineto(128.036312,236.11432833)
}
}
{
\newrgbcolor{curcolor}{0 0 0}
\pscustom[linewidth=0.51670998,linecolor=curcolor]
{
\newpath
\moveto(146.875592,223.31469733)
\lineto(146.875592,235.85095733)
}
}
{
\newrgbcolor{curcolor}{0 0 0}
\pscustom[linewidth=0.52059871,linecolor=curcolor]
{
\newpath
\moveto(165.625602,223.31469833)
\lineto(165.625602,236.04036233)
}
}
{
\newrgbcolor{curcolor}{0 0 0}
\pscustom[linewidth=0.51670998,linecolor=curcolor]
{
\newpath
\moveto(184.554172,223.22541233)
\lineto(184.554172,235.76167233)
}
}
{
\newrgbcolor{curcolor}{0 0 0}
\pscustom[linewidth=0.51670998,linecolor=curcolor]
{
\newpath
\moveto(203.661312,223.31469833)
\lineto(203.661312,235.85095833)
}
}
{
\newrgbcolor{curcolor}{0 0 0}
\pscustom[linewidth=0.52059871,linecolor=curcolor]
{
\newpath
\moveto(222.589882,223.31469733)
\lineto(222.589882,236.04036133)
}
}
{
\newrgbcolor{curcolor}{0 0 0}
\pscustom[linewidth=0.52573889,linecolor=curcolor]
{
\newpath
\moveto(258.240192,235.96647233)
\lineto(258.204692,222.98831833)
}
}
{
\newrgbcolor{curcolor}{0 1 0}
\pscustom[linestyle=none,fillstyle=solid,fillcolor=curcolor]
{
\newpath
\moveto(52.43154639,211.64116586)
\lineto(71.27951544,211.64116586)
\lineto(71.27951544,198.82485497)
\lineto(52.43154639,198.82485497)
\closepath
}
}
{
\newrgbcolor{curcolor}{0 0 0}
\pscustom[linestyle=none,fillstyle=solid,fillcolor=curcolor]
{
\newpath
\moveto(62.27863502,207.44255206)
\curveto(62.70861698,207.60969539)(63.1259524,207.81862454)(63.5306413,208.06933953)
\curveto(63.9353302,208.32841168)(64.31051053,208.65434116)(64.6561823,209.04712797)
\lineto(65.38968092,209.04712797)
\lineto(65.38968092,202.16500163)
\lineto(66.8693247,202.16500163)
\lineto(66.8693247,201.28749918)
\lineto(62.65803086,201.28749918)
\lineto(62.65803086,202.16500163)
\lineto(64.35266562,202.16500163)
\lineto(64.35266562,207.6055168)
\curveto(64.25992442,207.52194514)(64.14610566,207.4341949)(64.01120936,207.34226607)
\curveto(63.88474408,207.25869441)(63.74141677,207.17512275)(63.58122741,207.09155108)
\curveto(63.42946907,207.00797942)(63.26927972,206.92858634)(63.10065935,206.85337185)
\curveto(62.93203897,206.77815735)(62.76763411,206.71547861)(62.60744475,206.66533561)
\lineto(62.27863502,207.44255206)
\closepath
}
}
{
\newrgbcolor{curcolor}{0 0 0}
\pscustom[linestyle=none,fillstyle=solid,fillcolor=curcolor]
{
\newpath
\moveto(85.61147926,207.06647959)
\curveto(85.61147926,206.79905027)(85.55667764,206.53997812)(85.4470744,206.28926313)
\curveto(85.34590218,206.03854815)(85.20679037,205.79201174)(85.02973897,205.54965393)
\curveto(84.8611186,205.30729611)(84.66720517,205.06911687)(84.44799868,204.83511622)
\curveto(84.2287922,204.60111557)(84.0053702,204.3712935)(83.7777327,204.14565001)
\curveto(83.65126742,204.02029252)(83.50372459,203.86986353)(83.33510422,203.69436304)
\curveto(83.16648384,203.51886255)(83.00629449,203.33918347)(82.85453615,203.15532582)
\curveto(82.70277781,202.97146816)(82.57631253,202.79178909)(82.47514031,202.6162886)
\curveto(82.37396809,202.44078811)(82.32338197,202.29035912)(82.32338197,202.16500163)
\lineto(85.92764247,202.16500163)
\lineto(85.92764247,201.28749918)
\lineto(81.18519445,201.28749918)
\curveto(81.17676343,201.32928501)(81.17254792,201.37107084)(81.17254792,201.41285667)
\lineto(81.17254792,201.55074991)
\curveto(81.17254792,201.90175089)(81.23156505,202.22768037)(81.34959931,202.52853835)
\curveto(81.46763358,202.82939634)(81.61939191,203.11353999)(81.80487432,203.3809693)
\curveto(81.99035673,203.64839862)(82.19691669,203.89911361)(82.4245542,204.13311426)
\curveto(82.66062272,204.37547208)(82.89247573,204.60947273)(83.12011324,204.83511622)
\curveto(83.30559565,205.01897387)(83.48264704,205.19865295)(83.65126742,205.37415344)
\curveto(83.82831881,205.54965393)(83.98007715,205.72515442)(84.10654243,205.9006549)
\curveto(84.24143873,206.07615539)(84.34682646,206.25583447)(84.42270563,206.43969212)
\curveto(84.50701582,206.63190694)(84.54917091,206.82830035)(84.54917091,207.02887234)
\curveto(84.54917091,207.25451582)(84.51123132,207.44673065)(84.43535216,207.6055168)
\curveto(84.36790401,207.76430296)(84.27094729,207.89383904)(84.14448201,207.99412503)
\curveto(84.02644775,208.10276819)(83.88733594,208.18216127)(83.72714659,208.23230427)
\curveto(83.56695723,208.28244726)(83.39412135,208.30751876)(83.20863894,208.30751876)
\curveto(82.98943245,208.30751876)(82.787088,208.27826868)(82.60160559,208.21976852)
\curveto(82.4245542,208.16126835)(82.26436484,208.09023244)(82.12103752,208.00666078)
\curveto(81.98614123,207.92308912)(81.86810696,207.83951746)(81.76693474,207.75594579)
\curveto(81.66576251,207.6807313)(81.58988335,207.61805255)(81.53929723,207.56790956)
\lineto(81.02078958,208.29498301)
\curveto(81.08823773,208.37019751)(81.18940996,208.46212634)(81.32430626,208.5707695)
\curveto(81.45920256,208.67941266)(81.61939191,208.77969865)(81.80487432,208.87162748)
\curveto(81.99878775,208.97191347)(82.21377873,209.05548513)(82.44984725,209.12234246)
\curveto(82.68591578,209.18919979)(82.93884634,209.22262846)(83.20863894,209.22262846)
\curveto(84.02644775,209.22262846)(84.62926559,209.03459222)(85.01709245,208.65851974)
\curveto(85.41335033,208.29080443)(85.61147926,207.76012438)(85.61147926,207.06647959)
\closepath
}
}
{
\newrgbcolor{curcolor}{0 0 0}
\pscustom[linestyle=none,fillstyle=solid,fillcolor=curcolor]
{
\newpath
\moveto(101.95079254,202.03964413)
\curveto(102.61684302,202.03964413)(103.08898006,202.16918021)(103.36720368,202.42825236)
\curveto(103.65385832,202.69568168)(103.79718563,203.05086124)(103.79718563,203.49379105)
\curveto(103.79718563,203.7779347)(103.7381685,204.01611393)(103.62013424,204.20832876)
\curveto(103.50209998,204.40054358)(103.34612613,204.55515115)(103.1522127,204.67215148)
\curveto(102.95829927,204.7891518)(102.73487728,204.87272347)(102.48194672,204.92286646)
\curveto(102.22901616,204.97300946)(101.96343907,204.99808096)(101.68521545,204.99808096)
\lineto(101.41963836,204.99808096)
\lineto(101.41963836,205.83797616)
\lineto(101.78638767,205.83797616)
\curveto(101.97187009,205.83797616)(102.16156801,205.85469049)(102.35548144,205.88811916)
\curveto(102.55782589,205.92990499)(102.73909279,205.99676232)(102.89928214,206.08869114)
\curveto(103.06790252,206.18897714)(103.20279882,206.3226918)(103.30397104,206.48983512)
\curveto(103.40514326,206.65697844)(103.45572938,206.87008618)(103.45572938,207.12915833)
\curveto(103.45572938,207.55537381)(103.32083308,207.85623179)(103.05104048,208.03173228)
\curveto(102.7896789,208.21558993)(102.48194672,208.30751876)(102.12784393,208.30751876)
\curveto(101.76531013,208.30751876)(101.45757795,208.25319718)(101.20464738,208.14455402)
\curveto(100.95171682,208.04426803)(100.74094136,207.93980345)(100.57232098,207.83116029)
\lineto(100.16763209,208.62091249)
\curveto(100.34468348,208.74626999)(100.61026057,208.87580606)(100.96436335,209.00952072)
\curveto(101.32689716,209.15159255)(101.72737054,209.22262846)(102.16578352,209.22262846)
\curveto(102.57890343,209.22262846)(102.93300622,209.17248546)(103.22809187,209.07219947)
\curveto(103.52317753,208.97191347)(103.76346156,208.82984165)(103.94894397,208.64598399)
\curveto(104.1428574,208.46212634)(104.28618472,208.24484002)(104.37892592,207.99412503)
\curveto(104.47166713,207.75176721)(104.51803773,207.48433789)(104.51803773,207.19183708)
\curveto(104.51803773,206.78233594)(104.40843449,206.43551354)(104.189228,206.15136989)
\curveto(103.97845254,205.86722624)(103.70444443,205.64993992)(103.36720368,205.49951093)
\curveto(103.77189258,205.3825106)(104.12177985,205.15268853)(104.41686551,204.81004472)
\curveto(104.71195116,204.47575807)(104.85949399,204.02864968)(104.85949399,203.46871955)
\curveto(104.85949399,203.1344329)(104.80047686,202.82103917)(104.6824426,202.52853835)
\curveto(104.57283935,202.2443947)(104.40000347,201.9978583)(104.16393495,201.78892915)
\curveto(103.93629744,201.57999999)(103.63699628,201.41703525)(103.26603146,201.30003493)
\curveto(102.90349765,201.1830346)(102.46930019,201.12453444)(101.96343907,201.12453444)
\curveto(101.76952564,201.12453444)(101.56718119,201.14124877)(101.35640572,201.17467743)
\curveto(101.15406127,201.19974893)(100.96436335,201.23735618)(100.78731196,201.28749918)
\curveto(100.61026057,201.32928501)(100.45007121,201.37107084)(100.30674389,201.41285667)
\curveto(100.17184759,201.46299967)(100.07489088,201.50060692)(100.01587375,201.52567841)
\lineto(100.2182182,202.41571661)
\curveto(100.3531145,202.34885928)(100.56810547,202.2694662)(100.86319113,202.17753738)
\curveto(101.15827678,202.08560855)(101.52081059,202.03964413)(101.95079254,202.03964413)
\closepath
}
}
{
\newrgbcolor{curcolor}{0 0 0}
\pscustom[linestyle=none,fillstyle=solid,fillcolor=curcolor]
{
\newpath
\moveto(118.69479854,203.97014952)
\curveto(118.83812586,204.30443617)(119.03203929,204.68886581)(119.27653883,205.12343845)
\curveto(119.52946939,205.55801109)(119.81190852,206.00511948)(120.12385621,206.46476362)
\curveto(120.4358039,206.93276493)(120.76882914,207.38823048)(121.12293193,207.83116029)
\curveto(121.47703471,208.28244726)(121.83535301,208.68776982)(122.19788681,209.04712797)
\lineto(123.20960906,209.04712797)
\lineto(123.20960906,204.12057851)
\lineto(124.1328056,204.12057851)
\lineto(124.1328056,203.26814756)
\lineto(123.20960906,203.26814756)
\lineto(123.20960906,201.28749918)
\lineto(122.19788681,201.28749918)
\lineto(122.19788681,203.26814756)
\lineto(118.69479854,203.26814756)
\lineto(118.69479854,203.97014952)
\closepath
\moveto(122.19788681,207.81862454)
\curveto(121.97024931,207.57626672)(121.73839629,207.3088374)(121.50232777,207.01633659)
\curveto(121.27469026,206.72383577)(121.05126827,206.41462062)(120.83206178,206.08869114)
\curveto(120.6128553,205.77111883)(120.40629534,205.44518935)(120.21238191,205.1109027)
\curveto(120.0268995,204.77661605)(119.85827912,204.44650799)(119.70652079,204.12057851)
\lineto(122.19788681,204.12057851)
\lineto(122.19788681,207.81862454)
\closepath
}
}
{
\newrgbcolor{curcolor}{0 0 0}
\pscustom[linestyle=none,fillstyle=solid,fillcolor=curcolor]
{
\newpath
\moveto(139.56156503,206.0510839)
\curveto(140.69132153,206.00929806)(141.51334586,205.76276166)(142.027638,205.31147469)
\curveto(142.54193014,204.86018772)(142.79907621,204.25429317)(142.79907621,203.49379105)
\curveto(142.79907621,203.15114723)(142.74427458,202.83357492)(142.63467134,202.5410741)
\curveto(142.5250681,202.24857329)(142.35223222,201.9978583)(142.11616369,201.78892915)
\curveto(141.88852619,201.57999999)(141.59344053,201.41703525)(141.23090673,201.30003493)
\curveto(140.87680394,201.1830346)(140.45525301,201.12453444)(139.96625392,201.12453444)
\curveto(139.76390948,201.12453444)(139.56156503,201.14124877)(139.35922058,201.17467743)
\curveto(139.16530715,201.19974893)(138.97982474,201.2331776)(138.80277334,201.27496343)
\curveto(138.62572195,201.31674926)(138.46974811,201.35853509)(138.33485181,201.40032092)
\curveto(138.19995551,201.45046392)(138.10299879,201.49224975)(138.04398166,201.52567841)
\lineto(138.24632611,202.41571661)
\curveto(138.38122241,202.34885928)(138.58778237,202.2694662)(138.86600598,202.17753738)
\curveto(139.15266062,202.08560855)(139.51097891,202.03964413)(139.94096087,202.03964413)
\curveto(140.27820162,202.03964413)(140.56064074,202.07725138)(140.78827825,202.15246588)
\curveto(141.01591575,202.22768037)(141.19718265,202.32796637)(141.33207895,202.45332386)
\curveto(141.47540627,202.58703852)(141.5765785,202.73746751)(141.63559563,202.90461083)
\curveto(141.70304378,203.07175416)(141.73676785,203.24725465)(141.73676785,203.4311123)
\curveto(141.73676785,203.71525595)(141.68618174,203.96597094)(141.58500951,204.18325726)
\curveto(141.49226831,204.40890074)(141.32364793,204.59693698)(141.07914839,204.74736597)
\curveto(140.84307987,204.89779496)(140.51427014,205.01061671)(140.0927192,205.0858312)
\curveto(139.67959929,205.16940286)(139.15687613,205.2111887)(138.52454973,205.2111887)
\curveto(138.57513584,205.57890401)(138.61307542,205.92154782)(138.63836848,206.23912013)
\curveto(138.67209255,206.56504962)(138.69738561,206.87844335)(138.71424765,207.17930133)
\curveto(138.7395407,207.48851648)(138.75640274,207.79355304)(138.76483376,208.09441102)
\curveto(138.7816958,208.39526901)(138.79855784,208.71284132)(138.81541987,209.04712797)
\lineto(142.58408523,209.04712797)
\lineto(142.58408523,208.16962552)
\lineto(139.72596989,208.16962552)
\curveto(139.71753887,208.05262519)(139.70489234,207.89801762)(139.68803031,207.7058028)
\curveto(139.67959929,207.52194514)(139.66695276,207.32555174)(139.65009072,207.11662258)
\curveto(139.63322869,206.90769343)(139.61636665,206.70712144)(139.59950461,206.51490662)
\curveto(139.58264257,206.3226918)(139.56999605,206.16808422)(139.56156503,206.0510839)
\closepath
}
}
{
\newrgbcolor{curcolor}{0 0 0}
\pscustom[linestyle=none,fillstyle=solid,fillcolor=curcolor]
{
\newpath
\moveto(156.86201827,204.39636499)
\curveto(156.86201827,205.14850995)(156.96319049,205.81290466)(157.16553494,206.38954913)
\curveto(157.37631041,206.97455076)(157.67139606,207.46344498)(158.0507919,207.85623179)
\curveto(158.43861876,208.2490186)(158.9065403,208.54987658)(159.45455651,208.75880574)
\curveto(160.00257273,208.96773489)(160.6222526,209.07637805)(161.31359614,209.08473522)
\lineto(161.40212183,208.20723277)
\curveto(160.95527784,208.1988756)(160.54637344,208.1487326)(160.17540861,208.05680378)
\curveto(159.81287481,207.97323212)(159.48828059,207.83116029)(159.20162595,207.6305883)
\curveto(158.91497132,207.43837348)(158.67468728,207.1876585)(158.48077385,206.87844335)
\curveto(158.28686042,206.5692282)(158.1393176,206.18897714)(158.03814537,205.73769016)
\curveto(158.24048982,205.82961899)(158.45969631,205.90483349)(158.69576483,205.96333365)
\curveto(158.94026437,206.02183381)(159.19741044,206.0510839)(159.46720304,206.0510839)
\curveto(159.90561601,206.0510839)(160.27658084,205.98422657)(160.58009751,205.85051191)
\curveto(160.8920452,205.71679725)(161.14076025,205.53711818)(161.32624266,205.31147469)
\curveto(161.51172508,205.09418837)(161.64662138,204.8392948)(161.73093156,204.54679398)
\curveto(161.81524175,204.25429317)(161.85739684,203.95343519)(161.85739684,203.64422004)
\curveto(161.85739684,203.36007639)(161.81102624,203.06757557)(161.71828503,202.76671759)
\curveto(161.62554383,202.47421677)(161.48221651,202.20260887)(161.28830308,201.95189389)
\curveto(161.09438965,201.70953607)(160.8456746,201.50896408)(160.54215793,201.35017792)
\curveto(160.23864125,201.19974893)(159.87610745,201.12453444)(159.45455651,201.12453444)
\curveto(158.58616159,201.12453444)(157.93697315,201.41285667)(157.5069912,201.98950114)
\curveto(157.07700924,202.5661456)(156.86201827,203.36843355)(156.86201827,204.39636499)
\closepath
\moveto(159.35338429,205.19865295)
\curveto(159.08359169,205.19865295)(158.83487664,205.17358145)(158.60723914,205.12343845)
\curveto(158.38803265,205.07329545)(158.16461065,204.99808096)(157.93697315,204.89779496)
\curveto(157.92854213,204.8142233)(157.92432662,204.73065164)(157.92432662,204.64707998)
\lineto(157.92432662,204.39636499)
\curveto(157.92432662,204.07043551)(157.94540417,203.76122036)(157.98755926,203.46871955)
\curveto(158.03814537,203.1845759)(158.11824005,202.92968233)(158.22784329,202.70403884)
\curveto(158.34587756,202.48675252)(158.5018514,202.31125203)(158.69576483,202.17753738)
\curveto(158.88967826,202.05217988)(159.13839331,201.98950114)(159.44190999,201.98950114)
\curveto(159.69484055,201.98950114)(159.90561601,202.03964413)(160.07423639,202.13993013)
\curveto(160.24285676,202.24857329)(160.38196857,202.38228795)(160.49157181,202.5410741)
\curveto(160.60117506,202.69986026)(160.67705423,202.87536075)(160.71920932,203.06757557)
\curveto(160.76979543,203.26814756)(160.79508849,203.4561838)(160.79508849,203.63168429)
\curveto(160.79508849,204.14147143)(160.67705423,204.53007965)(160.4409857,204.79750897)
\curveto(160.2133482,205.06493829)(159.85081439,205.19865295)(159.35338429,205.19865295)
\closepath
}
}
{
\newrgbcolor{curcolor}{0 0 0}
\pscustom[linestyle=none,fillstyle=solid,fillcolor=curcolor]
{
\newpath
\moveto(177.10911254,201.28749918)
\curveto(177.15126763,201.88085798)(177.25665536,202.50764544)(177.42527574,203.16786157)
\curveto(177.60232713,203.83643486)(177.8131026,204.47993666)(178.05760214,205.09836695)
\curveto(178.30210168,205.72515442)(178.57189428,206.30179888)(178.86697994,206.82830035)
\curveto(179.16206559,207.36315898)(179.45293573,207.80191021)(179.73959037,208.14455402)
\lineto(175.94563196,208.14455402)
\lineto(175.94563196,209.04712797)
\lineto(180.90307095,209.04712797)
\lineto(180.90307095,208.18216127)
\curveto(180.65014039,207.88966045)(180.37613228,207.49687364)(180.08104663,207.00380084)
\curveto(179.78596097,206.51072804)(179.50352185,205.95497648)(179.23372925,205.33654619)
\curveto(178.97236767,204.72647306)(178.74473016,204.07043551)(178.55081673,203.36843355)
\curveto(178.3569033,202.67478876)(178.23465353,201.98114397)(178.18406742,201.28749918)
\lineto(177.10911254,201.28749918)
\closepath
}
}
{
\newrgbcolor{curcolor}{0 0 0}
\pscustom[linestyle=none,fillstyle=solid,fillcolor=curcolor]
{
\newpath
\moveto(199.79698672,203.30575481)
\curveto(199.79698672,202.65389585)(199.58621125,202.12739438)(199.16466032,201.7262504)
\curveto(198.7515404,201.32510643)(198.119214,201.12453444)(197.26768111,201.12453444)
\curveto(196.77868203,201.12453444)(196.37399313,201.18721318)(196.05361442,201.31257068)
\curveto(195.73323571,201.44628534)(195.47608964,201.61342866)(195.28217621,201.81400065)
\curveto(195.0966938,202.0229298)(194.9617975,202.25275187)(194.87748731,202.50346686)
\curveto(194.80160814,202.75418184)(194.76366856,203.00071824)(194.76366856,203.24307606)
\curveto(194.76366856,203.68600587)(194.88591833,204.07879268)(195.13041787,204.42143649)
\curveto(195.37491742,204.76408031)(195.66578756,205.03986679)(196.00302831,205.24879594)
\curveto(195.28639172,205.64993992)(194.92807343,206.26419163)(194.92807343,207.09155108)
\curveto(194.92807343,207.37569473)(194.98287505,207.64730263)(195.09247829,207.90637479)
\curveto(195.20208153,208.16544694)(195.35805538,208.39109042)(195.56039983,208.58330525)
\curveto(195.76274428,208.77552007)(196.00724382,208.93012764)(196.29389845,209.04712797)
\curveto(196.58898411,209.16412829)(196.91779384,209.22262846)(197.28032764,209.22262846)
\curveto(197.70187858,209.22262846)(198.06019687,209.15994971)(198.35528253,209.03459222)
\curveto(198.6587992,208.90923473)(198.90329874,208.74626999)(199.08878115,208.545698)
\curveto(199.27426356,208.35348318)(199.40915986,208.13619686)(199.49347005,207.89383904)
\curveto(199.57778024,207.65983838)(199.61993533,207.43001631)(199.61993533,207.20437283)
\curveto(199.61993533,206.76144302)(199.50611658,206.37701338)(199.27847907,206.0510839)
\curveto(199.05084157,205.73351158)(198.78947999,205.47861801)(198.49439433,205.28640319)
\curveto(199.36278926,204.87690205)(199.79698672,204.21668592)(199.79698672,203.30575481)
\closepath
\moveto(195.7753908,203.23054031)
\curveto(195.7753908,203.09682565)(195.80068386,202.95475383)(195.85126997,202.80432484)
\curveto(195.90185608,202.66225301)(195.98616627,202.52853835)(196.10420053,202.40318086)
\curveto(196.2222348,202.27782337)(196.37820864,202.17335879)(196.57212207,202.08978713)
\curveto(196.7660355,202.01457264)(197.00210402,201.97696539)(197.28032764,201.97696539)
\curveto(197.54168922,201.97696539)(197.76511122,202.01039405)(197.95059363,202.07725138)
\curveto(198.14450706,202.15246588)(198.3004809,202.24857329)(198.41851517,202.36557361)
\curveto(198.54498045,202.49093111)(198.63772165,202.62882435)(198.69673878,202.77925334)
\curveto(198.75575591,202.92968233)(198.78526448,203.08011132)(198.78526448,203.23054031)
\curveto(198.78526448,203.70689878)(198.6124286,204.07043551)(198.26675683,204.3211505)
\curveto(197.92108506,204.58022265)(197.44473251,204.77661605)(196.83769916,204.91033071)
\curveto(196.50045841,204.72647306)(196.23909683,204.49665099)(196.05361442,204.2208645)
\curveto(195.86813201,203.94507802)(195.7753908,203.61496996)(195.7753908,203.23054031)
\closepath
\moveto(198.60821309,207.20437283)
\curveto(198.60821309,207.31301599)(198.58292003,207.4341949)(198.53233392,207.56790956)
\curveto(198.48174781,207.70998138)(198.40165313,207.83533887)(198.29204988,207.94398203)
\curveto(198.19087766,208.06098236)(198.05598136,208.15708977)(197.88736099,208.23230427)
\curveto(197.71874061,208.31587593)(197.51639616,208.35766176)(197.28032764,208.35766176)
\curveto(197.0358281,208.35766176)(196.82926814,208.32005451)(196.66064777,208.24484002)
\curveto(196.50045841,208.16962552)(196.36556211,208.07351811)(196.25595887,207.95651778)
\curveto(196.14635563,207.84787462)(196.06626095,207.72251713)(196.01567484,207.58044531)
\curveto(195.97351974,207.44673065)(195.9524422,207.31301599)(195.9524422,207.17930133)
\curveto(195.9524422,206.83665752)(196.07469197,206.51490662)(196.31919151,206.21404864)
\curveto(196.57212207,205.92154782)(196.98524199,205.70844008)(197.55855126,205.57472542)
\curveto(197.87892997,205.75858308)(198.13186053,205.9758694)(198.31734294,206.22658439)
\curveto(198.51125637,206.47729937)(198.60821309,206.80322885)(198.60821309,207.20437283)
\closepath
}
}
{
\newrgbcolor{curcolor}{0 0 0}
\pscustom[linestyle=none,fillstyle=solid,fillcolor=curcolor]
{
\newpath
\moveto(218.74148094,205.93826215)
\curveto(218.74148094,204.39218641)(218.3620851,203.22636173)(217.60329342,202.44078811)
\curveto(216.84450174,201.66357166)(215.7020987,201.27078485)(214.17608432,201.26242768)
\lineto(214.13814473,202.13993013)
\curveto(215.08241883,202.13993013)(215.84121051,202.32378778)(216.41451978,202.69150309)
\curveto(216.99626007,203.06757557)(217.37987142,203.71107737)(217.56535383,204.62200848)
\curveto(217.36300939,204.53007965)(217.13958739,204.45486516)(216.89508785,204.39636499)
\curveto(216.65058831,204.346222)(216.39344224,204.3211505)(216.12364964,204.3211505)
\curveto(215.67680565,204.3211505)(215.30162531,204.38382924)(214.99810864,204.50918674)
\curveto(214.69459197,204.6429014)(214.45009243,204.81840189)(214.26461002,205.03568821)
\curveto(214.0791276,205.26133169)(213.9442313,205.51622526)(213.85992112,205.80036891)
\curveto(213.77561093,206.08451256)(213.73345584,206.38537054)(213.73345584,206.70294286)
\curveto(213.73345584,206.98708651)(213.77982644,207.27540874)(213.87256765,207.56790956)
\curveto(213.96530885,207.86876754)(214.10863617,208.14037544)(214.3025496,208.38273326)
\curveto(214.49646303,208.62509108)(214.74517808,208.82566306)(215.04869475,208.98444922)
\curveto(215.35221143,209.14323538)(215.71474523,209.22262846)(216.13629617,209.22262846)
\curveto(216.99626007,209.22262846)(217.64544851,208.93012764)(218.08386148,208.34512601)
\curveto(218.52227446,207.76012438)(218.74148094,206.95783643)(218.74148094,205.93826215)
\closepath
\moveto(216.23746839,205.17358145)
\curveto(216.50726099,205.17358145)(216.75597604,205.19865295)(216.98361354,205.24879594)
\curveto(217.21968207,205.29893894)(217.44731957,205.36997485)(217.66652606,205.46190368)
\curveto(217.67495708,205.54547534)(217.67917259,205.62486842)(217.67917259,205.70008292)
\lineto(217.67917259,205.93826215)
\curveto(217.67917259,206.26419163)(217.65387953,206.57340678)(217.60329342,206.8659076)
\curveto(217.56113833,207.15840841)(217.48104365,207.41330198)(217.36300939,207.6305883)
\curveto(217.25340614,207.84787462)(217.0974323,208.01919653)(216.89508785,208.14455402)
\curveto(216.70117442,208.27826868)(216.45245937,208.34512601)(216.14894269,208.34512601)
\curveto(215.89601213,208.34512601)(215.68523667,208.29080443)(215.51661629,208.18216127)
\curveto(215.34799592,208.08187528)(215.20888411,207.9523392)(215.09928087,207.79355304)
\curveto(214.98967762,207.64312405)(214.90958295,207.47180214)(214.85899683,207.27958732)
\curveto(214.81684174,207.0873725)(214.79576419,206.90351485)(214.79576419,206.72801436)
\curveto(214.79576419,206.21822722)(214.90958295,205.82961899)(215.13722045,205.56218967)
\curveto(215.37328897,205.30311752)(215.74003829,205.17358145)(216.23746839,205.17358145)
\closepath
}
}
{
\newrgbcolor{curcolor}{0 0 0}
\pscustom[linestyle=none,fillstyle=solid,fillcolor=curcolor]
{
\newpath
\moveto(236.15574527,202.08978713)
\curveto(236.15574527,201.83907215)(236.07143508,201.61760724)(235.90281471,201.42539242)
\curveto(235.73419434,201.2331776)(235.51077234,201.13707019)(235.23254872,201.13707019)
\curveto(234.94589409,201.13707019)(234.71825658,201.2331776)(234.54963621,201.42539242)
\curveto(234.38101583,201.61760724)(234.29670565,201.83907215)(234.29670565,202.08978713)
\curveto(234.29670565,202.34885928)(234.38101583,202.57450277)(234.54963621,202.76671759)
\curveto(234.71825658,202.95893241)(234.94589409,203.05503982)(235.23254872,203.05503982)
\curveto(235.51077234,203.05503982)(235.73419434,202.95893241)(235.90281471,202.76671759)
\curveto(236.07143508,202.57450277)(236.15574527,202.34885928)(236.15574527,202.08978713)
\closepath
}
}
{
\newrgbcolor{curcolor}{0 0 0}
\pscustom[linestyle=none,fillstyle=solid,fillcolor=curcolor]
{
\newpath
\moveto(242.47901791,202.08978713)
\curveto(242.47901791,201.83907215)(242.39470773,201.61760724)(242.22608735,201.42539242)
\curveto(242.05746698,201.2331776)(241.83404498,201.13707019)(241.55582137,201.13707019)
\curveto(241.26916673,201.13707019)(241.04152923,201.2331776)(240.87290885,201.42539242)
\curveto(240.70428848,201.61760724)(240.61997829,201.83907215)(240.61997829,202.08978713)
\curveto(240.61997829,202.34885928)(240.70428848,202.57450277)(240.87290885,202.76671759)
\curveto(241.04152923,202.95893241)(241.26916673,203.05503982)(241.55582137,203.05503982)
\curveto(241.83404498,203.05503982)(242.05746698,202.95893241)(242.22608735,202.76671759)
\curveto(242.39470773,202.57450277)(242.47901791,202.34885928)(242.47901791,202.08978713)
\closepath
}
}
{
\newrgbcolor{curcolor}{0 0 0}
\pscustom[linestyle=none,fillstyle=solid,fillcolor=curcolor]
{
\newpath
\moveto(248.80227523,202.08978713)
\curveto(248.80227523,201.83907215)(248.71796504,201.61760724)(248.54934467,201.42539242)
\curveto(248.3807243,201.2331776)(248.1573023,201.13707019)(247.87907868,201.13707019)
\curveto(247.59242405,201.13707019)(247.36478654,201.2331776)(247.19616617,201.42539242)
\curveto(247.02754579,201.61760724)(246.94323561,201.83907215)(246.94323561,202.08978713)
\curveto(246.94323561,202.34885928)(247.02754579,202.57450277)(247.19616617,202.76671759)
\curveto(247.36478654,202.95893241)(247.59242405,203.05503982)(247.87907868,203.05503982)
\curveto(248.1573023,203.05503982)(248.3807243,202.95893241)(248.54934467,202.76671759)
\curveto(248.71796504,202.57450277)(248.80227523,202.34885928)(248.80227523,202.08978713)
\closepath
}
}
{
\newrgbcolor{curcolor}{0 0 0}
\pscustom[linestyle=none,fillstyle=solid,fillcolor=curcolor]
{
\newpath
\moveto(266.02684164,206.0510839)
\curveto(267.15659814,206.00929806)(267.97862247,205.76276166)(268.49291461,205.31147469)
\curveto(269.00720675,204.86018772)(269.26435282,204.25429317)(269.26435282,203.49379105)
\curveto(269.26435282,203.15114723)(269.20955119,202.83357492)(269.09994795,202.5410741)
\curveto(268.99034471,202.24857329)(268.81750883,201.9978583)(268.5814403,201.78892915)
\curveto(268.3538028,201.57999999)(268.05871714,201.41703525)(267.69618334,201.30003493)
\curveto(267.34208055,201.1830346)(266.92052962,201.12453444)(266.43153053,201.12453444)
\curveto(266.22918609,201.12453444)(266.02684164,201.14124877)(265.82449719,201.17467743)
\curveto(265.63058376,201.19974893)(265.44510135,201.2331776)(265.26804995,201.27496343)
\curveto(265.09099856,201.31674926)(264.93502472,201.35853509)(264.80012842,201.40032092)
\curveto(264.66523212,201.45046392)(264.5682754,201.49224975)(264.50925827,201.52567841)
\lineto(264.71160272,202.41571661)
\curveto(264.84649902,202.34885928)(265.05305898,202.2694662)(265.33128259,202.17753738)
\curveto(265.61793723,202.08560855)(265.97625552,202.03964413)(266.40623748,202.03964413)
\curveto(266.74347823,202.03964413)(267.02591735,202.07725138)(267.25355486,202.15246588)
\curveto(267.48119236,202.22768037)(267.66245926,202.32796637)(267.79735556,202.45332386)
\curveto(267.94068288,202.58703852)(268.04185511,202.73746751)(268.10087224,202.90461083)
\curveto(268.16832039,203.07175416)(268.20204446,203.24725465)(268.20204446,203.4311123)
\curveto(268.20204446,203.71525595)(268.15145835,203.96597094)(268.05028612,204.18325726)
\curveto(267.95754492,204.40890074)(267.78892454,204.59693698)(267.544425,204.74736597)
\curveto(267.30835648,204.89779496)(266.97954675,205.01061671)(266.55799581,205.0858312)
\curveto(266.1448759,205.16940286)(265.62215274,205.2111887)(264.98982634,205.2111887)
\curveto(265.04041245,205.57890401)(265.07835203,205.92154782)(265.10364509,206.23912013)
\curveto(265.13736916,206.56504962)(265.16266222,206.87844335)(265.17952426,207.17930133)
\curveto(265.20481731,207.48851648)(265.22167935,207.79355304)(265.23011037,208.09441102)
\curveto(265.24697241,208.39526901)(265.26383444,208.71284132)(265.28069648,209.04712797)
\lineto(269.04936184,209.04712797)
\lineto(269.04936184,208.16962552)
\lineto(266.1912465,208.16962552)
\curveto(266.18281548,208.05262519)(266.17016895,207.89801762)(266.15330692,207.7058028)
\curveto(266.1448759,207.52194514)(266.13222937,207.32555174)(266.11536733,207.11662258)
\curveto(266.0985053,206.90769343)(266.08164326,206.70712144)(266.06478122,206.51490662)
\curveto(266.04791918,206.3226918)(266.03527266,206.16808422)(266.02684164,206.0510839)
\closepath
}
}
{
\newrgbcolor{curcolor}{0 0 0}
\pscustom[linestyle=none,fillstyle=solid,fillcolor=curcolor]
{
\newpath
\moveto(275.30940184,207.06647959)
\curveto(275.30940184,206.79905027)(275.25460022,206.53997812)(275.14499698,206.28926313)
\curveto(275.04382475,206.03854815)(274.90471294,205.79201174)(274.72766155,205.54965393)
\curveto(274.55904118,205.30729611)(274.36512775,205.06911687)(274.14592126,204.83511622)
\curveto(273.92671478,204.60111557)(273.70329278,204.3712935)(273.47565528,204.14565001)
\curveto(273.34919,204.02029252)(273.20164717,203.86986353)(273.03302679,203.69436304)
\curveto(272.86440642,203.51886255)(272.70421707,203.33918347)(272.55245873,203.15532582)
\curveto(272.40070039,202.97146816)(272.27423511,202.79178909)(272.17306289,202.6162886)
\curveto(272.07189066,202.44078811)(272.02130455,202.29035912)(272.02130455,202.16500163)
\lineto(275.62556504,202.16500163)
\lineto(275.62556504,201.28749918)
\lineto(270.88311703,201.28749918)
\curveto(270.87468601,201.32928501)(270.8704705,201.37107084)(270.8704705,201.41285667)
\lineto(270.8704705,201.55074991)
\curveto(270.8704705,201.90175089)(270.92948763,202.22768037)(271.04752189,202.52853835)
\curveto(271.16555615,202.82939634)(271.31731449,203.11353999)(271.5027969,203.3809693)
\curveto(271.68827931,203.64839862)(271.89483927,203.89911361)(272.12247678,204.13311426)
\curveto(272.3585453,204.37547208)(272.59039831,204.60947273)(272.81803582,204.83511622)
\curveto(273.00351823,205.01897387)(273.18056962,205.19865295)(273.34919,205.37415344)
\curveto(273.52624139,205.54965393)(273.67799972,205.72515442)(273.80446501,205.9006549)
\curveto(273.9393613,206.07615539)(274.04474904,206.25583447)(274.12062821,206.43969212)
\curveto(274.20493839,206.63190694)(274.24709349,206.82830035)(274.24709349,207.02887234)
\curveto(274.24709349,207.25451582)(274.2091539,207.44673065)(274.13327473,207.6055168)
\curveto(274.06582658,207.76430296)(273.96886987,207.89383904)(273.84240459,207.99412503)
\curveto(273.72437033,208.10276819)(273.58525852,208.18216127)(273.42506916,208.23230427)
\curveto(273.26487981,208.28244726)(273.09204393,208.30751876)(272.90656151,208.30751876)
\curveto(272.68735503,208.30751876)(272.48501058,208.27826868)(272.29952817,208.21976852)
\curveto(272.12247678,208.16126835)(271.96228742,208.09023244)(271.8189601,208.00666078)
\curveto(271.6840638,207.92308912)(271.56602954,207.83951746)(271.46485732,207.75594579)
\curveto(271.36368509,207.6807313)(271.28780592,207.61805255)(271.23721981,207.56790956)
\lineto(270.71871216,208.29498301)
\curveto(270.78616031,208.37019751)(270.88733254,208.46212634)(271.02222884,208.5707695)
\curveto(271.15712513,208.67941266)(271.31731449,208.77969865)(271.5027969,208.87162748)
\curveto(271.69671033,208.97191347)(271.91170131,209.05548513)(272.14776983,209.12234246)
\curveto(272.38383835,209.18919979)(272.63676892,209.22262846)(272.90656151,209.22262846)
\curveto(273.72437033,209.22262846)(274.32718816,209.03459222)(274.71501502,208.65851974)
\curveto(275.1112729,208.29080443)(275.30940184,207.76012438)(275.30940184,207.06647959)
\closepath
}
}
{
\newrgbcolor{curcolor}{0 0 0}
\pscustom[linewidth=0.56097835,linecolor=curcolor]
{
\newpath
\moveto(52.53490942,211.53314472)
\lineto(277.23287695,211.53314472)
\lineto(277.23287695,198.81402565)
\lineto(52.53490942,198.81402565)
\closepath
}
}
{
\newrgbcolor{curcolor}{0 0 0}
\pscustom[linewidth=0.51800942,linecolor=curcolor]
{
\newpath
\moveto(71.388227,198.82813433)
\lineto(71.388227,211.42752933)
}
}
{
\newrgbcolor{curcolor}{0 0 0}
\pscustom[linewidth=0.5154072,linecolor=curcolor]
{
\newpath
\moveto(90.26845,198.92244533)
\lineto(90.26845,211.39557133)
}
}
{
\newrgbcolor{curcolor}{0 0 0}
\pscustom[linewidth=0.51930571,linecolor=curcolor]
{
\newpath
\moveto(109.10774,198.92244433)
\lineto(109.10774,211.58497333)
}
}
{
\newrgbcolor{curcolor}{0 0 0}
\pscustom[linewidth=0.52573889,linecolor=curcolor]
{
\newpath
\moveto(128.03631,198.74387333)
\lineto(128.03631,211.72207533)
}
}
{
\newrgbcolor{curcolor}{0 0 0}
\pscustom[linewidth=0.51670998,linecolor=curcolor]
{
\newpath
\moveto(146.87559,198.92244433)
\lineto(146.87559,211.45870433)
}
}
{
\newrgbcolor{curcolor}{0 0 0}
\pscustom[linewidth=0.52059871,linecolor=curcolor]
{
\newpath
\moveto(165.6256,198.92244533)
\lineto(165.6256,211.64810933)
}
}
{
\newrgbcolor{curcolor}{0 0 0}
\pscustom[linewidth=0.51670998,linecolor=curcolor]
{
\newpath
\moveto(184.55417,198.83315933)
\lineto(184.55417,211.36941933)
}
}
{
\newrgbcolor{curcolor}{0 0 0}
\pscustom[linewidth=0.51670998,linecolor=curcolor]
{
\newpath
\moveto(203.66131,198.92244533)
\lineto(203.66131,211.45870533)
}
}
{
\newrgbcolor{curcolor}{0 0 0}
\pscustom[linewidth=0.52059871,linecolor=curcolor]
{
\newpath
\moveto(222.58988,198.92244433)
\lineto(222.58988,211.64810833)
}
}
{
\newrgbcolor{curcolor}{0 0 0}
\pscustom[linewidth=0.52573889,linecolor=curcolor]
{
\newpath
\moveto(258.24019,211.57421933)
\lineto(258.20469,198.59606533)
}
}
{
\newrgbcolor{curcolor}{0 1 0}
\pscustom[linestyle=none,fillstyle=solid,fillcolor=curcolor]
{
\newpath
\moveto(71.47393572,189.60860753)
\lineto(90.32190478,189.60860753)
\lineto(90.32190478,176.79229665)
\lineto(71.47393572,176.79229665)
\closepath
}
}
{
\newrgbcolor{curcolor}{0 0 0}
\pscustom[linestyle=none,fillstyle=solid,fillcolor=curcolor]
{
\newpath
\moveto(62.27863088,185.38325054)
\curveto(62.70861283,185.55039386)(63.12594826,185.75932302)(63.53063716,186.010038)
\curveto(63.93532605,186.26911015)(64.31050639,186.59503963)(64.65617815,186.98782644)
\lineto(65.38967678,186.98782644)
\lineto(65.38967678,180.1057001)
\lineto(66.86932056,180.1057001)
\lineto(66.86932056,179.22819765)
\lineto(62.65802672,179.22819765)
\lineto(62.65802672,180.1057001)
\lineto(64.35266148,180.1057001)
\lineto(64.35266148,185.54621528)
\curveto(64.25992027,185.46264362)(64.14610152,185.37489337)(64.01120522,185.28296454)
\curveto(63.88473994,185.19939288)(63.74141262,185.11582122)(63.58122327,185.03224956)
\curveto(63.42946493,184.9486779)(63.26927558,184.86928482)(63.1006552,184.79407032)
\curveto(62.93203483,184.71885583)(62.76762997,184.65617708)(62.60744061,184.60603408)
\lineto(62.27863088,185.38325054)
\closepath
}
}
{
\newrgbcolor{curcolor}{0 0 0}
\pscustom[linestyle=none,fillstyle=solid,fillcolor=curcolor]
{
\newpath
\moveto(76.32892258,183.99178237)
\curveto(77.45867909,183.94999654)(78.28070341,183.70346014)(78.79499555,183.25217316)
\curveto(79.30928769,182.80088619)(79.56643376,182.19499164)(79.56643376,181.43448952)
\curveto(79.56643376,181.09184571)(79.51163214,180.77427339)(79.40202889,180.48177258)
\curveto(79.29242565,180.18927176)(79.11958977,179.93855678)(78.88352125,179.72962762)
\curveto(78.65588374,179.52069847)(78.36079809,179.35773373)(77.99826428,179.2407334)
\curveto(77.6441615,179.12373308)(77.22261056,179.06523291)(76.73361148,179.06523291)
\curveto(76.53126703,179.06523291)(76.32892258,179.08194725)(76.12657813,179.11537591)
\curveto(75.9326647,179.14044741)(75.74718229,179.17387607)(75.5701309,179.2156619)
\curveto(75.3930795,179.25744773)(75.23710566,179.29923357)(75.10220936,179.3410194)
\curveto(74.96731306,179.39116239)(74.87035635,179.43294822)(74.81133921,179.46637689)
\lineto(75.01368366,180.35641509)
\curveto(75.14857996,180.28955776)(75.35513992,180.21016468)(75.63336354,180.11823585)
\curveto(75.92001817,180.02630702)(76.27833647,179.98034261)(76.70831842,179.98034261)
\curveto(77.04555917,179.98034261)(77.3279983,180.01794986)(77.5556358,180.09316435)
\curveto(77.78327331,180.16837885)(77.96454021,180.26866484)(78.09943651,180.39402233)
\curveto(78.24276382,180.52773699)(78.34393605,180.67816598)(78.40295318,180.84530931)
\curveto(78.47040133,181.01245263)(78.5041254,181.18795312)(78.5041254,181.37181078)
\curveto(78.5041254,181.65595443)(78.45353929,181.90666941)(78.35236707,182.12395573)
\curveto(78.25962586,182.34959922)(78.09100549,182.53763546)(77.84650595,182.68806445)
\curveto(77.61043742,182.83849344)(77.28162769,182.95131518)(76.86007676,183.02652968)
\curveto(76.44695684,183.11010134)(75.92423368,183.15188717)(75.29190728,183.15188717)
\curveto(75.34249339,183.51960248)(75.38043298,183.8622463)(75.40572603,184.17981861)
\curveto(75.43945011,184.50574809)(75.46474316,184.81914182)(75.4816052,185.1199998)
\curveto(75.50689826,185.42921495)(75.52376029,185.73425152)(75.53219131,186.0351095)
\curveto(75.54905335,186.33596748)(75.56591539,186.6535398)(75.58277743,186.98782644)
\lineto(79.35144278,186.98782644)
\lineto(79.35144278,186.110324)
\lineto(76.49332744,186.110324)
\curveto(76.48489643,185.99332367)(76.4722499,185.83871609)(76.45538786,185.64650127)
\curveto(76.44695684,185.46264362)(76.43431031,185.26625021)(76.41744828,185.05732106)
\curveto(76.40058624,184.8483919)(76.3837242,184.64781992)(76.36686216,184.45560509)
\curveto(76.35000013,184.26339027)(76.3373536,184.1087827)(76.32892258,183.99178237)
\closepath
}
}
{
\newrgbcolor{curcolor}{0 0 0}
\pscustom[linestyle=none,fillstyle=solid,fillcolor=curcolor]
{
\newpath
\moveto(81.24842295,185.38325054)
\curveto(81.6784049,185.55039386)(82.09574033,185.75932302)(82.50042922,186.010038)
\curveto(82.90511812,186.26911015)(83.28029845,186.59503963)(83.62597022,186.98782644)
\lineto(84.35946885,186.98782644)
\lineto(84.35946885,180.1057001)
\lineto(85.83911263,180.1057001)
\lineto(85.83911263,179.22819765)
\lineto(81.62781879,179.22819765)
\lineto(81.62781879,180.1057001)
\lineto(83.32245355,180.1057001)
\lineto(83.32245355,185.54621528)
\curveto(83.22971234,185.46264362)(83.11589359,185.37489337)(82.98099729,185.28296454)
\curveto(82.85453201,185.19939288)(82.71120469,185.11582122)(82.55101534,185.03224956)
\curveto(82.399257,184.9486779)(82.23906764,184.86928482)(82.07044727,184.79407032)
\curveto(81.9018269,184.71885583)(81.73742203,184.65617708)(81.57723268,184.60603408)
\lineto(81.24842295,185.38325054)
\closepath
}
}
{
\newrgbcolor{curcolor}{0 0 0}
\pscustom[linestyle=none,fillstyle=solid,fillcolor=curcolor]
{
\newpath
\moveto(101.9507884,179.98034261)
\curveto(102.61683887,179.98034261)(103.08897592,180.10987868)(103.36719954,180.36895084)
\curveto(103.65385417,180.63638015)(103.79718149,180.99155972)(103.79718149,181.43448952)
\curveto(103.79718149,181.71863317)(103.73816436,181.95681241)(103.6201301,182.14902723)
\curveto(103.50209584,182.34124205)(103.34612199,182.49584963)(103.15220856,182.61284995)
\curveto(102.95829513,182.72985028)(102.73487314,182.81342194)(102.48194258,182.86356494)
\curveto(102.22901201,182.91370793)(101.96343493,182.93877943)(101.68521131,182.93877943)
\lineto(101.41963422,182.93877943)
\lineto(101.41963422,183.77867463)
\lineto(101.78638353,183.77867463)
\curveto(101.97186594,183.77867463)(102.16156387,183.79538897)(102.3554773,183.82881763)
\curveto(102.55782174,183.87060346)(102.73908865,183.93746079)(102.899278,184.02938962)
\curveto(103.06789837,184.12967561)(103.20279467,184.26339027)(103.3039669,184.43053359)
\curveto(103.40513912,184.59767692)(103.45572523,184.81078466)(103.45572523,185.06985681)
\curveto(103.45572523,185.49607228)(103.32082894,185.79693026)(103.05103634,185.97243075)
\curveto(102.78967476,186.15628841)(102.48194258,186.24821724)(102.12783979,186.24821724)
\curveto(101.76530599,186.24821724)(101.4575738,186.19389566)(101.20464324,186.0852525)
\curveto(100.95171268,185.9849665)(100.74093721,185.88050193)(100.57231684,185.77185877)
\lineto(100.16762794,186.56161097)
\curveto(100.34467934,186.68696846)(100.61025642,186.81650454)(100.96435921,186.9502192)
\curveto(101.32689301,187.09229102)(101.7273664,187.16332693)(102.16577937,187.16332693)
\curveto(102.57889929,187.16332693)(102.93300208,187.11318394)(103.22808773,187.01289794)
\curveto(103.52317338,186.91261195)(103.76345742,186.77054012)(103.94893983,186.58668247)
\curveto(104.14285326,186.40282481)(104.28618058,186.18553849)(104.37892178,185.93482351)
\curveto(104.47166299,185.69246569)(104.51803359,185.42503637)(104.51803359,185.13253555)
\curveto(104.51803359,184.72303441)(104.40843035,184.37621201)(104.18922386,184.09206836)
\curveto(103.97844839,183.80792471)(103.70444029,183.59063839)(103.36719954,183.4402094)
\curveto(103.77188844,183.32320908)(104.12177571,183.09338701)(104.41686137,182.75074319)
\curveto(104.71194702,182.41645655)(104.85948985,181.96934816)(104.85948985,181.40941802)
\curveto(104.85948985,181.07513138)(104.80047272,180.76173765)(104.68243846,180.46923683)
\curveto(104.57283521,180.18509318)(104.39999933,179.93855678)(104.16393081,179.72962762)
\curveto(103.9362933,179.52069847)(103.63699214,179.35773373)(103.26602731,179.2407334)
\curveto(102.90349351,179.12373308)(102.46929605,179.06523291)(101.96343493,179.06523291)
\curveto(101.7695215,179.06523291)(101.56717705,179.08194725)(101.35640158,179.11537591)
\curveto(101.15405713,179.14044741)(100.96435921,179.17805466)(100.78730782,179.22819765)
\curveto(100.61025642,179.26998348)(100.45006707,179.31176931)(100.30673975,179.35355515)
\curveto(100.17184345,179.40369814)(100.07488674,179.44130539)(100.01586961,179.46637689)
\lineto(100.21821406,180.35641509)
\curveto(100.35311035,180.28955776)(100.56810133,180.21016468)(100.86318699,180.11823585)
\curveto(101.15827264,180.02630702)(101.52080644,179.98034261)(101.9507884,179.98034261)
\closepath
}
}
{
\newrgbcolor{curcolor}{0 0 0}
\pscustom[linestyle=none,fillstyle=solid,fillcolor=curcolor]
{
\newpath
\moveto(118.6947944,181.91084799)
\curveto(118.83812172,182.24513464)(119.03203515,182.62956428)(119.27653469,183.06413693)
\curveto(119.52946525,183.49870957)(119.81190438,183.94581796)(120.12385207,184.4054621)
\curveto(120.43579976,184.8734634)(120.768825,185.32892896)(121.12292779,185.77185877)
\curveto(121.47703057,186.22314574)(121.83534887,186.6284683)(122.19788267,186.98782644)
\lineto(123.20960491,186.98782644)
\lineto(123.20960491,182.06127699)
\lineto(124.13280146,182.06127699)
\lineto(124.13280146,181.20884604)
\lineto(123.20960491,181.20884604)
\lineto(123.20960491,179.22819765)
\lineto(122.19788267,179.22819765)
\lineto(122.19788267,181.20884604)
\lineto(118.6947944,181.20884604)
\lineto(118.6947944,181.91084799)
\closepath
\moveto(122.19788267,185.75932302)
\curveto(121.97024517,185.5169652)(121.73839215,185.24953588)(121.50232363,184.95703506)
\curveto(121.27468612,184.66453425)(121.05126413,184.3553191)(120.83205764,184.02938962)
\curveto(120.61285116,183.7118173)(120.4062912,183.38588782)(120.21237777,183.05160118)
\curveto(120.02689536,182.71731453)(119.85827498,182.38720647)(119.70651665,182.06127699)
\lineto(122.19788267,182.06127699)
\lineto(122.19788267,185.75932302)
\closepath
}
}
{
\newrgbcolor{curcolor}{0 0 0}
\pscustom[linestyle=none,fillstyle=solid,fillcolor=curcolor]
{
\newpath
\moveto(139.56156089,183.99178237)
\curveto(140.69131739,183.94999654)(141.51334171,183.70346014)(142.02763385,183.25217316)
\curveto(142.54192599,182.80088619)(142.79907206,182.19499164)(142.79907206,181.43448952)
\curveto(142.79907206,181.09184571)(142.74427044,180.77427339)(142.6346672,180.48177258)
\curveto(142.52506396,180.18927176)(142.35222807,179.93855678)(142.11615955,179.72962762)
\curveto(141.88852205,179.52069847)(141.59343639,179.35773373)(141.23090259,179.2407334)
\curveto(140.8767998,179.12373308)(140.45524887,179.06523291)(139.96624978,179.06523291)
\curveto(139.76390533,179.06523291)(139.56156089,179.08194725)(139.35921644,179.11537591)
\curveto(139.16530301,179.14044741)(138.9798206,179.17387607)(138.8027692,179.2156619)
\curveto(138.62571781,179.25744773)(138.46974396,179.29923357)(138.33484766,179.3410194)
\curveto(138.19995137,179.39116239)(138.10299465,179.43294822)(138.04397752,179.46637689)
\lineto(138.24632197,180.35641509)
\curveto(138.38121827,180.28955776)(138.58777823,180.21016468)(138.86600184,180.11823585)
\curveto(139.15265648,180.02630702)(139.51097477,179.98034261)(139.94095673,179.98034261)
\curveto(140.27819747,179.98034261)(140.5606366,180.01794986)(140.78827411,180.09316435)
\curveto(141.01591161,180.16837885)(141.19717851,180.26866484)(141.33207481,180.39402233)
\curveto(141.47540213,180.52773699)(141.57657435,180.67816598)(141.63559148,180.84530931)
\curveto(141.70303963,181.01245263)(141.73676371,181.18795312)(141.73676371,181.37181078)
\curveto(141.73676371,181.65595443)(141.6861776,181.90666941)(141.58500537,182.12395573)
\curveto(141.49226417,182.34959922)(141.32364379,182.53763546)(141.07914425,182.68806445)
\curveto(140.84307573,182.83849344)(140.514266,182.95131518)(140.09271506,183.02652968)
\curveto(139.67959515,183.11010134)(139.15687199,183.15188717)(138.52454559,183.15188717)
\curveto(138.5751317,183.51960248)(138.61307128,183.8622463)(138.63836434,184.17981861)
\curveto(138.67208841,184.50574809)(138.69738147,184.81914182)(138.71424351,185.1199998)
\curveto(138.73953656,185.42921495)(138.7563986,185.73425152)(138.76482962,186.0351095)
\curveto(138.78169166,186.33596748)(138.79855369,186.6535398)(138.81541573,186.98782644)
\lineto(142.58408109,186.98782644)
\lineto(142.58408109,186.110324)
\lineto(139.72596575,186.110324)
\curveto(139.71753473,185.99332367)(139.7048882,185.83871609)(139.68802617,185.64650127)
\curveto(139.67959515,185.46264362)(139.66694862,185.26625021)(139.65008658,185.05732106)
\curveto(139.63322454,184.8483919)(139.61636251,184.64781992)(139.59950047,184.45560509)
\curveto(139.58263843,184.26339027)(139.5699919,184.1087827)(139.56156089,183.99178237)
\closepath
}
}
{
\newrgbcolor{curcolor}{0 0 0}
\pscustom[linestyle=none,fillstyle=solid,fillcolor=curcolor]
{
\newpath
\moveto(156.86201412,182.33706347)
\curveto(156.86201412,183.08920842)(156.96318635,183.75360313)(157.1655308,184.3302476)
\curveto(157.37630626,184.91524923)(157.67139192,185.40414345)(158.05078776,185.79693026)
\curveto(158.43861462,186.18971707)(158.90653616,186.49057506)(159.45455237,186.69950421)
\curveto(160.00256859,186.90843336)(160.62224846,187.01707652)(161.31359199,187.02543369)
\lineto(161.40211769,186.14793124)
\curveto(160.9552737,186.13957408)(160.54636929,186.08943108)(160.17540447,185.99750225)
\curveto(159.81287067,185.91393059)(159.48827645,185.77185877)(159.20162181,185.57128678)
\curveto(158.91496718,185.37907196)(158.67468314,185.12835697)(158.48076971,184.81914182)
\curveto(158.28685628,184.50992667)(158.13931346,184.12967561)(158.03814123,183.67838864)
\curveto(158.24048568,183.77031747)(158.45969217,183.84553196)(158.69576069,183.90403213)
\curveto(158.94026023,183.96253229)(159.1974063,183.99178237)(159.4671989,183.99178237)
\curveto(159.90561187,183.99178237)(160.2765767,183.92492504)(160.58009337,183.79121038)
\curveto(160.89204106,183.65749572)(161.14075611,183.47781665)(161.32623852,183.25217316)
\curveto(161.51172093,183.03488684)(161.64661723,182.77999328)(161.73092742,182.48749246)
\curveto(161.81523761,182.19499164)(161.8573927,181.89413366)(161.8573927,181.58491851)
\curveto(161.8573927,181.30077486)(161.8110221,181.00827405)(161.71828089,180.70741607)
\curveto(161.62553969,180.41491525)(161.48221237,180.14330735)(161.28829894,179.89259236)
\curveto(161.09438551,179.65023454)(160.84567046,179.44966256)(160.54215378,179.2908764)
\curveto(160.23863711,179.14044741)(159.87610331,179.06523291)(159.45455237,179.06523291)
\curveto(158.58615745,179.06523291)(157.93696901,179.35355515)(157.50698705,179.93019961)
\curveto(157.0770051,180.50684408)(156.86201412,181.30913203)(156.86201412,182.33706347)
\closepath
\moveto(159.35338015,183.13935142)
\curveto(159.08358755,183.13935142)(158.8348725,183.11427992)(158.60723499,183.06413693)
\curveto(158.38802851,183.01399393)(158.16460651,182.93877943)(157.93696901,182.83849344)
\curveto(157.92853799,182.75492178)(157.92432248,182.67135012)(157.92432248,182.58777845)
\lineto(157.92432248,182.33706347)
\curveto(157.92432248,182.01113399)(157.94540003,181.70191884)(157.98755512,181.40941802)
\curveto(158.03814123,181.12527437)(158.11823591,180.87038081)(158.22783915,180.64473732)
\curveto(158.34587341,180.427451)(158.50184726,180.25195051)(158.69576069,180.11823585)
\curveto(158.88967412,179.99287836)(159.13838917,179.93019961)(159.44190584,179.93019961)
\curveto(159.69483641,179.93019961)(159.90561187,179.98034261)(160.07423225,180.0806286)
\curveto(160.24285262,180.18927176)(160.38196443,180.32298642)(160.49156767,180.48177258)
\curveto(160.60117091,180.64055874)(160.67705008,180.81605923)(160.71920518,181.00827405)
\curveto(160.76979129,181.20884604)(160.79508434,181.39688227)(160.79508434,181.57238276)
\curveto(160.79508434,182.0821699)(160.67705008,182.47077813)(160.44098156,182.73820745)
\curveto(160.21334405,183.00563676)(159.85081025,183.13935142)(159.35338015,183.13935142)
\closepath
}
}
{
\newrgbcolor{curcolor}{0 0 0}
\pscustom[linestyle=none,fillstyle=solid,fillcolor=curcolor]
{
\newpath
\moveto(177.1091084,179.22819765)
\curveto(177.15126349,179.82155645)(177.25665122,180.44834391)(177.4252716,181.10856004)
\curveto(177.60232299,181.77713334)(177.81309846,182.42063513)(178.057598,183.03906543)
\curveto(178.30209754,183.66585289)(178.57189014,184.24249736)(178.86697579,184.76899882)
\curveto(179.16206145,185.30385746)(179.45293159,185.74260868)(179.73958623,186.0852525)
\lineto(175.94562782,186.0852525)
\lineto(175.94562782,186.98782644)
\lineto(180.90306681,186.98782644)
\lineto(180.90306681,186.12285974)
\curveto(180.65013625,185.83035893)(180.37612814,185.43757212)(180.08104249,184.94449931)
\curveto(179.78595683,184.45142651)(179.50351771,183.89567496)(179.23372511,183.27724466)
\curveto(178.97236353,182.66717153)(178.74472602,182.01113399)(178.55081259,181.30913203)
\curveto(178.35689916,180.61548724)(178.23464939,179.92184245)(178.18406328,179.22819765)
\lineto(177.1091084,179.22819765)
\closepath
}
}
{
\newrgbcolor{curcolor}{0 0 0}
\pscustom[linestyle=none,fillstyle=solid,fillcolor=curcolor]
{
\newpath
\moveto(199.79698258,181.24645328)
\curveto(199.79698258,180.59459432)(199.58620711,180.06809285)(199.16465618,179.66694888)
\curveto(198.75153626,179.2658049)(198.11920986,179.06523291)(197.26767697,179.06523291)
\curveto(196.77867789,179.06523291)(196.37398899,179.12791166)(196.05361028,179.25326915)
\curveto(195.73323157,179.38698381)(195.4760855,179.55412713)(195.28217207,179.75469912)
\curveto(195.09668966,179.96362828)(194.96179336,180.19345035)(194.87748317,180.44416533)
\curveto(194.801604,180.69488032)(194.76366442,180.94141672)(194.76366442,181.18377454)
\curveto(194.76366442,181.62670434)(194.88591419,182.01949115)(195.13041373,182.36213497)
\curveto(195.37491327,182.70477878)(195.66578342,182.98056526)(196.00302417,183.18949442)
\curveto(195.28638758,183.59063839)(194.92806928,184.20489011)(194.92806928,185.03224956)
\curveto(194.92806928,185.31639321)(194.9828709,185.58800111)(195.09247415,185.84707326)
\curveto(195.20207739,186.10614541)(195.35805124,186.3317889)(195.56039569,186.52400372)
\curveto(195.76274013,186.71621854)(196.00723968,186.87082612)(196.29389431,186.98782644)
\curveto(196.58897997,187.10482677)(196.9177897,187.16332693)(197.2803235,187.16332693)
\curveto(197.70187443,187.16332693)(198.06019273,187.10064819)(198.35527838,186.97529069)
\curveto(198.65879506,186.8499332)(198.9032946,186.68696846)(199.08877701,186.48639647)
\curveto(199.27425942,186.29418165)(199.40915572,186.07689533)(199.49346591,185.83453751)
\curveto(199.57777609,185.60053686)(199.61993119,185.37071479)(199.61993119,185.1450713)
\curveto(199.61993119,184.7021415)(199.50611244,184.31771185)(199.27847493,183.99178237)
\curveto(199.05083743,183.67421006)(198.78947585,183.41931649)(198.49439019,183.22710167)
\curveto(199.36278512,182.81760052)(199.79698258,182.1573844)(199.79698258,181.24645328)
\closepath
\moveto(195.77538666,181.17123879)
\curveto(195.77538666,181.03752413)(195.80067972,180.8954523)(195.85126583,180.74502331)
\curveto(195.90185194,180.60295149)(195.98616213,180.46923683)(196.10419639,180.34387934)
\curveto(196.22223065,180.21852184)(196.3782045,180.11405727)(196.57211793,180.03048561)
\curveto(196.76603136,179.95527111)(197.00209988,179.91766386)(197.2803235,179.91766386)
\curveto(197.54168508,179.91766386)(197.76510707,179.95109253)(197.95058949,180.01794986)
\curveto(198.14450292,180.09316435)(198.30047676,180.18927176)(198.41851102,180.30627209)
\curveto(198.5449763,180.43162958)(198.63771751,180.56952282)(198.69673464,180.71995181)
\curveto(198.75575177,180.87038081)(198.78526034,181.0208098)(198.78526034,181.17123879)
\curveto(198.78526034,181.64759726)(198.61242445,182.01113399)(198.26675269,182.26184897)
\curveto(197.92108092,182.52092112)(197.44472836,182.71731453)(196.83769502,182.85102919)
\curveto(196.50045427,182.66717153)(196.23909269,182.43734946)(196.05361028,182.16156298)
\curveto(195.86812787,181.8857765)(195.77538666,181.55566843)(195.77538666,181.17123879)
\closepath
\moveto(198.60820894,185.1450713)
\curveto(198.60820894,185.25371446)(198.58291589,185.37489337)(198.53232978,185.50860803)
\curveto(198.48174366,185.65067986)(198.40164899,185.77603735)(198.29204574,185.88468051)
\curveto(198.19087352,186.00168083)(198.05597722,186.09778825)(197.88735685,186.17300274)
\curveto(197.71873647,186.2565744)(197.51639202,186.29836023)(197.2803235,186.29836023)
\curveto(197.03582396,186.29836023)(196.829264,186.26075299)(196.66064363,186.18553849)
\curveto(196.50045427,186.110324)(196.36555797,186.01421658)(196.25595473,185.89721626)
\curveto(196.14635148,185.7885731)(196.06625681,185.6632156)(196.0156707,185.52114378)
\curveto(195.9735156,185.38742912)(195.95243805,185.25371446)(195.95243805,185.1199998)
\curveto(195.95243805,184.77735599)(196.07468783,184.45560509)(196.31918737,184.15474711)
\curveto(196.57211793,183.8622463)(196.98523785,183.64913856)(197.55854712,183.5154239)
\curveto(197.87892583,183.69928155)(198.13185639,183.91656788)(198.3173388,184.16728286)
\curveto(198.51125223,184.41799785)(198.60820894,184.74392733)(198.60820894,185.1450713)
\closepath
}
}
{
\newrgbcolor{curcolor}{0 0 0}
\pscustom[linestyle=none,fillstyle=solid,fillcolor=curcolor]
{
\newpath
\moveto(218.7414768,183.87896063)
\curveto(218.7414768,182.33288489)(218.36208096,181.1670602)(217.60328928,180.38148658)
\curveto(216.84449759,179.60427013)(215.70209456,179.21148332)(214.17608018,179.20312615)
\lineto(214.13814059,180.0806286)
\curveto(215.08241469,180.0806286)(215.84120637,180.26448626)(216.41451564,180.63220157)
\curveto(216.99625593,181.00827405)(217.37986728,181.65177584)(217.56534969,182.56270696)
\curveto(217.36300524,182.47077813)(217.13958325,182.39556363)(216.89508371,182.33706347)
\curveto(216.65058416,182.28692047)(216.39343809,182.26184897)(216.1236455,182.26184897)
\curveto(215.67680151,182.26184897)(215.30162117,182.32452772)(214.9981045,182.44988521)
\curveto(214.69458783,182.58359987)(214.45008828,182.75910036)(214.26460587,182.97638668)
\curveto(214.07912346,183.20203017)(213.94422716,183.45692374)(213.85991698,183.74106739)
\curveto(213.77560679,184.02521104)(213.7334517,184.32606902)(213.7334517,184.64364133)
\curveto(213.7334517,184.92778498)(213.7798223,185.21610721)(213.8725635,185.50860803)
\curveto(213.96530471,185.80946601)(214.10863203,186.08107391)(214.30254546,186.32343173)
\curveto(214.49645889,186.56578955)(214.74517394,186.76636154)(215.04869061,186.9251477)
\curveto(215.35220729,187.08393385)(215.71474109,187.16332693)(216.13629202,187.16332693)
\curveto(216.99625593,187.16332693)(217.64544437,186.87082612)(218.08385734,186.28582448)
\curveto(218.52227031,185.70082285)(218.7414768,184.8985349)(218.7414768,183.87896063)
\closepath
\moveto(216.23746425,183.11427992)
\curveto(216.50725685,183.11427992)(216.7559719,183.13935142)(216.9836094,183.18949442)
\curveto(217.21967793,183.23963742)(217.44731543,183.31067333)(217.66652192,183.40260216)
\curveto(217.67495294,183.48617382)(217.67916845,183.5655669)(217.67916845,183.64078139)
\lineto(217.67916845,183.87896063)
\curveto(217.67916845,184.20489011)(217.65387539,184.51410526)(217.60328928,184.80660607)
\curveto(217.56113418,185.09910689)(217.48103951,185.35400046)(217.36300524,185.57128678)
\curveto(217.253402,185.7885731)(217.09742816,185.959895)(216.89508371,186.0852525)
\curveto(216.70117028,186.21896716)(216.45245523,186.28582448)(216.14893855,186.28582448)
\curveto(215.89600799,186.28582448)(215.68523252,186.2315029)(215.51661215,186.12285974)
\curveto(215.34799178,186.02257375)(215.20887997,185.89303767)(215.09927672,185.73425152)
\curveto(214.98967348,185.58382253)(214.9095788,185.41250062)(214.85899269,185.2202858)
\curveto(214.8168376,185.02807098)(214.79576005,184.84421332)(214.79576005,184.66871283)
\curveto(214.79576005,184.15892569)(214.9095788,183.77031747)(215.13721631,183.50288815)
\curveto(215.37328483,183.243816)(215.74003415,183.11427992)(216.23746425,183.11427992)
\closepath
}
}
{
\newrgbcolor{curcolor}{0 0 0}
\pscustom[linestyle=none,fillstyle=solid,fillcolor=curcolor]
{
\newpath
\moveto(236.15574113,180.03048561)
\curveto(236.15574113,179.77977062)(236.07143094,179.55830572)(235.90281057,179.36609089)
\curveto(235.73419019,179.17387607)(235.5107682,179.07776866)(235.23254458,179.07776866)
\curveto(234.94588995,179.07776866)(234.71825244,179.17387607)(234.54963207,179.36609089)
\curveto(234.38101169,179.55830572)(234.29670151,179.77977062)(234.29670151,180.03048561)
\curveto(234.29670151,180.28955776)(234.38101169,180.51520124)(234.54963207,180.70741607)
\curveto(234.71825244,180.89963089)(234.94588995,180.9957383)(235.23254458,180.9957383)
\curveto(235.5107682,180.9957383)(235.73419019,180.89963089)(235.90281057,180.70741607)
\curveto(236.07143094,180.51520124)(236.15574113,180.28955776)(236.15574113,180.03048561)
\closepath
}
}
{
\newrgbcolor{curcolor}{0 0 0}
\pscustom[linestyle=none,fillstyle=solid,fillcolor=curcolor]
{
\newpath
\moveto(242.47901377,180.03048561)
\curveto(242.47901377,179.77977062)(242.39470358,179.55830572)(242.22608321,179.36609089)
\curveto(242.05746284,179.17387607)(241.83404084,179.07776866)(241.55581722,179.07776866)
\curveto(241.26916259,179.07776866)(241.04152508,179.17387607)(240.87290471,179.36609089)
\curveto(240.70428434,179.55830572)(240.61997415,179.77977062)(240.61997415,180.03048561)
\curveto(240.61997415,180.28955776)(240.70428434,180.51520124)(240.87290471,180.70741607)
\curveto(241.04152508,180.89963089)(241.26916259,180.9957383)(241.55581722,180.9957383)
\curveto(241.83404084,180.9957383)(242.05746284,180.89963089)(242.22608321,180.70741607)
\curveto(242.39470358,180.51520124)(242.47901377,180.28955776)(242.47901377,180.03048561)
\closepath
}
}
{
\newrgbcolor{curcolor}{0 0 0}
\pscustom[linestyle=none,fillstyle=solid,fillcolor=curcolor]
{
\newpath
\moveto(248.80227109,180.03048561)
\curveto(248.80227109,179.77977062)(248.7179609,179.55830572)(248.54934053,179.36609089)
\curveto(248.38072015,179.17387607)(248.15729816,179.07776866)(247.87907454,179.07776866)
\curveto(247.59241991,179.07776866)(247.3647824,179.17387607)(247.19616203,179.36609089)
\curveto(247.02754165,179.55830572)(246.94323147,179.77977062)(246.94323147,180.03048561)
\curveto(246.94323147,180.28955776)(247.02754165,180.51520124)(247.19616203,180.70741607)
\curveto(247.3647824,180.89963089)(247.59241991,180.9957383)(247.87907454,180.9957383)
\curveto(248.15729816,180.9957383)(248.38072015,180.89963089)(248.54934053,180.70741607)
\curveto(248.7179609,180.51520124)(248.80227109,180.28955776)(248.80227109,180.03048561)
\closepath
}
}
{
\newrgbcolor{curcolor}{0 0 0}
\pscustom[linestyle=none,fillstyle=solid,fillcolor=curcolor]
{
\newpath
\moveto(266.0268375,183.99178237)
\curveto(267.156594,183.94999654)(267.97861832,183.70346014)(268.49291046,183.25217316)
\curveto(269.0072026,182.80088619)(269.26434867,182.19499164)(269.26434867,181.43448952)
\curveto(269.26434867,181.09184571)(269.20954705,180.77427339)(269.09994381,180.48177258)
\curveto(268.99034057,180.18927176)(268.81750468,179.93855678)(268.58143616,179.72962762)
\curveto(268.35379866,179.52069847)(268.058713,179.35773373)(267.6961792,179.2407334)
\curveto(267.34207641,179.12373308)(266.92052548,179.06523291)(266.43152639,179.06523291)
\curveto(266.22918194,179.06523291)(266.0268375,179.08194725)(265.82449305,179.11537591)
\curveto(265.63057962,179.14044741)(265.44509721,179.17387607)(265.26804581,179.2156619)
\curveto(265.09099442,179.25744773)(264.93502057,179.29923357)(264.80012427,179.3410194)
\curveto(264.66522798,179.39116239)(264.56827126,179.43294822)(264.50925413,179.46637689)
\lineto(264.71159858,180.35641509)
\curveto(264.84649488,180.28955776)(265.05305484,180.21016468)(265.33127845,180.11823585)
\curveto(265.61793309,180.02630702)(265.97625138,179.98034261)(266.40623334,179.98034261)
\curveto(266.74347408,179.98034261)(267.02591321,180.01794986)(267.25355072,180.09316435)
\curveto(267.48118822,180.16837885)(267.66245512,180.26866484)(267.79735142,180.39402233)
\curveto(267.94067874,180.52773699)(268.04185096,180.67816598)(268.10086809,180.84530931)
\curveto(268.16831624,181.01245263)(268.20204032,181.18795312)(268.20204032,181.37181078)
\curveto(268.20204032,181.65595443)(268.15145421,181.90666941)(268.05028198,182.12395573)
\curveto(267.95754078,182.34959922)(267.7889204,182.53763546)(267.54442086,182.68806445)
\curveto(267.30835234,182.83849344)(266.97954261,182.95131518)(266.55799167,183.02652968)
\curveto(266.14487176,183.11010134)(265.6221486,183.15188717)(264.9898222,183.15188717)
\curveto(265.04040831,183.51960248)(265.07834789,183.8622463)(265.10364095,184.17981861)
\curveto(265.13736502,184.50574809)(265.16265808,184.81914182)(265.17952012,185.1199998)
\curveto(265.20481317,185.42921495)(265.22167521,185.73425152)(265.23010623,186.0351095)
\curveto(265.24696827,186.33596748)(265.2638303,186.6535398)(265.28069234,186.98782644)
\lineto(269.0493577,186.98782644)
\lineto(269.0493577,186.110324)
\lineto(266.19124236,186.110324)
\curveto(266.18281134,185.99332367)(266.17016481,185.83871609)(266.15330278,185.64650127)
\curveto(266.14487176,185.46264362)(266.13222523,185.26625021)(266.11536319,185.05732106)
\curveto(266.09850115,184.8483919)(266.08163912,184.64781992)(266.06477708,184.45560509)
\curveto(266.04791504,184.26339027)(266.03526851,184.1087827)(266.0268375,183.99178237)
\closepath
}
}
{
\newrgbcolor{curcolor}{0 0 0}
\pscustom[linestyle=none,fillstyle=solid,fillcolor=curcolor]
{
\newpath
\moveto(275.3093977,185.00717806)
\curveto(275.3093977,184.73974874)(275.25459608,184.48067659)(275.14499284,184.22996161)
\curveto(275.04382061,183.97924662)(274.9047088,183.73271022)(274.72765741,183.4903524)
\curveto(274.55903704,183.24799458)(274.36512361,183.00981535)(274.14591712,182.77581469)
\curveto(273.92671063,182.54181404)(273.70328864,182.31199197)(273.47565113,182.08634848)
\curveto(273.34918585,181.96099099)(273.20164303,181.810562)(273.03302265,181.63506151)
\curveto(272.86440228,181.45956102)(272.70421292,181.27988195)(272.55245459,181.09602429)
\curveto(272.40069625,180.91216664)(272.27423097,180.73248756)(272.17305875,180.55698707)
\curveto(272.07188652,180.38148658)(272.02130041,180.23105759)(272.02130041,180.1057001)
\lineto(275.6255609,180.1057001)
\lineto(275.6255609,179.22819765)
\lineto(270.88311288,179.22819765)
\curveto(270.87468187,179.26998348)(270.87046636,179.31176931)(270.87046636,179.35355515)
\lineto(270.87046636,179.49144839)
\curveto(270.87046636,179.84244937)(270.92948349,180.16837885)(271.04751775,180.46923683)
\curveto(271.16555201,180.77009481)(271.31731035,181.05423846)(271.50279276,181.32166778)
\curveto(271.68827517,181.5890971)(271.89483513,181.83981208)(272.12247263,182.07381273)
\curveto(272.35854116,182.31617055)(272.59039417,182.55017121)(272.81803168,182.77581469)
\curveto(273.00351409,182.95967235)(273.18056548,183.13935142)(273.34918585,183.31485191)
\curveto(273.52623725,183.4903524)(273.67799558,183.66585289)(273.80446086,183.84135338)
\curveto(273.93935716,184.01685387)(274.0447449,184.19653294)(274.12062406,184.3803906)
\curveto(274.20493425,184.57260542)(274.24708934,184.76899882)(274.24708934,184.96957081)
\curveto(274.24708934,185.1952143)(274.20914976,185.38742912)(274.13327059,185.54621528)
\curveto(274.06582244,185.70500144)(273.96886573,185.83453751)(273.84240045,185.93482351)
\curveto(273.72436619,186.04346667)(273.58525438,186.12285974)(273.42506502,186.17300274)
\curveto(273.26487567,186.22314574)(273.09203978,186.24821724)(272.90655737,186.24821724)
\curveto(272.68735089,186.24821724)(272.48500644,186.21896716)(272.29952403,186.16046699)
\curveto(272.12247263,186.10196683)(271.96228328,186.03093092)(271.81895596,185.94735925)
\curveto(271.68405966,185.86378759)(271.5660254,185.78021593)(271.46485317,185.69664427)
\curveto(271.36368095,185.62142977)(271.28780178,185.55875103)(271.23721567,185.50860803)
\lineto(270.71870802,186.23568149)
\curveto(270.78615617,186.31089598)(270.88732839,186.40282481)(271.02222469,186.51146797)
\curveto(271.15712099,186.62011113)(271.31731035,186.72039713)(271.50279276,186.81232595)
\curveto(271.69670619,186.91261195)(271.91169717,186.99618361)(272.14776569,187.06304094)
\curveto(272.38383421,187.12989827)(272.63676477,187.16332693)(272.90655737,187.16332693)
\curveto(273.72436619,187.16332693)(274.32718402,186.97529069)(274.71501088,186.59921822)
\curveto(275.11126876,186.2315029)(275.3093977,185.70082285)(275.3093977,185.00717806)
\closepath
}
}
{
\newrgbcolor{curcolor}{0 0 0}
\pscustom[linewidth=0.56097835,linecolor=curcolor]
{
\newpath
\moveto(52.53490984,189.47385717)
\lineto(277.23287737,189.47385717)
\lineto(277.23287737,176.75473809)
\lineto(52.53490984,176.75473809)
\closepath
}
}
{
\newrgbcolor{curcolor}{0 0 0}
\pscustom[linewidth=0.51800942,linecolor=curcolor]
{
\newpath
\moveto(71.388226,176.76884333)
\lineto(71.388226,189.36823333)
}
}
{
\newrgbcolor{curcolor}{0 0 0}
\pscustom[linewidth=0.5154072,linecolor=curcolor]
{
\newpath
\moveto(90.26845,176.86315333)
\lineto(90.26845,189.33627333)
}
}
{
\newrgbcolor{curcolor}{0 0 0}
\pscustom[linewidth=0.51930571,linecolor=curcolor]
{
\newpath
\moveto(109.10774,176.86315333)
\lineto(109.10774,189.52568333)
}
}
{
\newrgbcolor{curcolor}{0 0 0}
\pscustom[linewidth=0.52573889,linecolor=curcolor]
{
\newpath
\moveto(128.03631,176.68458333)
\lineto(128.03631,189.66278333)
}
}
{
\newrgbcolor{curcolor}{0 0 0}
\pscustom[linewidth=0.51670998,linecolor=curcolor]
{
\newpath
\moveto(146.87559,176.86315333)
\lineto(146.87559,189.39941333)
}
}
{
\newrgbcolor{curcolor}{0 0 0}
\pscustom[linewidth=0.52059871,linecolor=curcolor]
{
\newpath
\moveto(165.6256,176.86315333)
\lineto(165.6256,189.58881333)
}
}
{
\newrgbcolor{curcolor}{0 0 0}
\pscustom[linewidth=0.51670998,linecolor=curcolor]
{
\newpath
\moveto(184.55417,176.77386333)
\lineto(184.55417,189.31012333)
}
}
{
\newrgbcolor{curcolor}{0 0 0}
\pscustom[linewidth=0.51670998,linecolor=curcolor]
{
\newpath
\moveto(203.66131,176.86315333)
\lineto(203.66131,189.39941333)
}
}
{
\newrgbcolor{curcolor}{0 0 0}
\pscustom[linewidth=0.52059871,linecolor=curcolor]
{
\newpath
\moveto(222.58988,176.86315333)
\lineto(222.58988,189.58881333)
}
}
{
\newrgbcolor{curcolor}{0 0 0}
\pscustom[linewidth=0.52573889,linecolor=curcolor]
{
\newpath
\moveto(258.24019,189.51492333)
\lineto(258.20469,176.53677333)
}
}
{
\newrgbcolor{curcolor}{0 1 0}
\pscustom[linestyle=none,fillstyle=solid,fillcolor=curcolor]
{
\newpath
\moveto(90.31035748,167.55105117)
\lineto(109.15832654,167.55105117)
\lineto(109.15832654,154.73474028)
\lineto(90.31035748,154.73474028)
\closepath
}
}
{
\newrgbcolor{curcolor}{0 0 0}
\pscustom[linestyle=none,fillstyle=solid,fillcolor=curcolor]
{
\newpath
\moveto(62.27863373,163.32398197)
\curveto(62.70861568,163.4911253)(63.12595111,163.70005445)(63.53064001,163.95076944)
\curveto(63.9353289,164.20984159)(64.31050924,164.53577107)(64.656181,164.92855788)
\lineto(65.38967963,164.92855788)
\lineto(65.38967963,158.04643154)
\lineto(66.86932341,158.04643154)
\lineto(66.86932341,157.16892909)
\lineto(62.65802957,157.16892909)
\lineto(62.65802957,158.04643154)
\lineto(64.35266433,158.04643154)
\lineto(64.35266433,163.48694672)
\curveto(64.25992312,163.40337505)(64.14610437,163.31562481)(64.01120807,163.22369598)
\curveto(63.88474279,163.14012432)(63.74141547,163.05655266)(63.58122612,162.972981)
\curveto(63.42946778,162.88940933)(63.26927843,162.81001626)(63.10065805,162.73480176)
\curveto(62.93203768,162.65958726)(62.76763282,162.59690852)(62.60744346,162.54676552)
\lineto(62.27863373,163.32398197)
\closepath
}
}
{
\newrgbcolor{curcolor}{0 0 0}
\pscustom[linestyle=none,fillstyle=solid,fillcolor=curcolor]
{
\newpath
\moveto(76.32892543,161.93251381)
\curveto(77.45868194,161.89072798)(78.28070626,161.64419157)(78.7949984,161.1929046)
\curveto(79.30929054,160.74161763)(79.56643661,160.13572308)(79.56643661,159.37522096)
\curveto(79.56643661,159.03257715)(79.51163499,158.71500483)(79.40203174,158.42250402)
\curveto(79.2924285,158.1300032)(79.11959262,157.87928821)(78.88352409,157.67035906)
\curveto(78.65588659,157.46142991)(78.36080094,157.29846517)(77.99826713,157.18146484)
\curveto(77.64416435,157.06446451)(77.22261341,157.00596435)(76.73361433,157.00596435)
\curveto(76.53126988,157.00596435)(76.32892543,157.02267868)(76.12658098,157.05610735)
\curveto(75.93266755,157.08117885)(75.74718514,157.11460751)(75.57013375,157.15639334)
\curveto(75.39308235,157.19817917)(75.23710851,157.239965)(75.10221221,157.28175083)
\curveto(74.96731591,157.33189383)(74.8703592,157.37367966)(74.81134206,157.40710833)
\lineto(75.01368651,158.29714652)
\curveto(75.14858281,158.23028919)(75.35514277,158.15089611)(75.63336639,158.05896729)
\curveto(75.92002102,157.96703846)(76.27833932,157.92107405)(76.70832127,157.92107405)
\curveto(77.04556202,157.92107405)(77.32800115,157.95868129)(77.55563865,158.03389579)
\curveto(77.78327616,158.10911028)(77.96454306,158.20939628)(78.09943936,158.33475377)
\curveto(78.24276667,158.46846843)(78.3439389,158.61889742)(78.40295603,158.78604074)
\curveto(78.47040418,158.95318407)(78.50412825,159.12868456)(78.50412825,159.31254221)
\curveto(78.50412825,159.59668586)(78.45354214,159.84740085)(78.35236992,160.06468717)
\curveto(78.25962871,160.29033065)(78.09100834,160.47836689)(77.8465088,160.62879588)
\curveto(77.61044027,160.77922488)(77.28163054,160.89204662)(76.86007961,160.96726111)
\curveto(76.44695969,161.05083278)(75.92423653,161.09261861)(75.29191013,161.09261861)
\curveto(75.34249624,161.46033392)(75.38043583,161.80297773)(75.40572888,162.12055005)
\curveto(75.43945296,162.44647953)(75.46474601,162.75987326)(75.48160805,163.06073124)
\curveto(75.50690111,163.36994639)(75.52376314,163.67498295)(75.53219416,163.97584094)
\curveto(75.5490562,164.27669892)(75.56591824,164.59427123)(75.58278028,164.92855788)
\lineto(79.35144563,164.92855788)
\lineto(79.35144563,164.05105543)
\lineto(76.49333029,164.05105543)
\curveto(76.48489928,163.93405511)(76.47225275,163.77944753)(76.45539071,163.58723271)
\curveto(76.44695969,163.40337505)(76.43431316,163.20698165)(76.41745113,162.99805249)
\curveto(76.40058909,162.78912334)(76.38372705,162.58855135)(76.36686501,162.39633653)
\curveto(76.35000298,162.20412171)(76.33735645,162.04951413)(76.32892543,161.93251381)
\closepath
}
}
{
\newrgbcolor{curcolor}{0 0 0}
\pscustom[linestyle=none,fillstyle=solid,fillcolor=curcolor]
{
\newpath
\moveto(81.2484258,163.32398197)
\curveto(81.67840775,163.4911253)(82.09574318,163.70005445)(82.50043207,163.95076944)
\curveto(82.90512097,164.20984159)(83.2803013,164.53577107)(83.62597307,164.92855788)
\lineto(84.3594717,164.92855788)
\lineto(84.3594717,158.04643154)
\lineto(85.83911548,158.04643154)
\lineto(85.83911548,157.16892909)
\lineto(81.62782164,157.16892909)
\lineto(81.62782164,158.04643154)
\lineto(83.3224564,158.04643154)
\lineto(83.3224564,163.48694672)
\curveto(83.22971519,163.40337505)(83.11589644,163.31562481)(82.98100014,163.22369598)
\curveto(82.85453486,163.14012432)(82.71120754,163.05655266)(82.55101819,162.972981)
\curveto(82.39925985,162.88940933)(82.23907049,162.81001626)(82.07045012,162.73480176)
\curveto(81.90182975,162.65958726)(81.73742488,162.59690852)(81.57723553,162.54676552)
\lineto(81.2484258,163.32398197)
\closepath
}
}
{
\newrgbcolor{curcolor}{0 0 0}
\pscustom[linestyle=none,fillstyle=solid,fillcolor=curcolor]
{
\newpath
\moveto(93.89495193,163.32398197)
\curveto(94.32493388,163.4911253)(94.7422693,163.70005445)(95.1469582,163.95076944)
\curveto(95.5516471,164.20984159)(95.92682743,164.53577107)(96.2724992,164.92855788)
\lineto(97.00599782,164.92855788)
\lineto(97.00599782,158.04643154)
\lineto(98.48564161,158.04643154)
\lineto(98.48564161,157.16892909)
\lineto(94.27434777,157.16892909)
\lineto(94.27434777,158.04643154)
\lineto(95.96898252,158.04643154)
\lineto(95.96898252,163.48694672)
\curveto(95.87624132,163.40337505)(95.76242257,163.31562481)(95.62752627,163.22369598)
\curveto(95.50106099,163.14012432)(95.35773367,163.05655266)(95.19754431,162.972981)
\curveto(95.04578598,162.88940933)(94.88559662,162.81001626)(94.71697625,162.73480176)
\curveto(94.54835587,162.65958726)(94.38395101,162.59690852)(94.22376165,162.54676552)
\lineto(93.89495193,163.32398197)
\closepath
}
}
{
\newrgbcolor{curcolor}{0 0 0}
\pscustom[linestyle=none,fillstyle=solid,fillcolor=curcolor]
{
\newpath
\moveto(99.72500231,159.85157943)
\curveto(99.86832963,160.18586608)(100.06224306,160.57029572)(100.3067426,161.00486836)
\curveto(100.55967316,161.439441)(100.84211229,161.88654939)(101.15405998,162.34619353)
\curveto(101.46600767,162.81419484)(101.79903291,163.26966039)(102.1531357,163.7125902)
\curveto(102.50723848,164.16387718)(102.86555678,164.56919973)(103.22809058,164.92855788)
\lineto(104.23981282,164.92855788)
\lineto(104.23981282,160.00200842)
\lineto(105.16300937,160.00200842)
\lineto(105.16300937,159.14957747)
\lineto(104.23981282,159.14957747)
\lineto(104.23981282,157.16892909)
\lineto(103.22809058,157.16892909)
\lineto(103.22809058,159.14957747)
\lineto(99.72500231,159.14957747)
\lineto(99.72500231,159.85157943)
\closepath
\moveto(103.22809058,163.70005445)
\curveto(103.00045308,163.45769663)(102.76860006,163.19026732)(102.53253154,162.8977665)
\curveto(102.30489403,162.60526568)(102.08147204,162.29605054)(101.86226555,161.97012106)
\curveto(101.64305907,161.65254874)(101.43649911,161.32661926)(101.24258568,160.99233261)
\curveto(101.05710327,160.65804597)(100.88848289,160.3279379)(100.73672456,160.00200842)
\lineto(103.22809058,160.00200842)
\lineto(103.22809058,163.70005445)
\closepath
}
}
{
\newrgbcolor{curcolor}{0 0 0}
\pscustom[linestyle=none,fillstyle=solid,fillcolor=curcolor]
{
\newpath
\moveto(118.69479725,159.85157943)
\curveto(118.83812457,160.18586608)(119.032038,160.57029572)(119.27653754,161.00486836)
\curveto(119.5294681,161.439441)(119.81190723,161.88654939)(120.12385492,162.34619353)
\curveto(120.43580261,162.81419484)(120.76882785,163.26966039)(121.12293064,163.7125902)
\curveto(121.47703342,164.16387718)(121.83535172,164.56919973)(122.19788552,164.92855788)
\lineto(123.20960776,164.92855788)
\lineto(123.20960776,160.00200842)
\lineto(124.13280431,160.00200842)
\lineto(124.13280431,159.14957747)
\lineto(123.20960776,159.14957747)
\lineto(123.20960776,157.16892909)
\lineto(122.19788552,157.16892909)
\lineto(122.19788552,159.14957747)
\lineto(118.69479725,159.14957747)
\lineto(118.69479725,159.85157943)
\closepath
\moveto(122.19788552,163.70005445)
\curveto(121.97024801,163.45769663)(121.738395,163.19026732)(121.50232648,162.8977665)
\curveto(121.27468897,162.60526568)(121.05126698,162.29605054)(120.83206049,161.97012106)
\curveto(120.612854,161.65254874)(120.40629405,161.32661926)(120.21238062,160.99233261)
\curveto(120.02689821,160.65804597)(119.85827783,160.3279379)(119.7065195,160.00200842)
\lineto(122.19788552,160.00200842)
\lineto(122.19788552,163.70005445)
\closepath
}
}
{
\newrgbcolor{curcolor}{0 0 0}
\pscustom[linestyle=none,fillstyle=solid,fillcolor=curcolor]
{
\newpath
\moveto(139.56156373,161.93251381)
\curveto(140.69132024,161.89072798)(141.51334456,161.64419157)(142.0276367,161.1929046)
\curveto(142.54192884,160.74161763)(142.79907491,160.13572308)(142.79907491,159.37522096)
\curveto(142.79907491,159.03257715)(142.74427329,158.71500483)(142.63467005,158.42250402)
\curveto(142.52506681,158.1300032)(142.35223092,157.87928821)(142.1161624,157.67035906)
\curveto(141.8885249,157.46142991)(141.59343924,157.29846517)(141.23090544,157.18146484)
\curveto(140.87680265,157.06446451)(140.45525172,157.00596435)(139.96625263,157.00596435)
\curveto(139.76390818,157.00596435)(139.56156373,157.02267868)(139.35921929,157.05610735)
\curveto(139.16530586,157.08117885)(138.97982344,157.11460751)(138.80277205,157.15639334)
\curveto(138.62572066,157.19817917)(138.46974681,157.239965)(138.33485051,157.28175083)
\curveto(138.19995422,157.33189383)(138.1029975,157.37367966)(138.04398037,157.40710833)
\lineto(138.24632482,158.29714652)
\curveto(138.38122112,158.23028919)(138.58778108,158.15089611)(138.86600469,158.05896729)
\curveto(139.15265933,157.96703846)(139.51097762,157.92107405)(139.94095958,157.92107405)
\curveto(140.27820032,157.92107405)(140.56063945,157.95868129)(140.78827696,158.03389579)
\curveto(141.01591446,158.10911028)(141.19718136,158.20939628)(141.33207766,158.33475377)
\curveto(141.47540498,158.46846843)(141.5765772,158.61889742)(141.63559433,158.78604074)
\curveto(141.70304248,158.95318407)(141.73676656,159.12868456)(141.73676656,159.31254221)
\curveto(141.73676656,159.59668586)(141.68618045,159.84740085)(141.58500822,160.06468717)
\curveto(141.49226702,160.29033065)(141.32364664,160.47836689)(141.0791471,160.62879588)
\curveto(140.84307858,160.77922488)(140.51426885,160.89204662)(140.09271791,160.96726111)
\curveto(139.679598,161.05083278)(139.15687484,161.09261861)(138.52454844,161.09261861)
\curveto(138.57513455,161.46033392)(138.61307413,161.80297773)(138.63836719,162.12055005)
\curveto(138.67209126,162.44647953)(138.69738432,162.75987326)(138.71424636,163.06073124)
\curveto(138.73953941,163.36994639)(138.75640145,163.67498295)(138.76483247,163.97584094)
\curveto(138.78169451,164.27669892)(138.79855654,164.59427123)(138.81541858,164.92855788)
\lineto(142.58408394,164.92855788)
\lineto(142.58408394,164.05105543)
\lineto(139.7259686,164.05105543)
\curveto(139.71753758,163.93405511)(139.70489105,163.77944753)(139.68802902,163.58723271)
\curveto(139.679598,163.40337505)(139.66695147,163.20698165)(139.65008943,162.99805249)
\curveto(139.63322739,162.78912334)(139.61636536,162.58855135)(139.59950332,162.39633653)
\curveto(139.58264128,162.20412171)(139.56999475,162.04951413)(139.56156373,161.93251381)
\closepath
}
}
{
\newrgbcolor{curcolor}{0 0 0}
\pscustom[linestyle=none,fillstyle=solid,fillcolor=curcolor]
{
\newpath
\moveto(156.86201697,160.27779491)
\curveto(156.86201697,161.02993986)(156.9631892,161.69433457)(157.16553365,162.27097904)
\curveto(157.37630911,162.85598067)(157.67139477,163.34487489)(158.05079061,163.7376617)
\curveto(158.43861747,164.13044851)(158.90653901,164.43130649)(159.45455522,164.64023565)
\curveto(160.00257144,164.8491648)(160.62225131,164.95780796)(161.31359484,164.96616513)
\lineto(161.40212054,164.08866268)
\curveto(160.95527655,164.08030551)(160.54637214,164.03016252)(160.17540732,163.93823369)
\curveto(159.81287352,163.85466203)(159.4882793,163.7125902)(159.20162466,163.51201821)
\curveto(158.91497003,163.31980339)(158.67468599,163.06908841)(158.48077256,162.75987326)
\curveto(158.28685913,162.45065811)(158.13931631,162.07040705)(158.03814408,161.61912008)
\curveto(158.24048853,161.7110489)(158.45969502,161.7862634)(158.69576354,161.84476356)
\curveto(158.94026308,161.90326373)(159.19740915,161.93251381)(159.46720175,161.93251381)
\curveto(159.90561472,161.93251381)(160.27657954,161.86565648)(160.58009622,161.73194182)
\curveto(160.89204391,161.59822716)(161.14075896,161.41854809)(161.32624137,161.1929046)
\curveto(161.51172378,160.97561828)(161.64662008,160.72072471)(161.73093027,160.4282239)
\curveto(161.81524046,160.13572308)(161.85739555,159.8348651)(161.85739555,159.52564995)
\curveto(161.85739555,159.2415063)(161.81102495,158.94900548)(161.71828374,158.6481475)
\curveto(161.62554254,158.35564669)(161.48221522,158.08403879)(161.28830179,157.8333238)
\curveto(161.09438836,157.59096598)(160.84567331,157.39039399)(160.54215663,157.23160784)
\curveto(160.23863996,157.08117885)(159.87610616,157.00596435)(159.45455522,157.00596435)
\curveto(158.5861603,157.00596435)(157.93697186,157.29428658)(157.5069899,157.87093105)
\curveto(157.07700795,158.44757551)(156.86201697,159.24986347)(156.86201697,160.27779491)
\closepath
\moveto(159.353383,161.08008286)
\curveto(159.0835904,161.08008286)(158.83487535,161.05501136)(158.60723784,161.00486836)
\curveto(158.38803136,160.95472537)(158.16460936,160.87951087)(157.93697186,160.77922488)
\curveto(157.92854084,160.69565321)(157.92432533,160.61208155)(157.92432533,160.52850989)
\lineto(157.92432533,160.27779491)
\curveto(157.92432533,159.95186542)(157.94540288,159.64265028)(157.98755797,159.35014946)
\curveto(158.03814408,159.06600581)(158.11823876,158.81111224)(158.227842,158.58546876)
\curveto(158.34587626,158.36818244)(158.50185011,158.19268195)(158.69576354,158.05896729)
\curveto(158.88967697,157.93360979)(159.13839202,157.87093105)(159.44190869,157.87093105)
\curveto(159.69483925,157.87093105)(159.90561472,157.92107405)(160.0742351,158.02136004)
\curveto(160.24285547,158.1300032)(160.38196728,158.26371786)(160.49157052,158.42250402)
\curveto(160.60117376,158.58129017)(160.67705293,158.75679066)(160.71920803,158.94900548)
\curveto(160.76979414,159.14957747)(160.79508719,159.33761371)(160.79508719,159.5131142)
\curveto(160.79508719,160.02290134)(160.67705293,160.41150956)(160.44098441,160.67893888)
\curveto(160.2133469,160.9463682)(159.8508131,161.08008286)(159.353383,161.08008286)
\closepath
}
}
{
\newrgbcolor{curcolor}{0 0 0}
\pscustom[linestyle=none,fillstyle=solid,fillcolor=curcolor]
{
\newpath
\moveto(177.10911125,157.16892909)
\curveto(177.15126634,157.76228789)(177.25665407,158.38907535)(177.42527445,159.04929148)
\curveto(177.60232584,159.71786477)(177.81310131,160.36136657)(178.05760085,160.97979686)
\curveto(178.30210039,161.60658433)(178.57189299,162.18322879)(178.86697864,162.70973026)
\curveto(179.1620643,163.2445889)(179.45293444,163.68334012)(179.73958908,164.02598393)
\lineto(175.94563067,164.02598393)
\lineto(175.94563067,164.92855788)
\lineto(180.90306966,164.92855788)
\lineto(180.90306966,164.06359118)
\curveto(180.6501391,163.77109037)(180.37613099,163.37830356)(180.08104534,162.88523075)
\curveto(179.78595968,162.39215795)(179.50352055,161.8364064)(179.23372796,161.2179761)
\curveto(178.97236638,160.60790297)(178.74472887,159.95186542)(178.55081544,159.24986347)
\curveto(178.35690201,158.55621867)(178.23465224,157.86257388)(178.18406613,157.16892909)
\lineto(177.10911125,157.16892909)
\closepath
}
}
{
\newrgbcolor{curcolor}{0 0 0}
\pscustom[linestyle=none,fillstyle=solid,fillcolor=curcolor]
{
\newpath
\moveto(199.79698543,159.18718472)
\curveto(199.79698543,158.53532576)(199.58620996,158.00882429)(199.16465903,157.60768031)
\curveto(198.75153911,157.20653634)(198.11921271,157.00596435)(197.26767982,157.00596435)
\curveto(196.77868074,157.00596435)(196.37399184,157.0686431)(196.05361313,157.19400059)
\curveto(195.73323442,157.32771525)(195.47608835,157.49485857)(195.28217492,157.69543056)
\curveto(195.09669251,157.90435971)(194.96179621,158.13418178)(194.87748602,158.38489677)
\curveto(194.80160685,158.63561175)(194.76366727,158.88214815)(194.76366727,159.12450597)
\curveto(194.76366727,159.56743578)(194.88591704,159.96022259)(195.13041658,160.3028664)
\curveto(195.37491612,160.64551022)(195.66578627,160.9212967)(196.00302702,161.13022585)
\curveto(195.28639043,161.53136983)(194.92807213,162.14562154)(194.92807213,162.972981)
\curveto(194.92807213,163.25712465)(194.98287375,163.52873255)(195.092477,163.7878047)
\curveto(195.20208024,164.04687685)(195.35805409,164.27252034)(195.56039854,164.46473516)
\curveto(195.76274298,164.65694998)(196.00724253,164.81155755)(196.29389716,164.92855788)
\curveto(196.58898282,165.04555821)(196.91779255,165.10405837)(197.28032635,165.10405837)
\curveto(197.70187728,165.10405837)(198.06019558,165.04137962)(198.35528123,164.91602213)
\curveto(198.65879791,164.79066464)(198.90329745,164.6276999)(199.08877986,164.42712791)
\curveto(199.27426227,164.23491309)(199.40915857,164.01762677)(199.49346876,163.77526895)
\curveto(199.57777894,163.5412683)(199.61993404,163.31144623)(199.61993404,163.08580274)
\curveto(199.61993404,162.64287293)(199.50611529,162.25844329)(199.27847778,161.93251381)
\curveto(199.05084028,161.61494149)(198.7894787,161.36004792)(198.49439304,161.1678331)
\curveto(199.36278797,160.75833196)(199.79698543,160.09811583)(199.79698543,159.18718472)
\closepath
\moveto(195.77538951,159.11197022)
\curveto(195.77538951,158.97825557)(195.80068257,158.83618374)(195.85126868,158.68575475)
\curveto(195.90185479,158.54368292)(195.98616498,158.40996827)(196.10419924,158.28461077)
\curveto(196.2222335,158.15925328)(196.37820735,158.0547887)(196.57212078,157.97121704)
\curveto(196.76603421,157.89600255)(197.00210273,157.8583953)(197.28032635,157.8583953)
\curveto(197.54168793,157.8583953)(197.76510992,157.89182396)(197.95059234,157.95868129)
\curveto(198.14450577,158.03389579)(198.30047961,158.1300032)(198.41851387,158.24700353)
\curveto(198.54497915,158.37236102)(198.63772036,158.51025426)(198.69673749,158.66068325)
\curveto(198.75575462,158.81111224)(198.78526319,158.96154123)(198.78526319,159.11197022)
\curveto(198.78526319,159.5883287)(198.6124273,159.95186542)(198.26675554,160.20258041)
\curveto(197.92108377,160.46165256)(197.44473121,160.65804597)(196.83769787,160.79176063)
\curveto(196.50045712,160.60790297)(196.23909554,160.3780809)(196.05361313,160.10229442)
\curveto(195.86813072,159.82650793)(195.77538951,159.49639987)(195.77538951,159.11197022)
\closepath
\moveto(198.60821179,163.08580274)
\curveto(198.60821179,163.1944459)(198.58291874,163.31562481)(198.53233263,163.44933947)
\curveto(198.48174651,163.59141129)(198.40165184,163.71676878)(198.29204859,163.82541195)
\curveto(198.19087637,163.94241227)(198.05598007,164.03851968)(197.8873597,164.11373418)
\curveto(197.71873932,164.19730584)(197.51639487,164.23909167)(197.28032635,164.23909167)
\curveto(197.03582681,164.23909167)(196.82926685,164.20148442)(196.66064648,164.12626993)
\curveto(196.50045712,164.05105543)(196.36556082,163.95494802)(196.25595758,163.83794769)
\curveto(196.14635433,163.72930453)(196.06625966,163.60394704)(196.01567354,163.46187522)
\curveto(195.97351845,163.32816056)(195.9524409,163.1944459)(195.9524409,163.06073124)
\curveto(195.9524409,162.71808743)(196.07469068,162.39633653)(196.31919022,162.09547855)
\curveto(196.57212078,161.80297773)(196.98524069,161.58986999)(197.55854997,161.45615534)
\curveto(197.87892868,161.64001299)(198.13185924,161.85729931)(198.31734165,162.1080143)
\curveto(198.51125508,162.35872928)(198.60821179,162.68465876)(198.60821179,163.08580274)
\closepath
}
}
{
\newrgbcolor{curcolor}{0 0 0}
\pscustom[linestyle=none,fillstyle=solid,fillcolor=curcolor]
{
\newpath
\moveto(218.74147965,161.81969206)
\curveto(218.74147965,160.27361632)(218.36208381,159.10779164)(217.60329213,158.32221802)
\curveto(216.84450044,157.54500157)(215.70209741,157.15221476)(214.17608303,157.14385759)
\lineto(214.13814344,158.02136004)
\curveto(215.08241754,158.02136004)(215.84120922,158.2052177)(216.41451849,158.57293301)
\curveto(216.99625878,158.94900548)(217.37987013,159.59250728)(217.56535254,160.50343839)
\curveto(217.36300809,160.41150956)(217.1395861,160.33629507)(216.89508656,160.27779491)
\curveto(216.65058701,160.22765191)(216.39344094,160.20258041)(216.12364835,160.20258041)
\curveto(215.67680435,160.20258041)(215.30162402,160.26525916)(214.99810735,160.39061665)
\curveto(214.69459068,160.52433131)(214.45009113,160.6998318)(214.26460872,160.91711812)
\curveto(214.07912631,161.1427616)(213.94423001,161.39765517)(213.85991983,161.68179882)
\curveto(213.77560964,161.96594247)(213.73345455,162.26680045)(213.73345455,162.58437277)
\curveto(213.73345455,162.86851642)(213.77982515,163.15683865)(213.87256635,163.44933947)
\curveto(213.96530756,163.75019745)(214.10863488,164.02180535)(214.30254831,164.26416317)
\curveto(214.49646174,164.50652099)(214.74517679,164.70709298)(215.04869346,164.86587913)
\curveto(215.35221014,165.02466529)(215.71474394,165.10405837)(216.13629487,165.10405837)
\curveto(216.99625878,165.10405837)(217.64544722,164.81155755)(218.08386019,164.22655592)
\curveto(218.52227316,163.64155429)(218.74147965,162.83926634)(218.74147965,161.81969206)
\closepath
\moveto(216.2374671,161.05501136)
\curveto(216.5072597,161.05501136)(216.75597475,161.08008286)(216.98361225,161.13022585)
\curveto(217.21968078,161.18036885)(217.44731828,161.25140476)(217.66652477,161.34333359)
\curveto(217.67495579,161.42690525)(217.6791713,161.50629833)(217.6791713,161.58151283)
\lineto(217.6791713,161.81969206)
\curveto(217.6791713,162.14562154)(217.65387824,162.45483669)(217.60329213,162.74733751)
\curveto(217.56113703,163.03983833)(217.48104236,163.29473189)(217.36300809,163.51201821)
\curveto(217.25340485,163.72930453)(217.09743101,163.90062644)(216.89508656,164.02598393)
\curveto(216.70117313,164.15969859)(216.45245807,164.22655592)(216.1489414,164.22655592)
\curveto(215.89601084,164.22655592)(215.68523537,164.17223434)(215.516615,164.06359118)
\curveto(215.34799463,163.96330519)(215.20888282,163.83376911)(215.09927957,163.67498295)
\curveto(214.98967633,163.52455396)(214.90958165,163.35323206)(214.85899554,163.16101723)
\curveto(214.81684045,162.96880241)(214.7957629,162.78494476)(214.7957629,162.60944427)
\curveto(214.7957629,162.09965713)(214.90958165,161.7110489)(215.13721916,161.44361959)
\curveto(215.37328768,161.18454744)(215.740037,161.05501136)(216.2374671,161.05501136)
\closepath
}
}
{
\newrgbcolor{curcolor}{0 0 0}
\pscustom[linestyle=none,fillstyle=solid,fillcolor=curcolor]
{
\newpath
\moveto(236.15574398,157.97121704)
\curveto(236.15574398,157.72050206)(236.07143379,157.49903715)(235.90281342,157.30682233)
\curveto(235.73419304,157.11460751)(235.51077105,157.0185001)(235.23254743,157.0185001)
\curveto(234.9458928,157.0185001)(234.71825529,157.11460751)(234.54963492,157.30682233)
\curveto(234.38101454,157.49903715)(234.29670436,157.72050206)(234.29670436,157.97121704)
\curveto(234.29670436,158.23028919)(234.38101454,158.45593268)(234.54963492,158.6481475)
\curveto(234.71825529,158.84036232)(234.9458928,158.93646974)(235.23254743,158.93646974)
\curveto(235.51077105,158.93646974)(235.73419304,158.84036232)(235.90281342,158.6481475)
\curveto(236.07143379,158.45593268)(236.15574398,158.23028919)(236.15574398,157.97121704)
\closepath
}
}
{
\newrgbcolor{curcolor}{0 0 0}
\pscustom[linestyle=none,fillstyle=solid,fillcolor=curcolor]
{
\newpath
\moveto(242.47901662,157.97121704)
\curveto(242.47901662,157.72050206)(242.39470643,157.49903715)(242.22608606,157.30682233)
\curveto(242.05746569,157.11460751)(241.83404369,157.0185001)(241.55582007,157.0185001)
\curveto(241.26916544,157.0185001)(241.04152793,157.11460751)(240.87290756,157.30682233)
\curveto(240.70428719,157.49903715)(240.619977,157.72050206)(240.619977,157.97121704)
\curveto(240.619977,158.23028919)(240.70428719,158.45593268)(240.87290756,158.6481475)
\curveto(241.04152793,158.84036232)(241.26916544,158.93646974)(241.55582007,158.93646974)
\curveto(241.83404369,158.93646974)(242.05746569,158.84036232)(242.22608606,158.6481475)
\curveto(242.39470643,158.45593268)(242.47901662,158.23028919)(242.47901662,157.97121704)
\closepath
}
}
{
\newrgbcolor{curcolor}{0 0 0}
\pscustom[linestyle=none,fillstyle=solid,fillcolor=curcolor]
{
\newpath
\moveto(248.80227394,157.97121704)
\curveto(248.80227394,157.72050206)(248.71796375,157.49903715)(248.54934338,157.30682233)
\curveto(248.380723,157.11460751)(248.15730101,157.0185001)(247.87907739,157.0185001)
\curveto(247.59242276,157.0185001)(247.36478525,157.11460751)(247.19616488,157.30682233)
\curveto(247.0275445,157.49903715)(246.94323432,157.72050206)(246.94323432,157.97121704)
\curveto(246.94323432,158.23028919)(247.0275445,158.45593268)(247.19616488,158.6481475)
\curveto(247.36478525,158.84036232)(247.59242276,158.93646974)(247.87907739,158.93646974)
\curveto(248.15730101,158.93646974)(248.380723,158.84036232)(248.54934338,158.6481475)
\curveto(248.71796375,158.45593268)(248.80227394,158.23028919)(248.80227394,157.97121704)
\closepath
}
}
{
\newrgbcolor{curcolor}{0 0 0}
\pscustom[linestyle=none,fillstyle=solid,fillcolor=curcolor]
{
\newpath
\moveto(266.02684034,161.93251381)
\curveto(267.15659685,161.89072798)(267.97862117,161.64419157)(268.49291331,161.1929046)
\curveto(269.00720545,160.74161763)(269.26435152,160.13572308)(269.26435152,159.37522096)
\curveto(269.26435152,159.03257715)(269.2095499,158.71500483)(269.09994666,158.42250402)
\curveto(268.99034342,158.1300032)(268.81750753,157.87928821)(268.58143901,157.67035906)
\curveto(268.35380151,157.46142991)(268.05871585,157.29846517)(267.69618205,157.18146484)
\curveto(267.34207926,157.06446451)(266.92052833,157.00596435)(266.43152924,157.00596435)
\curveto(266.22918479,157.00596435)(266.02684034,157.02267868)(265.8244959,157.05610735)
\curveto(265.63058247,157.08117885)(265.44510005,157.11460751)(265.26804866,157.15639334)
\curveto(265.09099727,157.19817917)(264.93502342,157.239965)(264.80012712,157.28175083)
\curveto(264.66523083,157.33189383)(264.56827411,157.37367966)(264.50925698,157.40710833)
\lineto(264.71160143,158.29714652)
\curveto(264.84649773,158.23028919)(265.05305769,158.15089611)(265.3312813,158.05896729)
\curveto(265.61793594,157.96703846)(265.97625423,157.92107405)(266.40623619,157.92107405)
\curveto(266.74347693,157.92107405)(267.02591606,157.95868129)(267.25355357,158.03389579)
\curveto(267.48119107,158.10911028)(267.66245797,158.20939628)(267.79735427,158.33475377)
\curveto(267.94068159,158.46846843)(268.04185381,158.61889742)(268.10087094,158.78604074)
\curveto(268.16831909,158.95318407)(268.20204317,159.12868456)(268.20204317,159.31254221)
\curveto(268.20204317,159.59668586)(268.15145706,159.84740085)(268.05028483,160.06468717)
\curveto(267.95754363,160.29033065)(267.78892325,160.47836689)(267.54442371,160.62879588)
\curveto(267.30835519,160.77922488)(266.97954546,160.89204662)(266.55799452,160.96726111)
\curveto(266.14487461,161.05083278)(265.62215145,161.09261861)(264.98982505,161.09261861)
\curveto(265.04041116,161.46033392)(265.07835074,161.80297773)(265.1036438,162.12055005)
\curveto(265.13736787,162.44647953)(265.16266093,162.75987326)(265.17952297,163.06073124)
\curveto(265.20481602,163.36994639)(265.22167806,163.67498295)(265.23010908,163.97584094)
\curveto(265.24697112,164.27669892)(265.26383315,164.59427123)(265.28069519,164.92855788)
\lineto(269.04936055,164.92855788)
\lineto(269.04936055,164.05105543)
\lineto(266.19124521,164.05105543)
\curveto(266.18281419,163.93405511)(266.17016766,163.77944753)(266.15330563,163.58723271)
\curveto(266.14487461,163.40337505)(266.13222808,163.20698165)(266.11536604,162.99805249)
\curveto(266.098504,162.78912334)(266.08164197,162.58855135)(266.06477993,162.39633653)
\curveto(266.04791789,162.20412171)(266.03527136,162.04951413)(266.02684034,161.93251381)
\closepath
}
}
{
\newrgbcolor{curcolor}{0 0 0}
\pscustom[linestyle=none,fillstyle=solid,fillcolor=curcolor]
{
\newpath
\moveto(275.30940055,162.9479095)
\curveto(275.30940055,162.68048018)(275.25459893,162.42140803)(275.14499569,162.17069304)
\curveto(275.04382346,161.91997806)(274.90471165,161.67344166)(274.72766026,161.43108384)
\curveto(274.55903989,161.18872602)(274.36512646,160.95054678)(274.14591997,160.71654613)
\curveto(273.92671348,160.48254548)(273.70329149,160.25272341)(273.47565398,160.02707992)
\curveto(273.3491887,159.90172243)(273.20164588,159.75129344)(273.0330255,159.57579295)
\curveto(272.86440513,159.40029246)(272.70421577,159.22061338)(272.55245744,159.03675573)
\curveto(272.4006991,158.85289807)(272.27423382,158.673219)(272.1730616,158.49771851)
\curveto(272.07188937,158.32221802)(272.02130326,158.17178903)(272.02130326,158.04643154)
\lineto(275.62556375,158.04643154)
\lineto(275.62556375,157.16892909)
\lineto(270.88311573,157.16892909)
\curveto(270.87468472,157.21071492)(270.87046921,157.25250075)(270.87046921,157.29428658)
\lineto(270.87046921,157.43217982)
\curveto(270.87046921,157.7831808)(270.92948634,158.10911028)(271.0475206,158.40996827)
\curveto(271.16555486,158.71082625)(271.3173132,158.9949699)(271.50279561,159.26239922)
\curveto(271.68827802,159.52982853)(271.89483798,159.78054352)(272.12247548,160.01454417)
\curveto(272.35854401,160.25690199)(272.59039702,160.49090264)(272.81803453,160.71654613)
\curveto(273.00351694,160.90040379)(273.18056833,161.08008286)(273.3491887,161.25558335)
\curveto(273.5262401,161.43108384)(273.67799843,161.60658433)(273.80446371,161.78208482)
\curveto(273.93936001,161.95758531)(274.04474775,162.13726438)(274.12062691,162.32112203)
\curveto(274.2049371,162.51333686)(274.24709219,162.70973026)(274.24709219,162.91030225)
\curveto(274.24709219,163.13594574)(274.20915261,163.32816056)(274.13327344,163.48694672)
\curveto(274.06582529,163.64573287)(273.96886858,163.77526895)(273.8424033,163.87555494)
\curveto(273.72436904,163.9841981)(273.58525723,164.06359118)(273.42506787,164.11373418)
\curveto(273.26487852,164.16387718)(273.09204263,164.18894867)(272.90656022,164.18894867)
\curveto(272.68735374,164.18894867)(272.48500929,164.15969859)(272.29952688,164.10119843)
\curveto(272.12247548,164.04269827)(271.96228613,163.97166235)(271.81895881,163.88809069)
\curveto(271.68406251,163.80451903)(271.56602825,163.72094737)(271.46485602,163.63737571)
\curveto(271.3636838,163.56216121)(271.28780463,163.49948246)(271.23721852,163.44933947)
\lineto(270.71871087,164.17641292)
\curveto(270.78615902,164.25162742)(270.88733124,164.34355625)(271.02222754,164.45219941)
\curveto(271.15712384,164.56084257)(271.3173132,164.66112856)(271.50279561,164.75305739)
\curveto(271.69670904,164.85334338)(271.91170002,164.93691505)(272.14776854,165.00377238)
\curveto(272.38383706,165.0706297)(272.63676762,165.10405837)(272.90656022,165.10405837)
\curveto(273.72436904,165.10405837)(274.32718687,164.91602213)(274.71501373,164.53994965)
\curveto(275.11127161,164.17223434)(275.30940055,163.64155429)(275.30940055,162.9479095)
\closepath
}
}
{
\newrgbcolor{curcolor}{0 0 0}
\pscustom[linewidth=0.56097835,linecolor=curcolor]
{
\newpath
\moveto(52.53490964,167.41457656)
\lineto(277.23287717,167.41457656)
\lineto(277.23287717,154.69545749)
\lineto(52.53490964,154.69545749)
\closepath
}
}
{
\newrgbcolor{curcolor}{0 0 0}
\pscustom[linewidth=0.51800942,linecolor=curcolor]
{
\newpath
\moveto(71.388228,154.70955633)
\lineto(71.388228,167.30895633)
}
}
{
\newrgbcolor{curcolor}{0 0 0}
\pscustom[linewidth=0.5154072,linecolor=curcolor]
{
\newpath
\moveto(90.26845,154.80387633)
\lineto(90.26845,167.27699633)
}
}
{
\newrgbcolor{curcolor}{0 0 0}
\pscustom[linewidth=0.51930571,linecolor=curcolor]
{
\newpath
\moveto(109.10774,154.80386633)
\lineto(109.10774,167.46639633)
}
}
{
\newrgbcolor{curcolor}{0 0 0}
\pscustom[linewidth=0.52573889,linecolor=curcolor]
{
\newpath
\moveto(128.03631,154.62529633)
\lineto(128.03631,167.60350633)
}
}
{
\newrgbcolor{curcolor}{0 0 0}
\pscustom[linewidth=0.51670998,linecolor=curcolor]
{
\newpath
\moveto(146.87559,154.80386633)
\lineto(146.87559,167.34012633)
}
}
{
\newrgbcolor{curcolor}{0 0 0}
\pscustom[linewidth=0.52059871,linecolor=curcolor]
{
\newpath
\moveto(165.6256,154.80387633)
\lineto(165.6256,167.52953633)
}
}
{
\newrgbcolor{curcolor}{0 0 0}
\pscustom[linewidth=0.51670998,linecolor=curcolor]
{
\newpath
\moveto(184.55417,154.71458633)
\lineto(184.55417,167.25084633)
}
}
{
\newrgbcolor{curcolor}{0 0 0}
\pscustom[linewidth=0.51670998,linecolor=curcolor]
{
\newpath
\moveto(203.66131,154.80387633)
\lineto(203.66131,167.34013633)
}
}
{
\newrgbcolor{curcolor}{0 0 0}
\pscustom[linewidth=0.52059871,linecolor=curcolor]
{
\newpath
\moveto(222.58988,154.80386633)
\lineto(222.58988,167.52953633)
}
}
{
\newrgbcolor{curcolor}{0 0 0}
\pscustom[linewidth=0.52573889,linecolor=curcolor]
{
\newpath
\moveto(258.24019,167.45564633)
\lineto(258.20469,154.47749633)
}
}
{
\newrgbcolor{curcolor}{0 1 0}
\pscustom[linestyle=none,fillstyle=solid,fillcolor=curcolor]
{
\newpath
\moveto(109.1057789,145.425075)
\lineto(127.95374795,145.425075)
\lineto(127.95374795,132.60876412)
\lineto(109.1057789,132.60876412)
\closepath
}
}
{
\newrgbcolor{curcolor}{0 0 0}
\pscustom[linestyle=none,fillstyle=solid,fillcolor=curcolor]
{
\newpath
\moveto(62.27863229,141.26470255)
\curveto(62.70861425,141.43184588)(63.12594967,141.64077503)(63.53063857,141.89149002)
\curveto(63.93532747,142.15056217)(64.3105078,142.47649165)(64.65617956,142.86927846)
\lineto(65.38967819,142.86927846)
\lineto(65.38967819,135.98715212)
\lineto(66.86932197,135.98715212)
\lineto(66.86932197,135.10964967)
\lineto(62.65802813,135.10964967)
\lineto(62.65802813,135.98715212)
\lineto(64.35266289,135.98715212)
\lineto(64.35266289,141.42766729)
\curveto(64.25992169,141.34409563)(64.14610293,141.25634539)(64.01120663,141.16441656)
\curveto(63.88474135,141.0808449)(63.74141404,140.99727324)(63.58122468,140.91370157)
\curveto(63.42946634,140.83012991)(63.26927699,140.75073683)(63.10065662,140.67552234)
\curveto(62.93203624,140.60030784)(62.76763138,140.5376291)(62.60744202,140.4874861)
\lineto(62.27863229,141.26470255)
\closepath
}
}
{
\newrgbcolor{curcolor}{0 0 0}
\pscustom[linestyle=none,fillstyle=solid,fillcolor=curcolor]
{
\newpath
\moveto(76.32892399,139.87323439)
\curveto(77.4586805,139.83144855)(78.28070482,139.58491215)(78.79499696,139.13362518)
\curveto(79.3092891,138.68233821)(79.56643517,138.07644366)(79.56643517,137.31594154)
\curveto(79.56643517,136.97329772)(79.51163355,136.65572541)(79.40203031,136.36322459)
\curveto(79.29242706,136.07072378)(79.11959118,135.82000879)(78.88352266,135.61107964)
\curveto(78.65588515,135.40215048)(78.3607995,135.23918574)(77.99826569,135.12218542)
\curveto(77.64416291,135.00518509)(77.22261197,134.94668493)(76.73361289,134.94668493)
\curveto(76.53126844,134.94668493)(76.32892399,134.96339926)(76.12657954,134.99682792)
\curveto(75.93266611,135.02189942)(75.7471837,135.05532809)(75.57013231,135.09711392)
\curveto(75.39308092,135.13889975)(75.23710707,135.18068558)(75.10221077,135.22247141)
\curveto(74.96731447,135.27261441)(74.87035776,135.31440024)(74.81134063,135.3478289)
\lineto(75.01368507,136.2378671)
\curveto(75.14858137,136.17100977)(75.35514133,136.09161669)(75.63336495,135.99968786)
\curveto(75.92001958,135.90775904)(76.27833788,135.86179462)(76.70831983,135.86179462)
\curveto(77.04556058,135.86179462)(77.32799971,135.89940187)(77.55563721,135.97461637)
\curveto(77.78327472,136.04983086)(77.96454162,136.15011686)(78.09943792,136.27547435)
\curveto(78.24276524,136.40918901)(78.34393746,136.559618)(78.40295459,136.72676132)
\curveto(78.47040274,136.89390465)(78.50412682,137.06940513)(78.50412682,137.25326279)
\curveto(78.50412682,137.53740644)(78.4535407,137.78812143)(78.35236848,138.00540775)
\curveto(78.25962727,138.23105123)(78.0910069,138.41908747)(77.84650736,138.56951646)
\curveto(77.61043883,138.71994545)(77.2816291,138.8327672)(76.86007817,138.90798169)
\curveto(76.44695825,138.99155335)(75.92423509,139.03333918)(75.29190869,139.03333918)
\curveto(75.3424948,139.4010545)(75.38043439,139.74369831)(75.40572744,140.06127062)
\curveto(75.43945152,140.3872001)(75.46474458,140.70059384)(75.48160661,141.00145182)
\curveto(75.50689967,141.31066697)(75.52376171,141.61570353)(75.53219272,141.91656151)
\curveto(75.54905476,142.2174195)(75.5659168,142.53499181)(75.58277884,142.86927846)
\lineto(79.35144419,142.86927846)
\lineto(79.35144419,141.99177601)
\lineto(76.49332886,141.99177601)
\curveto(76.48489784,141.87477568)(76.47225131,141.72016811)(76.45538927,141.52795329)
\curveto(76.44695825,141.34409563)(76.43431173,141.14770223)(76.41744969,140.93877307)
\curveto(76.40058765,140.72984392)(76.38372561,140.52927193)(76.36686358,140.33705711)
\curveto(76.35000154,140.14484229)(76.33735501,139.99023471)(76.32892399,139.87323439)
\closepath
}
}
{
\newrgbcolor{curcolor}{0 0 0}
\pscustom[linestyle=none,fillstyle=solid,fillcolor=curcolor]
{
\newpath
\moveto(81.24842436,141.26470255)
\curveto(81.67840631,141.43184588)(82.09574174,141.64077503)(82.50043063,141.89149002)
\curveto(82.90511953,142.15056217)(83.28029986,142.47649165)(83.62597163,142.86927846)
\lineto(84.35947026,142.86927846)
\lineto(84.35947026,135.98715212)
\lineto(85.83911404,135.98715212)
\lineto(85.83911404,135.10964967)
\lineto(81.6278202,135.10964967)
\lineto(81.6278202,135.98715212)
\lineto(83.32245496,135.98715212)
\lineto(83.32245496,141.42766729)
\curveto(83.22971375,141.34409563)(83.115895,141.25634539)(82.9809987,141.16441656)
\curveto(82.85453342,141.0808449)(82.7112061,140.99727324)(82.55101675,140.91370157)
\curveto(82.39925841,140.83012991)(82.23906906,140.75073683)(82.07044868,140.67552234)
\curveto(81.90182831,140.60030784)(81.73742344,140.5376291)(81.57723409,140.4874861)
\lineto(81.24842436,141.26470255)
\closepath
}
}
{
\newrgbcolor{curcolor}{0 0 0}
\pscustom[linestyle=none,fillstyle=solid,fillcolor=curcolor]
{
\newpath
\moveto(93.89495049,141.26470255)
\curveto(94.32493244,141.43184588)(94.74226787,141.64077503)(95.14695676,141.89149002)
\curveto(95.55164566,142.15056217)(95.92682599,142.47649165)(96.27249776,142.86927846)
\lineto(97.00599639,142.86927846)
\lineto(97.00599639,135.98715212)
\lineto(98.48564017,135.98715212)
\lineto(98.48564017,135.10964967)
\lineto(94.27434633,135.10964967)
\lineto(94.27434633,135.98715212)
\lineto(95.96898109,135.98715212)
\lineto(95.96898109,141.42766729)
\curveto(95.87623988,141.34409563)(95.76242113,141.25634539)(95.62752483,141.16441656)
\curveto(95.50105955,141.0808449)(95.35773223,140.99727324)(95.19754288,140.91370157)
\curveto(95.04578454,140.83012991)(94.88559518,140.75073683)(94.71697481,140.67552234)
\curveto(94.54835444,140.60030784)(94.38394957,140.5376291)(94.22376022,140.4874861)
\lineto(93.89495049,141.26470255)
\closepath
}
}
{
\newrgbcolor{curcolor}{0 0 0}
\pscustom[linestyle=none,fillstyle=solid,fillcolor=curcolor]
{
\newpath
\moveto(99.72500087,137.79230001)
\curveto(99.86832819,138.12658666)(100.06224162,138.5110163)(100.30674116,138.94558894)
\curveto(100.55967172,139.38016158)(100.84211085,139.82726997)(101.15405854,140.28691411)
\curveto(101.46600623,140.75491542)(101.79903147,141.21038097)(102.15313426,141.65331078)
\curveto(102.50723704,142.10459775)(102.86555534,142.50992031)(103.22808914,142.86927846)
\lineto(104.23981138,142.86927846)
\lineto(104.23981138,137.942729)
\lineto(105.16300793,137.942729)
\lineto(105.16300793,137.09029805)
\lineto(104.23981138,137.09029805)
\lineto(104.23981138,135.10964967)
\lineto(103.22808914,135.10964967)
\lineto(103.22808914,137.09029805)
\lineto(99.72500087,137.09029805)
\lineto(99.72500087,137.79230001)
\closepath
\moveto(103.22808914,141.64077503)
\curveto(103.00045164,141.39841721)(102.76859862,141.13098789)(102.5325301,140.83848708)
\curveto(102.30489259,140.54598626)(102.0814706,140.23677111)(101.86226411,139.91084163)
\curveto(101.64305763,139.59326932)(101.43649767,139.26733984)(101.24258424,138.93305319)
\curveto(101.05710183,138.59876654)(100.88848145,138.26865848)(100.73672312,137.942729)
\lineto(103.22808914,137.942729)
\lineto(103.22808914,141.64077503)
\closepath
}
}
{
\newrgbcolor{curcolor}{0 0 0}
\pscustom[linestyle=none,fillstyle=solid,fillcolor=curcolor]
{
\newpath
\moveto(117.2277976,140.88863007)
\curveto(117.2277976,140.62120076)(117.17299598,140.36212861)(117.06339274,140.11141362)
\curveto(116.96222051,139.86069864)(116.8231087,139.61416223)(116.64605731,139.37180441)
\curveto(116.47743694,139.1294466)(116.28352351,138.89126736)(116.06431702,138.65726671)
\curveto(115.84511054,138.42326605)(115.62168854,138.19344398)(115.39405104,137.9678005)
\curveto(115.26758576,137.84244301)(115.12004293,137.69201401)(114.95142255,137.51651352)
\curveto(114.78280218,137.34101304)(114.62261282,137.16133396)(114.47085449,136.97747631)
\curveto(114.31909615,136.79361865)(114.19263087,136.61393958)(114.09145865,136.43843909)
\curveto(113.99028642,136.2629386)(113.93970031,136.11250961)(113.93970031,135.98715212)
\lineto(117.5439608,135.98715212)
\lineto(117.5439608,135.10964967)
\lineto(112.80151279,135.10964967)
\curveto(112.79308177,135.1514355)(112.78886626,135.19322133)(112.78886626,135.23500716)
\lineto(112.78886626,135.3729004)
\curveto(112.78886626,135.72390138)(112.84788339,136.04983086)(112.96591765,136.35068884)
\curveto(113.08395191,136.65154683)(113.23571025,136.93569048)(113.42119266,137.20311979)
\curveto(113.60667507,137.47054911)(113.81323503,137.7212641)(114.04087253,137.95526475)
\curveto(114.27694106,138.19762257)(114.50879407,138.43162322)(114.73643158,138.65726671)
\curveto(114.92191399,138.84112436)(115.09896538,139.02080344)(115.26758576,139.19630393)
\curveto(115.44463715,139.37180441)(115.59639548,139.5473049)(115.72286076,139.72280539)
\curveto(115.85775706,139.89830588)(115.9631448,140.07798496)(116.03902397,140.26184261)
\curveto(116.12333415,140.45405743)(116.16548925,140.65045084)(116.16548925,140.85102283)
\curveto(116.16548925,141.07666631)(116.12754966,141.26888114)(116.05167049,141.42766729)
\curveto(115.98422234,141.58645345)(115.88726563,141.71598953)(115.76080035,141.81627552)
\curveto(115.64276609,141.92491868)(115.50365428,142.00431176)(115.34346492,142.05445476)
\curveto(115.18327557,142.10459775)(115.01043968,142.12966925)(114.82495727,142.12966925)
\curveto(114.60575079,142.12966925)(114.40340634,142.10041917)(114.21792393,142.04191901)
\curveto(114.04087253,141.98341884)(113.88068318,141.91238293)(113.73735586,141.82881127)
\curveto(113.60245956,141.74523961)(113.4844253,141.66166795)(113.38325308,141.57809628)
\curveto(113.28208085,141.50288179)(113.20620168,141.44020304)(113.15561557,141.39006005)
\lineto(112.63710792,142.1171335)
\curveto(112.70455607,142.192348)(112.8057283,142.28427683)(112.94062459,142.39291999)
\curveto(113.07552089,142.50156315)(113.23571025,142.60184914)(113.42119266,142.69377797)
\curveto(113.61510609,142.79406396)(113.83009707,142.87763562)(114.06616559,142.94449295)
\curveto(114.30223411,143.01135028)(114.55516468,143.04477895)(114.82495727,143.04477895)
\curveto(115.64276609,143.04477895)(116.24558392,142.85674271)(116.63341078,142.48067023)
\curveto(117.02966866,142.11295492)(117.2277976,141.58227487)(117.2277976,140.88863007)
\closepath
}
}
{
\newrgbcolor{curcolor}{0 0 0}
\pscustom[linestyle=none,fillstyle=solid,fillcolor=curcolor]
{
\newpath
\moveto(122.13465144,139.13362518)
\curveto(122.13465144,138.91633886)(122.06720329,138.72830262)(121.93230699,138.56951646)
\curveto(121.80584171,138.41073031)(121.63722134,138.33133723)(121.42644587,138.33133723)
\curveto(121.20723938,138.33133723)(121.03018799,138.41073031)(120.89529169,138.56951646)
\curveto(120.76039539,138.72830262)(120.69294724,138.91633886)(120.69294724,139.13362518)
\curveto(120.69294724,139.3509115)(120.76039539,139.54312632)(120.89529169,139.71026964)
\curveto(121.03018799,139.87741297)(121.20723938,139.96098463)(121.42644587,139.96098463)
\curveto(121.63722134,139.96098463)(121.80584171,139.87741297)(121.93230699,139.71026964)
\curveto(122.06720329,139.54312632)(122.13465144,139.3509115)(122.13465144,139.13362518)
\closepath
\moveto(118.82126109,138.99573194)
\curveto(118.82126109,140.29944986)(119.04468309,141.29813122)(119.49152708,141.99177601)
\curveto(119.94680209,142.69377797)(120.583344,143.04477895)(121.40115281,143.04477895)
\curveto(122.22739265,143.04477895)(122.86393456,142.69377797)(123.31077855,141.99177601)
\curveto(123.75762254,141.29813122)(123.98104454,140.29944986)(123.98104454,138.99573194)
\curveto(123.98104454,137.69201401)(123.75762254,136.68915407)(123.31077855,135.98715212)
\curveto(122.86393456,135.29350732)(122.22739265,134.94668493)(121.40115281,134.94668493)
\curveto(120.583344,134.94668493)(119.94680209,135.29350732)(119.49152708,135.98715212)
\curveto(119.04468309,136.68915407)(118.82126109,137.69201401)(118.82126109,138.99573194)
\closepath
\moveto(122.91873618,138.99573194)
\curveto(122.91873618,139.42194741)(122.89344312,139.82309139)(122.84285701,140.19916387)
\curveto(122.7922709,140.58359351)(122.70796071,140.91788016)(122.58992645,141.20202381)
\curveto(122.47189219,141.48616746)(122.31591834,141.71181094)(122.12200491,141.87895427)
\curveto(121.92809148,142.04609759)(121.68780745,142.12966925)(121.40115281,142.12966925)
\curveto(121.11449818,142.12966925)(120.87421415,142.04609759)(120.68030072,141.87895427)
\curveto(120.48638729,141.71181094)(120.33041344,141.48616746)(120.21237918,141.20202381)
\curveto(120.09434492,140.91788016)(120.01003473,140.58359351)(119.95944862,140.19916387)
\curveto(119.90886251,139.82309139)(119.88356945,139.42194741)(119.88356945,138.99573194)
\curveto(119.88356945,138.56951646)(119.90886251,138.1641939)(119.95944862,137.77976426)
\curveto(120.01003473,137.40369178)(120.09434492,137.07358372)(120.21237918,136.78944007)
\curveto(120.33041344,136.50529642)(120.48638729,136.27965293)(120.68030072,136.11250961)
\curveto(120.87421415,135.94536628)(121.11449818,135.86179462)(121.40115281,135.86179462)
\curveto(121.68780745,135.86179462)(121.92809148,135.94536628)(122.12200491,136.11250961)
\curveto(122.31591834,136.27965293)(122.47189219,136.50529642)(122.58992645,136.78944007)
\curveto(122.70796071,137.07358372)(122.7922709,137.40369178)(122.84285701,137.77976426)
\curveto(122.89344312,138.1641939)(122.91873618,138.56951646)(122.91873618,138.99573194)
\closepath
}
}
{
\newrgbcolor{curcolor}{0 0 0}
\pscustom[linestyle=none,fillstyle=solid,fillcolor=curcolor]
{
\newpath
\moveto(139.5615623,139.87323439)
\curveto(140.6913188,139.83144855)(141.51334312,139.58491215)(142.02763526,139.13362518)
\curveto(142.54192741,138.68233821)(142.79907348,138.07644366)(142.79907348,137.31594154)
\curveto(142.79907348,136.97329772)(142.74427185,136.65572541)(142.63466861,136.36322459)
\curveto(142.52506537,136.07072378)(142.35222948,135.82000879)(142.11616096,135.61107964)
\curveto(141.88852346,135.40215048)(141.5934378,135.23918574)(141.230904,135.12218542)
\curveto(140.87680121,135.00518509)(140.45525028,134.94668493)(139.96625119,134.94668493)
\curveto(139.76390675,134.94668493)(139.5615623,134.96339926)(139.35921785,134.99682792)
\curveto(139.16530442,135.02189942)(138.97982201,135.05532809)(138.80277061,135.09711392)
\curveto(138.62571922,135.13889975)(138.46974538,135.18068558)(138.33484908,135.22247141)
\curveto(138.19995278,135.27261441)(138.10299606,135.31440024)(138.04397893,135.3478289)
\lineto(138.24632338,136.2378671)
\curveto(138.38121968,136.17100977)(138.58777964,136.09161669)(138.86600325,135.99968786)
\curveto(139.15265789,135.90775904)(139.51097618,135.86179462)(139.94095814,135.86179462)
\curveto(140.27819889,135.86179462)(140.56063801,135.89940187)(140.78827552,135.97461637)
\curveto(141.01591302,136.04983086)(141.19717992,136.15011686)(141.33207622,136.27547435)
\curveto(141.47540354,136.40918901)(141.57657576,136.559618)(141.6355929,136.72676132)
\curveto(141.70304105,136.89390465)(141.73676512,137.06940513)(141.73676512,137.25326279)
\curveto(141.73676512,137.53740644)(141.68617901,137.78812143)(141.58500678,138.00540775)
\curveto(141.49226558,138.23105123)(141.3236452,138.41908747)(141.07914566,138.56951646)
\curveto(140.84307714,138.71994545)(140.51426741,138.8327672)(140.09271647,138.90798169)
\curveto(139.67959656,138.99155335)(139.1568734,139.03333918)(138.524547,139.03333918)
\curveto(138.57513311,139.4010545)(138.61307269,139.74369831)(138.63836575,140.06127062)
\curveto(138.67208982,140.3872001)(138.69738288,140.70059384)(138.71424492,141.00145182)
\curveto(138.73953797,141.31066697)(138.75640001,141.61570353)(138.76483103,141.91656151)
\curveto(138.78169307,142.2174195)(138.7985551,142.53499181)(138.81541714,142.86927846)
\lineto(142.5840825,142.86927846)
\lineto(142.5840825,141.99177601)
\lineto(139.72596716,141.99177601)
\curveto(139.71753614,141.87477568)(139.70488961,141.72016811)(139.68802758,141.52795329)
\curveto(139.67959656,141.34409563)(139.66695003,141.14770223)(139.65008799,140.93877307)
\curveto(139.63322596,140.72984392)(139.61636392,140.52927193)(139.59950188,140.33705711)
\curveto(139.58263984,140.14484229)(139.56999332,139.99023471)(139.5615623,139.87323439)
\closepath
}
}
{
\newrgbcolor{curcolor}{0 0 0}
\pscustom[linestyle=none,fillstyle=solid,fillcolor=curcolor]
{
\newpath
\moveto(156.86201553,138.21851548)
\curveto(156.86201553,138.97066044)(156.96318776,139.63505515)(157.16553221,140.21169962)
\curveto(157.37630767,140.79670125)(157.67139333,141.28559547)(158.05078917,141.67838228)
\curveto(158.43861603,142.07116909)(158.90653757,142.37202707)(159.45455378,142.58095622)
\curveto(160.00257,142.78988538)(160.62224987,142.89852854)(161.31359341,142.90688571)
\lineto(161.4021191,142.02938326)
\curveto(160.95527511,142.02102609)(160.5463707,141.97088309)(160.17540588,141.87895427)
\curveto(159.81287208,141.7953826)(159.48827786,141.65331078)(159.20162322,141.45273879)
\curveto(158.91496859,141.26052397)(158.67468455,141.00980898)(158.48077112,140.70059384)
\curveto(158.28685769,140.39137869)(158.13931487,140.01112763)(158.03814264,139.55984065)
\curveto(158.24048709,139.65176948)(158.45969358,139.72698398)(158.6957621,139.78548414)
\curveto(158.94026164,139.8439843)(159.19740771,139.87323439)(159.46720031,139.87323439)
\curveto(159.90561328,139.87323439)(160.27657811,139.80637706)(160.58009478,139.6726624)
\curveto(160.89204247,139.53894774)(161.14075752,139.35926867)(161.32623993,139.13362518)
\curveto(161.51172235,138.91633886)(161.64661864,138.66144529)(161.73092883,138.36894447)
\curveto(161.81523902,138.07644366)(161.85739411,137.77558568)(161.85739411,137.46637053)
\curveto(161.85739411,137.18222688)(161.81102351,136.88972606)(161.7182823,136.58886808)
\curveto(161.6255411,136.29636726)(161.48221378,136.02475936)(161.28830035,135.77404438)
\curveto(161.09438692,135.53168656)(160.84567187,135.33111457)(160.5421552,135.17232841)
\curveto(160.23863852,135.02189942)(159.87610472,134.94668493)(159.45455378,134.94668493)
\curveto(158.58615886,134.94668493)(157.93697042,135.23500716)(157.50698846,135.81165163)
\curveto(157.07700651,136.38829609)(156.86201553,137.19058404)(156.86201553,138.21851548)
\closepath
\moveto(159.35338156,139.02080344)
\curveto(159.08358896,139.02080344)(158.83487391,138.99573194)(158.6072364,138.94558894)
\curveto(158.38802992,138.89544594)(158.16460792,138.82023145)(157.93697042,138.71994545)
\curveto(157.9285394,138.63637379)(157.92432389,138.55280213)(157.92432389,138.46923047)
\lineto(157.92432389,138.21851548)
\curveto(157.92432389,137.892586)(157.94540144,137.58337085)(157.98755653,137.29087004)
\curveto(158.03814264,137.00672639)(158.11823732,136.75183282)(158.22784056,136.52618933)
\curveto(158.34587482,136.30890301)(158.50184867,136.13340252)(158.6957621,135.99968786)
\curveto(158.88967553,135.87433037)(159.13839058,135.81165163)(159.44190726,135.81165163)
\curveto(159.69483782,135.81165163)(159.90561328,135.86179462)(160.07423366,135.96208062)
\curveto(160.24285403,136.07072378)(160.38196584,136.20443844)(160.49156908,136.36322459)
\curveto(160.60117233,136.52201075)(160.67705149,136.69751124)(160.71920659,136.88972606)
\curveto(160.7697927,137.09029805)(160.79508576,137.27833429)(160.79508576,137.45383478)
\curveto(160.79508576,137.96362192)(160.67705149,138.35223014)(160.44098297,138.61965946)
\curveto(160.21334547,138.88708878)(159.85081166,139.02080344)(159.35338156,139.02080344)
\closepath
}
}
{
\newrgbcolor{curcolor}{0 0 0}
\pscustom[linestyle=none,fillstyle=solid,fillcolor=curcolor]
{
\newpath
\moveto(177.10910981,135.10964967)
\curveto(177.1512649,135.70300847)(177.25665263,136.32979593)(177.42527301,136.99001206)
\curveto(177.6023244,137.65858535)(177.81309987,138.30208714)(178.05759941,138.92051744)
\curveto(178.30209895,139.5473049)(178.57189155,140.12394937)(178.8669772,140.65045084)
\curveto(179.16206286,141.18530947)(179.452933,141.6240607)(179.73958764,141.96670451)
\lineto(175.94562923,141.96670451)
\lineto(175.94562923,142.86927846)
\lineto(180.90306822,142.86927846)
\lineto(180.90306822,142.00431176)
\curveto(180.65013766,141.71181094)(180.37612955,141.31902413)(180.0810439,140.82595133)
\curveto(179.78595824,140.33287852)(179.50351912,139.77712697)(179.23372652,139.15869668)
\curveto(178.97236494,138.54862355)(178.74472743,137.892586)(178.550814,137.19058404)
\curveto(178.35690057,136.49693925)(178.2346508,135.80329446)(178.18406469,135.10964967)
\lineto(177.10910981,135.10964967)
\closepath
}
}
{
\newrgbcolor{curcolor}{0 0 0}
\pscustom[linestyle=none,fillstyle=solid,fillcolor=curcolor]
{
\newpath
\moveto(199.79698399,137.1279053)
\curveto(199.79698399,136.47604634)(199.58620852,135.94954487)(199.16465759,135.54840089)
\curveto(198.75153767,135.14725692)(198.11921127,134.94668493)(197.26767838,134.94668493)
\curveto(196.7786793,134.94668493)(196.3739904,135.00936367)(196.05361169,135.13472117)
\curveto(195.73323298,135.26843582)(195.47608691,135.43557915)(195.28217348,135.63615114)
\curveto(195.09669107,135.84508029)(194.96179477,136.07490236)(194.87748458,136.32561735)
\curveto(194.80160541,136.57633233)(194.76366583,136.82286873)(194.76366583,137.06522655)
\curveto(194.76366583,137.50815636)(194.8859156,137.90094317)(195.13041514,138.24358698)
\curveto(195.37491469,138.58623079)(195.66578483,138.86201728)(196.00302558,139.07094643)
\curveto(195.28638899,139.47209041)(194.92807069,140.08634212)(194.92807069,140.91370157)
\curveto(194.92807069,141.19784522)(194.98287232,141.46945312)(195.09247556,141.72852528)
\curveto(195.2020788,141.98759743)(195.35805265,142.21324091)(195.5603971,142.40545573)
\curveto(195.76274155,142.59767056)(196.00724109,142.75227813)(196.29389572,142.86927846)
\curveto(196.58898138,142.98627878)(196.91779111,143.04477895)(197.28032491,143.04477895)
\curveto(197.70187585,143.04477895)(198.06019414,142.9821002)(198.35527979,142.85674271)
\curveto(198.65879647,142.73138522)(198.90329601,142.56842048)(199.08877842,142.36784849)
\curveto(199.27426083,142.17563367)(199.40915713,141.95834734)(199.49346732,141.71598953)
\curveto(199.57777751,141.48198887)(199.6199326,141.2521668)(199.6199326,141.02652332)
\curveto(199.6199326,140.58359351)(199.50611385,140.19916387)(199.27847634,139.87323439)
\curveto(199.05083884,139.55566207)(198.78947726,139.3007685)(198.4943916,139.10855368)
\curveto(199.36278653,138.69905254)(199.79698399,138.03883641)(199.79698399,137.1279053)
\closepath
\moveto(195.77538807,137.0526908)
\curveto(195.77538807,136.91897614)(195.80068113,136.77690432)(195.85126724,136.62647533)
\curveto(195.90185335,136.4844035)(195.98616354,136.35068884)(196.1041978,136.22533135)
\curveto(196.22223206,136.09997386)(196.37820591,135.99550928)(196.57211934,135.91193762)
\curveto(196.76603277,135.83672312)(197.00210129,135.79911588)(197.28032491,135.79911588)
\curveto(197.54168649,135.79911588)(197.76510849,135.83254454)(197.9505909,135.89940187)
\curveto(198.14450433,135.97461637)(198.30047817,136.07072378)(198.41851243,136.1877241)
\curveto(198.54497772,136.3130816)(198.63771892,136.45097484)(198.69673605,136.60140383)
\curveto(198.75575318,136.75183282)(198.78526175,136.90226181)(198.78526175,137.0526908)
\curveto(198.78526175,137.52904927)(198.61242586,137.892586)(198.2667541,138.14330099)
\curveto(197.92108233,138.40237314)(197.44472978,138.59876654)(196.83769643,138.7324812)
\curveto(196.50045568,138.54862355)(196.2390941,138.31880148)(196.05361169,138.04301499)
\curveto(195.86812928,137.76722851)(195.77538807,137.43712045)(195.77538807,137.0526908)
\closepath
\moveto(198.60821036,141.02652332)
\curveto(198.60821036,141.13516648)(198.5829173,141.25634539)(198.53233119,141.39006005)
\curveto(198.48174507,141.53213187)(198.4016504,141.65748936)(198.29204715,141.76613252)
\curveto(198.19087493,141.88313285)(198.05597863,141.97924026)(197.88735826,142.05445476)
\curveto(197.71873788,142.13802642)(197.51639343,142.17981225)(197.28032491,142.17981225)
\curveto(197.03582537,142.17981225)(196.82926541,142.142205)(196.66064504,142.06699051)
\curveto(196.50045568,141.99177601)(196.36555938,141.8956686)(196.25595614,141.77866827)
\curveto(196.1463529,141.67002511)(196.06625822,141.54466762)(196.01567211,141.40259579)
\curveto(195.97351701,141.26888114)(195.95243947,141.13516648)(195.95243947,141.00145182)
\curveto(195.95243947,140.65880801)(196.07468924,140.33705711)(196.31918878,140.03619913)
\curveto(196.57211934,139.74369831)(196.98523926,139.53059057)(197.55854853,139.39687591)
\curveto(197.87892724,139.58073357)(198.1318578,139.79801989)(198.31734021,140.04873487)
\curveto(198.51125364,140.29944986)(198.60821036,140.62537934)(198.60821036,141.02652332)
\closepath
}
}
{
\newrgbcolor{curcolor}{0 0 0}
\pscustom[linestyle=none,fillstyle=solid,fillcolor=curcolor]
{
\newpath
\moveto(218.74147821,139.76041264)
\curveto(218.74147821,138.2143369)(218.36208237,137.04851222)(217.60329069,136.2629386)
\curveto(216.84449901,135.48572215)(215.70209597,135.09293534)(214.17608159,135.08457817)
\lineto(214.138142,135.96208062)
\curveto(215.0824161,135.96208062)(215.84120778,136.14593827)(216.41451705,136.51365358)
\curveto(216.99625734,136.88972606)(217.37986869,137.53322786)(217.5653511,138.44415897)
\curveto(217.36300666,138.35223014)(217.13958466,138.27701565)(216.89508512,138.21851548)
\curveto(216.65058558,138.16837249)(216.39343951,138.14330099)(216.12364691,138.14330099)
\curveto(215.67680292,138.14330099)(215.30162258,138.20597973)(214.99810591,138.33133723)
\curveto(214.69458924,138.46505189)(214.4500897,138.64055237)(214.26460728,138.8578387)
\curveto(214.07912487,139.08348218)(213.94422857,139.33837575)(213.85991839,139.6225194)
\curveto(213.7756082,139.90666305)(213.73345311,140.20752103)(213.73345311,140.52509335)
\curveto(213.73345311,140.809237)(213.77982371,141.09755923)(213.87256492,141.39006005)
\curveto(213.96530612,141.69091803)(214.10863344,141.96252593)(214.30254687,142.20488375)
\curveto(214.4964603,142.44724157)(214.74517535,142.64781355)(215.04869202,142.80659971)
\curveto(215.3522087,142.96538587)(215.7147425,143.04477895)(216.13629344,143.04477895)
\curveto(216.99625734,143.04477895)(217.64544578,142.75227813)(218.08385875,142.1672765)
\curveto(218.52227173,141.58227487)(218.74147821,140.77998691)(218.74147821,139.76041264)
\closepath
\moveto(216.23746566,138.99573194)
\curveto(216.50725826,138.99573194)(216.75597331,139.02080344)(216.98361081,139.07094643)
\curveto(217.21967934,139.12108943)(217.44731684,139.19212534)(217.66652333,139.28405417)
\curveto(217.67495435,139.36762583)(217.67916986,139.44701891)(217.67916986,139.52223341)
\lineto(217.67916986,139.76041264)
\curveto(217.67916986,140.08634212)(217.6538768,140.39555727)(217.60329069,140.68805809)
\curveto(217.56113559,140.9805589)(217.48104092,141.23545247)(217.36300666,141.45273879)
\curveto(217.25340341,141.67002511)(217.09742957,141.84134702)(216.89508512,141.96670451)
\curveto(216.70117169,142.10041917)(216.45245664,142.1672765)(216.14893996,142.1672765)
\curveto(215.8960094,142.1672765)(215.68523393,142.11295492)(215.51661356,142.00431176)
\curveto(215.34799319,141.90402576)(215.20888138,141.77448969)(215.09927814,141.61570353)
\curveto(214.98967489,141.46527454)(214.90958021,141.29395263)(214.8589941,141.10173781)
\curveto(214.81683901,140.90952299)(214.79576146,140.72566533)(214.79576146,140.55016484)
\curveto(214.79576146,140.04037771)(214.90958021,139.65176948)(215.13721772,139.38434016)
\curveto(215.37328624,139.12526801)(215.74003556,138.99573194)(216.23746566,138.99573194)
\closepath
}
}
{
\newrgbcolor{curcolor}{0 0 0}
\pscustom[linestyle=none,fillstyle=solid,fillcolor=curcolor]
{
\newpath
\moveto(236.15574254,135.91193762)
\curveto(236.15574254,135.66122263)(236.07143235,135.43975773)(235.90281198,135.24754291)
\curveto(235.7341916,135.05532809)(235.51076961,134.95922068)(235.23254599,134.95922068)
\curveto(234.94589136,134.95922068)(234.71825385,135.05532809)(234.54963348,135.24754291)
\curveto(234.3810131,135.43975773)(234.29670292,135.66122263)(234.29670292,135.91193762)
\curveto(234.29670292,136.17100977)(234.3810131,136.39665326)(234.54963348,136.58886808)
\curveto(234.71825385,136.7810829)(234.94589136,136.87719031)(235.23254599,136.87719031)
\curveto(235.51076961,136.87719031)(235.7341916,136.7810829)(235.90281198,136.58886808)
\curveto(236.07143235,136.39665326)(236.15574254,136.17100977)(236.15574254,135.91193762)
\closepath
}
}
{
\newrgbcolor{curcolor}{0 0 0}
\pscustom[linestyle=none,fillstyle=solid,fillcolor=curcolor]
{
\newpath
\moveto(242.47901518,135.91193762)
\curveto(242.47901518,135.66122263)(242.394705,135.43975773)(242.22608462,135.24754291)
\curveto(242.05746425,135.05532809)(241.83404225,134.95922068)(241.55581864,134.95922068)
\curveto(241.269164,134.95922068)(241.04152649,135.05532809)(240.87290612,135.24754291)
\curveto(240.70428575,135.43975773)(240.61997556,135.66122263)(240.61997556,135.91193762)
\curveto(240.61997556,136.17100977)(240.70428575,136.39665326)(240.87290612,136.58886808)
\curveto(241.04152649,136.7810829)(241.269164,136.87719031)(241.55581864,136.87719031)
\curveto(241.83404225,136.87719031)(242.05746425,136.7810829)(242.22608462,136.58886808)
\curveto(242.394705,136.39665326)(242.47901518,136.17100977)(242.47901518,135.91193762)
\closepath
}
}
{
\newrgbcolor{curcolor}{0 0 0}
\pscustom[linestyle=none,fillstyle=solid,fillcolor=curcolor]
{
\newpath
\moveto(248.8022725,135.91193762)
\curveto(248.8022725,135.66122263)(248.71796231,135.43975773)(248.54934194,135.24754291)
\curveto(248.38072156,135.05532809)(248.15729957,134.95922068)(247.87907595,134.95922068)
\curveto(247.59242132,134.95922068)(247.36478381,135.05532809)(247.19616344,135.24754291)
\curveto(247.02754306,135.43975773)(246.94323288,135.66122263)(246.94323288,135.91193762)
\curveto(246.94323288,136.17100977)(247.02754306,136.39665326)(247.19616344,136.58886808)
\curveto(247.36478381,136.7810829)(247.59242132,136.87719031)(247.87907595,136.87719031)
\curveto(248.15729957,136.87719031)(248.38072156,136.7810829)(248.54934194,136.58886808)
\curveto(248.71796231,136.39665326)(248.8022725,136.17100977)(248.8022725,135.91193762)
\closepath
}
}
{
\newrgbcolor{curcolor}{0 0 0}
\pscustom[linestyle=none,fillstyle=solid,fillcolor=curcolor]
{
\newpath
\moveto(266.02683891,139.87323439)
\curveto(267.15659541,139.83144855)(267.97861973,139.58491215)(268.49291187,139.13362518)
\curveto(269.00720402,138.68233821)(269.26435009,138.07644366)(269.26435009,137.31594154)
\curveto(269.26435009,136.97329772)(269.20954846,136.65572541)(269.09994522,136.36322459)
\curveto(268.99034198,136.07072378)(268.81750609,135.82000879)(268.58143757,135.61107964)
\curveto(268.35380007,135.40215048)(268.05871441,135.23918574)(267.69618061,135.12218542)
\curveto(267.34207782,135.00518509)(266.92052689,134.94668493)(266.4315278,134.94668493)
\curveto(266.22918336,134.94668493)(266.02683891,134.96339926)(265.82449446,134.99682792)
\curveto(265.63058103,135.02189942)(265.44509862,135.05532809)(265.26804722,135.09711392)
\curveto(265.09099583,135.13889975)(264.93502199,135.18068558)(264.80012569,135.22247141)
\curveto(264.66522939,135.27261441)(264.56827267,135.31440024)(264.50925554,135.3478289)
\lineto(264.71159999,136.2378671)
\curveto(264.84649629,136.17100977)(265.05305625,136.09161669)(265.33127986,135.99968786)
\curveto(265.6179345,135.90775904)(265.97625279,135.86179462)(266.40623475,135.86179462)
\curveto(266.7434755,135.86179462)(267.02591462,135.89940187)(267.25355213,135.97461637)
\curveto(267.48118963,136.04983086)(267.66245653,136.15011686)(267.79735283,136.27547435)
\curveto(267.94068015,136.40918901)(268.04185237,136.559618)(268.10086951,136.72676132)
\curveto(268.16831766,136.89390465)(268.20204173,137.06940513)(268.20204173,137.25326279)
\curveto(268.20204173,137.53740644)(268.15145562,137.78812143)(268.05028339,138.00540775)
\curveto(267.95754219,138.23105123)(267.78892181,138.41908747)(267.54442227,138.56951646)
\curveto(267.30835375,138.71994545)(266.97954402,138.8327672)(266.55799308,138.90798169)
\curveto(266.14487317,138.99155335)(265.62215001,139.03333918)(264.98982361,139.03333918)
\curveto(265.04040972,139.4010545)(265.0783493,139.74369831)(265.10364236,140.06127062)
\curveto(265.13736643,140.3872001)(265.16265949,140.70059384)(265.17952153,141.00145182)
\curveto(265.20481458,141.31066697)(265.22167662,141.61570353)(265.23010764,141.91656151)
\curveto(265.24696968,142.2174195)(265.26383171,142.53499181)(265.28069375,142.86927846)
\lineto(269.04935911,142.86927846)
\lineto(269.04935911,141.99177601)
\lineto(266.19124377,141.99177601)
\curveto(266.18281275,141.87477568)(266.17016622,141.72016811)(266.15330419,141.52795329)
\curveto(266.14487317,141.34409563)(266.13222664,141.14770223)(266.1153646,140.93877307)
\curveto(266.09850257,140.72984392)(266.08164053,140.52927193)(266.06477849,140.33705711)
\curveto(266.04791645,140.14484229)(266.03526992,139.99023471)(266.02683891,139.87323439)
\closepath
}
}
{
\newrgbcolor{curcolor}{0 0 0}
\pscustom[linestyle=none,fillstyle=solid,fillcolor=curcolor]
{
\newpath
\moveto(275.30939911,140.88863007)
\curveto(275.30939911,140.62120076)(275.25459749,140.36212861)(275.14499425,140.11141362)
\curveto(275.04382202,139.86069864)(274.90471021,139.61416223)(274.72765882,139.37180441)
\curveto(274.55903845,139.1294466)(274.36512502,138.89126736)(274.14591853,138.65726671)
\curveto(273.92671205,138.42326605)(273.70329005,138.19344398)(273.47565255,137.9678005)
\curveto(273.34918726,137.84244301)(273.20164444,137.69201401)(273.03302406,137.51651352)
\curveto(272.86440369,137.34101304)(272.70421433,137.16133396)(272.552456,136.97747631)
\curveto(272.40069766,136.79361865)(272.27423238,136.61393958)(272.17306016,136.43843909)
\curveto(272.07188793,136.2629386)(272.02130182,136.11250961)(272.02130182,135.98715212)
\lineto(275.62556231,135.98715212)
\lineto(275.62556231,135.10964967)
\lineto(270.8831143,135.10964967)
\curveto(270.87468328,135.1514355)(270.87046777,135.19322133)(270.87046777,135.23500716)
\lineto(270.87046777,135.3729004)
\curveto(270.87046777,135.72390138)(270.9294849,136.04983086)(271.04751916,136.35068884)
\curveto(271.16555342,136.65154683)(271.31731176,136.93569048)(271.50279417,137.20311979)
\curveto(271.68827658,137.47054911)(271.89483654,137.7212641)(272.12247404,137.95526475)
\curveto(272.35854257,138.19762257)(272.59039558,138.43162322)(272.81803309,138.65726671)
\curveto(273.0035155,138.84112436)(273.18056689,139.02080344)(273.34918726,139.19630393)
\curveto(273.52623866,139.37180441)(273.67799699,139.5473049)(273.80446227,139.72280539)
\curveto(273.93935857,139.89830588)(274.04474631,140.07798496)(274.12062548,140.26184261)
\curveto(274.20493566,140.45405743)(274.24709076,140.65045084)(274.24709076,140.85102283)
\curveto(274.24709076,141.07666631)(274.20915117,141.26888114)(274.133272,141.42766729)
\curveto(274.06582385,141.58645345)(273.96886714,141.71598953)(273.84240186,141.81627552)
\curveto(273.7243676,141.92491868)(273.58525579,142.00431176)(273.42506643,142.05445476)
\curveto(273.26487708,142.10459775)(273.09204119,142.12966925)(272.90655878,142.12966925)
\curveto(272.6873523,142.12966925)(272.48500785,142.10041917)(272.29952544,142.04191901)
\curveto(272.12247404,141.98341884)(271.96228469,141.91238293)(271.81895737,141.82881127)
\curveto(271.68406107,141.74523961)(271.56602681,141.66166795)(271.46485459,141.57809628)
\curveto(271.36368236,141.50288179)(271.28780319,141.44020304)(271.23721708,141.39006005)
\lineto(270.71870943,142.1171335)
\curveto(270.78615758,142.192348)(270.88732981,142.28427683)(271.0222261,142.39291999)
\curveto(271.1571224,142.50156315)(271.31731176,142.60184914)(271.50279417,142.69377797)
\curveto(271.6967076,142.79406396)(271.91169858,142.87763562)(272.1477671,142.94449295)
\curveto(272.38383562,143.01135028)(272.63676618,143.04477895)(272.90655878,143.04477895)
\curveto(273.7243676,143.04477895)(274.32718543,142.85674271)(274.71501229,142.48067023)
\curveto(275.11127017,142.11295492)(275.30939911,141.58227487)(275.30939911,140.88863007)
\closepath
}
}
{
\newrgbcolor{curcolor}{0 0 0}
\pscustom[linewidth=0.56097835,linecolor=curcolor]
{
\newpath
\moveto(52.53490945,145.35529656)
\lineto(277.23287698,145.35529656)
\lineto(277.23287698,132.63617749)
\lineto(52.53490945,132.63617749)
\closepath
}
}
{
\newrgbcolor{curcolor}{0 0 0}
\pscustom[linewidth=0.51800942,linecolor=curcolor]
{
\newpath
\moveto(71.388224,132.65027633)
\lineto(71.388224,145.24967633)
}
}
{
\newrgbcolor{curcolor}{0 0 0}
\pscustom[linewidth=0.5154072,linecolor=curcolor]
{
\newpath
\moveto(90.268446,132.74459633)
\lineto(90.268446,145.21771633)
}
}
{
\newrgbcolor{curcolor}{0 0 0}
\pscustom[linewidth=0.51930571,linecolor=curcolor]
{
\newpath
\moveto(109.107736,132.74458633)
\lineto(109.107736,145.40711633)
}
}
{
\newrgbcolor{curcolor}{0 0 0}
\pscustom[linewidth=0.52573889,linecolor=curcolor]
{
\newpath
\moveto(128.036306,132.56601633)
\lineto(128.036306,145.54422633)
}
}
{
\newrgbcolor{curcolor}{0 0 0}
\pscustom[linewidth=0.51670998,linecolor=curcolor]
{
\newpath
\moveto(146.875586,132.74458633)
\lineto(146.875586,145.28084633)
}
}
{
\newrgbcolor{curcolor}{0 0 0}
\pscustom[linewidth=0.52059871,linecolor=curcolor]
{
\newpath
\moveto(165.625596,132.74459633)
\lineto(165.625596,145.47025633)
}
}
{
\newrgbcolor{curcolor}{0 0 0}
\pscustom[linewidth=0.51670998,linecolor=curcolor]
{
\newpath
\moveto(184.554166,132.65530633)
\lineto(184.554166,145.19156633)
}
}
{
\newrgbcolor{curcolor}{0 0 0}
\pscustom[linewidth=0.51670998,linecolor=curcolor]
{
\newpath
\moveto(203.661306,132.74459633)
\lineto(203.661306,145.28085633)
}
}
{
\newrgbcolor{curcolor}{0 0 0}
\pscustom[linewidth=0.52059871,linecolor=curcolor]
{
\newpath
\moveto(222.589876,132.74458633)
\lineto(222.589876,145.47025633)
}
}
{
\newrgbcolor{curcolor}{0 0 0}
\pscustom[linewidth=0.52573889,linecolor=curcolor]
{
\newpath
\moveto(258.240186,145.39636633)
\lineto(258.204686,132.41821633)
}
}
{
\newrgbcolor{curcolor}{0 1 0}
\pscustom[linestyle=none,fillstyle=solid,fillcolor=curcolor]
{
\newpath
\moveto(127.98238066,123.3893304)
\lineto(146.83034971,123.3893304)
\lineto(146.83034971,110.57301952)
\lineto(127.98238066,110.57301952)
\closepath
}
}
{
\newrgbcolor{curcolor}{0 0 0}
\pscustom[linestyle=none,fillstyle=solid,fillcolor=curcolor]
{
\newpath
\moveto(62.27863295,119.20543298)
\curveto(62.70861491,119.37257631)(63.12595033,119.58150546)(63.53063923,119.83222045)
\curveto(63.93532813,120.0912926)(64.31050846,120.41722208)(64.65618023,120.81000889)
\lineto(65.38967885,120.81000889)
\lineto(65.38967885,113.92788255)
\lineto(66.86932263,113.92788255)
\lineto(66.86932263,113.0503801)
\lineto(62.6580288,113.0503801)
\lineto(62.6580288,113.92788255)
\lineto(64.35266355,113.92788255)
\lineto(64.35266355,119.36839772)
\curveto(64.25992235,119.28482606)(64.1461036,119.19707582)(64.0112073,119.10514699)
\curveto(63.88474202,119.02157533)(63.7414147,118.93800367)(63.58122534,118.854432)
\curveto(63.42946701,118.77086034)(63.26927765,118.69146726)(63.10065728,118.61625277)
\curveto(62.9320369,118.54103827)(62.76763204,118.47835953)(62.60744268,118.42821653)
\lineto(62.27863295,119.20543298)
\closepath
}
}
{
\newrgbcolor{curcolor}{0 0 0}
\pscustom[linestyle=none,fillstyle=solid,fillcolor=curcolor]
{
\newpath
\moveto(76.32892465,117.81396482)
\curveto(77.45868116,117.77217898)(78.28070548,117.52564258)(78.79499762,117.07435561)
\curveto(79.30928976,116.62306864)(79.56643583,116.01717409)(79.56643583,115.25667197)
\curveto(79.56643583,114.91402815)(79.51163421,114.59645584)(79.40203097,114.30395502)
\curveto(79.29242772,114.01145421)(79.11959184,113.76073922)(78.88352332,113.55181007)
\curveto(78.65588581,113.34288091)(78.36080016,113.17991617)(77.99826635,113.06291585)
\curveto(77.64416357,112.94591552)(77.22261263,112.88741536)(76.73361355,112.88741536)
\curveto(76.5312691,112.88741536)(76.32892465,112.90412969)(76.1265802,112.93755835)
\curveto(75.93266677,112.96262985)(75.74718436,112.99605852)(75.57013297,113.03784435)
\curveto(75.39308158,113.07963018)(75.23710773,113.12141601)(75.10221143,113.16320184)
\curveto(74.96731513,113.21334484)(74.87035842,113.25513067)(74.81134129,113.28855933)
\lineto(75.01368574,114.17859753)
\curveto(75.14858204,114.1117402)(75.35514199,114.03234712)(75.63336561,113.94041829)
\curveto(75.92002025,113.84848947)(76.27833854,113.80252505)(76.70832049,113.80252505)
\curveto(77.04556124,113.80252505)(77.32800037,113.8401323)(77.55563787,113.9153468)
\curveto(77.78327538,113.99056129)(77.96454228,114.09084729)(78.09943858,114.21620478)
\curveto(78.2427659,114.34991944)(78.34393812,114.50034843)(78.40295525,114.66749175)
\curveto(78.4704034,114.83463507)(78.50412748,115.01013556)(78.50412748,115.19399322)
\curveto(78.50412748,115.47813687)(78.45354136,115.72885186)(78.35236914,115.94613818)
\curveto(78.25962793,116.17178166)(78.09100756,116.3598179)(77.84650802,116.51024689)
\curveto(77.61043949,116.66067588)(77.28162977,116.77349763)(76.86007883,116.84871212)
\curveto(76.44695891,116.93228378)(75.92423576,116.97406961)(75.29190935,116.97406961)
\curveto(75.34249547,117.34178493)(75.38043505,117.68442874)(75.40572811,118.00200105)
\curveto(75.43945218,118.32793053)(75.46474524,118.64132427)(75.48160727,118.94218225)
\curveto(75.50690033,119.2513974)(75.52376237,119.55643396)(75.53219339,119.85729194)
\curveto(75.54905542,120.15814993)(75.56591746,120.47572224)(75.5827795,120.81000889)
\lineto(79.35144486,120.81000889)
\lineto(79.35144486,119.93250644)
\lineto(76.49332952,119.93250644)
\curveto(76.4848985,119.81550611)(76.47225197,119.66089854)(76.45538993,119.46868372)
\curveto(76.44695891,119.28482606)(76.43431239,119.08843266)(76.41745035,118.8795035)
\curveto(76.40058831,118.67057435)(76.38372627,118.47000236)(76.36686424,118.27778754)
\curveto(76.3500022,118.08557272)(76.33735567,117.93096514)(76.32892465,117.81396482)
\closepath
}
}
{
\newrgbcolor{curcolor}{0 0 0}
\pscustom[linestyle=none,fillstyle=solid,fillcolor=curcolor]
{
\newpath
\moveto(81.24842502,119.20543298)
\curveto(81.67840697,119.37257631)(82.0957424,119.58150546)(82.5004313,119.83222045)
\curveto(82.90512019,120.0912926)(83.28030053,120.41722208)(83.62597229,120.81000889)
\lineto(84.35947092,120.81000889)
\lineto(84.35947092,113.92788255)
\lineto(85.8391147,113.92788255)
\lineto(85.8391147,113.0503801)
\lineto(81.62782086,113.0503801)
\lineto(81.62782086,113.92788255)
\lineto(83.32245562,113.92788255)
\lineto(83.32245562,119.36839772)
\curveto(83.22971441,119.28482606)(83.11589566,119.19707582)(82.98099936,119.10514699)
\curveto(82.85453408,119.02157533)(82.71120676,118.93800367)(82.55101741,118.854432)
\curveto(82.39925907,118.77086034)(82.23906972,118.69146726)(82.07044934,118.61625277)
\curveto(81.90182897,118.54103827)(81.7374241,118.47835953)(81.57723475,118.42821653)
\lineto(81.24842502,119.20543298)
\closepath
}
}
{
\newrgbcolor{curcolor}{0 0 0}
\pscustom[linestyle=none,fillstyle=solid,fillcolor=curcolor]
{
\newpath
\moveto(93.89495115,119.20543298)
\curveto(94.3249331,119.37257631)(94.74226853,119.58150546)(95.14695742,119.83222045)
\curveto(95.55164632,120.0912926)(95.92682665,120.41722208)(96.27249842,120.81000889)
\lineto(97.00599705,120.81000889)
\lineto(97.00599705,113.92788255)
\lineto(98.48564083,113.92788255)
\lineto(98.48564083,113.0503801)
\lineto(94.27434699,113.0503801)
\lineto(94.27434699,113.92788255)
\lineto(95.96898175,113.92788255)
\lineto(95.96898175,119.36839772)
\curveto(95.87624054,119.28482606)(95.76242179,119.19707582)(95.62752549,119.10514699)
\curveto(95.50106021,119.02157533)(95.35773289,118.93800367)(95.19754354,118.854432)
\curveto(95.0457852,118.77086034)(94.88559585,118.69146726)(94.71697547,118.61625277)
\curveto(94.5483551,118.54103827)(94.38395023,118.47835953)(94.22376088,118.42821653)
\lineto(93.89495115,119.20543298)
\closepath
}
}
{
\newrgbcolor{curcolor}{0 0 0}
\pscustom[linestyle=none,fillstyle=solid,fillcolor=curcolor]
{
\newpath
\moveto(99.72500153,115.73303044)
\curveto(99.86832885,116.06731709)(100.06224228,116.45174673)(100.30674182,116.88631937)
\curveto(100.55967239,117.32089201)(100.84211151,117.7680004)(101.1540592,118.22764454)
\curveto(101.4660069,118.69564585)(101.79903213,119.1511114)(102.15313492,119.59404121)
\curveto(102.5072377,120.04532818)(102.865556,120.45065074)(103.2280898,120.81000889)
\lineto(104.23981205,120.81000889)
\lineto(104.23981205,115.88345943)
\lineto(105.16300859,115.88345943)
\lineto(105.16300859,115.03102848)
\lineto(104.23981205,115.03102848)
\lineto(104.23981205,113.0503801)
\lineto(103.2280898,113.0503801)
\lineto(103.2280898,115.03102848)
\lineto(99.72500153,115.03102848)
\lineto(99.72500153,115.73303044)
\closepath
\moveto(103.2280898,119.58150546)
\curveto(103.0004523,119.33914764)(102.76859928,119.07171832)(102.53253076,118.77921751)
\curveto(102.30489326,118.48671669)(102.08147126,118.17750154)(101.86226477,117.85157206)
\curveto(101.64305829,117.53399975)(101.43649833,117.20807027)(101.2425849,116.87378362)
\curveto(101.05710249,116.53949697)(100.88848211,116.20938891)(100.73672378,115.88345943)
\lineto(103.2280898,115.88345943)
\lineto(103.2280898,119.58150546)
\closepath
}
}
{
\newrgbcolor{curcolor}{0 0 0}
\pscustom[linestyle=none,fillstyle=solid,fillcolor=curcolor]
{
\newpath
\moveto(117.22779826,118.8293605)
\curveto(117.22779826,118.56193119)(117.17299664,118.30285904)(117.0633934,118.05214405)
\curveto(116.96222117,117.80142907)(116.82310937,117.55489266)(116.64605797,117.31253484)
\curveto(116.4774376,117.07017703)(116.28352417,116.83199779)(116.06431768,116.59799714)
\curveto(115.8451112,116.36399648)(115.6216892,116.13417441)(115.3940517,115.90853093)
\curveto(115.26758642,115.78317344)(115.12004359,115.63274444)(114.95142322,115.45724395)
\curveto(114.78280284,115.28174347)(114.62261349,115.10206439)(114.47085515,114.91820674)
\curveto(114.31909681,114.73434908)(114.19263153,114.55467001)(114.09145931,114.37916952)
\curveto(113.99028708,114.20366903)(113.93970097,114.05324004)(113.93970097,113.92788255)
\lineto(117.54396146,113.92788255)
\lineto(117.54396146,113.0503801)
\lineto(112.80151345,113.0503801)
\curveto(112.79308243,113.09216593)(112.78886692,113.13395176)(112.78886692,113.17573759)
\lineto(112.78886692,113.31363083)
\curveto(112.78886692,113.66463181)(112.84788405,113.99056129)(112.96591831,114.29141927)
\curveto(113.08395257,114.59227726)(113.23571091,114.87642091)(113.42119332,115.14385022)
\curveto(113.60667573,115.41127954)(113.81323569,115.66199453)(114.0408732,115.89599518)
\curveto(114.27694172,116.138353)(114.50879473,116.37235365)(114.73643224,116.59799714)
\curveto(114.92191465,116.78185479)(115.09896604,116.96153387)(115.26758642,117.13703436)
\curveto(115.44463781,117.31253484)(115.59639615,117.48803533)(115.72286143,117.66353582)
\curveto(115.85775773,117.83903631)(115.96314546,118.01871539)(116.03902463,118.20257304)
\curveto(116.12333481,118.39478786)(116.16548991,118.59118127)(116.16548991,118.79175326)
\curveto(116.16548991,119.01739674)(116.12755032,119.20961157)(116.05167116,119.36839772)
\curveto(115.98422301,119.52718388)(115.88726629,119.65671996)(115.76080101,119.75700595)
\curveto(115.64276675,119.86564911)(115.50365494,119.94504219)(115.34346559,119.99518519)
\curveto(115.18327623,120.04532818)(115.01044035,120.07039968)(114.82495794,120.07039968)
\curveto(114.60575145,120.07039968)(114.403407,120.0411496)(114.21792459,119.98264944)
\curveto(114.0408732,119.92414927)(113.88068384,119.85311336)(113.73735652,119.7695417)
\curveto(113.60246022,119.68597004)(113.48442596,119.60239838)(113.38325374,119.51882671)
\curveto(113.28208151,119.44361222)(113.20620235,119.38093347)(113.15561623,119.33079048)
\lineto(112.63710858,120.05786393)
\curveto(112.70455673,120.13307843)(112.80572896,120.22500726)(112.94062526,120.33365042)
\curveto(113.07552156,120.44229358)(113.23571091,120.54257957)(113.42119332,120.6345084)
\curveto(113.61510675,120.73479439)(113.83009773,120.81836605)(114.06616625,120.88522338)
\curveto(114.30223478,120.95208071)(114.55516534,120.98550938)(114.82495794,120.98550938)
\curveto(115.64276675,120.98550938)(116.24558459,120.79747314)(116.63341145,120.42140066)
\curveto(117.02966932,120.05368535)(117.22779826,119.5230053)(117.22779826,118.8293605)
\closepath
}
}
{
\newrgbcolor{curcolor}{0 0 0}
\pscustom[linestyle=none,fillstyle=solid,fillcolor=curcolor]
{
\newpath
\moveto(122.1346521,117.07435561)
\curveto(122.1346521,116.85706929)(122.06720395,116.66903305)(121.93230765,116.51024689)
\curveto(121.80584237,116.35146074)(121.637222,116.27206766)(121.42644653,116.27206766)
\curveto(121.20724005,116.27206766)(121.03018865,116.35146074)(120.89529235,116.51024689)
\curveto(120.76039606,116.66903305)(120.69294791,116.85706929)(120.69294791,117.07435561)
\curveto(120.69294791,117.29164193)(120.76039606,117.48385675)(120.89529235,117.65100007)
\curveto(121.03018865,117.8181434)(121.20724005,117.90171506)(121.42644653,117.90171506)
\curveto(121.637222,117.90171506)(121.80584237,117.8181434)(121.93230765,117.65100007)
\curveto(122.06720395,117.48385675)(122.1346521,117.29164193)(122.1346521,117.07435561)
\closepath
\moveto(118.82126176,116.93646237)
\curveto(118.82126176,118.24018029)(119.04468375,119.23886165)(119.49152774,119.93250644)
\curveto(119.94680275,120.6345084)(120.58334466,120.98550938)(121.40115348,120.98550938)
\curveto(122.22739331,120.98550938)(122.86393522,120.6345084)(123.31077921,119.93250644)
\curveto(123.7576232,119.23886165)(123.9810452,118.24018029)(123.9810452,116.93646237)
\curveto(123.9810452,115.63274444)(123.7576232,114.6298845)(123.31077921,113.92788255)
\curveto(122.86393522,113.23423775)(122.22739331,112.88741536)(121.40115348,112.88741536)
\curveto(120.58334466,112.88741536)(119.94680275,113.23423775)(119.49152774,113.92788255)
\curveto(119.04468375,114.6298845)(118.82126176,115.63274444)(118.82126176,116.93646237)
\closepath
\moveto(122.91873684,116.93646237)
\curveto(122.91873684,117.36267784)(122.89344379,117.76382182)(122.84285767,118.1398943)
\curveto(122.79227156,118.52432394)(122.70796137,118.85861059)(122.58992711,119.14275424)
\curveto(122.47189285,119.42689789)(122.315919,119.65254137)(122.12200557,119.8196847)
\curveto(121.92809214,119.98682802)(121.68780811,120.07039968)(121.40115348,120.07039968)
\curveto(121.11449884,120.07039968)(120.87421481,119.98682802)(120.68030138,119.8196847)
\curveto(120.48638795,119.65254137)(120.3304141,119.42689789)(120.21237984,119.14275424)
\curveto(120.09434558,118.85861059)(120.01003539,118.52432394)(119.95944928,118.1398943)
\curveto(119.90886317,117.76382182)(119.88357011,117.36267784)(119.88357011,116.93646237)
\curveto(119.88357011,116.51024689)(119.90886317,116.10492433)(119.95944928,115.72049469)
\curveto(120.01003539,115.34442221)(120.09434558,115.01431415)(120.21237984,114.7301705)
\curveto(120.3304141,114.44602685)(120.48638795,114.22038336)(120.68030138,114.05324004)
\curveto(120.87421481,113.88609671)(121.11449884,113.80252505)(121.40115348,113.80252505)
\curveto(121.68780811,113.80252505)(121.92809214,113.88609671)(122.12200557,114.05324004)
\curveto(122.315919,114.22038336)(122.47189285,114.44602685)(122.58992711,114.7301705)
\curveto(122.70796137,115.01431415)(122.79227156,115.34442221)(122.84285767,115.72049469)
\curveto(122.89344379,116.10492433)(122.91873684,116.51024689)(122.91873684,116.93646237)
\closepath
}
}
{
\newrgbcolor{curcolor}{0 0 0}
\pscustom[linestyle=none,fillstyle=solid,fillcolor=curcolor]
{
\newpath
\moveto(131.83453337,119.20543298)
\curveto(132.26451532,119.37257631)(132.68185074,119.58150546)(133.08653964,119.83222045)
\curveto(133.49122854,120.0912926)(133.86640887,120.41722208)(134.21208064,120.81000889)
\lineto(134.94557926,120.81000889)
\lineto(134.94557926,113.92788255)
\lineto(136.42522305,113.92788255)
\lineto(136.42522305,113.0503801)
\lineto(132.21392921,113.0503801)
\lineto(132.21392921,113.92788255)
\lineto(133.90856396,113.92788255)
\lineto(133.90856396,119.36839772)
\curveto(133.81582276,119.28482606)(133.70200401,119.19707582)(133.56710771,119.10514699)
\curveto(133.44064243,119.02157533)(133.29731511,118.93800367)(133.13712575,118.854432)
\curveto(132.98536742,118.77086034)(132.82517806,118.69146726)(132.65655769,118.61625277)
\curveto(132.48793731,118.54103827)(132.32353245,118.47835953)(132.16334309,118.42821653)
\lineto(131.83453337,119.20543298)
\closepath
}
}
{
\newrgbcolor{curcolor}{0 0 0}
\pscustom[linestyle=none,fillstyle=solid,fillcolor=curcolor]
{
\newpath
\moveto(141.10443938,117.07435561)
\curveto(141.10443938,116.85706929)(141.03699123,116.66903305)(140.90209493,116.51024689)
\curveto(140.77562965,116.35146074)(140.60700928,116.27206766)(140.39623381,116.27206766)
\curveto(140.17702732,116.27206766)(139.99997593,116.35146074)(139.86507963,116.51024689)
\curveto(139.73018333,116.66903305)(139.66273518,116.85706929)(139.66273518,117.07435561)
\curveto(139.66273518,117.29164193)(139.73018333,117.48385675)(139.86507963,117.65100007)
\curveto(139.99997593,117.8181434)(140.17702732,117.90171506)(140.39623381,117.90171506)
\curveto(140.60700928,117.90171506)(140.77562965,117.8181434)(140.90209493,117.65100007)
\curveto(141.03699123,117.48385675)(141.10443938,117.29164193)(141.10443938,117.07435561)
\closepath
\moveto(137.79104903,116.93646237)
\curveto(137.79104903,118.24018029)(138.01447103,119.23886165)(138.46131502,119.93250644)
\curveto(138.91659003,120.6345084)(139.55313194,120.98550938)(140.37094075,120.98550938)
\curveto(141.19718059,120.98550938)(141.8337225,120.6345084)(142.28056649,119.93250644)
\curveto(142.72741048,119.23886165)(142.95083247,118.24018029)(142.95083247,116.93646237)
\curveto(142.95083247,115.63274444)(142.72741048,114.6298845)(142.28056649,113.92788255)
\curveto(141.8337225,113.23423775)(141.19718059,112.88741536)(140.37094075,112.88741536)
\curveto(139.55313194,112.88741536)(138.91659003,113.23423775)(138.46131502,113.92788255)
\curveto(138.01447103,114.6298845)(137.79104903,115.63274444)(137.79104903,116.93646237)
\closepath
\moveto(141.88852412,116.93646237)
\curveto(141.88852412,117.36267784)(141.86323106,117.76382182)(141.81264495,118.1398943)
\curveto(141.76205884,118.52432394)(141.67774865,118.85861059)(141.55971439,119.14275424)
\curveto(141.44168013,119.42689789)(141.28570628,119.65254137)(141.09179285,119.8196847)
\curveto(140.89787942,119.98682802)(140.65759539,120.07039968)(140.37094075,120.07039968)
\curveto(140.08428612,120.07039968)(139.84400208,119.98682802)(139.65008865,119.8196847)
\curveto(139.45617522,119.65254137)(139.30020138,119.42689789)(139.18216712,119.14275424)
\curveto(139.06413285,118.85861059)(138.97982267,118.52432394)(138.92923656,118.1398943)
\curveto(138.87865044,117.76382182)(138.85335739,117.36267784)(138.85335739,116.93646237)
\curveto(138.85335739,116.51024689)(138.87865044,116.10492433)(138.92923656,115.72049469)
\curveto(138.97982267,115.34442221)(139.06413285,115.01431415)(139.18216712,114.7301705)
\curveto(139.30020138,114.44602685)(139.45617522,114.22038336)(139.65008865,114.05324004)
\curveto(139.84400208,113.88609671)(140.08428612,113.80252505)(140.37094075,113.80252505)
\curveto(140.65759539,113.80252505)(140.89787942,113.88609671)(141.09179285,114.05324004)
\curveto(141.28570628,114.22038336)(141.44168013,114.44602685)(141.55971439,114.7301705)
\curveto(141.67774865,115.01431415)(141.76205884,115.34442221)(141.81264495,115.72049469)
\curveto(141.86323106,116.10492433)(141.88852412,116.51024689)(141.88852412,116.93646237)
\closepath
}
}
{
\newrgbcolor{curcolor}{0 0 0}
\pscustom[linestyle=none,fillstyle=solid,fillcolor=curcolor]
{
\newpath
\moveto(156.8620162,116.15924591)
\curveto(156.8620162,116.91139087)(156.96318842,117.57578558)(157.16553287,118.15243004)
\curveto(157.37630834,118.73743168)(157.67139399,119.2263259)(158.05078983,119.61911271)
\curveto(158.43861669,120.01189952)(158.90653823,120.3127575)(159.45455445,120.52168665)
\curveto(160.00257066,120.73061581)(160.62225053,120.83925897)(161.31359407,120.84761614)
\lineto(161.40211976,119.97011369)
\curveto(160.95527577,119.96175652)(160.54637137,119.91161352)(160.17540654,119.8196847)
\curveto(159.81287274,119.73611303)(159.48827852,119.59404121)(159.20162388,119.39346922)
\curveto(158.91496925,119.2012544)(158.67468522,118.95053941)(158.48077179,118.64132427)
\curveto(158.28685836,118.33210912)(158.13931553,117.95185806)(158.0381433,117.50057108)
\curveto(158.24048775,117.59249991)(158.45969424,117.66771441)(158.69576276,117.72621457)
\curveto(158.9402623,117.78471473)(159.19740837,117.81396482)(159.46720097,117.81396482)
\curveto(159.90561395,117.81396482)(160.27657877,117.74710749)(160.58009544,117.61339283)
\curveto(160.89204313,117.47967817)(161.14075818,117.2999991)(161.3262406,117.07435561)
\curveto(161.51172301,116.85706929)(161.64661931,116.60217572)(161.73092949,116.3096749)
\curveto(161.81523968,116.01717409)(161.85739477,115.71631611)(161.85739477,115.40710096)
\curveto(161.85739477,115.12295731)(161.81102417,114.83045649)(161.71828297,114.52959851)
\curveto(161.62554176,114.23709769)(161.48221444,113.96548979)(161.28830101,113.71477481)
\curveto(161.09438758,113.47241699)(160.84567253,113.271845)(160.54215586,113.11305884)
\curveto(160.23863918,112.96262985)(159.87610538,112.88741536)(159.45455445,112.88741536)
\curveto(158.58615952,112.88741536)(157.93697108,113.17573759)(157.50698913,113.75238206)
\curveto(157.07700717,114.32902652)(156.8620162,115.13131447)(156.8620162,116.15924591)
\closepath
\moveto(159.35338222,116.96153387)
\curveto(159.08358962,116.96153387)(158.83487457,116.93646237)(158.60723707,116.88631937)
\curveto(158.38803058,116.83617637)(158.16460858,116.76096188)(157.93697108,116.66067588)
\curveto(157.92854006,116.57710422)(157.92432455,116.49353256)(157.92432455,116.4099609)
\lineto(157.92432455,116.15924591)
\curveto(157.92432455,115.83331643)(157.9454021,115.52410128)(157.98755719,115.23160047)
\curveto(158.0381433,114.94745682)(158.11823798,114.69256325)(158.22784122,114.46691976)
\curveto(158.34587549,114.24963344)(158.50184933,114.07413295)(158.69576276,113.94041829)
\curveto(158.88967619,113.8150608)(159.13839124,113.75238206)(159.44190792,113.75238206)
\curveto(159.69483848,113.75238206)(159.90561395,113.80252505)(160.07423432,113.90281105)
\curveto(160.24285469,114.01145421)(160.3819665,114.14516887)(160.49156974,114.30395502)
\curveto(160.60117299,114.46274118)(160.67705216,114.63824167)(160.71920725,114.83045649)
\curveto(160.76979336,115.03102848)(160.79508642,115.21906472)(160.79508642,115.39456521)
\curveto(160.79508642,115.90435234)(160.67705216,116.29296057)(160.44098363,116.56038989)
\curveto(160.21334613,116.82781921)(159.85081232,116.96153387)(159.35338222,116.96153387)
\closepath
}
}
{
\newrgbcolor{curcolor}{0 0 0}
\pscustom[linestyle=none,fillstyle=solid,fillcolor=curcolor]
{
\newpath
\moveto(177.10911047,113.0503801)
\curveto(177.15126556,113.6437389)(177.2566533,114.27052636)(177.42527367,114.93074249)
\curveto(177.60232506,115.59931578)(177.81310053,116.24281757)(178.05760007,116.86124787)
\curveto(178.30209961,117.48803533)(178.57189221,118.0646798)(178.86697787,118.59118127)
\curveto(179.16206352,119.1260399)(179.45293367,119.56479113)(179.7395883,119.90743494)
\lineto(175.94562989,119.90743494)
\lineto(175.94562989,120.81000889)
\lineto(180.90306888,120.81000889)
\lineto(180.90306888,119.94504219)
\curveto(180.65013832,119.65254137)(180.37613021,119.25975456)(180.08104456,118.76668176)
\curveto(179.7859589,118.27360895)(179.50351978,117.7178574)(179.23372718,117.09942711)
\curveto(178.9723656,116.48935398)(178.7447281,115.83331643)(178.55081467,115.13131447)
\curveto(178.35690124,114.43766968)(178.23465146,113.74402489)(178.18406535,113.0503801)
\lineto(177.10911047,113.0503801)
\closepath
}
}
{
\newrgbcolor{curcolor}{0 0 0}
\pscustom[linestyle=none,fillstyle=solid,fillcolor=curcolor]
{
\newpath
\moveto(199.79698465,115.06863573)
\curveto(199.79698465,114.41677677)(199.58620919,113.8902753)(199.16465825,113.48913132)
\curveto(198.75153833,113.08798735)(198.11921193,112.88741536)(197.26767904,112.88741536)
\curveto(196.77867996,112.88741536)(196.37399106,112.9500941)(196.05361235,113.0754516)
\curveto(195.73323364,113.20916625)(195.47608757,113.37630958)(195.28217414,113.57688157)
\curveto(195.09669173,113.78581072)(194.96179543,114.01563279)(194.87748524,114.26634778)
\curveto(194.80160608,114.51706276)(194.76366649,114.76359916)(194.76366649,115.00595698)
\curveto(194.76366649,115.44888679)(194.88591626,115.8416736)(195.1304158,116.18431741)
\curveto(195.37491535,116.52696122)(195.66578549,116.80274771)(196.00302624,117.01167686)
\curveto(195.28638965,117.41282084)(194.92807136,118.02707255)(194.92807136,118.854432)
\curveto(194.92807136,119.13857565)(194.98287298,119.41018355)(195.09247622,119.66925571)
\curveto(195.20207946,119.92832786)(195.35805331,120.15397134)(195.56039776,120.34618616)
\curveto(195.76274221,120.53840099)(196.00724175,120.69300856)(196.29389638,120.81000889)
\curveto(196.58898204,120.92700921)(196.91779177,120.98550938)(197.28032557,120.98550938)
\curveto(197.70187651,120.98550938)(198.0601948,120.92283063)(198.35528046,120.79747314)
\curveto(198.65879713,120.67211565)(198.90329667,120.50915091)(199.08877908,120.30857892)
\curveto(199.27426149,120.1163641)(199.40915779,119.89907777)(199.49346798,119.65671996)
\curveto(199.57777817,119.4227193)(199.61993326,119.19289723)(199.61993326,118.96725375)
\curveto(199.61993326,118.52432394)(199.50611451,118.1398943)(199.278477,117.81396482)
\curveto(199.0508395,117.4963925)(198.78947792,117.24149893)(198.49439226,117.04928411)
\curveto(199.36278719,116.63978297)(199.79698465,115.97956684)(199.79698465,115.06863573)
\closepath
\moveto(195.77538874,114.99342123)
\curveto(195.77538874,114.85970657)(195.80068179,114.71763475)(195.8512679,114.56720576)
\curveto(195.90185402,114.42513393)(195.9861642,114.29141927)(196.10419846,114.16606178)
\curveto(196.22223273,114.04070429)(196.37820657,113.93623971)(196.57212,113.85266805)
\curveto(196.76603343,113.77745355)(197.00210196,113.73984631)(197.28032557,113.73984631)
\curveto(197.54168715,113.73984631)(197.76510915,113.77327497)(197.95059156,113.8401323)
\curveto(198.14450499,113.9153468)(198.30047883,114.01145421)(198.4185131,114.12845453)
\curveto(198.54497838,114.25381203)(198.63771958,114.39170527)(198.69673671,114.54213426)
\curveto(198.75575384,114.69256325)(198.78526241,114.84299224)(198.78526241,114.99342123)
\curveto(198.78526241,115.4697797)(198.61242653,115.83331643)(198.26675476,116.08403142)
\curveto(197.92108299,116.34310357)(197.44473044,116.53949697)(196.83769709,116.67321163)
\curveto(196.50045634,116.48935398)(196.23909476,116.25953191)(196.05361235,115.98374542)
\curveto(195.86812994,115.70795894)(195.77538874,115.37785088)(195.77538874,114.99342123)
\closepath
\moveto(198.60821102,118.96725375)
\curveto(198.60821102,119.07589691)(198.58291796,119.19707582)(198.53233185,119.33079048)
\curveto(198.48174574,119.4728623)(198.40165106,119.59821979)(198.29204782,119.70686295)
\curveto(198.19087559,119.82386328)(198.05597929,119.91997069)(197.88735892,119.99518519)
\curveto(197.71873854,120.07875685)(197.5163941,120.12054268)(197.28032557,120.12054268)
\curveto(197.03582603,120.12054268)(196.82926607,120.08293543)(196.6606457,120.00772093)
\curveto(196.50045634,119.93250644)(196.36556004,119.83639903)(196.2559568,119.7193987)
\curveto(196.14635356,119.61075554)(196.06625888,119.48539805)(196.01567277,119.34332622)
\curveto(195.97351767,119.20961157)(195.95244013,119.07589691)(195.95244013,118.94218225)
\curveto(195.95244013,118.59953844)(196.0746899,118.27778754)(196.31918944,117.97692956)
\curveto(196.57212,117.68442874)(196.98523992,117.471321)(197.55854919,117.33760634)
\curveto(197.8789279,117.521464)(198.13185846,117.73875032)(198.31734087,117.9894653)
\curveto(198.5112543,118.24018029)(198.60821102,118.56610977)(198.60821102,118.96725375)
\closepath
}
}
{
\newrgbcolor{curcolor}{0 0 0}
\pscustom[linestyle=none,fillstyle=solid,fillcolor=curcolor]
{
\newpath
\moveto(218.74147887,117.70114307)
\curveto(218.74147887,116.15506733)(218.36208303,114.98924265)(217.60329135,114.20366903)
\curveto(216.84449967,113.42645258)(215.70209663,113.03366577)(214.17608225,113.0253086)
\lineto(214.13814267,113.90281105)
\curveto(215.08241676,113.90281105)(215.84120844,114.0866687)(216.41451771,114.45438401)
\curveto(216.996258,114.83045649)(217.37986935,115.47395829)(217.56535177,116.3848894)
\curveto(217.36300732,116.29296057)(217.13958532,116.21774608)(216.89508578,116.15924591)
\curveto(216.65058624,116.10910292)(216.39344017,116.08403142)(216.12364757,116.08403142)
\curveto(215.67680358,116.08403142)(215.30162325,116.14671016)(214.99810657,116.27206766)
\curveto(214.6945899,116.40578232)(214.45009036,116.5812828)(214.26460795,116.79856913)
\curveto(214.07912553,117.02421261)(213.94422924,117.27910618)(213.85991905,117.56324983)
\curveto(213.77560886,117.84739348)(213.73345377,118.14825146)(213.73345377,118.46582378)
\curveto(213.73345377,118.74996743)(213.77982437,119.03828966)(213.87256558,119.33079048)
\curveto(213.96530678,119.63164846)(214.1086341,119.90325636)(214.30254753,120.14561418)
\curveto(214.49646096,120.387972)(214.74517601,120.58854398)(215.04869269,120.74733014)
\curveto(215.35220936,120.9061163)(215.71474316,120.98550938)(216.1362941,120.98550938)
\curveto(216.996258,120.98550938)(217.64544644,120.69300856)(218.08385942,120.10800693)
\curveto(218.52227239,119.5230053)(218.74147887,118.72071734)(218.74147887,117.70114307)
\closepath
\moveto(216.23746632,116.93646237)
\curveto(216.50725892,116.93646237)(216.75597397,116.96153387)(216.98361148,117.01167686)
\curveto(217.21968,117.06181986)(217.4473175,117.13285577)(217.66652399,117.2247846)
\curveto(217.67495501,117.30835626)(217.67917052,117.38774934)(217.67917052,117.46296384)
\lineto(217.67917052,117.70114307)
\curveto(217.67917052,118.02707255)(217.65387746,118.3362877)(217.60329135,118.62878852)
\curveto(217.56113626,118.92128933)(217.48104158,119.1761829)(217.36300732,119.39346922)
\curveto(217.25340407,119.61075554)(217.09743023,119.78207745)(216.89508578,119.90743494)
\curveto(216.70117235,120.0411496)(216.4524573,120.10800693)(216.14894062,120.10800693)
\curveto(215.89601006,120.10800693)(215.6852346,120.05368535)(215.51661422,119.94504219)
\curveto(215.34799385,119.84475619)(215.20888204,119.71522012)(215.0992788,119.55643396)
\curveto(214.98967555,119.40600497)(214.90958088,119.23468306)(214.85899476,119.04246824)
\curveto(214.81683967,118.85025342)(214.79576212,118.66639576)(214.79576212,118.49089527)
\curveto(214.79576212,117.98110814)(214.90958088,117.59249991)(215.13721838,117.32507059)
\curveto(215.3732869,117.06599844)(215.74003622,116.93646237)(216.23746632,116.93646237)
\closepath
}
}
{
\newrgbcolor{curcolor}{0 0 0}
\pscustom[linestyle=none,fillstyle=solid,fillcolor=curcolor]
{
\newpath
\moveto(236.1557432,113.85266805)
\curveto(236.1557432,113.60195306)(236.07143301,113.38048816)(235.90281264,113.18827334)
\curveto(235.73419227,112.99605852)(235.51077027,112.89995111)(235.23254665,112.89995111)
\curveto(234.94589202,112.89995111)(234.71825451,112.99605852)(234.54963414,113.18827334)
\curveto(234.38101377,113.38048816)(234.29670358,113.60195306)(234.29670358,113.85266805)
\curveto(234.29670358,114.1117402)(234.38101377,114.33738369)(234.54963414,114.52959851)
\curveto(234.71825451,114.72181333)(234.94589202,114.81792074)(235.23254665,114.81792074)
\curveto(235.51077027,114.81792074)(235.73419227,114.72181333)(235.90281264,114.52959851)
\curveto(236.07143301,114.33738369)(236.1557432,114.1117402)(236.1557432,113.85266805)
\closepath
}
}
{
\newrgbcolor{curcolor}{0 0 0}
\pscustom[linestyle=none,fillstyle=solid,fillcolor=curcolor]
{
\newpath
\moveto(242.47901584,113.85266805)
\curveto(242.47901584,113.60195306)(242.39470566,113.38048816)(242.22608528,113.18827334)
\curveto(242.05746491,112.99605852)(241.83404291,112.89995111)(241.5558193,112.89995111)
\curveto(241.26916466,112.89995111)(241.04152716,112.99605852)(240.87290678,113.18827334)
\curveto(240.70428641,113.38048816)(240.61997622,113.60195306)(240.61997622,113.85266805)
\curveto(240.61997622,114.1117402)(240.70428641,114.33738369)(240.87290678,114.52959851)
\curveto(241.04152716,114.72181333)(241.26916466,114.81792074)(241.5558193,114.81792074)
\curveto(241.83404291,114.81792074)(242.05746491,114.72181333)(242.22608528,114.52959851)
\curveto(242.39470566,114.33738369)(242.47901584,114.1117402)(242.47901584,113.85266805)
\closepath
}
}
{
\newrgbcolor{curcolor}{0 0 0}
\pscustom[linestyle=none,fillstyle=solid,fillcolor=curcolor]
{
\newpath
\moveto(248.80227316,113.85266805)
\curveto(248.80227316,113.60195306)(248.71796297,113.38048816)(248.5493426,113.18827334)
\curveto(248.38072223,112.99605852)(248.15730023,112.89995111)(247.87907661,112.89995111)
\curveto(247.59242198,112.89995111)(247.36478447,112.99605852)(247.1961641,113.18827334)
\curveto(247.02754373,113.38048816)(246.94323354,113.60195306)(246.94323354,113.85266805)
\curveto(246.94323354,114.1117402)(247.02754373,114.33738369)(247.1961641,114.52959851)
\curveto(247.36478447,114.72181333)(247.59242198,114.81792074)(247.87907661,114.81792074)
\curveto(248.15730023,114.81792074)(248.38072223,114.72181333)(248.5493426,114.52959851)
\curveto(248.71796297,114.33738369)(248.80227316,114.1117402)(248.80227316,113.85266805)
\closepath
}
}
{
\newrgbcolor{curcolor}{0 0 0}
\pscustom[linestyle=none,fillstyle=solid,fillcolor=curcolor]
{
\newpath
\moveto(266.02683957,117.81396482)
\curveto(267.15659607,117.77217898)(267.9786204,117.52564258)(268.49291254,117.07435561)
\curveto(269.00720468,116.62306864)(269.26435075,116.01717409)(269.26435075,115.25667197)
\curveto(269.26435075,114.91402815)(269.20954913,114.59645584)(269.09994588,114.30395502)
\curveto(268.99034264,114.01145421)(268.81750676,113.76073922)(268.58143823,113.55181007)
\curveto(268.35380073,113.34288091)(268.05871507,113.17991617)(267.69618127,113.06291585)
\curveto(267.34207848,112.94591552)(266.92052755,112.88741536)(266.43152847,112.88741536)
\curveto(266.22918402,112.88741536)(266.02683957,112.90412969)(265.82449512,112.93755835)
\curveto(265.63058169,112.96262985)(265.44509928,112.99605852)(265.26804789,113.03784435)
\curveto(265.09099649,113.07963018)(264.93502265,113.12141601)(264.80012635,113.16320184)
\curveto(264.66523005,113.21334484)(264.56827333,113.25513067)(264.5092562,113.28855933)
\lineto(264.71160065,114.17859753)
\curveto(264.84649695,114.1117402)(265.05305691,114.03234712)(265.33128053,113.94041829)
\curveto(265.61793516,113.84848947)(265.97625346,113.80252505)(266.40623541,113.80252505)
\curveto(266.74347616,113.80252505)(267.02591528,113.8401323)(267.25355279,113.9153468)
\curveto(267.48119029,113.99056129)(267.66245719,114.09084729)(267.79735349,114.21620478)
\curveto(267.94068081,114.34991944)(268.04185304,114.50034843)(268.10087017,114.66749175)
\curveto(268.16831832,114.83463507)(268.20204239,115.01013556)(268.20204239,115.19399322)
\curveto(268.20204239,115.47813687)(268.15145628,115.72885186)(268.05028406,115.94613818)
\curveto(267.95754285,116.17178166)(267.78892248,116.3598179)(267.54442293,116.51024689)
\curveto(267.30835441,116.66067588)(266.97954468,116.77349763)(266.55799375,116.84871212)
\curveto(266.14487383,116.93228378)(265.62215067,116.97406961)(264.98982427,116.97406961)
\curveto(265.04041038,117.34178493)(265.07834996,117.68442874)(265.10364302,118.00200105)
\curveto(265.1373671,118.32793053)(265.16266015,118.64132427)(265.17952219,118.94218225)
\curveto(265.20481525,119.2513974)(265.22167728,119.55643396)(265.2301083,119.85729194)
\curveto(265.24697034,120.15814993)(265.26383238,120.47572224)(265.28069441,120.81000889)
\lineto(269.04935977,120.81000889)
\lineto(269.04935977,119.93250644)
\lineto(266.19124443,119.93250644)
\curveto(266.18281341,119.81550611)(266.17016689,119.66089854)(266.15330485,119.46868372)
\curveto(266.14487383,119.28482606)(266.1322273,119.08843266)(266.11536526,118.8795035)
\curveto(266.09850323,118.67057435)(266.08164119,118.47000236)(266.06477915,118.27778754)
\curveto(266.04791711,118.08557272)(266.03527059,117.93096514)(266.02683957,117.81396482)
\closepath
}
}
{
\newrgbcolor{curcolor}{0 0 0}
\pscustom[linestyle=none,fillstyle=solid,fillcolor=curcolor]
{
\newpath
\moveto(275.30939977,118.8293605)
\curveto(275.30939977,118.56193119)(275.25459815,118.30285904)(275.14499491,118.05214405)
\curveto(275.04382268,117.80142907)(274.90471088,117.55489266)(274.72765948,117.31253484)
\curveto(274.55903911,117.07017703)(274.36512568,116.83199779)(274.14591919,116.59799714)
\curveto(273.92671271,116.36399648)(273.70329071,116.13417441)(273.47565321,115.90853093)
\curveto(273.34918793,115.78317344)(273.2016451,115.63274444)(273.03302473,115.45724395)
\curveto(272.86440435,115.28174347)(272.704215,115.10206439)(272.55245666,114.91820674)
\curveto(272.40069832,114.73434908)(272.27423304,114.55467001)(272.17306082,114.37916952)
\curveto(272.07188859,114.20366903)(272.02130248,114.05324004)(272.02130248,113.92788255)
\lineto(275.62556297,113.92788255)
\lineto(275.62556297,113.0503801)
\lineto(270.88311496,113.0503801)
\curveto(270.87468394,113.09216593)(270.87046843,113.13395176)(270.87046843,113.17573759)
\lineto(270.87046843,113.31363083)
\curveto(270.87046843,113.66463181)(270.92948556,113.99056129)(271.04751982,114.29141927)
\curveto(271.16555408,114.59227726)(271.31731242,114.87642091)(271.50279483,115.14385022)
\curveto(271.68827724,115.41127954)(271.8948372,115.66199453)(272.12247471,115.89599518)
\curveto(272.35854323,116.138353)(272.59039624,116.37235365)(272.81803375,116.59799714)
\curveto(273.00351616,116.78185479)(273.18056755,116.96153387)(273.34918793,117.13703436)
\curveto(273.52623932,117.31253484)(273.67799766,117.48803533)(273.80446294,117.66353582)
\curveto(273.93935924,117.83903631)(274.04474697,118.01871539)(274.12062614,118.20257304)
\curveto(274.20493632,118.39478786)(274.24709142,118.59118127)(274.24709142,118.79175326)
\curveto(274.24709142,119.01739674)(274.20915183,119.20961157)(274.13327267,119.36839772)
\curveto(274.06582452,119.52718388)(273.9688678,119.65671996)(273.84240252,119.75700595)
\curveto(273.72436826,119.86564911)(273.58525645,119.94504219)(273.42506709,119.99518519)
\curveto(273.26487774,120.04532818)(273.09204186,120.07039968)(272.90655944,120.07039968)
\curveto(272.68735296,120.07039968)(272.48500851,120.0411496)(272.2995261,119.98264944)
\curveto(272.12247471,119.92414927)(271.96228535,119.85311336)(271.81895803,119.7695417)
\curveto(271.68406173,119.68597004)(271.56602747,119.60239838)(271.46485525,119.51882671)
\curveto(271.36368302,119.44361222)(271.28780386,119.38093347)(271.23721774,119.33079048)
\lineto(270.71871009,120.05786393)
\curveto(270.78615824,120.13307843)(270.88733047,120.22500726)(271.02222677,120.33365042)
\curveto(271.15712307,120.44229358)(271.31731242,120.54257957)(271.50279483,120.6345084)
\curveto(271.69670826,120.73479439)(271.91169924,120.81836605)(272.14776776,120.88522338)
\curveto(272.38383629,120.95208071)(272.63676685,120.98550938)(272.90655944,120.98550938)
\curveto(273.72436826,120.98550938)(274.3271861,120.79747314)(274.71501296,120.42140066)
\curveto(275.11127083,120.05368535)(275.30939977,119.5230053)(275.30939977,118.8293605)
\closepath
}
}
{
\newrgbcolor{curcolor}{0 0 0}
\pscustom[linewidth=0.56097835,linecolor=curcolor]
{
\newpath
\moveto(52.53490904,123.29602291)
\lineto(277.23287657,123.29602291)
\lineto(277.23287657,110.57690384)
\lineto(52.53490904,110.57690384)
\closepath
}
}
{
\newrgbcolor{curcolor}{0 0 0}
\pscustom[linewidth=0.51800942,linecolor=curcolor]
{
\newpath
\moveto(71.388226,110.59101433)
\lineto(71.388226,123.19041433)
}
}
{
\newrgbcolor{curcolor}{0 0 0}
\pscustom[linewidth=0.5154072,linecolor=curcolor]
{
\newpath
\moveto(90.268446,110.68533433)
\lineto(90.268446,123.15845433)
}
}
{
\newrgbcolor{curcolor}{0 0 0}
\pscustom[linewidth=0.51930571,linecolor=curcolor]
{
\newpath
\moveto(109.107736,110.68532433)
\lineto(109.107736,123.34785433)
}
}
{
\newrgbcolor{curcolor}{0 0 0}
\pscustom[linewidth=0.52573889,linecolor=curcolor]
{
\newpath
\moveto(128.036306,110.50675433)
\lineto(128.036306,123.48496433)
}
}
{
\newrgbcolor{curcolor}{0 0 0}
\pscustom[linewidth=0.51670998,linecolor=curcolor]
{
\newpath
\moveto(146.875586,110.68532433)
\lineto(146.875586,123.22158433)
}
}
{
\newrgbcolor{curcolor}{0 0 0}
\pscustom[linewidth=0.52059871,linecolor=curcolor]
{
\newpath
\moveto(165.625596,110.68533433)
\lineto(165.625596,123.41099433)
}
}
{
\newrgbcolor{curcolor}{0 0 0}
\pscustom[linewidth=0.51670998,linecolor=curcolor]
{
\newpath
\moveto(184.554166,110.59604433)
\lineto(184.554166,123.13230433)
}
}
{
\newrgbcolor{curcolor}{0 0 0}
\pscustom[linewidth=0.51670998,linecolor=curcolor]
{
\newpath
\moveto(203.661306,110.68533433)
\lineto(203.661306,123.22159433)
}
}
{
\newrgbcolor{curcolor}{0 0 0}
\pscustom[linewidth=0.52059871,linecolor=curcolor]
{
\newpath
\moveto(222.589876,110.68532433)
\lineto(222.589876,123.41099433)
}
}
{
\newrgbcolor{curcolor}{0 0 0}
\pscustom[linewidth=0.52573889,linecolor=curcolor]
{
\newpath
\moveto(258.240186,123.33710433)
\lineto(258.204686,110.35895433)
}
}
{
\newrgbcolor{curcolor}{0 1 0}
\pscustom[linestyle=none,fillstyle=solid,fillcolor=curcolor]
{
\newpath
\moveto(146.67390976,101.47981546)
\lineto(165.52187882,101.47981546)
\lineto(165.52187882,88.66350458)
\lineto(146.67390976,88.66350458)
\closepath
}
}
{
\newrgbcolor{curcolor}{0 0 0}
\pscustom[linestyle=none,fillstyle=solid,fillcolor=curcolor]
{
\newpath
\moveto(62.27863777,97.14615134)
\curveto(62.70861973,97.31329467)(63.12595515,97.52222382)(63.53064405,97.77293881)
\curveto(63.93533295,98.03201096)(64.31051328,98.35794044)(64.65618505,98.75072725)
\lineto(65.38968367,98.75072725)
\lineto(65.38968367,91.86860091)
\lineto(66.86932745,91.86860091)
\lineto(66.86932745,90.99109846)
\lineto(62.65803361,90.99109846)
\lineto(62.65803361,91.86860091)
\lineto(64.35266837,91.86860091)
\lineto(64.35266837,97.30911608)
\curveto(64.25992717,97.22554442)(64.14610841,97.13779418)(64.01121212,97.04586535)
\curveto(63.88474683,96.96229369)(63.74141952,96.87872203)(63.58123016,96.79515036)
\curveto(63.42947183,96.7115787)(63.26928247,96.63218562)(63.1006621,96.55697113)
\curveto(62.93204172,96.48175663)(62.76763686,96.41907789)(62.6074475,96.36893489)
\lineto(62.27863777,97.14615134)
\closepath
}
}
{
\newrgbcolor{curcolor}{0 0 0}
\pscustom[linestyle=none,fillstyle=solid,fillcolor=curcolor]
{
\newpath
\moveto(76.32892947,95.75468318)
\curveto(77.45868598,95.71289734)(78.2807103,95.46636094)(78.79500244,95.01507397)
\curveto(79.30929458,94.563787)(79.56644065,93.95789245)(79.56644065,93.19739033)
\curveto(79.56644065,92.85474651)(79.51163903,92.5371742)(79.40203579,92.24467338)
\curveto(79.29243254,91.95217257)(79.11959666,91.70145758)(78.88352814,91.49252843)
\curveto(78.65589063,91.28359927)(78.36080498,91.12063453)(77.99827117,91.00363421)
\curveto(77.64416839,90.88663388)(77.22261745,90.82813372)(76.73361837,90.82813372)
\curveto(76.53127392,90.82813372)(76.32892947,90.84484805)(76.12658502,90.87827671)
\curveto(75.93267159,90.90334821)(75.74718918,90.93677688)(75.57013779,90.97856271)
\curveto(75.3930864,91.02034854)(75.23711255,91.06213437)(75.10221625,91.1039202)
\curveto(74.96731995,91.1540632)(74.87036324,91.19584903)(74.81134611,91.22927769)
\lineto(75.01369056,92.11931589)
\curveto(75.14858685,92.05245856)(75.35514681,91.97306548)(75.63337043,91.88113666)
\curveto(75.92002507,91.78920783)(76.27834336,91.74324341)(76.70832531,91.74324341)
\curveto(77.04556606,91.74324341)(77.32800519,91.78085066)(77.55564269,91.85606516)
\curveto(77.7832802,91.93127965)(77.9645471,92.03156565)(78.0994434,92.15692314)
\curveto(78.24277072,92.2906378)(78.34394294,92.44106679)(78.40296007,92.60821011)
\curveto(78.47040822,92.77535344)(78.5041323,92.95085392)(78.5041323,93.13471158)
\curveto(78.5041323,93.41885523)(78.45354618,93.66957022)(78.35237396,93.88685654)
\curveto(78.25963275,94.11250002)(78.09101238,94.30053626)(77.84651284,94.45096525)
\curveto(77.61044431,94.60139424)(77.28163459,94.71421599)(76.86008365,94.78943048)
\curveto(76.44696373,94.87300214)(75.92424058,94.91478798)(75.29191417,94.91478798)
\curveto(75.34250029,95.28250329)(75.38043987,95.6251471)(75.40573293,95.94271941)
\curveto(75.439457,96.26864889)(75.46475006,96.58204263)(75.48161209,96.88290061)
\curveto(75.50690515,97.19211576)(75.52376719,97.49715232)(75.53219821,97.7980103)
\curveto(75.54906024,98.09886829)(75.56592228,98.4164406)(75.58278432,98.75072725)
\lineto(79.35144967,98.75072725)
\lineto(79.35144967,97.8732248)
\lineto(76.49333434,97.8732248)
\curveto(76.48490332,97.75622447)(76.47225679,97.6016169)(76.45539475,97.40940208)
\curveto(76.44696373,97.22554442)(76.43431721,97.02915102)(76.41745517,96.82022186)
\curveto(76.40059313,96.61129271)(76.38373109,96.41072072)(76.36686906,96.2185059)
\curveto(76.35000702,96.02629108)(76.33736049,95.8716835)(76.32892947,95.75468318)
\closepath
}
}
{
\newrgbcolor{curcolor}{0 0 0}
\pscustom[linestyle=none,fillstyle=solid,fillcolor=curcolor]
{
\newpath
\moveto(81.24842984,97.14615134)
\curveto(81.67841179,97.31329467)(82.09574722,97.52222382)(82.50043612,97.77293881)
\curveto(82.90512501,98.03201096)(83.28030535,98.35794044)(83.62597711,98.75072725)
\lineto(84.35947574,98.75072725)
\lineto(84.35947574,91.86860091)
\lineto(85.83911952,91.86860091)
\lineto(85.83911952,90.99109846)
\lineto(81.62782568,90.99109846)
\lineto(81.62782568,91.86860091)
\lineto(83.32246044,91.86860091)
\lineto(83.32246044,97.30911608)
\curveto(83.22971923,97.22554442)(83.11590048,97.13779418)(82.98100418,97.04586535)
\curveto(82.8545389,96.96229369)(82.71121158,96.87872203)(82.55102223,96.79515036)
\curveto(82.39926389,96.7115787)(82.23907454,96.63218562)(82.07045416,96.55697113)
\curveto(81.90183379,96.48175663)(81.73742892,96.41907789)(81.57723957,96.36893489)
\lineto(81.24842984,97.14615134)
\closepath
}
}
{
\newrgbcolor{curcolor}{0 0 0}
\pscustom[linestyle=none,fillstyle=solid,fillcolor=curcolor]
{
\newpath
\moveto(93.89495597,97.14615134)
\curveto(94.32493792,97.31329467)(94.74227335,97.52222382)(95.14696224,97.77293881)
\curveto(95.55165114,98.03201096)(95.92683147,98.35794044)(96.27250324,98.75072725)
\lineto(97.00600187,98.75072725)
\lineto(97.00600187,91.86860091)
\lineto(98.48564565,91.86860091)
\lineto(98.48564565,90.99109846)
\lineto(94.27435181,90.99109846)
\lineto(94.27435181,91.86860091)
\lineto(95.96898657,91.86860091)
\lineto(95.96898657,97.30911608)
\curveto(95.87624536,97.22554442)(95.76242661,97.13779418)(95.62753031,97.04586535)
\curveto(95.50106503,96.96229369)(95.35773771,96.87872203)(95.19754836,96.79515036)
\curveto(95.04579002,96.7115787)(94.88560066,96.63218562)(94.71698029,96.55697113)
\curveto(94.54835992,96.48175663)(94.38395505,96.41907789)(94.2237657,96.36893489)
\lineto(93.89495597,97.14615134)
\closepath
}
}
{
\newrgbcolor{curcolor}{0 0 0}
\pscustom[linestyle=none,fillstyle=solid,fillcolor=curcolor]
{
\newpath
\moveto(99.72500635,93.6737488)
\curveto(99.86833367,94.00803545)(100.0622471,94.39246509)(100.30674664,94.82703773)
\curveto(100.55967721,95.26161037)(100.84211633,95.70871876)(101.15406402,96.1683629)
\curveto(101.46601171,96.63636421)(101.79903695,97.09182976)(102.15313974,97.53475957)
\curveto(102.50724252,97.98604654)(102.86556082,98.3913691)(103.22809462,98.75072725)
\lineto(104.23981687,98.75072725)
\lineto(104.23981687,93.82417779)
\lineto(105.16301341,93.82417779)
\lineto(105.16301341,92.97174684)
\lineto(104.23981687,92.97174684)
\lineto(104.23981687,90.99109846)
\lineto(103.22809462,90.99109846)
\lineto(103.22809462,92.97174684)
\lineto(99.72500635,92.97174684)
\lineto(99.72500635,93.6737488)
\closepath
\moveto(103.22809462,97.52222382)
\curveto(103.00045712,97.279866)(102.7686041,97.01243668)(102.53253558,96.71993587)
\curveto(102.30489808,96.42743505)(102.08147608,96.1182199)(101.86226959,95.79229042)
\curveto(101.64306311,95.47471811)(101.43650315,95.14878863)(101.24258972,94.81450198)
\curveto(101.05710731,94.48021533)(100.88848693,94.15010727)(100.7367286,93.82417779)
\lineto(103.22809462,93.82417779)
\lineto(103.22809462,97.52222382)
\closepath
}
}
{
\newrgbcolor{curcolor}{0 0 0}
\pscustom[linestyle=none,fillstyle=solid,fillcolor=curcolor]
{
\newpath
\moveto(117.22780308,96.77007887)
\curveto(117.22780308,96.50264955)(117.17300146,96.2435774)(117.06339822,95.99286241)
\curveto(116.96222599,95.74214743)(116.82311419,95.49561102)(116.64606279,95.25325321)
\curveto(116.47744242,95.01089539)(116.28352899,94.77271615)(116.0643225,94.5387155)
\curveto(115.84511602,94.30471484)(115.62169402,94.07489277)(115.39405652,93.84924929)
\curveto(115.26759124,93.7238918)(115.12004841,93.5734628)(114.95142804,93.39796232)
\curveto(114.78280766,93.22246183)(114.62261831,93.04278275)(114.47085997,92.8589251)
\curveto(114.31910163,92.67506744)(114.19263635,92.49538837)(114.09146413,92.31988788)
\curveto(113.9902919,92.14438739)(113.93970579,91.9939584)(113.93970579,91.86860091)
\lineto(117.54396628,91.86860091)
\lineto(117.54396628,90.99109846)
\lineto(112.80151827,90.99109846)
\curveto(112.79308725,91.03288429)(112.78887174,91.07467012)(112.78887174,91.11645595)
\lineto(112.78887174,91.25434919)
\curveto(112.78887174,91.60535017)(112.84788887,91.93127965)(112.96592313,92.23213763)
\curveto(113.08395739,92.53299562)(113.23571573,92.81713927)(113.42119814,93.08456858)
\curveto(113.60668055,93.3519979)(113.81324051,93.60271289)(114.04087802,93.83671354)
\curveto(114.27694654,94.07907136)(114.50879955,94.31307201)(114.73643706,94.5387155)
\curveto(114.92191947,94.72257315)(115.09897086,94.90225223)(115.26759124,95.07775272)
\curveto(115.44464263,95.25325321)(115.59640097,95.42875369)(115.72286625,95.60425418)
\curveto(115.85776254,95.77975467)(115.96315028,95.95943375)(116.03902945,96.1432914)
\curveto(116.12333963,96.33550622)(116.16549473,96.53189963)(116.16549473,96.73247162)
\curveto(116.16549473,96.9581151)(116.12755514,97.15032993)(116.05167597,97.30911608)
\curveto(115.98422783,97.46790224)(115.88727111,97.59743832)(115.76080583,97.69772431)
\curveto(115.64277157,97.80636747)(115.50365976,97.88576055)(115.3434704,97.93590355)
\curveto(115.18328105,97.98604654)(115.01044517,98.01111804)(114.82496275,98.01111804)
\curveto(114.60575627,98.01111804)(114.40341182,97.98186796)(114.21792941,97.9233678)
\curveto(114.04087802,97.86486763)(113.88068866,97.79383172)(113.73736134,97.71026006)
\curveto(113.60246504,97.6266884)(113.48443078,97.54311674)(113.38325856,97.45954507)
\curveto(113.28208633,97.38433058)(113.20620716,97.32165183)(113.15562105,97.27150884)
\lineto(112.6371134,97.99858229)
\curveto(112.70456155,98.07379679)(112.80573378,98.16572562)(112.94063008,98.27436878)
\curveto(113.07552638,98.38301194)(113.23571573,98.48329793)(113.42119814,98.57522676)
\curveto(113.61511157,98.67551275)(113.83010255,98.75908441)(114.06617107,98.82594174)
\curveto(114.3022396,98.89279907)(114.55517016,98.92622774)(114.82496275,98.92622774)
\curveto(115.64277157,98.92622774)(116.2455894,98.7381915)(116.63341626,98.36211902)
\curveto(117.02967414,97.99440371)(117.22780308,97.46372366)(117.22780308,96.77007887)
\closepath
}
}
{
\newrgbcolor{curcolor}{0 0 0}
\pscustom[linestyle=none,fillstyle=solid,fillcolor=curcolor]
{
\newpath
\moveto(122.13465692,95.01507397)
\curveto(122.13465692,94.79778765)(122.06720877,94.60975141)(121.93231247,94.45096525)
\curveto(121.80584719,94.2921791)(121.63722682,94.21278602)(121.42645135,94.21278602)
\curveto(121.20724487,94.21278602)(121.03019347,94.2921791)(120.89529717,94.45096525)
\curveto(120.76040087,94.60975141)(120.69295272,94.79778765)(120.69295272,95.01507397)
\curveto(120.69295272,95.23236029)(120.76040087,95.42457511)(120.89529717,95.59171844)
\curveto(121.03019347,95.75886176)(121.20724487,95.84243342)(121.42645135,95.84243342)
\curveto(121.63722682,95.84243342)(121.80584719,95.75886176)(121.93231247,95.59171844)
\curveto(122.06720877,95.42457511)(122.13465692,95.23236029)(122.13465692,95.01507397)
\closepath
\moveto(118.82126657,94.87718073)
\curveto(118.82126657,96.18089865)(119.04468857,97.17958001)(119.49153256,97.8732248)
\curveto(119.94680757,98.57522676)(120.58334948,98.92622774)(121.4011583,98.92622774)
\curveto(122.22739813,98.92622774)(122.86394004,98.57522676)(123.31078403,97.8732248)
\curveto(123.75762802,97.17958001)(123.98105002,96.18089865)(123.98105002,94.87718073)
\curveto(123.98105002,93.5734628)(123.75762802,92.57060286)(123.31078403,91.86860091)
\curveto(122.86394004,91.17495611)(122.22739813,90.82813372)(121.4011583,90.82813372)
\curveto(120.58334948,90.82813372)(119.94680757,91.17495611)(119.49153256,91.86860091)
\curveto(119.04468857,92.57060286)(118.82126657,93.5734628)(118.82126657,94.87718073)
\closepath
\moveto(122.91874166,94.87718073)
\curveto(122.91874166,95.3033962)(122.8934486,95.70454018)(122.84286249,96.08061266)
\curveto(122.79227638,96.4650423)(122.70796619,96.79932895)(122.58993193,97.0834726)
\curveto(122.47189767,97.36761625)(122.31592382,97.59325973)(122.12201039,97.76040306)
\curveto(121.92809696,97.92754638)(121.68781293,98.01111804)(121.4011583,98.01111804)
\curveto(121.11450366,98.01111804)(120.87421963,97.92754638)(120.6803062,97.76040306)
\curveto(120.48639277,97.59325973)(120.33041892,97.36761625)(120.21238466,97.0834726)
\curveto(120.0943504,96.79932895)(120.01004021,96.4650423)(119.9594541,96.08061266)
\curveto(119.90886799,95.70454018)(119.88357493,95.3033962)(119.88357493,94.87718073)
\curveto(119.88357493,94.45096525)(119.90886799,94.04564269)(119.9594541,93.66121305)
\curveto(120.01004021,93.28514057)(120.0943504,92.95503251)(120.21238466,92.67088886)
\curveto(120.33041892,92.38674521)(120.48639277,92.16110172)(120.6803062,91.9939584)
\curveto(120.87421963,91.82681507)(121.11450366,91.74324341)(121.4011583,91.74324341)
\curveto(121.68781293,91.74324341)(121.92809696,91.82681507)(122.12201039,91.9939584)
\curveto(122.31592382,92.16110172)(122.47189767,92.38674521)(122.58993193,92.67088886)
\curveto(122.70796619,92.95503251)(122.79227638,93.28514057)(122.84286249,93.66121305)
\curveto(122.8934486,94.04564269)(122.91874166,94.45096525)(122.91874166,94.87718073)
\closepath
}
}
{
\newrgbcolor{curcolor}{0 0 0}
\pscustom[linestyle=none,fillstyle=solid,fillcolor=curcolor]
{
\newpath
\moveto(131.83453818,97.14615134)
\curveto(132.26452014,97.31329467)(132.68185556,97.52222382)(133.08654446,97.77293881)
\curveto(133.49123336,98.03201096)(133.86641369,98.35794044)(134.21208546,98.75072725)
\lineto(134.94558408,98.75072725)
\lineto(134.94558408,91.86860091)
\lineto(136.42522786,91.86860091)
\lineto(136.42522786,90.99109846)
\lineto(132.21393403,90.99109846)
\lineto(132.21393403,91.86860091)
\lineto(133.90856878,91.86860091)
\lineto(133.90856878,97.30911608)
\curveto(133.81582758,97.22554442)(133.70200883,97.13779418)(133.56711253,97.04586535)
\curveto(133.44064725,96.96229369)(133.29731993,96.87872203)(133.13713057,96.79515036)
\curveto(132.98537224,96.7115787)(132.82518288,96.63218562)(132.65656251,96.55697113)
\curveto(132.48794213,96.48175663)(132.32353727,96.41907789)(132.16334791,96.36893489)
\lineto(131.83453818,97.14615134)
\closepath
}
}
{
\newrgbcolor{curcolor}{0 0 0}
\pscustom[linestyle=none,fillstyle=solid,fillcolor=curcolor]
{
\newpath
\moveto(141.1044442,95.01507397)
\curveto(141.1044442,94.79778765)(141.03699605,94.60975141)(140.90209975,94.45096525)
\curveto(140.77563447,94.2921791)(140.6070141,94.21278602)(140.39623863,94.21278602)
\curveto(140.17703214,94.21278602)(139.99998075,94.2921791)(139.86508445,94.45096525)
\curveto(139.73018815,94.60975141)(139.66274,94.79778765)(139.66274,95.01507397)
\curveto(139.66274,95.23236029)(139.73018815,95.42457511)(139.86508445,95.59171844)
\curveto(139.99998075,95.75886176)(140.17703214,95.84243342)(140.39623863,95.84243342)
\curveto(140.6070141,95.84243342)(140.77563447,95.75886176)(140.90209975,95.59171844)
\curveto(141.03699605,95.42457511)(141.1044442,95.23236029)(141.1044442,95.01507397)
\closepath
\moveto(137.79105385,94.87718073)
\curveto(137.79105385,96.18089865)(138.01447585,97.17958001)(138.46131984,97.8732248)
\curveto(138.91659485,98.57522676)(139.55313676,98.92622774)(140.37094557,98.92622774)
\curveto(141.1971854,98.92622774)(141.83372732,98.57522676)(142.28057131,97.8732248)
\curveto(142.7274153,97.17958001)(142.95083729,96.18089865)(142.95083729,94.87718073)
\curveto(142.95083729,93.5734628)(142.7274153,92.57060286)(142.28057131,91.86860091)
\curveto(141.83372732,91.17495611)(141.1971854,90.82813372)(140.37094557,90.82813372)
\curveto(139.55313676,90.82813372)(138.91659485,91.17495611)(138.46131984,91.86860091)
\curveto(138.01447585,92.57060286)(137.79105385,93.5734628)(137.79105385,94.87718073)
\closepath
\moveto(141.88852894,94.87718073)
\curveto(141.88852894,95.3033962)(141.86323588,95.70454018)(141.81264977,96.08061266)
\curveto(141.76206366,96.4650423)(141.67775347,96.79932895)(141.55971921,97.0834726)
\curveto(141.44168495,97.36761625)(141.2857111,97.59325973)(141.09179767,97.76040306)
\curveto(140.89788424,97.92754638)(140.65760021,98.01111804)(140.37094557,98.01111804)
\curveto(140.08429094,98.01111804)(139.8440069,97.92754638)(139.65009347,97.76040306)
\curveto(139.45618004,97.59325973)(139.3002062,97.36761625)(139.18217194,97.0834726)
\curveto(139.06413767,96.79932895)(138.97982749,96.4650423)(138.92924138,96.08061266)
\curveto(138.87865526,95.70454018)(138.85336221,95.3033962)(138.85336221,94.87718073)
\curveto(138.85336221,94.45096525)(138.87865526,94.04564269)(138.92924138,93.66121305)
\curveto(138.97982749,93.28514057)(139.06413767,92.95503251)(139.18217194,92.67088886)
\curveto(139.3002062,92.38674521)(139.45618004,92.16110172)(139.65009347,91.9939584)
\curveto(139.8440069,91.82681507)(140.08429094,91.74324341)(140.37094557,91.74324341)
\curveto(140.65760021,91.74324341)(140.89788424,91.82681507)(141.09179767,91.9939584)
\curveto(141.2857111,92.16110172)(141.44168495,92.38674521)(141.55971921,92.67088886)
\curveto(141.67775347,92.95503251)(141.76206366,93.28514057)(141.81264977,93.66121305)
\curveto(141.86323588,94.04564269)(141.88852894,94.45096525)(141.88852894,94.87718073)
\closepath
}
}
{
\newrgbcolor{curcolor}{0 0 0}
\pscustom[linestyle=none,fillstyle=solid,fillcolor=curcolor]
{
\newpath
\moveto(152.53690747,91.74324341)
\curveto(153.20295794,91.74324341)(153.67509499,91.87277949)(153.95331861,92.13185164)
\curveto(154.23997324,92.39928096)(154.38330056,92.75446052)(154.38330056,93.19739033)
\curveto(154.38330056,93.48153398)(154.32428343,93.71971321)(154.20624917,93.91192803)
\curveto(154.08821491,94.10414286)(153.93224106,94.25875043)(153.73832763,94.37575076)
\curveto(153.5444142,94.49275108)(153.32099221,94.57632275)(153.06806164,94.62646574)
\curveto(152.81513108,94.67660874)(152.54955399,94.70168024)(152.27133038,94.70168024)
\lineto(152.00575329,94.70168024)
\lineto(152.00575329,95.54157544)
\lineto(152.3725026,95.54157544)
\curveto(152.55798501,95.54157544)(152.74768293,95.55828977)(152.94159636,95.59171844)
\curveto(153.14394081,95.63350427)(153.32520771,95.7003616)(153.48539707,95.79229042)
\curveto(153.65401744,95.89257642)(153.78891374,96.02629108)(153.89008597,96.1934344)
\curveto(153.99125819,96.36057772)(154.0418443,96.57368546)(154.0418443,96.83275761)
\curveto(154.0418443,97.25897309)(153.906948,97.55983107)(153.63715541,97.73533156)
\curveto(153.37579383,97.91918921)(153.06806164,98.01111804)(152.71395886,98.01111804)
\curveto(152.35142506,98.01111804)(152.04369287,97.95679646)(151.79076231,97.8481533)
\curveto(151.53783175,97.74786731)(151.32705628,97.64340273)(151.15843591,97.53475957)
\lineto(150.75374701,98.32451177)
\curveto(150.9307984,98.44986927)(151.19637549,98.57940534)(151.55047828,98.71312)
\curveto(151.91301208,98.85519182)(152.31348547,98.92622774)(152.75189844,98.92622774)
\curveto(153.16501836,98.92622774)(153.51912114,98.87608474)(153.8142068,98.77579875)
\curveto(154.10929245,98.67551275)(154.34957649,98.53344093)(154.5350589,98.34958327)
\curveto(154.72897233,98.16572562)(154.87229965,97.9484393)(154.96504085,97.69772431)
\curveto(155.05778206,97.45536649)(155.10415266,97.18793717)(155.10415266,96.89543636)
\curveto(155.10415266,96.48593522)(154.99454942,96.13911282)(154.77534293,95.85496917)
\curveto(154.56456746,95.57082552)(154.29055936,95.3535392)(153.95331861,95.20311021)
\curveto(154.3580075,95.08610988)(154.70789478,94.85628781)(155.00298044,94.513644)
\curveto(155.29806609,94.17935735)(155.44560892,93.73224896)(155.44560892,93.17231883)
\curveto(155.44560892,92.83803218)(155.38659179,92.52463845)(155.26855752,92.23213763)
\curveto(155.15895428,91.94799398)(154.9861184,91.70145758)(154.75004987,91.49252843)
\curveto(154.52241237,91.28359927)(154.22311121,91.12063453)(153.85214638,91.00363421)
\curveto(153.48961258,90.88663388)(153.05541512,90.82813372)(152.54955399,90.82813372)
\curveto(152.35564056,90.82813372)(152.15329612,90.84484805)(151.94252065,90.87827671)
\curveto(151.7401762,90.90334821)(151.55047828,90.94095546)(151.37342689,90.99109846)
\curveto(151.19637549,91.03288429)(151.03618614,91.07467012)(150.89285882,91.11645595)
\curveto(150.75796252,91.16659895)(150.66100581,91.2042062)(150.60198868,91.22927769)
\lineto(150.80433312,92.11931589)
\curveto(150.93922942,92.05245856)(151.1542204,91.97306548)(151.44930605,91.88113666)
\curveto(151.74439171,91.78920783)(152.10692551,91.74324341)(152.53690747,91.74324341)
\closepath
}
}
{
\newrgbcolor{curcolor}{0 0 0}
\pscustom[linestyle=none,fillstyle=solid,fillcolor=curcolor]
{
\newpath
\moveto(158.13932035,90.99109846)
\curveto(158.18147544,91.58445726)(158.28686318,92.21124472)(158.45548355,92.87146085)
\curveto(158.63253494,93.54003414)(158.84331041,94.18353594)(159.08780995,94.80196623)
\curveto(159.33230949,95.42875369)(159.60210209,96.00539816)(159.89718775,96.53189963)
\curveto(160.1922734,97.06675826)(160.48314355,97.50550949)(160.76979818,97.8481533)
\lineto(156.97583977,97.8481533)
\lineto(156.97583977,98.75072725)
\lineto(161.93327876,98.75072725)
\lineto(161.93327876,97.88576055)
\curveto(161.6803482,97.59325973)(161.40634009,97.20047292)(161.11125444,96.70740012)
\curveto(160.81616878,96.21432731)(160.53372966,95.65857576)(160.26393706,95.04014547)
\curveto(160.00257548,94.43007234)(159.77493797,93.77403479)(159.58102454,93.07203283)
\curveto(159.38711111,92.37838804)(159.26486134,91.68474325)(159.21427523,90.99109846)
\lineto(158.13932035,90.99109846)
\closepath
}
}
{
\newrgbcolor{curcolor}{0 0 0}
\pscustom[linestyle=none,fillstyle=solid,fillcolor=curcolor]
{
\newpath
\moveto(177.10911529,90.99109846)
\curveto(177.15127038,91.58445726)(177.25665811,92.21124472)(177.42527849,92.87146085)
\curveto(177.60232988,93.54003414)(177.81310535,94.18353594)(178.05760489,94.80196623)
\curveto(178.30210443,95.42875369)(178.57189703,96.00539816)(178.86698269,96.53189963)
\curveto(179.16206834,97.06675826)(179.45293849,97.50550949)(179.73959312,97.8481533)
\lineto(175.94563471,97.8481533)
\lineto(175.94563471,98.75072725)
\lineto(180.9030737,98.75072725)
\lineto(180.9030737,97.88576055)
\curveto(180.65014314,97.59325973)(180.37613503,97.20047292)(180.08104938,96.70740012)
\curveto(179.78596372,96.21432731)(179.5035246,95.65857576)(179.233732,95.04014547)
\curveto(178.97237042,94.43007234)(178.74473291,93.77403479)(178.55081948,93.07203283)
\curveto(178.35690605,92.37838804)(178.23465628,91.68474325)(178.18407017,90.99109846)
\lineto(177.10911529,90.99109846)
\closepath
}
}
{
\newrgbcolor{curcolor}{0 0 0}
\pscustom[linestyle=none,fillstyle=solid,fillcolor=curcolor]
{
\newpath
\moveto(199.79698947,93.00935409)
\curveto(199.79698947,92.35749513)(199.58621401,91.83099366)(199.16466307,91.42984968)
\curveto(198.75154315,91.02870571)(198.11921675,90.82813372)(197.26768386,90.82813372)
\curveto(196.77868478,90.82813372)(196.37399588,90.89081246)(196.05361717,91.01616996)
\curveto(195.73323846,91.14988462)(195.47609239,91.31702794)(195.28217896,91.51759993)
\curveto(195.09669655,91.72652908)(194.96180025,91.95635115)(194.87749006,92.20706614)
\curveto(194.8016109,92.45778112)(194.76367131,92.70431752)(194.76367131,92.94667534)
\curveto(194.76367131,93.38960515)(194.88592108,93.78239196)(195.13042062,94.12503577)
\curveto(195.37492017,94.46767958)(195.66579031,94.74346607)(196.00303106,94.95239522)
\curveto(195.28639447,95.3535392)(194.92807618,95.96779091)(194.92807618,96.79515036)
\curveto(194.92807618,97.07929401)(194.9828778,97.35090191)(195.09248104,97.60997407)
\curveto(195.20208428,97.86904622)(195.35805813,98.0946897)(195.56040258,98.28690453)
\curveto(195.76274703,98.47911935)(196.00724657,98.63372692)(196.2939012,98.75072725)
\curveto(196.58898686,98.86772757)(196.91779659,98.92622774)(197.28033039,98.92622774)
\curveto(197.70188133,98.92622774)(198.06019962,98.86354899)(198.35528528,98.7381915)
\curveto(198.65880195,98.61283401)(198.90330149,98.44986927)(199.0887839,98.24929728)
\curveto(199.27426631,98.05708246)(199.40916261,97.83979614)(199.4934728,97.59743832)
\curveto(199.57778299,97.36343766)(199.61993808,97.13361559)(199.61993808,96.90797211)
\curveto(199.61993808,96.4650423)(199.50611933,96.08061266)(199.27848182,95.75468318)
\curveto(199.05084432,95.43711086)(198.78948274,95.18221729)(198.49439708,94.99000247)
\curveto(199.36279201,94.58050133)(199.79698947,93.9202852)(199.79698947,93.00935409)
\closepath
\moveto(195.77539355,92.93413959)
\curveto(195.77539355,92.80042493)(195.80068661,92.65835311)(195.85127272,92.50792412)
\curveto(195.90185883,92.36585229)(195.98616902,92.23213763)(196.10420328,92.10678014)
\curveto(196.22223755,91.98142265)(196.37821139,91.87695807)(196.57212482,91.79338641)
\curveto(196.76603825,91.71817191)(197.00210677,91.68056467)(197.28033039,91.68056467)
\curveto(197.54169197,91.68056467)(197.76511397,91.71399333)(197.95059638,91.78085066)
\curveto(198.14450981,91.85606516)(198.30048365,91.95217257)(198.41851792,92.06917289)
\curveto(198.5449832,92.19453039)(198.6377244,92.33242363)(198.69674153,92.48285262)
\curveto(198.75575866,92.63328161)(198.78526723,92.7837106)(198.78526723,92.93413959)
\curveto(198.78526723,93.41049806)(198.61243135,93.77403479)(198.26675958,94.02474978)
\curveto(197.92108781,94.28382193)(197.44473526,94.48021533)(196.83770191,94.61392999)
\curveto(196.50046116,94.43007234)(196.23909958,94.20025027)(196.05361717,93.92446378)
\curveto(195.86813476,93.6486773)(195.77539355,93.31856924)(195.77539355,92.93413959)
\closepath
\moveto(198.60821584,96.90797211)
\curveto(198.60821584,97.01661527)(198.58292278,97.13779418)(198.53233667,97.27150884)
\curveto(198.48175056,97.41358066)(198.40165588,97.53893815)(198.29205264,97.64758131)
\curveto(198.19088041,97.76458164)(198.05598411,97.86068905)(197.88736374,97.93590355)
\curveto(197.71874336,98.01947521)(197.51639892,98.06126104)(197.28033039,98.06126104)
\curveto(197.03583085,98.06126104)(196.82927089,98.02365379)(196.66065052,97.9484393)
\curveto(196.50046116,97.8732248)(196.36556486,97.77711739)(196.25596162,97.66011706)
\curveto(196.14635838,97.5514739)(196.0662637,97.42611641)(196.01567759,97.28404458)
\curveto(195.97352249,97.15032993)(195.95244495,97.01661527)(195.95244495,96.88290061)
\curveto(195.95244495,96.5402568)(196.07469472,96.2185059)(196.31919426,95.91764792)
\curveto(196.57212482,95.6251471)(196.98524474,95.41203936)(197.55855401,95.2783247)
\curveto(197.87893272,95.46218236)(198.13186328,95.67946868)(198.31734569,95.93018366)
\curveto(198.51125912,96.18089865)(198.60821584,96.50682813)(198.60821584,96.90797211)
\closepath
}
}
{
\newrgbcolor{curcolor}{0 0 0}
\pscustom[linestyle=none,fillstyle=solid,fillcolor=curcolor]
{
\newpath
\moveto(218.74148369,95.64186143)
\curveto(218.74148369,94.09578569)(218.36208785,92.92996101)(217.60329617,92.14438739)
\curveto(216.84450449,91.36717094)(215.70210145,90.97438413)(214.17608707,90.96602696)
\lineto(214.13814749,91.84352941)
\curveto(215.08242158,91.84352941)(215.84121326,92.02738706)(216.41452253,92.39510237)
\curveto(216.99626282,92.77117485)(217.37987417,93.41467665)(217.56535659,94.32560776)
\curveto(217.36301214,94.23367893)(217.13959014,94.15846444)(216.8950906,94.09996427)
\curveto(216.65059106,94.04982128)(216.39344499,94.02474978)(216.12365239,94.02474978)
\curveto(215.6768084,94.02474978)(215.30162807,94.08742852)(214.99811139,94.21278602)
\curveto(214.69459472,94.34650068)(214.45009518,94.52200117)(214.26461277,94.73928749)
\curveto(214.07913035,94.96493097)(213.94423406,95.21982454)(213.85992387,95.50396819)
\curveto(213.77561368,95.78811184)(213.73345859,96.08896982)(213.73345859,96.40654214)
\curveto(213.73345859,96.69068579)(213.77982919,96.97900802)(213.8725704,97.27150884)
\curveto(213.9653116,97.57236682)(214.10863892,97.84397472)(214.30255235,98.08633254)
\curveto(214.49646578,98.32869036)(214.74518083,98.52926234)(215.0486975,98.6880485)
\curveto(215.35221418,98.84683466)(215.71474798,98.92622774)(216.13629892,98.92622774)
\curveto(216.99626282,98.92622774)(217.64545126,98.63372692)(218.08386423,98.04872529)
\curveto(218.52227721,97.46372366)(218.74148369,96.6614357)(218.74148369,95.64186143)
\closepath
\moveto(216.23747114,94.87718073)
\curveto(216.50726374,94.87718073)(216.75597879,94.90225223)(216.9836163,94.95239522)
\curveto(217.21968482,95.00253822)(217.44732232,95.07357413)(217.66652881,95.16550296)
\curveto(217.67495983,95.24907462)(217.67917534,95.3284677)(217.67917534,95.4036822)
\lineto(217.67917534,95.64186143)
\curveto(217.67917534,95.96779091)(217.65388228,96.27700606)(217.60329617,96.56950688)
\curveto(217.56114108,96.86200769)(217.4810464,97.11690126)(217.36301214,97.33418758)
\curveto(217.25340889,97.5514739)(217.09743505,97.72279581)(216.8950906,97.8481533)
\curveto(216.70117717,97.98186796)(216.45246212,98.04872529)(216.14894544,98.04872529)
\curveto(215.89601488,98.04872529)(215.68523942,97.99440371)(215.51661904,97.88576055)
\curveto(215.34799867,97.78547455)(215.20888686,97.65593848)(215.09928362,97.49715232)
\curveto(214.98968037,97.34672333)(214.9095857,97.17540142)(214.85899958,96.9831866)
\curveto(214.81684449,96.79097178)(214.79576694,96.60711412)(214.79576694,96.43161364)
\curveto(214.79576694,95.9218265)(214.9095857,95.53321827)(215.1372232,95.26578895)
\curveto(215.37329172,95.0067168)(215.74004104,94.87718073)(216.23747114,94.87718073)
\closepath
}
}
{
\newrgbcolor{curcolor}{0 0 0}
\pscustom[linestyle=none,fillstyle=solid,fillcolor=curcolor]
{
\newpath
\moveto(236.15574802,91.79338641)
\curveto(236.15574802,91.54267143)(236.07143783,91.32120652)(235.90281746,91.1289917)
\curveto(235.73419709,90.93677688)(235.51077509,90.84066947)(235.23255147,90.84066947)
\curveto(234.94589684,90.84066947)(234.71825933,90.93677688)(234.54963896,91.1289917)
\curveto(234.38101859,91.32120652)(234.2967084,91.54267143)(234.2967084,91.79338641)
\curveto(234.2967084,92.05245856)(234.38101859,92.27810205)(234.54963896,92.47031687)
\curveto(234.71825933,92.66253169)(234.94589684,92.7586391)(235.23255147,92.7586391)
\curveto(235.51077509,92.7586391)(235.73419709,92.66253169)(235.90281746,92.47031687)
\curveto(236.07143783,92.27810205)(236.15574802,92.05245856)(236.15574802,91.79338641)
\closepath
}
}
{
\newrgbcolor{curcolor}{0 0 0}
\pscustom[linestyle=none,fillstyle=solid,fillcolor=curcolor]
{
\newpath
\moveto(242.47902066,91.79338641)
\curveto(242.47902066,91.54267143)(242.39471048,91.32120652)(242.2260901,91.1289917)
\curveto(242.05746973,90.93677688)(241.83404773,90.84066947)(241.55582412,90.84066947)
\curveto(241.26916948,90.84066947)(241.04153198,90.93677688)(240.8729116,91.1289917)
\curveto(240.70429123,91.32120652)(240.61998104,91.54267143)(240.61998104,91.79338641)
\curveto(240.61998104,92.05245856)(240.70429123,92.27810205)(240.8729116,92.47031687)
\curveto(241.04153198,92.66253169)(241.26916948,92.7586391)(241.55582412,92.7586391)
\curveto(241.83404773,92.7586391)(242.05746973,92.66253169)(242.2260901,92.47031687)
\curveto(242.39471048,92.27810205)(242.47902066,92.05245856)(242.47902066,91.79338641)
\closepath
}
}
{
\newrgbcolor{curcolor}{0 0 0}
\pscustom[linestyle=none,fillstyle=solid,fillcolor=curcolor]
{
\newpath
\moveto(248.80227798,91.79338641)
\curveto(248.80227798,91.54267143)(248.71796779,91.32120652)(248.54934742,91.1289917)
\curveto(248.38072705,90.93677688)(248.15730505,90.84066947)(247.87908143,90.84066947)
\curveto(247.5924268,90.84066947)(247.36478929,90.93677688)(247.19616892,91.1289917)
\curveto(247.02754855,91.32120652)(246.94323836,91.54267143)(246.94323836,91.79338641)
\curveto(246.94323836,92.05245856)(247.02754855,92.27810205)(247.19616892,92.47031687)
\curveto(247.36478929,92.66253169)(247.5924268,92.7586391)(247.87908143,92.7586391)
\curveto(248.15730505,92.7586391)(248.38072705,92.66253169)(248.54934742,92.47031687)
\curveto(248.71796779,92.27810205)(248.80227798,92.05245856)(248.80227798,91.79338641)
\closepath
}
}
{
\newrgbcolor{curcolor}{0 0 0}
\pscustom[linestyle=none,fillstyle=solid,fillcolor=curcolor]
{
\newpath
\moveto(266.02684439,95.75468318)
\curveto(267.15660089,95.71289734)(267.97862522,95.46636094)(268.49291736,95.01507397)
\curveto(269.0072095,94.563787)(269.26435557,93.95789245)(269.26435557,93.19739033)
\curveto(269.26435557,92.85474651)(269.20955395,92.5371742)(269.0999507,92.24467338)
\curveto(268.99034746,91.95217257)(268.81751158,91.70145758)(268.58144305,91.49252843)
\curveto(268.35380555,91.28359927)(268.05871989,91.12063453)(267.69618609,91.00363421)
\curveto(267.3420833,90.88663388)(266.92053237,90.82813372)(266.43153328,90.82813372)
\curveto(266.22918884,90.82813372)(266.02684439,90.84484805)(265.82449994,90.87827671)
\curveto(265.63058651,90.90334821)(265.4451041,90.93677688)(265.2680527,90.97856271)
\curveto(265.09100131,91.02034854)(264.93502747,91.06213437)(264.80013117,91.1039202)
\curveto(264.66523487,91.1540632)(264.56827815,91.19584903)(264.50926102,91.22927769)
\lineto(264.71160547,92.11931589)
\curveto(264.84650177,92.05245856)(265.05306173,91.97306548)(265.33128534,91.88113666)
\curveto(265.61793998,91.78920783)(265.97625828,91.74324341)(266.40624023,91.74324341)
\curveto(266.74348098,91.74324341)(267.0259201,91.78085066)(267.25355761,91.85606516)
\curveto(267.48119511,91.93127965)(267.66246201,92.03156565)(267.79735831,92.15692314)
\curveto(267.94068563,92.2906378)(268.04185786,92.44106679)(268.10087499,92.60821011)
\curveto(268.16832314,92.77535344)(268.20204721,92.95085392)(268.20204721,93.13471158)
\curveto(268.20204721,93.41885523)(268.1514611,93.66957022)(268.05028887,93.88685654)
\curveto(267.95754767,94.11250002)(267.78892729,94.30053626)(267.54442775,94.45096525)
\curveto(267.30835923,94.60139424)(266.9795495,94.71421599)(266.55799857,94.78943048)
\curveto(266.14487865,94.87300214)(265.62215549,94.91478798)(264.98982909,94.91478798)
\curveto(265.0404152,95.28250329)(265.07835478,95.6251471)(265.10364784,95.94271941)
\curveto(265.13737191,96.26864889)(265.16266497,96.58204263)(265.17952701,96.88290061)
\curveto(265.20482006,97.19211576)(265.2216821,97.49715232)(265.23011312,97.7980103)
\curveto(265.24697516,98.09886829)(265.2638372,98.4164406)(265.28069923,98.75072725)
\lineto(269.04936459,98.75072725)
\lineto(269.04936459,97.8732248)
\lineto(266.19124925,97.8732248)
\curveto(266.18281823,97.75622447)(266.17017171,97.6016169)(266.15330967,97.40940208)
\curveto(266.14487865,97.22554442)(266.13223212,97.02915102)(266.11537008,96.82022186)
\curveto(266.09850805,96.61129271)(266.08164601,96.41072072)(266.06478397,96.2185059)
\curveto(266.04792193,96.02629108)(266.03527541,95.8716835)(266.02684439,95.75468318)
\closepath
}
}
{
\newrgbcolor{curcolor}{0 0 0}
\pscustom[linestyle=none,fillstyle=solid,fillcolor=curcolor]
{
\newpath
\moveto(275.30940459,96.77007887)
\curveto(275.30940459,96.50264955)(275.25460297,96.2435774)(275.14499973,95.99286241)
\curveto(275.0438275,95.74214743)(274.9047157,95.49561102)(274.7276643,95.25325321)
\curveto(274.55904393,95.01089539)(274.3651305,94.77271615)(274.14592401,94.5387155)
\curveto(273.92671753,94.30471484)(273.70329553,94.07489277)(273.47565803,93.84924929)
\curveto(273.34919275,93.7238918)(273.20164992,93.5734628)(273.03302954,93.39796232)
\curveto(272.86440917,93.22246183)(272.70421982,93.04278275)(272.55246148,92.8589251)
\curveto(272.40070314,92.67506744)(272.27423786,92.49538837)(272.17306564,92.31988788)
\curveto(272.07189341,92.14438739)(272.0213073,91.9939584)(272.0213073,91.86860091)
\lineto(275.62556779,91.86860091)
\lineto(275.62556779,90.99109846)
\lineto(270.88311978,90.99109846)
\curveto(270.87468876,91.03288429)(270.87047325,91.07467012)(270.87047325,91.11645595)
\lineto(270.87047325,91.25434919)
\curveto(270.87047325,91.60535017)(270.92949038,91.93127965)(271.04752464,92.23213763)
\curveto(271.1655589,92.53299562)(271.31731724,92.81713927)(271.50279965,93.08456858)
\curveto(271.68828206,93.3519979)(271.89484202,93.60271289)(272.12247953,93.83671354)
\curveto(272.35854805,94.07907136)(272.59040106,94.31307201)(272.81803857,94.5387155)
\curveto(273.00352098,94.72257315)(273.18057237,94.90225223)(273.34919275,95.07775272)
\curveto(273.52624414,95.25325321)(273.67800247,95.42875369)(273.80446776,95.60425418)
\curveto(273.93936405,95.77975467)(274.04475179,95.95943375)(274.12063096,96.1432914)
\curveto(274.20494114,96.33550622)(274.24709624,96.53189963)(274.24709624,96.73247162)
\curveto(274.24709624,96.9581151)(274.20915665,97.15032993)(274.13327748,97.30911608)
\curveto(274.06582933,97.46790224)(273.96887262,97.59743832)(273.84240734,97.69772431)
\curveto(273.72437308,97.80636747)(273.58526127,97.88576055)(273.42507191,97.93590355)
\curveto(273.26488256,97.98604654)(273.09204668,98.01111804)(272.90656426,98.01111804)
\curveto(272.68735778,98.01111804)(272.48501333,97.98186796)(272.29953092,97.9233678)
\curveto(272.12247953,97.86486763)(271.96229017,97.79383172)(271.81896285,97.71026006)
\curveto(271.68406655,97.6266884)(271.56603229,97.54311674)(271.46486007,97.45954507)
\curveto(271.36368784,97.38433058)(271.28780867,97.32165183)(271.23722256,97.27150884)
\lineto(270.71871491,97.99858229)
\curveto(270.78616306,98.07379679)(270.88733529,98.16572562)(271.02223159,98.27436878)
\curveto(271.15712788,98.38301194)(271.31731724,98.48329793)(271.50279965,98.57522676)
\curveto(271.69671308,98.67551275)(271.91170406,98.75908441)(272.14777258,98.82594174)
\curveto(272.38384111,98.89279907)(272.63677167,98.92622774)(272.90656426,98.92622774)
\curveto(273.72437308,98.92622774)(274.32719091,98.7381915)(274.71501777,98.36211902)
\curveto(275.11127565,97.99440371)(275.30940459,97.46372366)(275.30940459,96.77007887)
\closepath
}
}
{
\newrgbcolor{curcolor}{0 0 0}
\pscustom[linewidth=0.56097835,linecolor=curcolor]
{
\newpath
\moveto(52.53490982,101.23674295)
\lineto(277.23287735,101.23674295)
\lineto(277.23287735,88.51762388)
\lineto(52.53490982,88.51762388)
\closepath
}
}
{
\newrgbcolor{curcolor}{0 0 0}
\pscustom[linewidth=0.51800942,linecolor=curcolor]
{
\newpath
\moveto(71.388231,88.53172633)
\lineto(71.388231,101.13112633)
}
}
{
\newrgbcolor{curcolor}{0 0 0}
\pscustom[linewidth=0.5154072,linecolor=curcolor]
{
\newpath
\moveto(90.268454,88.62603633)
\lineto(90.268454,101.09916633)
}
}
{
\newrgbcolor{curcolor}{0 0 0}
\pscustom[linewidth=0.51930571,linecolor=curcolor]
{
\newpath
\moveto(109.107744,88.62603633)
\lineto(109.107744,101.28856633)
}
}
{
\newrgbcolor{curcolor}{0 0 0}
\pscustom[linewidth=0.52573889,linecolor=curcolor]
{
\newpath
\moveto(128.036314,88.44746633)
\lineto(128.036314,101.42566633)
}
}
{
\newrgbcolor{curcolor}{0 0 0}
\pscustom[linewidth=0.51670998,linecolor=curcolor]
{
\newpath
\moveto(146.875594,88.62603633)
\lineto(146.875594,101.16229633)
}
}
{
\newrgbcolor{curcolor}{0 0 0}
\pscustom[linewidth=0.52059871,linecolor=curcolor]
{
\newpath
\moveto(165.625604,88.62603633)
\lineto(165.625604,101.35170633)
}
}
{
\newrgbcolor{curcolor}{0 0 0}
\pscustom[linewidth=0.51670998,linecolor=curcolor]
{
\newpath
\moveto(184.554174,88.53675633)
\lineto(184.554174,101.07301633)
}
}
{
\newrgbcolor{curcolor}{0 0 0}
\pscustom[linewidth=0.51670998,linecolor=curcolor]
{
\newpath
\moveto(203.661314,88.62603633)
\lineto(203.661314,101.16230633)
}
}
{
\newrgbcolor{curcolor}{0 0 0}
\pscustom[linewidth=0.52059871,linecolor=curcolor]
{
\newpath
\moveto(222.589884,88.62603633)
\lineto(222.589884,101.35170633)
}
}
{
\newrgbcolor{curcolor}{0 0 0}
\pscustom[linewidth=0.52573889,linecolor=curcolor]
{
\newpath
\moveto(258.240194,101.27781633)
\lineto(258.204694,88.29966633)
}
}
{
\newrgbcolor{curcolor}{0 1 0}
\pscustom[linestyle=none,fillstyle=solid,fillcolor=curcolor]
{
\newpath
\moveto(165.41372393,79.25280985)
\lineto(184.26169299,79.25280985)
\lineto(184.26169299,66.43649897)
\lineto(165.41372393,66.43649897)
\closepath
}
}
{
\newrgbcolor{curcolor}{0 0 0}
\pscustom[linestyle=none,fillstyle=solid,fillcolor=curcolor]
{
\newpath
\moveto(62.27864119,75.08685558)
\curveto(62.70862315,75.25399891)(63.12595857,75.46292806)(63.53064747,75.71364305)
\curveto(63.93533637,75.9727152)(64.3105167,76.29864468)(64.65618846,76.69143149)
\lineto(65.38968709,76.69143149)
\lineto(65.38968709,69.80930515)
\lineto(66.86933087,69.80930515)
\lineto(66.86933087,68.9318027)
\lineto(62.65803703,68.9318027)
\lineto(62.65803703,69.80930515)
\lineto(64.35267179,69.80930515)
\lineto(64.35267179,75.24982032)
\curveto(64.25993059,75.16624866)(64.14611183,75.07849842)(64.01121553,74.98656959)
\curveto(63.88475025,74.90299793)(63.74142294,74.81942627)(63.58123358,74.73585461)
\curveto(63.42947524,74.65228294)(63.26928589,74.57288986)(63.10066551,74.49767537)
\curveto(62.93204514,74.42246087)(62.76764028,74.35978213)(62.60745092,74.30963913)
\lineto(62.27864119,75.08685558)
\closepath
}
}
{
\newrgbcolor{curcolor}{0 0 0}
\pscustom[linestyle=none,fillstyle=solid,fillcolor=curcolor]
{
\newpath
\moveto(76.32893289,73.69538742)
\curveto(77.4586894,73.65360159)(78.28071372,73.40706518)(78.79500586,72.95577821)
\curveto(79.309298,72.50449124)(79.56644407,71.89859669)(79.56644407,71.13809457)
\curveto(79.56644407,70.79545076)(79.51164245,70.47787844)(79.40203921,70.18537762)
\curveto(79.29243596,69.89287681)(79.11960008,69.64216182)(78.88353156,69.43323267)
\curveto(78.65589405,69.22430352)(78.3608084,69.06133877)(77.99827459,68.94433845)
\curveto(77.64417181,68.82733812)(77.22262087,68.76883796)(76.73362179,68.76883796)
\curveto(76.53127734,68.76883796)(76.32893289,68.78555229)(76.12658844,68.81898096)
\curveto(75.93267501,68.84405245)(75.7471926,68.87748112)(75.57014121,68.91926695)
\curveto(75.39308982,68.96105278)(75.23711597,69.00283861)(75.10221967,69.04462444)
\curveto(74.96732337,69.09476744)(74.87036666,69.13655327)(74.81134953,69.16998194)
\lineto(75.01369397,70.06002013)
\curveto(75.14859027,69.9931628)(75.35515023,69.91376972)(75.63337385,69.8218409)
\curveto(75.92002848,69.72991207)(76.27834678,69.68394765)(76.70832873,69.68394765)
\curveto(77.04556948,69.68394765)(77.32800861,69.7215549)(77.55564611,69.7967694)
\curveto(77.78328362,69.87198389)(77.96455052,69.97226989)(78.09944682,70.09762738)
\curveto(78.24277414,70.23134204)(78.34394636,70.38177103)(78.40296349,70.54891435)
\curveto(78.47041164,70.71605768)(78.50413571,70.89155817)(78.50413571,71.07541582)
\curveto(78.50413571,71.35955947)(78.4535496,71.61027446)(78.35237738,71.82756078)
\curveto(78.25963617,72.05320426)(78.0910158,72.2412405)(77.84651626,72.39166949)
\curveto(77.61044773,72.54209849)(77.281638,72.65492023)(76.86008707,72.73013472)
\curveto(76.44696715,72.81370639)(75.92424399,72.85549222)(75.29191759,72.85549222)
\curveto(75.3425037,73.22320753)(75.38044329,73.56585134)(75.40573634,73.88342366)
\curveto(75.43946042,74.20935314)(75.46475347,74.52274687)(75.48161551,74.82360485)
\curveto(75.50690857,75.13282)(75.52377061,75.43785656)(75.53220162,75.73871455)
\curveto(75.54906366,76.03957253)(75.5659257,76.35714484)(75.58278774,76.69143149)
\lineto(79.35145309,76.69143149)
\lineto(79.35145309,75.81392904)
\lineto(76.49333776,75.81392904)
\curveto(76.48490674,75.69692871)(76.47226021,75.54232114)(76.45539817,75.35010632)
\curveto(76.44696715,75.16624866)(76.43432062,74.96985526)(76.41745859,74.7609261)
\curveto(76.40059655,74.55199695)(76.38373451,74.35142496)(76.36687248,74.15921014)
\curveto(76.35001044,73.96699532)(76.33736391,73.81238774)(76.32893289,73.69538742)
\closepath
}
}
{
\newrgbcolor{curcolor}{0 0 0}
\pscustom[linestyle=none,fillstyle=solid,fillcolor=curcolor]
{
\newpath
\moveto(81.24843326,75.08685558)
\curveto(81.67841521,75.25399891)(82.09575064,75.46292806)(82.50043953,75.71364305)
\curveto(82.90512843,75.9727152)(83.28030876,76.29864468)(83.62598053,76.69143149)
\lineto(84.35947916,76.69143149)
\lineto(84.35947916,69.80930515)
\lineto(85.83912294,69.80930515)
\lineto(85.83912294,68.9318027)
\lineto(81.6278291,68.9318027)
\lineto(81.6278291,69.80930515)
\lineto(83.32246386,69.80930515)
\lineto(83.32246386,75.24982032)
\curveto(83.22972265,75.16624866)(83.1159039,75.07849842)(82.9810076,74.98656959)
\curveto(82.85454232,74.90299793)(82.711215,74.81942627)(82.55102565,74.73585461)
\curveto(82.39926731,74.65228294)(82.23907796,74.57288986)(82.07045758,74.49767537)
\curveto(81.90183721,74.42246087)(81.73743234,74.35978213)(81.57724299,74.30963913)
\lineto(81.24843326,75.08685558)
\closepath
}
}
{
\newrgbcolor{curcolor}{0 0 0}
\pscustom[linestyle=none,fillstyle=solid,fillcolor=curcolor]
{
\newpath
\moveto(93.89495939,75.08685558)
\curveto(94.32494134,75.25399891)(94.74227677,75.46292806)(95.14696566,75.71364305)
\curveto(95.55165456,75.9727152)(95.92683489,76.29864468)(96.27250666,76.69143149)
\lineto(97.00600529,76.69143149)
\lineto(97.00600529,69.80930515)
\lineto(98.48564907,69.80930515)
\lineto(98.48564907,68.9318027)
\lineto(94.27435523,68.9318027)
\lineto(94.27435523,69.80930515)
\lineto(95.96898999,69.80930515)
\lineto(95.96898999,75.24982032)
\curveto(95.87624878,75.16624866)(95.76243003,75.07849842)(95.62753373,74.98656959)
\curveto(95.50106845,74.90299793)(95.35774113,74.81942627)(95.19755178,74.73585461)
\curveto(95.04579344,74.65228294)(94.88560408,74.57288986)(94.71698371,74.49767537)
\curveto(94.54836334,74.42246087)(94.38395847,74.35978213)(94.22376912,74.30963913)
\lineto(93.89495939,75.08685558)
\closepath
}
}
{
\newrgbcolor{curcolor}{0 0 0}
\pscustom[linestyle=none,fillstyle=solid,fillcolor=curcolor]
{
\newpath
\moveto(99.72500977,71.61445304)
\curveto(99.86833709,71.94873969)(100.06225052,72.33316933)(100.30675006,72.76774197)
\curveto(100.55968062,73.20231461)(100.84211975,73.649423)(101.15406744,74.10906714)
\curveto(101.46601513,74.57706845)(101.79904037,75.032534)(102.15314316,75.47546381)
\curveto(102.50724594,75.92675078)(102.86556424,76.33207334)(103.22809804,76.69143149)
\lineto(104.23982028,76.69143149)
\lineto(104.23982028,71.76488203)
\lineto(105.16301683,71.76488203)
\lineto(105.16301683,70.91245108)
\lineto(104.23982028,70.91245108)
\lineto(104.23982028,68.9318027)
\lineto(103.22809804,68.9318027)
\lineto(103.22809804,70.91245108)
\lineto(99.72500977,70.91245108)
\lineto(99.72500977,71.61445304)
\closepath
\moveto(103.22809804,75.46292806)
\curveto(103.00046054,75.22057024)(102.76860752,74.95314093)(102.532539,74.66064011)
\curveto(102.30490149,74.36813929)(102.0814795,74.05892415)(101.86227301,73.73299466)
\curveto(101.64306653,73.41542235)(101.43650657,73.08949287)(101.24259314,72.75520622)
\curveto(101.05711073,72.42091958)(100.88849035,72.09081151)(100.73673202,71.76488203)
\lineto(103.22809804,71.76488203)
\lineto(103.22809804,75.46292806)
\closepath
}
}
{
\newrgbcolor{curcolor}{0 0 0}
\pscustom[linestyle=none,fillstyle=solid,fillcolor=curcolor]
{
\newpath
\moveto(117.2278065,74.71078311)
\curveto(117.2278065,74.44335379)(117.17300488,74.18428164)(117.06340164,73.93356665)
\curveto(116.96222941,73.68285167)(116.8231176,73.43631527)(116.64606621,73.19395745)
\curveto(116.47744584,72.95159963)(116.28353241,72.71342039)(116.06432592,72.47941974)
\curveto(115.84511944,72.24541909)(115.62169744,72.01559702)(115.39405994,71.78995353)
\curveto(115.26759465,71.66459604)(115.12005183,71.51416705)(114.95143145,71.33866656)
\curveto(114.78281108,71.16316607)(114.62262172,70.98348699)(114.47086339,70.79962934)
\curveto(114.31910505,70.61577168)(114.19263977,70.43609261)(114.09146755,70.26059212)
\curveto(113.99029532,70.08509163)(113.93970921,69.93466264)(113.93970921,69.80930515)
\lineto(117.5439697,69.80930515)
\lineto(117.5439697,68.9318027)
\lineto(112.80152169,68.9318027)
\curveto(112.79309067,68.97358853)(112.78887516,69.01537436)(112.78887516,69.05716019)
\lineto(112.78887516,69.19505343)
\curveto(112.78887516,69.54605441)(112.84789229,69.87198389)(112.96592655,70.17284188)
\curveto(113.08396081,70.47369986)(113.23571915,70.75784351)(113.42120156,71.02527283)
\curveto(113.60668397,71.29270214)(113.81324393,71.54341713)(114.04088143,71.77741778)
\curveto(114.27694996,72.0197756)(114.50880297,72.25377625)(114.73644048,72.47941974)
\curveto(114.92192289,72.66327739)(115.09897428,72.84295647)(115.26759465,73.01845696)
\curveto(115.44464605,73.19395745)(115.59640438,73.36945794)(115.72286966,73.54495843)
\curveto(115.85776596,73.72045892)(115.9631537,73.90013799)(116.03903287,74.08399564)
\curveto(116.12334305,74.27621047)(116.16549815,74.47260387)(116.16549815,74.67317586)
\curveto(116.16549815,74.89881935)(116.12755856,75.09103417)(116.05167939,75.24982032)
\curveto(115.98423124,75.40860648)(115.88727453,75.53814256)(115.76080925,75.63842855)
\curveto(115.64277499,75.74707171)(115.50366318,75.82646479)(115.34347382,75.87660779)
\curveto(115.18328447,75.92675078)(115.01044858,75.95182228)(114.82496617,75.95182228)
\curveto(114.60575969,75.95182228)(114.40341524,75.9225722)(114.21793283,75.86407204)
\curveto(114.04088143,75.80557188)(113.88069208,75.73453596)(113.73736476,75.6509643)
\curveto(113.60246846,75.56739264)(113.4844342,75.48382098)(113.38326198,75.40024932)
\curveto(113.28208975,75.32503482)(113.20621058,75.26235607)(113.15562447,75.21221308)
\lineto(112.63711682,75.93928653)
\curveto(112.70456497,76.01450103)(112.8057372,76.10642986)(112.94063349,76.21507302)
\curveto(113.07552979,76.32371618)(113.23571915,76.42400217)(113.42120156,76.515931)
\curveto(113.61511499,76.61621699)(113.83010597,76.69978866)(114.06617449,76.76664598)
\curveto(114.30224301,76.83350331)(114.55517358,76.86693198)(114.82496617,76.86693198)
\curveto(115.64277499,76.86693198)(116.24559282,76.67889574)(116.63341968,76.30282326)
\curveto(117.02967756,75.93510795)(117.2278065,75.4044279)(117.2278065,74.71078311)
\closepath
}
}
{
\newrgbcolor{curcolor}{0 0 0}
\pscustom[linestyle=none,fillstyle=solid,fillcolor=curcolor]
{
\newpath
\moveto(122.13466034,72.95577821)
\curveto(122.13466034,72.73849189)(122.06721219,72.55045565)(121.93231589,72.39166949)
\curveto(121.80585061,72.23288334)(121.63723024,72.15349026)(121.42645477,72.15349026)
\curveto(121.20724828,72.15349026)(121.03019689,72.23288334)(120.89530059,72.39166949)
\curveto(120.76040429,72.55045565)(120.69295614,72.73849189)(120.69295614,72.95577821)
\curveto(120.69295614,73.17306453)(120.76040429,73.36527935)(120.89530059,73.53242268)
\curveto(121.03019689,73.699566)(121.20724828,73.78313766)(121.42645477,73.78313766)
\curveto(121.63723024,73.78313766)(121.80585061,73.699566)(121.93231589,73.53242268)
\curveto(122.06721219,73.36527935)(122.13466034,73.17306453)(122.13466034,72.95577821)
\closepath
\moveto(118.82126999,72.81788497)
\curveto(118.82126999,74.12160289)(119.04469199,75.12028425)(119.49153598,75.81392904)
\curveto(119.94681099,76.515931)(120.5833529,76.86693198)(121.40116171,76.86693198)
\curveto(122.22740155,76.86693198)(122.86394346,76.515931)(123.31078745,75.81392904)
\curveto(123.75763144,75.12028425)(123.98105344,74.12160289)(123.98105344,72.81788497)
\curveto(123.98105344,71.51416705)(123.75763144,70.51130711)(123.31078745,69.80930515)
\curveto(122.86394346,69.11566036)(122.22740155,68.76883796)(121.40116171,68.76883796)
\curveto(120.5833529,68.76883796)(119.94681099,69.11566036)(119.49153598,69.80930515)
\curveto(119.04469199,70.51130711)(118.82126999,71.51416705)(118.82126999,72.81788497)
\closepath
\moveto(122.91874508,72.81788497)
\curveto(122.91874508,73.24410044)(122.89345202,73.64524442)(122.84286591,74.0213169)
\curveto(122.7922798,74.40574654)(122.70796961,74.74003319)(122.58993535,75.02417684)
\curveto(122.47190109,75.30832049)(122.31592724,75.53396397)(122.12201381,75.7011073)
\curveto(121.92810038,75.86825062)(121.68781635,75.95182228)(121.40116171,75.95182228)
\curveto(121.11450708,75.95182228)(120.87422305,75.86825062)(120.68030962,75.7011073)
\curveto(120.48639619,75.53396397)(120.33042234,75.30832049)(120.21238808,75.02417684)
\curveto(120.09435382,74.74003319)(120.01004363,74.40574654)(119.95945752,74.0213169)
\curveto(119.9088714,73.64524442)(119.88357835,73.24410044)(119.88357835,72.81788497)
\curveto(119.88357835,72.39166949)(119.9088714,71.98634693)(119.95945752,71.60191729)
\curveto(120.01004363,71.22584481)(120.09435382,70.89573675)(120.21238808,70.6115931)
\curveto(120.33042234,70.32744945)(120.48639619,70.10180596)(120.68030962,69.93466264)
\curveto(120.87422305,69.76751932)(121.11450708,69.68394765)(121.40116171,69.68394765)
\curveto(121.68781635,69.68394765)(121.92810038,69.76751932)(122.12201381,69.93466264)
\curveto(122.31592724,70.10180596)(122.47190109,70.32744945)(122.58993535,70.6115931)
\curveto(122.70796961,70.89573675)(122.7922798,71.22584481)(122.84286591,71.60191729)
\curveto(122.89345202,71.98634693)(122.91874508,72.39166949)(122.91874508,72.81788497)
\closepath
}
}
{
\newrgbcolor{curcolor}{0 0 0}
\pscustom[linestyle=none,fillstyle=solid,fillcolor=curcolor]
{
\newpath
\moveto(131.8345416,75.08685558)
\curveto(132.26452356,75.25399891)(132.68185898,75.46292806)(133.08654788,75.71364305)
\curveto(133.49123678,75.9727152)(133.86641711,76.29864468)(134.21208888,76.69143149)
\lineto(134.9455875,76.69143149)
\lineto(134.9455875,69.80930515)
\lineto(136.42523128,69.80930515)
\lineto(136.42523128,68.9318027)
\lineto(132.21393744,68.9318027)
\lineto(132.21393744,69.80930515)
\lineto(133.9085722,69.80930515)
\lineto(133.9085722,75.24982032)
\curveto(133.815831,75.16624866)(133.70201224,75.07849842)(133.56711595,74.98656959)
\curveto(133.44065066,74.90299793)(133.29732335,74.81942627)(133.13713399,74.73585461)
\curveto(132.98537566,74.65228294)(132.8251863,74.57288986)(132.65656593,74.49767537)
\curveto(132.48794555,74.42246087)(132.32354069,74.35978213)(132.16335133,74.30963913)
\lineto(131.8345416,75.08685558)
\closepath
}
}
{
\newrgbcolor{curcolor}{0 0 0}
\pscustom[linestyle=none,fillstyle=solid,fillcolor=curcolor]
{
\newpath
\moveto(141.10444762,72.95577821)
\curveto(141.10444762,72.73849189)(141.03699947,72.55045565)(140.90210317,72.39166949)
\curveto(140.77563789,72.23288334)(140.60701751,72.15349026)(140.39624205,72.15349026)
\curveto(140.17703556,72.15349026)(139.99998417,72.23288334)(139.86508787,72.39166949)
\curveto(139.73019157,72.55045565)(139.66274342,72.73849189)(139.66274342,72.95577821)
\curveto(139.66274342,73.17306453)(139.73019157,73.36527935)(139.86508787,73.53242268)
\curveto(139.99998417,73.699566)(140.17703556,73.78313766)(140.39624205,73.78313766)
\curveto(140.60701751,73.78313766)(140.77563789,73.699566)(140.90210317,73.53242268)
\curveto(141.03699947,73.36527935)(141.10444762,73.17306453)(141.10444762,72.95577821)
\closepath
\moveto(137.79105727,72.81788497)
\curveto(137.79105727,74.12160289)(138.01447927,75.12028425)(138.46132326,75.81392904)
\curveto(138.91659827,76.515931)(139.55314018,76.86693198)(140.37094899,76.86693198)
\curveto(141.19718882,76.86693198)(141.83373073,76.515931)(142.28057473,75.81392904)
\curveto(142.72741872,75.12028425)(142.95084071,74.12160289)(142.95084071,72.81788497)
\curveto(142.95084071,71.51416705)(142.72741872,70.51130711)(142.28057473,69.80930515)
\curveto(141.83373073,69.11566036)(141.19718882,68.76883796)(140.37094899,68.76883796)
\curveto(139.55314018,68.76883796)(138.91659827,69.11566036)(138.46132326,69.80930515)
\curveto(138.01447927,70.51130711)(137.79105727,71.51416705)(137.79105727,72.81788497)
\closepath
\moveto(141.88853236,72.81788497)
\curveto(141.88853236,73.24410044)(141.8632393,73.64524442)(141.81265319,74.0213169)
\curveto(141.76206708,74.40574654)(141.67775689,74.74003319)(141.55972263,75.02417684)
\curveto(141.44168837,75.30832049)(141.28571452,75.53396397)(141.09180109,75.7011073)
\curveto(140.89788766,75.86825062)(140.65760363,75.95182228)(140.37094899,75.95182228)
\curveto(140.08429436,75.95182228)(139.84401032,75.86825062)(139.65009689,75.7011073)
\curveto(139.45618346,75.53396397)(139.30020962,75.30832049)(139.18217535,75.02417684)
\curveto(139.06414109,74.74003319)(138.97983091,74.40574654)(138.92924479,74.0213169)
\curveto(138.87865868,73.64524442)(138.85336563,73.24410044)(138.85336563,72.81788497)
\curveto(138.85336563,72.39166949)(138.87865868,71.98634693)(138.92924479,71.60191729)
\curveto(138.97983091,71.22584481)(139.06414109,70.89573675)(139.18217535,70.6115931)
\curveto(139.30020962,70.32744945)(139.45618346,70.10180596)(139.65009689,69.93466264)
\curveto(139.84401032,69.76751932)(140.08429436,69.68394765)(140.37094899,69.68394765)
\curveto(140.65760363,69.68394765)(140.89788766,69.76751932)(141.09180109,69.93466264)
\curveto(141.28571452,70.10180596)(141.44168837,70.32744945)(141.55972263,70.6115931)
\curveto(141.67775689,70.89573675)(141.76206708,71.22584481)(141.81265319,71.60191729)
\curveto(141.8632393,71.98634693)(141.88853236,72.39166949)(141.88853236,72.81788497)
\closepath
}
}
{
\newrgbcolor{curcolor}{0 0 0}
\pscustom[linestyle=none,fillstyle=solid,fillcolor=curcolor]
{
\newpath
\moveto(152.53691089,69.68394765)
\curveto(153.20296136,69.68394765)(153.67509841,69.81348373)(153.95332203,70.07255588)
\curveto(154.23997666,70.3399852)(154.38330398,70.69516476)(154.38330398,71.13809457)
\curveto(154.38330398,71.42223822)(154.32428685,71.66041745)(154.20625259,71.85263228)
\curveto(154.08821833,72.0448471)(153.93224448,72.19945467)(153.73833105,72.316455)
\curveto(153.54441762,72.43345532)(153.32099562,72.51702699)(153.06806506,72.56716998)
\curveto(152.8151345,72.61731298)(152.54955741,72.64238448)(152.2713338,72.64238448)
\lineto(152.00575671,72.64238448)
\lineto(152.00575671,73.48227968)
\lineto(152.37250602,73.48227968)
\curveto(152.55798843,73.48227968)(152.74768635,73.49899401)(152.94159978,73.53242268)
\curveto(153.14394423,73.57420851)(153.32521113,73.64106584)(153.48540049,73.73299466)
\curveto(153.65402086,73.83328066)(153.78891716,73.96699532)(153.89008939,74.13413864)
\curveto(153.99126161,74.30128196)(154.04184772,74.5143897)(154.04184772,74.77346185)
\curveto(154.04184772,75.19967733)(153.90695142,75.50053531)(153.63715883,75.6760358)
\curveto(153.37579725,75.85989346)(153.06806506,75.95182228)(152.71396228,75.95182228)
\curveto(152.35142847,75.95182228)(152.04369629,75.8975007)(151.79076573,75.78885754)
\curveto(151.53783517,75.68857155)(151.3270597,75.58410697)(151.15843933,75.47546381)
\lineto(150.75375043,76.26521601)
\curveto(150.93080182,76.39057351)(151.19637891,76.52010958)(151.5504817,76.65382424)
\curveto(151.9130155,76.79589607)(152.31348889,76.86693198)(152.75190186,76.86693198)
\curveto(153.16502178,76.86693198)(153.51912456,76.81678898)(153.81421022,76.71650299)
\curveto(154.10929587,76.61621699)(154.3495799,76.47414517)(154.53506232,76.29028751)
\curveto(154.72897575,76.10642986)(154.87230306,75.88914354)(154.96504427,75.63842855)
\curveto(155.05778548,75.39607073)(155.10415608,75.12864142)(155.10415608,74.8361406)
\curveto(155.10415608,74.42663946)(154.99455284,74.07981706)(154.77534635,73.79567341)
\curveto(154.56457088,73.51152976)(154.29056277,73.29424344)(153.95332203,73.14381445)
\curveto(154.35801092,73.02681412)(154.7078982,72.79699205)(155.00298385,72.45434824)
\curveto(155.29806951,72.12006159)(155.44561234,71.6729532)(155.44561234,71.11302307)
\curveto(155.44561234,70.77873642)(155.3865952,70.46534269)(155.26856094,70.17284188)
\curveto(155.1589577,69.88869823)(154.98612182,69.64216182)(154.75005329,69.43323267)
\curveto(154.52241579,69.22430352)(154.22311462,69.06133877)(153.8521498,68.94433845)
\curveto(153.489616,68.82733812)(153.05541854,68.76883796)(152.54955741,68.76883796)
\curveto(152.35564398,68.76883796)(152.15329953,68.78555229)(151.94252407,68.81898096)
\curveto(151.74017962,68.84405245)(151.5504817,68.8816597)(151.37343031,68.9318027)
\curveto(151.19637891,68.97358853)(151.03618956,69.01537436)(150.89286224,69.05716019)
\curveto(150.75796594,69.10730319)(150.66100923,69.14491044)(150.60199209,69.16998194)
\lineto(150.80433654,70.06002013)
\curveto(150.93923284,69.9931628)(151.15422382,69.91376972)(151.44930947,69.8218409)
\curveto(151.74439513,69.72991207)(152.10692893,69.68394765)(152.53691089,69.68394765)
\closepath
}
}
{
\newrgbcolor{curcolor}{0 0 0}
\pscustom[linestyle=none,fillstyle=solid,fillcolor=curcolor]
{
\newpath
\moveto(158.13932377,68.9318027)
\curveto(158.18147886,69.5251615)(158.28686659,70.15194896)(158.45548697,70.81216509)
\curveto(158.63253836,71.48073838)(158.84331383,72.12424018)(159.08781337,72.74267047)
\curveto(159.33231291,73.36945794)(159.60210551,73.9461024)(159.89719116,74.47260387)
\curveto(160.19227682,75.00746251)(160.48314696,75.44621373)(160.7698016,75.78885754)
\lineto(156.97584319,75.78885754)
\lineto(156.97584319,76.69143149)
\lineto(161.93328218,76.69143149)
\lineto(161.93328218,75.82646479)
\curveto(161.68035162,75.53396397)(161.40634351,75.14117716)(161.11125786,74.64810436)
\curveto(160.8161722,74.15503156)(160.53373308,73.59928001)(160.26394048,72.98084971)
\curveto(160.0025789,72.37077658)(159.77494139,71.71473903)(159.58102796,71.01273708)
\curveto(159.38711453,70.31909228)(159.26486476,69.62544749)(159.21427865,68.9318027)
\lineto(158.13932377,68.9318027)
\closepath
}
}
{
\newrgbcolor{curcolor}{0 0 0}
\pscustom[linestyle=none,fillstyle=solid,fillcolor=curcolor]
{
\newpath
\moveto(180.8019049,73.58256567)
\curveto(180.8019049,72.03648993)(180.42250905,70.87066525)(179.66371737,70.08509163)
\curveto(178.90492569,69.30787518)(177.76252266,68.91508837)(176.23650827,68.9067312)
\lineto(176.19856869,69.78423365)
\curveto(177.14284278,69.78423365)(177.90163446,69.9680913)(178.47494374,70.33580662)
\curveto(179.05668403,70.71187909)(179.44029538,71.35538089)(179.62577779,72.266312)
\curveto(179.42343334,72.17438317)(179.20001134,72.09916868)(178.9555118,72.04066851)
\curveto(178.71101226,71.99052552)(178.45386619,71.96545402)(178.18407359,71.96545402)
\curveto(177.7372296,71.96545402)(177.36204927,72.02813277)(177.05853259,72.15349026)
\curveto(176.75501592,72.28720492)(176.51051638,72.46270541)(176.32503397,72.67999173)
\curveto(176.13955156,72.90563521)(176.00465526,73.16052878)(175.92034507,73.44467243)
\curveto(175.83603488,73.72881608)(175.79387979,74.02967406)(175.79387979,74.34724638)
\curveto(175.79387979,74.63139003)(175.84025039,74.91971226)(175.9329916,75.21221308)
\curveto(176.0257328,75.51307106)(176.16906012,75.78467896)(176.36297355,76.02703678)
\curveto(176.55688698,76.2693946)(176.80560203,76.46996659)(177.10911871,76.62875274)
\curveto(177.41263538,76.7875389)(177.77516918,76.86693198)(178.19672012,76.86693198)
\curveto(179.05668403,76.86693198)(179.70587246,76.57443116)(180.14428544,75.98942953)
\curveto(180.58269841,75.4044279)(180.8019049,74.60213995)(180.8019049,73.58256567)
\closepath
\moveto(178.29789234,72.81788497)
\curveto(178.56768494,72.81788497)(178.81639999,72.84295647)(179.0440375,72.89309946)
\curveto(179.28010602,72.94324246)(179.50774353,73.01427837)(179.72695001,73.1062072)
\curveto(179.73538103,73.18977886)(179.73959654,73.26917194)(179.73959654,73.34438644)
\lineto(179.73959654,73.58256567)
\curveto(179.73959654,73.90849515)(179.71430348,74.2177103)(179.66371737,74.51021112)
\curveto(179.62156228,74.80271193)(179.5414676,75.0576055)(179.42343334,75.27489182)
\curveto(179.3138301,75.49217814)(179.15785625,75.66350005)(178.9555118,75.78885754)
\curveto(178.76159837,75.9225722)(178.51288332,75.98942953)(178.20936665,75.98942953)
\curveto(177.95643609,75.98942953)(177.74566062,75.93510795)(177.57704024,75.82646479)
\curveto(177.40841987,75.7261788)(177.26930806,75.59664272)(177.15970482,75.43785656)
\curveto(177.05010158,75.28742757)(176.9700069,75.11610567)(176.91942079,74.92389084)
\curveto(176.87726569,74.73167602)(176.85618815,74.54781837)(176.85618815,74.37231788)
\curveto(176.85618815,73.86253074)(176.9700069,73.47392251)(177.1976444,73.2064932)
\curveto(177.43371293,72.94742104)(177.80046224,72.81788497)(178.29789234,72.81788497)
\closepath
}
}
{
\newrgbcolor{curcolor}{0 0 0}
\pscustom[linestyle=none,fillstyle=solid,fillcolor=curcolor]
{
\newpath
\moveto(199.79699289,70.95005833)
\curveto(199.79699289,70.29819937)(199.58621742,69.7716979)(199.16466649,69.37055392)
\curveto(198.75154657,68.96940995)(198.11922017,68.76883796)(197.26768728,68.76883796)
\curveto(196.7786882,68.76883796)(196.3739993,68.83151671)(196.05362059,68.9568742)
\curveto(195.73324188,69.09058886)(195.47609581,69.25773218)(195.28218238,69.45830417)
\curveto(195.09669997,69.66723332)(194.96180367,69.89705539)(194.87749348,70.14777038)
\curveto(194.80161431,70.39848536)(194.76367473,70.64502176)(194.76367473,70.88737958)
\curveto(194.76367473,71.33030939)(194.8859245,71.7230962)(195.13042404,72.06574001)
\curveto(195.37492359,72.40838383)(195.66579373,72.68417031)(196.00303448,72.89309946)
\curveto(195.28639789,73.29424344)(194.92807959,73.90849515)(194.92807959,74.73585461)
\curveto(194.92807959,75.01999825)(194.98288122,75.29160616)(195.09248446,75.55067831)
\curveto(195.2020877,75.80975046)(195.35806155,76.03539394)(195.560406,76.22760877)
\curveto(195.76275045,76.41982359)(196.00724999,76.57443116)(196.29390462,76.69143149)
\curveto(196.58899028,76.80843182)(196.91780001,76.86693198)(197.28033381,76.86693198)
\curveto(197.70188475,76.86693198)(198.06020304,76.80425323)(198.35528869,76.67889574)
\curveto(198.65880537,76.55353825)(198.90330491,76.39057351)(199.08878732,76.19000152)
\curveto(199.27426973,75.9977867)(199.40916603,75.78050038)(199.49347622,75.53814256)
\curveto(199.57778641,75.3041419)(199.6199415,75.07431984)(199.6199415,74.84867635)
\curveto(199.6199415,74.40574654)(199.50612275,74.0213169)(199.27848524,73.69538742)
\curveto(199.05084774,73.3778151)(198.78948616,73.12292153)(198.4944005,72.93070671)
\curveto(199.36279543,72.52120557)(199.79699289,71.86098944)(199.79699289,70.95005833)
\closepath
\moveto(195.77539697,70.87484383)
\curveto(195.77539697,70.74112918)(195.80069003,70.59905735)(195.85127614,70.44862836)
\curveto(195.90186225,70.30655653)(195.98617244,70.17284188)(196.1042067,70.04748438)
\curveto(196.22224096,69.92212689)(196.37821481,69.81766231)(196.57212824,69.73409065)
\curveto(196.76604167,69.65887616)(197.00211019,69.62126891)(197.28033381,69.62126891)
\curveto(197.54169539,69.62126891)(197.76511739,69.65469757)(197.9505998,69.7215549)
\curveto(198.14451323,69.7967694)(198.30048707,69.89287681)(198.41852133,70.00987714)
\curveto(198.54498661,70.13523463)(198.63772782,70.27312787)(198.69674495,70.42355686)
\curveto(198.75576208,70.57398585)(198.78527065,70.72441484)(198.78527065,70.87484383)
\curveto(198.78527065,71.35120231)(198.61243476,71.71473903)(198.266763,71.96545402)
\curveto(197.92109123,72.22452617)(197.44473868,72.42091958)(196.83770533,72.55463423)
\curveto(196.50046458,72.37077658)(196.239103,72.14095451)(196.05362059,71.86516803)
\curveto(195.86813818,71.58938154)(195.77539697,71.25927348)(195.77539697,70.87484383)
\closepath
\moveto(198.60821926,74.84867635)
\curveto(198.60821926,74.95731951)(198.5829262,75.07849842)(198.53234009,75.21221308)
\curveto(198.48175397,75.3542849)(198.4016593,75.47964239)(198.29205605,75.58828555)
\curveto(198.19088383,75.70528588)(198.05598753,75.80139329)(197.88736716,75.87660779)
\curveto(197.71874678,75.96017945)(197.51640233,76.00196528)(197.28033381,76.00196528)
\curveto(197.03583427,76.00196528)(196.82927431,75.96435803)(196.66065394,75.88914354)
\curveto(196.50046458,75.81392904)(196.36556828,75.71782163)(196.25596504,75.6008213)
\curveto(196.1463618,75.49217814)(196.06626712,75.36682065)(196.01568101,75.22474883)
\curveto(195.97352591,75.09103417)(195.95244837,74.95731951)(195.95244837,74.82360485)
\curveto(195.95244837,74.48096104)(196.07469814,74.15921014)(196.31919768,73.85835216)
\curveto(196.57212824,73.56585134)(196.98524816,73.3527436)(197.55855743,73.21902895)
\curveto(197.87893614,73.4028866)(198.1318667,73.62017292)(198.31734911,73.87088791)
\curveto(198.51126254,74.12160289)(198.60821926,74.44753237)(198.60821926,74.84867635)
\closepath
}
}
{
\newrgbcolor{curcolor}{0 0 0}
\pscustom[linestyle=none,fillstyle=solid,fillcolor=curcolor]
{
\newpath
\moveto(215.04870092,68.9318027)
\curveto(215.09085602,69.5251615)(215.19624375,70.15194896)(215.36486412,70.81216509)
\curveto(215.54191552,71.48073838)(215.75269098,72.12424018)(215.99719053,72.74267047)
\curveto(216.24169007,73.36945794)(216.51148267,73.9461024)(216.80656832,74.47260387)
\curveto(217.10165398,75.00746251)(217.39252412,75.44621373)(217.67917876,75.78885754)
\lineto(213.88522034,75.78885754)
\lineto(213.88522034,76.69143149)
\lineto(218.84265934,76.69143149)
\lineto(218.84265934,75.82646479)
\curveto(218.58972878,75.53396397)(218.31572067,75.14117716)(218.02063501,74.64810436)
\curveto(217.72554936,74.15503156)(217.44311023,73.59928001)(217.17331763,72.98084971)
\curveto(216.91195605,72.37077658)(216.68431855,71.71473903)(216.49040512,71.01273708)
\curveto(216.29649169,70.31909228)(216.17424192,69.62544749)(216.12365581,68.9318027)
\lineto(215.04870092,68.9318027)
\closepath
}
}
{
\newrgbcolor{curcolor}{0 0 0}
\pscustom[linestyle=none,fillstyle=solid,fillcolor=curcolor]
{
\newpath
\moveto(236.15575144,69.73409065)
\curveto(236.15575144,69.48337567)(236.07144125,69.26191076)(235.90282088,69.06969594)
\curveto(235.7342005,68.87748112)(235.51077851,68.78137371)(235.23255489,68.78137371)
\curveto(234.94590026,68.78137371)(234.71826275,68.87748112)(234.54964238,69.06969594)
\curveto(234.381022,69.26191076)(234.29671182,69.48337567)(234.29671182,69.73409065)
\curveto(234.29671182,69.9931628)(234.381022,70.21880629)(234.54964238,70.41102111)
\curveto(234.71826275,70.60323593)(234.94590026,70.69934334)(235.23255489,70.69934334)
\curveto(235.51077851,70.69934334)(235.7342005,70.60323593)(235.90282088,70.41102111)
\curveto(236.07144125,70.21880629)(236.15575144,69.9931628)(236.15575144,69.73409065)
\closepath
}
}
{
\newrgbcolor{curcolor}{0 0 0}
\pscustom[linestyle=none,fillstyle=solid,fillcolor=curcolor]
{
\newpath
\moveto(242.47902408,69.73409065)
\curveto(242.47902408,69.48337567)(242.3947139,69.26191076)(242.22609352,69.06969594)
\curveto(242.05747315,68.87748112)(241.83405115,68.78137371)(241.55582754,68.78137371)
\curveto(241.2691729,68.78137371)(241.04153539,68.87748112)(240.87291502,69.06969594)
\curveto(240.70429465,69.26191076)(240.61998446,69.48337567)(240.61998446,69.73409065)
\curveto(240.61998446,69.9931628)(240.70429465,70.21880629)(240.87291502,70.41102111)
\curveto(241.04153539,70.60323593)(241.2691729,70.69934334)(241.55582754,70.69934334)
\curveto(241.83405115,70.69934334)(242.05747315,70.60323593)(242.22609352,70.41102111)
\curveto(242.3947139,70.21880629)(242.47902408,69.9931628)(242.47902408,69.73409065)
\closepath
}
}
{
\newrgbcolor{curcolor}{0 0 0}
\pscustom[linestyle=none,fillstyle=solid,fillcolor=curcolor]
{
\newpath
\moveto(248.8022814,69.73409065)
\curveto(248.8022814,69.48337567)(248.71797121,69.26191076)(248.54935084,69.06969594)
\curveto(248.38073046,68.87748112)(248.15730847,68.78137371)(247.87908485,68.78137371)
\curveto(247.59243022,68.78137371)(247.36479271,68.87748112)(247.19617234,69.06969594)
\curveto(247.02755196,69.26191076)(246.94324178,69.48337567)(246.94324178,69.73409065)
\curveto(246.94324178,69.9931628)(247.02755196,70.21880629)(247.19617234,70.41102111)
\curveto(247.36479271,70.60323593)(247.59243022,70.69934334)(247.87908485,70.69934334)
\curveto(248.15730847,70.69934334)(248.38073046,70.60323593)(248.54935084,70.41102111)
\curveto(248.71797121,70.21880629)(248.8022814,69.9931628)(248.8022814,69.73409065)
\closepath
}
}
{
\newrgbcolor{curcolor}{0 0 0}
\pscustom[linestyle=none,fillstyle=solid,fillcolor=curcolor]
{
\newpath
\moveto(266.02684781,73.69538742)
\curveto(267.15660431,73.65360159)(267.97862863,73.40706518)(268.49292077,72.95577821)
\curveto(269.00721292,72.50449124)(269.26435899,71.89859669)(269.26435899,71.13809457)
\curveto(269.26435899,70.79545076)(269.20955736,70.47787844)(269.09995412,70.18537762)
\curveto(268.99035088,69.89287681)(268.81751499,69.64216182)(268.58144647,69.43323267)
\curveto(268.35380897,69.22430352)(268.05872331,69.06133877)(267.69618951,68.94433845)
\curveto(267.34208672,68.82733812)(266.92053579,68.76883796)(266.4315367,68.76883796)
\curveto(266.22919225,68.76883796)(266.02684781,68.78555229)(265.82450336,68.81898096)
\curveto(265.63058993,68.84405245)(265.44510752,68.87748112)(265.26805612,68.91926695)
\curveto(265.09100473,68.96105278)(264.93503088,69.00283861)(264.80013459,69.04462444)
\curveto(264.66523829,69.09476744)(264.56828157,69.13655327)(264.50926444,69.16998194)
\lineto(264.71160889,70.06002013)
\curveto(264.84650519,69.9931628)(265.05306515,69.91376972)(265.33128876,69.8218409)
\curveto(265.6179434,69.72991207)(265.97626169,69.68394765)(266.40624365,69.68394765)
\curveto(266.7434844,69.68394765)(267.02592352,69.7215549)(267.25356103,69.7967694)
\curveto(267.48119853,69.87198389)(267.66246543,69.97226989)(267.79736173,70.09762738)
\curveto(267.94068905,70.23134204)(268.04186127,70.38177103)(268.10087841,70.54891435)
\curveto(268.16832655,70.71605768)(268.20205063,70.89155817)(268.20205063,71.07541582)
\curveto(268.20205063,71.35955947)(268.15146452,71.61027446)(268.05029229,71.82756078)
\curveto(267.95755109,72.05320426)(267.78893071,72.2412405)(267.54443117,72.39166949)
\curveto(267.30836265,72.54209849)(266.97955292,72.65492023)(266.55800198,72.73013472)
\curveto(266.14488207,72.81370639)(265.62215891,72.85549222)(264.98983251,72.85549222)
\curveto(265.04041862,73.22320753)(265.0783582,73.56585134)(265.10365126,73.88342366)
\curveto(265.13737533,74.20935314)(265.16266839,74.52274687)(265.17953043,74.82360485)
\curveto(265.20482348,75.13282)(265.22168552,75.43785656)(265.23011654,75.73871455)
\curveto(265.24697858,76.03957253)(265.26384061,76.35714484)(265.28070265,76.69143149)
\lineto(269.04936801,76.69143149)
\lineto(269.04936801,75.81392904)
\lineto(266.19125267,75.81392904)
\curveto(266.18282165,75.69692871)(266.17017512,75.54232114)(266.15331309,75.35010632)
\curveto(266.14488207,75.16624866)(266.13223554,74.96985526)(266.1153735,74.7609261)
\curveto(266.09851146,74.55199695)(266.08164943,74.35142496)(266.06478739,74.15921014)
\curveto(266.04792535,73.96699532)(266.03527882,73.81238774)(266.02684781,73.69538742)
\closepath
}
}
{
\newrgbcolor{curcolor}{0 0 0}
\pscustom[linestyle=none,fillstyle=solid,fillcolor=curcolor]
{
\newpath
\moveto(275.30940801,74.71078311)
\curveto(275.30940801,74.44335379)(275.25460639,74.18428164)(275.14500315,73.93356665)
\curveto(275.04383092,73.68285167)(274.90471911,73.43631527)(274.72766772,73.19395745)
\curveto(274.55904735,72.95159963)(274.36513392,72.71342039)(274.14592743,72.47941974)
\curveto(273.92672095,72.24541909)(273.70329895,72.01559702)(273.47566144,71.78995353)
\curveto(273.34919616,71.66459604)(273.20165334,71.51416705)(273.03303296,71.33866656)
\curveto(272.86441259,71.16316607)(272.70422323,70.98348699)(272.5524649,70.79962934)
\curveto(272.40070656,70.61577168)(272.27424128,70.43609261)(272.17306906,70.26059212)
\curveto(272.07189683,70.08509163)(272.02131072,69.93466264)(272.02131072,69.80930515)
\lineto(275.62557121,69.80930515)
\lineto(275.62557121,68.9318027)
\lineto(270.8831232,68.9318027)
\curveto(270.87469218,68.97358853)(270.87047667,69.01537436)(270.87047667,69.05716019)
\lineto(270.87047667,69.19505343)
\curveto(270.87047667,69.54605441)(270.9294938,69.87198389)(271.04752806,70.17284188)
\curveto(271.16556232,70.47369986)(271.31732066,70.75784351)(271.50280307,71.02527283)
\curveto(271.68828548,71.29270214)(271.89484544,71.54341713)(272.12248294,71.77741778)
\curveto(272.35855147,72.0197756)(272.59040448,72.25377625)(272.81804199,72.47941974)
\curveto(273.0035244,72.66327739)(273.18057579,72.84295647)(273.34919616,73.01845696)
\curveto(273.52624756,73.19395745)(273.67800589,73.36945794)(273.80447117,73.54495843)
\curveto(273.93936747,73.72045892)(274.04475521,73.90013799)(274.12063438,74.08399564)
\curveto(274.20494456,74.27621047)(274.24709966,74.47260387)(274.24709966,74.67317586)
\curveto(274.24709966,74.89881935)(274.20916007,75.09103417)(274.1332809,75.24982032)
\curveto(274.06583275,75.40860648)(273.96887604,75.53814256)(273.84241076,75.63842855)
\curveto(273.7243765,75.74707171)(273.58526469,75.82646479)(273.42507533,75.87660779)
\curveto(273.26488598,75.92675078)(273.09205009,75.95182228)(272.90656768,75.95182228)
\curveto(272.6873612,75.95182228)(272.48501675,75.9225722)(272.29953434,75.86407204)
\curveto(272.12248294,75.80557188)(271.96229359,75.73453596)(271.81896627,75.6509643)
\curveto(271.68406997,75.56739264)(271.56603571,75.48382098)(271.46486349,75.40024932)
\curveto(271.36369126,75.32503482)(271.28781209,75.26235607)(271.23722598,75.21221308)
\lineto(270.71871833,75.93928653)
\curveto(270.78616648,76.01450103)(270.88733871,76.10642986)(271.022235,76.21507302)
\curveto(271.1571313,76.32371618)(271.31732066,76.42400217)(271.50280307,76.515931)
\curveto(271.6967165,76.61621699)(271.91170748,76.69978866)(272.147776,76.76664598)
\curveto(272.38384452,76.83350331)(272.63677508,76.86693198)(272.90656768,76.86693198)
\curveto(273.7243765,76.86693198)(274.32719433,76.67889574)(274.71502119,76.30282326)
\curveto(275.11127907,75.93510795)(275.30940801,75.4044279)(275.30940801,74.71078311)
\closepath
}
}
{
\newrgbcolor{curcolor}{0 0 0}
\pscustom[linewidth=0.56097835,linecolor=curcolor]
{
\newpath
\moveto(52.53490992,79.17745972)
\lineto(277.23287745,79.17745972)
\lineto(277.23287745,66.45834065)
\lineto(52.53490992,66.45834065)
\closepath
}
}
{
\newrgbcolor{curcolor}{0 0 0}
\pscustom[linewidth=0.51800942,linecolor=curcolor]
{
\newpath
\moveto(71.3882275,66.47244933)
\lineto(71.3882275,79.07184433)
}
}
{
\newrgbcolor{curcolor}{0 0 0}
\pscustom[linewidth=0.5154072,linecolor=curcolor]
{
\newpath
\moveto(90.2684505,66.56676033)
\lineto(90.2684505,79.03988633)
}
}
{
\newrgbcolor{curcolor}{0 0 0}
\pscustom[linewidth=0.51930571,linecolor=curcolor]
{
\newpath
\moveto(109.1077405,66.56675933)
\lineto(109.1077405,79.22928833)
}
}
{
\newrgbcolor{curcolor}{0 0 0}
\pscustom[linewidth=0.52573889,linecolor=curcolor]
{
\newpath
\moveto(128.0363105,66.38818833)
\lineto(128.0363105,79.36639033)
}
}
{
\newrgbcolor{curcolor}{0 0 0}
\pscustom[linewidth=0.51670998,linecolor=curcolor]
{
\newpath
\moveto(146.8755905,66.56675933)
\lineto(146.8755905,79.10301933)
}
}
{
\newrgbcolor{curcolor}{0 0 0}
\pscustom[linewidth=0.52059871,linecolor=curcolor]
{
\newpath
\moveto(165.6256005,66.56676033)
\lineto(165.6256005,79.29242433)
}
}
{
\newrgbcolor{curcolor}{0 0 0}
\pscustom[linewidth=0.51670998,linecolor=curcolor]
{
\newpath
\moveto(184.5541705,66.47747433)
\lineto(184.5541705,79.01373433)
}
}
{
\newrgbcolor{curcolor}{0 0 0}
\pscustom[linewidth=0.51670998,linecolor=curcolor]
{
\newpath
\moveto(203.6613105,66.56676033)
\lineto(203.6613105,79.10302033)
}
}
{
\newrgbcolor{curcolor}{0 0 0}
\pscustom[linewidth=0.52059871,linecolor=curcolor]
{
\newpath
\moveto(222.5898805,66.56675933)
\lineto(222.5898805,79.29242333)
}
}
{
\newrgbcolor{curcolor}{0 0 0}
\pscustom[linewidth=0.52573889,linecolor=curcolor]
{
\newpath
\moveto(258.2401905,79.21853433)
\lineto(258.2046905,66.24038033)
}
}
{
\newrgbcolor{curcolor}{0 1 0}
\pscustom[linestyle=none,fillstyle=solid,fillcolor=curcolor]
{
\newpath
\moveto(184.67776624,57.13225034)
\lineto(203.52573529,57.13225034)
\lineto(203.52573529,44.31593945)
\lineto(184.67776624,44.31593945)
\closepath
}
}
{
\newrgbcolor{curcolor}{0 0 0}
\pscustom[linestyle=none,fillstyle=solid,fillcolor=curcolor]
{
\newpath
\moveto(62.27863542,53.02759206)
\curveto(62.70861738,53.19473539)(63.1259528,53.40366454)(63.5306417,53.65437953)
\curveto(63.9353306,53.91345168)(64.31051093,54.23938116)(64.6561827,54.63216797)
\lineto(65.38968132,54.63216797)
\lineto(65.38968132,47.75004163)
\lineto(66.8693251,47.75004163)
\lineto(66.8693251,46.87253918)
\lineto(62.65803126,46.87253918)
\lineto(62.65803126,47.75004163)
\lineto(64.35266602,47.75004163)
\lineto(64.35266602,53.1905568)
\curveto(64.25992482,53.10698514)(64.14610606,53.0192349)(64.01120976,52.92730607)
\curveto(63.88474448,52.84373441)(63.74141717,52.76016275)(63.58122781,52.67659108)
\curveto(63.42946947,52.59301942)(63.26928012,52.51362634)(63.10065975,52.43841185)
\curveto(62.93203937,52.36319735)(62.76763451,52.30051861)(62.60744515,52.25037561)
\lineto(62.27863542,53.02759206)
\closepath
}
}
{
\newrgbcolor{curcolor}{0 0 0}
\pscustom[linestyle=none,fillstyle=solid,fillcolor=curcolor]
{
\newpath
\moveto(76.32892712,51.6361239)
\curveto(77.45868363,51.59433806)(78.28070795,51.34780166)(78.79500009,50.89651469)
\curveto(79.30929223,50.44522772)(79.5664383,49.83933317)(79.5664383,49.07883105)
\curveto(79.5664383,48.73618723)(79.51163668,48.41861492)(79.40203344,48.1261141)
\curveto(79.29243019,47.83361329)(79.11959431,47.5828983)(78.88352579,47.37396915)
\curveto(78.65588828,47.16503999)(78.36080263,47.00207525)(77.99826882,46.88507493)
\curveto(77.64416604,46.7680746)(77.2226151,46.70957444)(76.73361602,46.70957444)
\curveto(76.53127157,46.70957444)(76.32892712,46.72628877)(76.12658267,46.75971743)
\curveto(75.93266924,46.78478893)(75.74718683,46.8182176)(75.57013544,46.86000343)
\curveto(75.39308405,46.90178926)(75.2371102,46.94357509)(75.1022139,46.98536092)
\curveto(74.9673176,47.03550392)(74.87036089,47.07728975)(74.81134376,47.11071841)
\lineto(75.01368821,48.00075661)
\curveto(75.1485845,47.93389928)(75.35514446,47.8545062)(75.63336808,47.76257738)
\curveto(75.92002272,47.67064855)(76.27834101,47.62468413)(76.70832296,47.62468413)
\curveto(77.04556371,47.62468413)(77.32800284,47.66229138)(77.55564034,47.73750588)
\curveto(77.78327785,47.81272037)(77.96454475,47.91300637)(78.09944105,48.03836386)
\curveto(78.24276837,48.17207852)(78.34394059,48.32250751)(78.40295772,48.48965083)
\curveto(78.47040587,48.65679416)(78.50412995,48.83229465)(78.50412995,49.0161523)
\curveto(78.50412995,49.30029595)(78.45354383,49.55101094)(78.35237161,49.76829726)
\curveto(78.2596304,49.99394074)(78.09101003,50.18197698)(77.84651049,50.33240597)
\curveto(77.61044196,50.48283496)(77.28163223,50.59565671)(76.8600813,50.6708712)
\curveto(76.44696138,50.75444286)(75.92423822,50.7962287)(75.29191182,50.7962287)
\curveto(75.34249793,51.16394401)(75.38043752,51.50658782)(75.40573057,51.82416013)
\curveto(75.43945465,52.15008962)(75.46474771,52.46348335)(75.48160974,52.76434133)
\curveto(75.5069028,53.07355648)(75.52376484,53.37859304)(75.53219586,53.67945102)
\curveto(75.54905789,53.98030901)(75.56591993,54.29788132)(75.58278197,54.63216797)
\lineto(79.35144732,54.63216797)
\lineto(79.35144732,53.75466552)
\lineto(76.49333199,53.75466552)
\curveto(76.48490097,53.63766519)(76.47225444,53.48305762)(76.4553924,53.2908428)
\curveto(76.44696138,53.10698514)(76.43431486,52.91059174)(76.41745282,52.70166258)
\curveto(76.40059078,52.49273343)(76.38372874,52.29216144)(76.36686671,52.09994662)
\curveto(76.35000467,51.9077318)(76.33735814,51.75312422)(76.32892712,51.6361239)
\closepath
}
}
{
\newrgbcolor{curcolor}{0 0 0}
\pscustom[linestyle=none,fillstyle=solid,fillcolor=curcolor]
{
\newpath
\moveto(81.24842749,53.02759206)
\curveto(81.67840944,53.19473539)(82.09574487,53.40366454)(82.50043377,53.65437953)
\curveto(82.90512266,53.91345168)(83.28030299,54.23938116)(83.62597476,54.63216797)
\lineto(84.35947339,54.63216797)
\lineto(84.35947339,47.75004163)
\lineto(85.83911717,47.75004163)
\lineto(85.83911717,46.87253918)
\lineto(81.62782333,46.87253918)
\lineto(81.62782333,47.75004163)
\lineto(83.32245809,47.75004163)
\lineto(83.32245809,53.1905568)
\curveto(83.22971688,53.10698514)(83.11589813,53.0192349)(82.98100183,52.92730607)
\curveto(82.85453655,52.84373441)(82.71120923,52.76016275)(82.55101988,52.67659108)
\curveto(82.39926154,52.59301942)(82.23907219,52.51362634)(82.07045181,52.43841185)
\curveto(81.90183144,52.36319735)(81.73742657,52.30051861)(81.57723722,52.25037561)
\lineto(81.24842749,53.02759206)
\closepath
}
}
{
\newrgbcolor{curcolor}{0 0 0}
\pscustom[linestyle=none,fillstyle=solid,fillcolor=curcolor]
{
\newpath
\moveto(93.89495362,53.02759206)
\curveto(94.32493557,53.19473539)(94.742271,53.40366454)(95.14695989,53.65437953)
\curveto(95.55164879,53.91345168)(95.92682912,54.23938116)(96.27250089,54.63216797)
\lineto(97.00599952,54.63216797)
\lineto(97.00599952,47.75004163)
\lineto(98.4856433,47.75004163)
\lineto(98.4856433,46.87253918)
\lineto(94.27434946,46.87253918)
\lineto(94.27434946,47.75004163)
\lineto(95.96898422,47.75004163)
\lineto(95.96898422,53.1905568)
\curveto(95.87624301,53.10698514)(95.76242426,53.0192349)(95.62752796,52.92730607)
\curveto(95.50106268,52.84373441)(95.35773536,52.76016275)(95.19754601,52.67659108)
\curveto(95.04578767,52.59301942)(94.88559831,52.51362634)(94.71697794,52.43841185)
\curveto(94.54835757,52.36319735)(94.3839527,52.30051861)(94.22376335,52.25037561)
\lineto(93.89495362,53.02759206)
\closepath
}
}
{
\newrgbcolor{curcolor}{0 0 0}
\pscustom[linestyle=none,fillstyle=solid,fillcolor=curcolor]
{
\newpath
\moveto(99.725004,49.55518952)
\curveto(99.86833132,49.88947617)(100.06224475,50.27390581)(100.30674429,50.70847845)
\curveto(100.55967485,51.14305109)(100.84211398,51.59015948)(101.15406167,52.04980362)
\curveto(101.46600936,52.51780493)(101.7990346,52.97327048)(102.15313739,53.41620029)
\curveto(102.50724017,53.86748726)(102.86555847,54.27280982)(103.22809227,54.63216797)
\lineto(104.23981452,54.63216797)
\lineto(104.23981452,49.70561851)
\lineto(105.16301106,49.70561851)
\lineto(105.16301106,48.85318756)
\lineto(104.23981452,48.85318756)
\lineto(104.23981452,46.87253918)
\lineto(103.22809227,46.87253918)
\lineto(103.22809227,48.85318756)
\lineto(99.725004,48.85318756)
\lineto(99.725004,49.55518952)
\closepath
\moveto(103.22809227,53.40366454)
\curveto(103.00045477,53.16130672)(102.76860175,52.8938774)(102.53253323,52.60137659)
\curveto(102.30489572,52.30887577)(102.08147373,51.99966062)(101.86226724,51.67373114)
\curveto(101.64306076,51.35615883)(101.4365008,51.03022935)(101.24258737,50.6959427)
\curveto(101.05710496,50.36165605)(100.88848458,50.03154799)(100.73672625,49.70561851)
\lineto(103.22809227,49.70561851)
\lineto(103.22809227,53.40366454)
\closepath
}
}
{
\newrgbcolor{curcolor}{0 0 0}
\pscustom[linestyle=none,fillstyle=solid,fillcolor=curcolor]
{
\newpath
\moveto(117.22780073,52.65151959)
\curveto(117.22780073,52.38409027)(117.17299911,52.12501812)(117.06339587,51.87430313)
\curveto(116.96222364,51.62358815)(116.82311184,51.37705174)(116.64606044,51.13469393)
\curveto(116.47744007,50.89233611)(116.28352664,50.65415687)(116.06432015,50.42015622)
\curveto(115.84511367,50.18615557)(115.62169167,49.9563335)(115.39405417,49.73069001)
\curveto(115.26758889,49.60533252)(115.12004606,49.45490353)(114.95142568,49.27940304)
\curveto(114.78280531,49.10390255)(114.62261596,48.92422347)(114.47085762,48.74036582)
\curveto(114.31909928,48.55650816)(114.192634,48.37682909)(114.09146178,48.2013286)
\curveto(113.99028955,48.02582811)(113.93970344,47.87539912)(113.93970344,47.75004163)
\lineto(117.54396393,47.75004163)
\lineto(117.54396393,46.87253918)
\lineto(112.80151592,46.87253918)
\curveto(112.7930849,46.91432501)(112.78886939,46.95611084)(112.78886939,46.99789667)
\lineto(112.78886939,47.13578991)
\curveto(112.78886939,47.48679089)(112.84788652,47.81272037)(112.96592078,48.11357835)
\curveto(113.08395504,48.41443634)(113.23571338,48.69857999)(113.42119579,48.9660093)
\curveto(113.6066782,49.23343862)(113.81323816,49.48415361)(114.04087567,49.71815426)
\curveto(114.27694419,49.96051208)(114.5087972,50.19451273)(114.73643471,50.42015622)
\curveto(114.92191712,50.60401387)(115.09896851,50.78369295)(115.26758889,50.95919344)
\curveto(115.44464028,51.13469393)(115.59639861,51.31019442)(115.7228639,51.4856949)
\curveto(115.85776019,51.66119539)(115.96314793,51.84087447)(116.0390271,52.02473212)
\curveto(116.12333728,52.21694694)(116.16549238,52.41334035)(116.16549238,52.61391234)
\curveto(116.16549238,52.83955582)(116.12755279,53.03177065)(116.05167362,53.1905568)
\curveto(115.98422547,53.34934296)(115.88726876,53.47887904)(115.76080348,53.57916503)
\curveto(115.64276922,53.68780819)(115.50365741,53.76720127)(115.34346805,53.81734427)
\curveto(115.1832787,53.86748726)(115.01044282,53.89255876)(114.8249604,53.89255876)
\curveto(114.60575392,53.89255876)(114.40340947,53.86330868)(114.21792706,53.80480852)
\curveto(114.04087567,53.74630835)(113.88068631,53.67527244)(113.73735899,53.59170078)
\curveto(113.60246269,53.50812912)(113.48442843,53.42455746)(113.38325621,53.34098579)
\curveto(113.28208398,53.2657713)(113.20620481,53.20309255)(113.1556187,53.15294956)
\lineto(112.63711105,53.88002301)
\curveto(112.7045592,53.95523751)(112.80573143,54.04716634)(112.94062773,54.1558095)
\curveto(113.07552402,54.26445266)(113.23571338,54.36473865)(113.42119579,54.45666748)
\curveto(113.61510922,54.55695347)(113.8301002,54.64052513)(114.06616872,54.70738246)
\curveto(114.30223725,54.77423979)(114.55516781,54.80766846)(114.8249604,54.80766846)
\curveto(115.64276922,54.80766846)(116.24558705,54.61963222)(116.63341391,54.24355974)
\curveto(117.02967179,53.87584443)(117.22780073,53.34516438)(117.22780073,52.65151959)
\closepath
}
}
{
\newrgbcolor{curcolor}{0 0 0}
\pscustom[linestyle=none,fillstyle=solid,fillcolor=curcolor]
{
\newpath
\moveto(122.13465457,50.89651469)
\curveto(122.13465457,50.67922837)(122.06720642,50.49119213)(121.93231012,50.33240597)
\curveto(121.80584484,50.17361982)(121.63722447,50.09422674)(121.426449,50.09422674)
\curveto(121.20724252,50.09422674)(121.03019112,50.17361982)(120.89529482,50.33240597)
\curveto(120.76039852,50.49119213)(120.69295037,50.67922837)(120.69295037,50.89651469)
\curveto(120.69295037,51.11380101)(120.76039852,51.30601583)(120.89529482,51.47315916)
\curveto(121.03019112,51.64030248)(121.20724252,51.72387414)(121.426449,51.72387414)
\curveto(121.63722447,51.72387414)(121.80584484,51.64030248)(121.93231012,51.47315916)
\curveto(122.06720642,51.30601583)(122.13465457,51.11380101)(122.13465457,50.89651469)
\closepath
\moveto(118.82126422,50.75862145)
\curveto(118.82126422,52.06233937)(119.04468622,53.06102073)(119.49153021,53.75466552)
\curveto(119.94680522,54.45666748)(120.58334713,54.80766846)(121.40115595,54.80766846)
\curveto(122.22739578,54.80766846)(122.86393769,54.45666748)(123.31078168,53.75466552)
\curveto(123.75762567,53.06102073)(123.98104767,52.06233937)(123.98104767,50.75862145)
\curveto(123.98104767,49.45490353)(123.75762567,48.45204358)(123.31078168,47.75004163)
\curveto(122.86393769,47.05639683)(122.22739578,46.70957444)(121.40115595,46.70957444)
\curveto(120.58334713,46.70957444)(119.94680522,47.05639683)(119.49153021,47.75004163)
\curveto(119.04468622,48.45204358)(118.82126422,49.45490353)(118.82126422,50.75862145)
\closepath
\moveto(122.91873931,50.75862145)
\curveto(122.91873931,51.18483692)(122.89344625,51.5859809)(122.84286014,51.96205338)
\curveto(122.79227403,52.34648302)(122.70796384,52.68076967)(122.58992958,52.96491332)
\curveto(122.47189532,53.24905697)(122.31592147,53.47470045)(122.12200804,53.64184378)
\curveto(121.92809461,53.8089871)(121.68781058,53.89255876)(121.40115595,53.89255876)
\curveto(121.11450131,53.89255876)(120.87421728,53.8089871)(120.68030385,53.64184378)
\curveto(120.48639042,53.47470045)(120.33041657,53.24905697)(120.21238231,52.96491332)
\curveto(120.09434805,52.68076967)(120.01003786,52.34648302)(119.95945175,51.96205338)
\curveto(119.90886564,51.5859809)(119.88357258,51.18483692)(119.88357258,50.75862145)
\curveto(119.88357258,50.33240597)(119.90886564,49.92708341)(119.95945175,49.54265377)
\curveto(120.01003786,49.16658129)(120.09434805,48.83647323)(120.21238231,48.55232958)
\curveto(120.33041657,48.26818593)(120.48639042,48.04254244)(120.68030385,47.87539912)
\curveto(120.87421728,47.7082558)(121.11450131,47.62468413)(121.40115595,47.62468413)
\curveto(121.68781058,47.62468413)(121.92809461,47.7082558)(122.12200804,47.87539912)
\curveto(122.31592147,48.04254244)(122.47189532,48.26818593)(122.58992958,48.55232958)
\curveto(122.70796384,48.83647323)(122.79227403,49.16658129)(122.84286014,49.54265377)
\curveto(122.89344625,49.92708341)(122.91873931,50.33240597)(122.91873931,50.75862145)
\closepath
}
}
{
\newrgbcolor{curcolor}{0 0 0}
\pscustom[linestyle=none,fillstyle=solid,fillcolor=curcolor]
{
\newpath
\moveto(131.83453583,53.02759206)
\curveto(132.26451779,53.19473539)(132.68185321,53.40366454)(133.08654211,53.65437953)
\curveto(133.49123101,53.91345168)(133.86641134,54.23938116)(134.21208311,54.63216797)
\lineto(134.94558173,54.63216797)
\lineto(134.94558173,47.75004163)
\lineto(136.42522551,47.75004163)
\lineto(136.42522551,46.87253918)
\lineto(132.21393168,46.87253918)
\lineto(132.21393168,47.75004163)
\lineto(133.90856643,47.75004163)
\lineto(133.90856643,53.1905568)
\curveto(133.81582523,53.10698514)(133.70200648,53.0192349)(133.56711018,52.92730607)
\curveto(133.4406449,52.84373441)(133.29731758,52.76016275)(133.13712822,52.67659108)
\curveto(132.98536989,52.59301942)(132.82518053,52.51362634)(132.65656016,52.43841185)
\curveto(132.48793978,52.36319735)(132.32353492,52.30051861)(132.16334556,52.25037561)
\lineto(131.83453583,53.02759206)
\closepath
}
}
{
\newrgbcolor{curcolor}{0 0 0}
\pscustom[linestyle=none,fillstyle=solid,fillcolor=curcolor]
{
\newpath
\moveto(141.10444185,50.89651469)
\curveto(141.10444185,50.67922837)(141.0369937,50.49119213)(140.9020974,50.33240597)
\curveto(140.77563212,50.17361982)(140.60701175,50.09422674)(140.39623628,50.09422674)
\curveto(140.17702979,50.09422674)(139.9999784,50.17361982)(139.8650821,50.33240597)
\curveto(139.7301858,50.49119213)(139.66273765,50.67922837)(139.66273765,50.89651469)
\curveto(139.66273765,51.11380101)(139.7301858,51.30601583)(139.8650821,51.47315916)
\curveto(139.9999784,51.64030248)(140.17702979,51.72387414)(140.39623628,51.72387414)
\curveto(140.60701175,51.72387414)(140.77563212,51.64030248)(140.9020974,51.47315916)
\curveto(141.0369937,51.30601583)(141.10444185,51.11380101)(141.10444185,50.89651469)
\closepath
\moveto(137.7910515,50.75862145)
\curveto(137.7910515,52.06233937)(138.0144735,53.06102073)(138.46131749,53.75466552)
\curveto(138.9165925,54.45666748)(139.55313441,54.80766846)(140.37094322,54.80766846)
\curveto(141.19718305,54.80766846)(141.83372497,54.45666748)(142.28056896,53.75466552)
\curveto(142.72741295,53.06102073)(142.95083494,52.06233937)(142.95083494,50.75862145)
\curveto(142.95083494,49.45490353)(142.72741295,48.45204358)(142.28056896,47.75004163)
\curveto(141.83372497,47.05639683)(141.19718305,46.70957444)(140.37094322,46.70957444)
\curveto(139.55313441,46.70957444)(138.9165925,47.05639683)(138.46131749,47.75004163)
\curveto(138.0144735,48.45204358)(137.7910515,49.45490353)(137.7910515,50.75862145)
\closepath
\moveto(141.88852659,50.75862145)
\curveto(141.88852659,51.18483692)(141.86323353,51.5859809)(141.81264742,51.96205338)
\curveto(141.76206131,52.34648302)(141.67775112,52.68076967)(141.55971686,52.96491332)
\curveto(141.4416826,53.24905697)(141.28570875,53.47470045)(141.09179532,53.64184378)
\curveto(140.89788189,53.8089871)(140.65759786,53.89255876)(140.37094322,53.89255876)
\curveto(140.08428859,53.89255876)(139.84400455,53.8089871)(139.65009112,53.64184378)
\curveto(139.45617769,53.47470045)(139.30020385,53.24905697)(139.18216959,52.96491332)
\curveto(139.06413532,52.68076967)(138.97982514,52.34648302)(138.92923902,51.96205338)
\curveto(138.87865291,51.5859809)(138.85335986,51.18483692)(138.85335986,50.75862145)
\curveto(138.85335986,50.33240597)(138.87865291,49.92708341)(138.92923902,49.54265377)
\curveto(138.97982514,49.16658129)(139.06413532,48.83647323)(139.18216959,48.55232958)
\curveto(139.30020385,48.26818593)(139.45617769,48.04254244)(139.65009112,47.87539912)
\curveto(139.84400455,47.7082558)(140.08428859,47.62468413)(140.37094322,47.62468413)
\curveto(140.65759786,47.62468413)(140.89788189,47.7082558)(141.09179532,47.87539912)
\curveto(141.28570875,48.04254244)(141.4416826,48.26818593)(141.55971686,48.55232958)
\curveto(141.67775112,48.83647323)(141.76206131,49.16658129)(141.81264742,49.54265377)
\curveto(141.86323353,49.92708341)(141.88852659,50.33240597)(141.88852659,50.75862145)
\closepath
}
}
{
\newrgbcolor{curcolor}{0 0 0}
\pscustom[linestyle=none,fillstyle=solid,fillcolor=curcolor]
{
\newpath
\moveto(152.53690512,47.62468413)
\curveto(153.20295559,47.62468413)(153.67509264,47.75422021)(153.95331626,48.01329236)
\curveto(154.23997089,48.28072168)(154.38329821,48.63590124)(154.38329821,49.07883105)
\curveto(154.38329821,49.3629747)(154.32428108,49.60115393)(154.20624682,49.79336876)
\curveto(154.08821256,49.98558358)(153.93223871,50.14019115)(153.73832528,50.25719148)
\curveto(153.54441185,50.3741918)(153.32098985,50.45776347)(153.06805929,50.50790646)
\curveto(152.81512873,50.55804946)(152.54955164,50.58312096)(152.27132803,50.58312096)
\lineto(152.00575094,50.58312096)
\lineto(152.00575094,51.42301616)
\lineto(152.37250025,51.42301616)
\curveto(152.55798266,51.42301616)(152.74768058,51.43973049)(152.94159401,51.47315916)
\curveto(153.14393846,51.51494499)(153.32520536,51.58180232)(153.48539472,51.67373114)
\curveto(153.65401509,51.77401714)(153.78891139,51.9077318)(153.89008362,52.07487512)
\curveto(153.99125584,52.24201844)(154.04184195,52.45512618)(154.04184195,52.71419833)
\curveto(154.04184195,53.14041381)(153.90694565,53.44127179)(153.63715306,53.61677228)
\curveto(153.37579148,53.80062993)(153.06805929,53.89255876)(152.71395651,53.89255876)
\curveto(152.3514227,53.89255876)(152.04369052,53.83823718)(151.79075996,53.72959402)
\curveto(151.5378294,53.62930803)(151.32705393,53.52484345)(151.15843356,53.41620029)
\lineto(150.75374466,54.20595249)
\curveto(150.93079605,54.33130999)(151.19637314,54.46084606)(151.55047593,54.59456072)
\curveto(151.91300973,54.73663255)(152.31348312,54.80766846)(152.75189609,54.80766846)
\curveto(153.16501601,54.80766846)(153.51911879,54.75752546)(153.81420445,54.65723947)
\curveto(154.1092901,54.55695347)(154.34957414,54.41488165)(154.53505655,54.23102399)
\curveto(154.72896998,54.04716634)(154.87229729,53.82988002)(154.9650385,53.57916503)
\curveto(155.05777971,53.33680721)(155.10415031,53.06937789)(155.10415031,52.77687708)
\curveto(155.10415031,52.36737594)(154.99454707,52.02055354)(154.77534058,51.73640989)
\curveto(154.56456511,51.45226624)(154.290557,51.23497992)(153.95331626,51.08455093)
\curveto(154.35800515,50.9675506)(154.70789243,50.73772853)(155.00297808,50.39508472)
\curveto(155.29806374,50.06079807)(155.44560657,49.61368968)(155.44560657,49.05375955)
\curveto(155.44560657,48.7194729)(155.38658944,48.40607917)(155.26855517,48.11357835)
\curveto(155.15895193,47.8294347)(154.98611605,47.5828983)(154.75004752,47.37396915)
\curveto(154.52241002,47.16503999)(154.22310886,47.00207525)(153.85214403,46.88507493)
\curveto(153.48961023,46.7680746)(153.05541277,46.70957444)(152.54955164,46.70957444)
\curveto(152.35563821,46.70957444)(152.15329377,46.72628877)(151.9425183,46.75971743)
\curveto(151.74017385,46.78478893)(151.55047593,46.82239618)(151.37342454,46.87253918)
\curveto(151.19637314,46.91432501)(151.03618379,46.95611084)(150.89285647,46.99789667)
\curveto(150.75796017,47.04803967)(150.66100346,47.08564692)(150.60198633,47.11071841)
\lineto(150.80433077,48.00075661)
\curveto(150.93922707,47.93389928)(151.15421805,47.8545062)(151.4493037,47.76257738)
\curveto(151.74438936,47.67064855)(152.10692316,47.62468413)(152.53690512,47.62468413)
\closepath
}
}
{
\newrgbcolor{curcolor}{0 0 0}
\pscustom[linestyle=none,fillstyle=solid,fillcolor=curcolor]
{
\newpath
\moveto(158.139318,46.87253918)
\curveto(158.18147309,47.46589798)(158.28686082,48.09268544)(158.4554812,48.75290157)
\curveto(158.63253259,49.42147486)(158.84330806,50.06497666)(159.0878076,50.68340695)
\curveto(159.33230714,51.31019442)(159.60209974,51.88683888)(159.8971854,52.41334035)
\curveto(160.19227105,52.94819898)(160.4831412,53.38695021)(160.76979583,53.72959402)
\lineto(156.97583742,53.72959402)
\lineto(156.97583742,54.63216797)
\lineto(161.93327641,54.63216797)
\lineto(161.93327641,53.76720127)
\curveto(161.68034585,53.47470045)(161.40633774,53.08191364)(161.11125209,52.58884084)
\curveto(160.81616643,52.09576804)(160.53372731,51.54001648)(160.26393471,50.92158619)
\curveto(160.00257313,50.31151306)(159.77493562,49.65547551)(159.58102219,48.95347355)
\curveto(159.38710876,48.25982876)(159.26485899,47.56618397)(159.21427288,46.87253918)
\lineto(158.139318,46.87253918)
\closepath
}
}
{
\newrgbcolor{curcolor}{0 0 0}
\pscustom[linestyle=none,fillstyle=solid,fillcolor=curcolor]
{
\newpath
\moveto(180.80189913,51.52330215)
\curveto(180.80189913,49.97722641)(180.42250328,48.81140173)(179.6637116,48.02582811)
\curveto(178.90491992,47.24861166)(177.76251689,46.85582485)(176.2365025,46.84746768)
\lineto(176.19856292,47.72497013)
\curveto(177.14283701,47.72497013)(177.90162869,47.90882778)(178.47493797,48.27654309)
\curveto(179.05667826,48.65261557)(179.44028961,49.29611737)(179.62577202,50.20704848)
\curveto(179.42342757,50.11511965)(179.20000557,50.03990516)(178.95550603,49.98140499)
\curveto(178.71100649,49.931262)(178.45386042,49.9061905)(178.18406782,49.9061905)
\curveto(177.73722383,49.9061905)(177.3620435,49.96886924)(177.05852683,50.09422674)
\curveto(176.75501015,50.2279414)(176.51051061,50.40344189)(176.3250282,50.62072821)
\curveto(176.13954579,50.84637169)(176.00464949,51.10126526)(175.9203393,51.38540891)
\curveto(175.83602911,51.66955256)(175.79387402,51.97041054)(175.79387402,52.28798286)
\curveto(175.79387402,52.57212651)(175.84024462,52.86044874)(175.93298583,53.15294956)
\curveto(176.02572703,53.45380754)(176.16905435,53.72541544)(176.36296778,53.96777326)
\curveto(176.55688121,54.21013108)(176.80559626,54.41070306)(177.10911294,54.56948922)
\curveto(177.41262961,54.72827538)(177.77516341,54.80766846)(178.19671435,54.80766846)
\curveto(179.05667826,54.80766846)(179.7058667,54.51516764)(180.14427967,53.93016601)
\curveto(180.58269264,53.34516438)(180.80189913,52.54287643)(180.80189913,51.52330215)
\closepath
\moveto(178.29788657,50.75862145)
\curveto(178.56767917,50.75862145)(178.81639422,50.78369295)(179.04403173,50.83383594)
\curveto(179.28010025,50.88397894)(179.50773776,50.95501485)(179.72694424,51.04694368)
\curveto(179.73537526,51.13051534)(179.73959077,51.20990842)(179.73959077,51.28512292)
\lineto(179.73959077,51.52330215)
\curveto(179.73959077,51.84923163)(179.71429771,52.15844678)(179.6637116,52.4509476)
\curveto(179.62155651,52.74344841)(179.54146183,52.99834198)(179.42342757,53.2156283)
\curveto(179.31382433,53.43291462)(179.15785048,53.60423653)(178.95550603,53.72959402)
\curveto(178.7615926,53.86330868)(178.51287755,53.93016601)(178.20936088,53.93016601)
\curveto(177.95643032,53.93016601)(177.74565485,53.87584443)(177.57703448,53.76720127)
\curveto(177.4084141,53.66691528)(177.26930229,53.5373792)(177.15969905,53.37859304)
\curveto(177.05009581,53.22816405)(176.97000113,53.05684214)(176.91941502,52.86462732)
\curveto(176.87725992,52.6724125)(176.85618238,52.48855485)(176.85618238,52.31305436)
\curveto(176.85618238,51.80326722)(176.97000113,51.41465899)(177.19763863,51.14722967)
\curveto(177.43370716,50.88815752)(177.80045647,50.75862145)(178.29788657,50.75862145)
\closepath
}
}
{
\newrgbcolor{curcolor}{0 0 0}
\pscustom[linestyle=none,fillstyle=solid,fillcolor=curcolor]
{
\newpath
\moveto(190.47648733,47.62468413)
\curveto(191.14253781,47.62468413)(191.61467486,47.75422021)(191.89289847,48.01329236)
\curveto(192.17955311,48.28072168)(192.32288043,48.63590124)(192.32288043,49.07883105)
\curveto(192.32288043,49.3629747)(192.2638633,49.60115393)(192.14582903,49.79336876)
\curveto(192.02779477,49.98558358)(191.87182093,50.14019115)(191.6779075,50.25719148)
\curveto(191.48399407,50.3741918)(191.26057207,50.45776347)(191.00764151,50.50790646)
\curveto(190.75471095,50.55804946)(190.48913386,50.58312096)(190.21091024,50.58312096)
\lineto(189.94533315,50.58312096)
\lineto(189.94533315,51.42301616)
\lineto(190.31208247,51.42301616)
\curveto(190.49756488,51.42301616)(190.6872628,51.43973049)(190.88117623,51.47315916)
\curveto(191.08352068,51.51494499)(191.26478758,51.58180232)(191.42497694,51.67373114)
\curveto(191.59359731,51.77401714)(191.72849361,51.9077318)(191.82966583,52.07487512)
\curveto(191.93083806,52.24201844)(191.98142417,52.45512618)(191.98142417,52.71419833)
\curveto(191.98142417,53.14041381)(191.84652787,53.44127179)(191.57673527,53.61677228)
\curveto(191.31537369,53.80062993)(191.00764151,53.89255876)(190.65353873,53.89255876)
\curveto(190.29100492,53.89255876)(189.98327274,53.83823718)(189.73034218,53.72959402)
\curveto(189.47741162,53.62930803)(189.26663615,53.52484345)(189.09801578,53.41620029)
\lineto(188.69332688,54.20595249)
\curveto(188.87037827,54.33130999)(189.13595536,54.46084606)(189.49005815,54.59456072)
\curveto(189.85259195,54.73663255)(190.25306534,54.80766846)(190.69147831,54.80766846)
\curveto(191.10459823,54.80766846)(191.45870101,54.75752546)(191.75378667,54.65723947)
\curveto(192.04887232,54.55695347)(192.28915635,54.41488165)(192.47463876,54.23102399)
\curveto(192.66855219,54.04716634)(192.81187951,53.82988002)(192.90462072,53.57916503)
\curveto(192.99736192,53.33680721)(193.04373253,53.06937789)(193.04373253,52.77687708)
\curveto(193.04373253,52.36737594)(192.93412928,52.02055354)(192.7149228,51.73640989)
\curveto(192.50414733,51.45226624)(192.23013922,51.23497992)(191.89289847,51.08455093)
\curveto(192.29758737,50.9675506)(192.64747465,50.73772853)(192.9425603,50.39508472)
\curveto(193.23764596,50.06079807)(193.38518878,49.61368968)(193.38518878,49.05375955)
\curveto(193.38518878,48.7194729)(193.32617165,48.40607917)(193.20813739,48.11357835)
\curveto(193.09853415,47.8294347)(192.92569826,47.5828983)(192.68962974,47.37396915)
\curveto(192.46199224,47.16503999)(192.16269107,47.00207525)(191.79172625,46.88507493)
\curveto(191.42919245,46.7680746)(190.99499498,46.70957444)(190.48913386,46.70957444)
\curveto(190.29522043,46.70957444)(190.09287598,46.72628877)(189.88210051,46.75971743)
\curveto(189.67975607,46.78478893)(189.49005815,46.82239618)(189.31300675,46.87253918)
\curveto(189.13595536,46.91432501)(188.975766,46.95611084)(188.83243869,46.99789667)
\curveto(188.69754239,47.04803967)(188.60058567,47.08564692)(188.54156854,47.11071841)
\lineto(188.74391299,48.00075661)
\curveto(188.87880929,47.93389928)(189.09380027,47.8545062)(189.38888592,47.76257738)
\curveto(189.68397158,47.67064855)(190.04650538,47.62468413)(190.47648733,47.62468413)
\closepath
}
}
{
\newrgbcolor{curcolor}{0 0 0}
\pscustom[linestyle=none,fillstyle=solid,fillcolor=curcolor]
{
\newpath
\moveto(196.79975998,47.62468413)
\curveto(197.46581045,47.62468413)(197.9379475,47.75422021)(198.21617112,48.01329236)
\curveto(198.50282575,48.28072168)(198.64615307,48.63590124)(198.64615307,49.07883105)
\curveto(198.64615307,49.3629747)(198.58713594,49.60115393)(198.46910168,49.79336876)
\curveto(198.35106742,49.98558358)(198.19509357,50.14019115)(198.00118014,50.25719148)
\curveto(197.80726671,50.3741918)(197.58384471,50.45776347)(197.33091415,50.50790646)
\curveto(197.07798359,50.55804946)(196.8124065,50.58312096)(196.53418289,50.58312096)
\lineto(196.2686058,50.58312096)
\lineto(196.2686058,51.42301616)
\lineto(196.63535511,51.42301616)
\curveto(196.82083752,51.42301616)(197.01053544,51.43973049)(197.20444887,51.47315916)
\curveto(197.40679332,51.51494499)(197.58806022,51.58180232)(197.74824958,51.67373114)
\curveto(197.91686995,51.77401714)(198.05176625,51.9077318)(198.15293848,52.07487512)
\curveto(198.2541107,52.24201844)(198.30469681,52.45512618)(198.30469681,52.71419833)
\curveto(198.30469681,53.14041381)(198.16980051,53.44127179)(197.90000792,53.61677228)
\curveto(197.63864634,53.80062993)(197.33091415,53.89255876)(196.97681137,53.89255876)
\curveto(196.61427756,53.89255876)(196.30654538,53.83823718)(196.05361482,53.72959402)
\curveto(195.80068426,53.62930803)(195.58990879,53.52484345)(195.42128842,53.41620029)
\lineto(195.01659952,54.20595249)
\curveto(195.19365091,54.33130999)(195.459228,54.46084606)(195.81333079,54.59456072)
\curveto(196.17586459,54.73663255)(196.57633798,54.80766846)(197.01475095,54.80766846)
\curveto(197.42787087,54.80766846)(197.78197365,54.75752546)(198.07705931,54.65723947)
\curveto(198.37214496,54.55695347)(198.612429,54.41488165)(198.79791141,54.23102399)
\curveto(198.99182484,54.04716634)(199.13515215,53.82988002)(199.22789336,53.57916503)
\curveto(199.32063457,53.33680721)(199.36700517,53.06937789)(199.36700517,52.77687708)
\curveto(199.36700517,52.36737594)(199.25740193,52.02055354)(199.03819544,51.73640989)
\curveto(198.82741997,51.45226624)(198.55341186,51.23497992)(198.21617112,51.08455093)
\curveto(198.62086001,50.9675506)(198.97074729,50.73772853)(199.26583294,50.39508472)
\curveto(199.5609186,50.06079807)(199.70846143,49.61368968)(199.70846143,49.05375955)
\curveto(199.70846143,48.7194729)(199.6494443,48.40607917)(199.53141003,48.11357835)
\curveto(199.42180679,47.8294347)(199.24897091,47.5828983)(199.01290238,47.37396915)
\curveto(198.78526488,47.16503999)(198.48596371,47.00207525)(198.11499889,46.88507493)
\curveto(197.75246509,46.7680746)(197.31826763,46.70957444)(196.8124065,46.70957444)
\curveto(196.61849307,46.70957444)(196.41614863,46.72628877)(196.20537316,46.75971743)
\curveto(196.00302871,46.78478893)(195.81333079,46.82239618)(195.6362794,46.87253918)
\curveto(195.459228,46.91432501)(195.29903865,46.95611084)(195.15571133,46.99789667)
\curveto(195.02081503,47.04803967)(194.92385832,47.08564692)(194.86484118,47.11071841)
\lineto(195.06718563,48.00075661)
\curveto(195.20208193,47.93389928)(195.41707291,47.8545062)(195.71215856,47.76257738)
\curveto(196.00724422,47.67064855)(196.36977802,47.62468413)(196.79975998,47.62468413)
\closepath
}
}
{
\newrgbcolor{curcolor}{0 0 0}
\pscustom[linestyle=none,fillstyle=solid,fillcolor=curcolor]
{
\newpath
\moveto(215.04869515,46.87253918)
\curveto(215.09085025,47.46589798)(215.19623798,48.09268544)(215.36485836,48.75290157)
\curveto(215.54190975,49.42147486)(215.75268522,50.06497666)(215.99718476,50.68340695)
\curveto(216.2416843,51.31019442)(216.5114769,51.88683888)(216.80656255,52.41334035)
\curveto(217.10164821,52.94819898)(217.39251835,53.38695021)(217.67917299,53.72959402)
\lineto(213.88521457,53.72959402)
\lineto(213.88521457,54.63216797)
\lineto(218.84265357,54.63216797)
\lineto(218.84265357,53.76720127)
\curveto(218.58972301,53.47470045)(218.3157149,53.08191364)(218.02062924,52.58884084)
\curveto(217.72554359,52.09576804)(217.44310446,51.54001648)(217.17331187,50.92158619)
\curveto(216.91195029,50.31151306)(216.68431278,49.65547551)(216.49039935,48.95347355)
\curveto(216.29648592,48.25982876)(216.17423615,47.56618397)(216.12365004,46.87253918)
\lineto(215.04869515,46.87253918)
\closepath
}
}
{
\newrgbcolor{curcolor}{0 0 0}
\pscustom[linestyle=none,fillstyle=solid,fillcolor=curcolor]
{
\newpath
\moveto(236.15574567,47.67482713)
\curveto(236.15574567,47.42411215)(236.07143548,47.20264724)(235.90281511,47.01043242)
\curveto(235.73419474,46.8182176)(235.51077274,46.72211019)(235.23254912,46.72211019)
\curveto(234.94589449,46.72211019)(234.71825698,46.8182176)(234.54963661,47.01043242)
\curveto(234.38101623,47.20264724)(234.29670605,47.42411215)(234.29670605,47.67482713)
\curveto(234.29670605,47.93389928)(234.38101623,48.15954277)(234.54963661,48.35175759)
\curveto(234.71825698,48.54397241)(234.94589449,48.64007982)(235.23254912,48.64007982)
\curveto(235.51077274,48.64007982)(235.73419474,48.54397241)(235.90281511,48.35175759)
\curveto(236.07143548,48.15954277)(236.15574567,47.93389928)(236.15574567,47.67482713)
\closepath
}
}
{
\newrgbcolor{curcolor}{0 0 0}
\pscustom[linestyle=none,fillstyle=solid,fillcolor=curcolor]
{
\newpath
\moveto(242.47901831,47.67482713)
\curveto(242.47901831,47.42411215)(242.39470813,47.20264724)(242.22608775,47.01043242)
\curveto(242.05746738,46.8182176)(241.83404538,46.72211019)(241.55582177,46.72211019)
\curveto(241.26916713,46.72211019)(241.04152963,46.8182176)(240.87290925,47.01043242)
\curveto(240.70428888,47.20264724)(240.61997869,47.42411215)(240.61997869,47.67482713)
\curveto(240.61997869,47.93389928)(240.70428888,48.15954277)(240.87290925,48.35175759)
\curveto(241.04152963,48.54397241)(241.26916713,48.64007982)(241.55582177,48.64007982)
\curveto(241.83404538,48.64007982)(242.05746738,48.54397241)(242.22608775,48.35175759)
\curveto(242.39470813,48.15954277)(242.47901831,47.93389928)(242.47901831,47.67482713)
\closepath
}
}
{
\newrgbcolor{curcolor}{0 0 0}
\pscustom[linestyle=none,fillstyle=solid,fillcolor=curcolor]
{
\newpath
\moveto(248.80227563,47.67482713)
\curveto(248.80227563,47.42411215)(248.71796544,47.20264724)(248.54934507,47.01043242)
\curveto(248.3807247,46.8182176)(248.1573027,46.72211019)(247.87907908,46.72211019)
\curveto(247.59242445,46.72211019)(247.36478694,46.8182176)(247.19616657,47.01043242)
\curveto(247.02754619,47.20264724)(246.94323601,47.42411215)(246.94323601,47.67482713)
\curveto(246.94323601,47.93389928)(247.02754619,48.15954277)(247.19616657,48.35175759)
\curveto(247.36478694,48.54397241)(247.59242445,48.64007982)(247.87907908,48.64007982)
\curveto(248.1573027,48.64007982)(248.3807247,48.54397241)(248.54934507,48.35175759)
\curveto(248.71796544,48.15954277)(248.80227563,47.93389928)(248.80227563,47.67482713)
\closepath
}
}
{
\newrgbcolor{curcolor}{0 0 0}
\pscustom[linestyle=none,fillstyle=solid,fillcolor=curcolor]
{
\newpath
\moveto(266.02684204,51.6361239)
\curveto(267.15659854,51.59433806)(267.97862287,51.34780166)(268.49291501,50.89651469)
\curveto(269.00720715,50.44522772)(269.26435322,49.83933317)(269.26435322,49.07883105)
\curveto(269.26435322,48.73618723)(269.20955159,48.41861492)(269.09994835,48.1261141)
\curveto(268.99034511,47.83361329)(268.81750923,47.5828983)(268.5814407,47.37396915)
\curveto(268.3538032,47.16503999)(268.05871754,47.00207525)(267.69618374,46.88507493)
\curveto(267.34208095,46.7680746)(266.92053002,46.70957444)(266.43153093,46.70957444)
\curveto(266.22918649,46.70957444)(266.02684204,46.72628877)(265.82449759,46.75971743)
\curveto(265.63058416,46.78478893)(265.44510175,46.8182176)(265.26805035,46.86000343)
\curveto(265.09099896,46.90178926)(264.93502512,46.94357509)(264.80012882,46.98536092)
\curveto(264.66523252,47.03550392)(264.5682758,47.07728975)(264.50925867,47.11071841)
\lineto(264.71160312,48.00075661)
\curveto(264.84649942,47.93389928)(265.05305938,47.8545062)(265.33128299,47.76257738)
\curveto(265.61793763,47.67064855)(265.97625592,47.62468413)(266.40623788,47.62468413)
\curveto(266.74347863,47.62468413)(267.02591775,47.66229138)(267.25355526,47.73750588)
\curveto(267.48119276,47.81272037)(267.66245966,47.91300637)(267.79735596,48.03836386)
\curveto(267.94068328,48.17207852)(268.04185551,48.32250751)(268.10087264,48.48965083)
\curveto(268.16832079,48.65679416)(268.20204486,48.83229465)(268.20204486,49.0161523)
\curveto(268.20204486,49.30029595)(268.15145875,49.55101094)(268.05028652,49.76829726)
\curveto(267.95754532,49.99394074)(267.78892494,50.18197698)(267.5444254,50.33240597)
\curveto(267.30835688,50.48283496)(266.97954715,50.59565671)(266.55799621,50.6708712)
\curveto(266.1448763,50.75444286)(265.62215314,50.7962287)(264.98982674,50.7962287)
\curveto(265.04041285,51.16394401)(265.07835243,51.50658782)(265.10364549,51.82416013)
\curveto(265.13736956,52.15008962)(265.16266262,52.46348335)(265.17952466,52.76434133)
\curveto(265.20481771,53.07355648)(265.22167975,53.37859304)(265.23011077,53.67945102)
\curveto(265.24697281,53.98030901)(265.26383484,54.29788132)(265.28069688,54.63216797)
\lineto(269.04936224,54.63216797)
\lineto(269.04936224,53.75466552)
\lineto(266.1912469,53.75466552)
\curveto(266.18281588,53.63766519)(266.17016935,53.48305762)(266.15330732,53.2908428)
\curveto(266.1448763,53.10698514)(266.13222977,52.91059174)(266.11536773,52.70166258)
\curveto(266.0985057,52.49273343)(266.08164366,52.29216144)(266.06478162,52.09994662)
\curveto(266.04791958,51.9077318)(266.03527306,51.75312422)(266.02684204,51.6361239)
\closepath
}
}
{
\newrgbcolor{curcolor}{0 0 0}
\pscustom[linestyle=none,fillstyle=solid,fillcolor=curcolor]
{
\newpath
\moveto(275.30940224,52.65151959)
\curveto(275.30940224,52.38409027)(275.25460062,52.12501812)(275.14499738,51.87430313)
\curveto(275.04382515,51.62358815)(274.90471334,51.37705174)(274.72766195,51.13469393)
\curveto(274.55904158,50.89233611)(274.36512815,50.65415687)(274.14592166,50.42015622)
\curveto(273.92671518,50.18615557)(273.70329318,49.9563335)(273.47565568,49.73069001)
\curveto(273.3491904,49.60533252)(273.20164757,49.45490353)(273.03302719,49.27940304)
\curveto(272.86440682,49.10390255)(272.70421747,48.92422347)(272.55245913,48.74036582)
\curveto(272.40070079,48.55650816)(272.27423551,48.37682909)(272.17306329,48.2013286)
\curveto(272.07189106,48.02582811)(272.02130495,47.87539912)(272.02130495,47.75004163)
\lineto(275.62556544,47.75004163)
\lineto(275.62556544,46.87253918)
\lineto(270.88311743,46.87253918)
\curveto(270.87468641,46.91432501)(270.8704709,46.95611084)(270.8704709,46.99789667)
\lineto(270.8704709,47.13578991)
\curveto(270.8704709,47.48679089)(270.92948803,47.81272037)(271.04752229,48.11357835)
\curveto(271.16555655,48.41443634)(271.31731489,48.69857999)(271.5027973,48.9660093)
\curveto(271.68827971,49.23343862)(271.89483967,49.48415361)(272.12247718,49.71815426)
\curveto(272.3585457,49.96051208)(272.59039871,50.19451273)(272.81803622,50.42015622)
\curveto(273.00351863,50.60401387)(273.18057002,50.78369295)(273.3491904,50.95919344)
\curveto(273.52624179,51.13469393)(273.67800012,51.31019442)(273.80446541,51.4856949)
\curveto(273.9393617,51.66119539)(274.04474944,51.84087447)(274.12062861,52.02473212)
\curveto(274.20493879,52.21694694)(274.24709389,52.41334035)(274.24709389,52.61391234)
\curveto(274.24709389,52.83955582)(274.2091543,53.03177065)(274.13327513,53.1905568)
\curveto(274.06582698,53.34934296)(273.96887027,53.47887904)(273.84240499,53.57916503)
\curveto(273.72437073,53.68780819)(273.58525892,53.76720127)(273.42506956,53.81734427)
\curveto(273.26488021,53.86748726)(273.09204433,53.89255876)(272.90656191,53.89255876)
\curveto(272.68735543,53.89255876)(272.48501098,53.86330868)(272.29952857,53.80480852)
\curveto(272.12247718,53.74630835)(271.96228782,53.67527244)(271.8189605,53.59170078)
\curveto(271.6840642,53.50812912)(271.56602994,53.42455746)(271.46485772,53.34098579)
\curveto(271.36368549,53.2657713)(271.28780632,53.20309255)(271.23722021,53.15294956)
\lineto(270.71871256,53.88002301)
\curveto(270.78616071,53.95523751)(270.88733294,54.04716634)(271.02222924,54.1558095)
\curveto(271.15712553,54.26445266)(271.31731489,54.36473865)(271.5027973,54.45666748)
\curveto(271.69671073,54.55695347)(271.91170171,54.64052513)(272.14777023,54.70738246)
\curveto(272.38383875,54.77423979)(272.63676932,54.80766846)(272.90656191,54.80766846)
\curveto(273.72437073,54.80766846)(274.32718856,54.61963222)(274.71501542,54.24355974)
\curveto(275.1112733,53.87584443)(275.30940224,53.34516438)(275.30940224,52.65151959)
\closepath
}
}
{
\newrgbcolor{curcolor}{0 0 0}
\pscustom[linewidth=0.56097835,linecolor=curcolor]
{
\newpath
\moveto(52.53490982,57.11818472)
\lineto(277.23287735,57.11818472)
\lineto(277.23287735,44.39906565)
\lineto(52.53490982,44.39906565)
\closepath
}
}
{
\newrgbcolor{curcolor}{0 0 0}
\pscustom[linewidth=0.51800942,linecolor=curcolor]
{
\newpath
\moveto(71.3882274,44.41317433)
\lineto(71.3882274,57.01256933)
}
}
{
\newrgbcolor{curcolor}{0 0 0}
\pscustom[linewidth=0.5154072,linecolor=curcolor]
{
\newpath
\moveto(90.2684504,44.50748533)
\lineto(90.2684504,56.98061133)
}
}
{
\newrgbcolor{curcolor}{0 0 0}
\pscustom[linewidth=0.51930571,linecolor=curcolor]
{
\newpath
\moveto(109.1077404,44.50748433)
\lineto(109.1077404,57.17001333)
}
}
{
\newrgbcolor{curcolor}{0 0 0}
\pscustom[linewidth=0.52573889,linecolor=curcolor]
{
\newpath
\moveto(128.0363104,44.32891333)
\lineto(128.0363104,57.30711533)
}
}
{
\newrgbcolor{curcolor}{0 0 0}
\pscustom[linewidth=0.51670998,linecolor=curcolor]
{
\newpath
\moveto(146.8755904,44.50748433)
\lineto(146.8755904,57.04374433)
}
}
{
\newrgbcolor{curcolor}{0 0 0}
\pscustom[linewidth=0.52059871,linecolor=curcolor]
{
\newpath
\moveto(165.6256004,44.50748533)
\lineto(165.6256004,57.23314933)
}
}
{
\newrgbcolor{curcolor}{0 0 0}
\pscustom[linewidth=0.51670998,linecolor=curcolor]
{
\newpath
\moveto(184.5541704,44.41819933)
\lineto(184.5541704,56.95445933)
}
}
{
\newrgbcolor{curcolor}{0 0 0}
\pscustom[linewidth=0.51670998,linecolor=curcolor]
{
\newpath
\moveto(203.6613104,44.50748533)
\lineto(203.6613104,57.04374533)
}
}
{
\newrgbcolor{curcolor}{0 0 0}
\pscustom[linewidth=0.52059871,linecolor=curcolor]
{
\newpath
\moveto(222.5898804,44.50748433)
\lineto(222.5898804,57.23314833)
}
}
{
\newrgbcolor{curcolor}{0 0 0}
\pscustom[linewidth=0.52573889,linecolor=curcolor]
{
\newpath
\moveto(258.2401904,57.15925933)
\lineto(258.2046904,44.18110533)
}
}
{
\newrgbcolor{curcolor}{0 1 0}
\pscustom[linestyle=none,fillstyle=solid,fillcolor=curcolor]
{
\newpath
\moveto(203.61685315,35.32180102)
\lineto(222.4648222,35.32180102)
\lineto(222.4648222,22.50549013)
\lineto(203.61685315,22.50549013)
\closepath
}
}
{
\newrgbcolor{curcolor}{0 0 0}
\pscustom[linestyle=none,fillstyle=solid,fillcolor=curcolor]
{
\newpath
\moveto(62.27863502,30.96831406)
\curveto(62.70861698,31.13545739)(63.1259524,31.34438654)(63.5306413,31.59510153)
\curveto(63.9353302,31.85417368)(64.31051053,32.18010316)(64.6561823,32.57288997)
\lineto(65.38968092,32.57288997)
\lineto(65.38968092,25.69076363)
\lineto(66.8693247,25.69076363)
\lineto(66.8693247,24.81326118)
\lineto(62.65803086,24.81326118)
\lineto(62.65803086,25.69076363)
\lineto(64.35266562,25.69076363)
\lineto(64.35266562,31.1312788)
\curveto(64.25992442,31.04770714)(64.14610566,30.9599569)(64.01120936,30.86802807)
\curveto(63.88474408,30.78445641)(63.74141677,30.70088475)(63.58122741,30.61731308)
\curveto(63.42946907,30.53374142)(63.26927972,30.45434834)(63.10065935,30.37913385)
\curveto(62.93203897,30.30391935)(62.76763411,30.24124061)(62.60744475,30.19109761)
\lineto(62.27863502,30.96831406)
\closepath
}
}
{
\newrgbcolor{curcolor}{0 0 0}
\pscustom[linestyle=none,fillstyle=solid,fillcolor=curcolor]
{
\newpath
\moveto(76.32892672,29.5768459)
\curveto(77.45868323,29.53506006)(78.28070755,29.28852366)(78.79499969,28.83723669)
\curveto(79.30929183,28.38594972)(79.5664379,27.78005517)(79.5664379,27.01955305)
\curveto(79.5664379,26.67690923)(79.51163628,26.35933692)(79.40203304,26.0668361)
\curveto(79.29242979,25.77433529)(79.11959391,25.5236203)(78.88352539,25.31469115)
\curveto(78.65588788,25.10576199)(78.36080223,24.94279725)(77.99826842,24.82579693)
\curveto(77.64416564,24.7087966)(77.2226147,24.65029644)(76.73361562,24.65029644)
\curveto(76.53127117,24.65029644)(76.32892672,24.66701077)(76.12658227,24.70043943)
\curveto(75.93266884,24.72551093)(75.74718643,24.7589396)(75.57013504,24.80072543)
\curveto(75.39308365,24.84251126)(75.2371098,24.88429709)(75.1022135,24.92608292)
\curveto(74.9673172,24.97622592)(74.87036049,25.01801175)(74.81134336,25.05144041)
\lineto(75.01368781,25.94147861)
\curveto(75.1485841,25.87462128)(75.35514406,25.7952282)(75.63336768,25.70329938)
\curveto(75.92002232,25.61137055)(76.27834061,25.56540613)(76.70832256,25.56540613)
\curveto(77.04556331,25.56540613)(77.32800244,25.60301338)(77.55563994,25.67822788)
\curveto(77.78327745,25.75344237)(77.96454435,25.85372837)(78.09944065,25.97908586)
\curveto(78.24276797,26.11280052)(78.34394019,26.26322951)(78.40295732,26.43037283)
\curveto(78.47040547,26.59751616)(78.50412955,26.77301665)(78.50412955,26.9568743)
\curveto(78.50412955,27.24101795)(78.45354343,27.49173294)(78.35237121,27.70901926)
\curveto(78.25963,27.93466274)(78.09100963,28.12269898)(77.84651009,28.27312797)
\curveto(77.61044156,28.42355696)(77.28163183,28.53637871)(76.8600809,28.6115932)
\curveto(76.44696098,28.69516486)(75.92423782,28.7369507)(75.29191142,28.7369507)
\curveto(75.34249753,29.10466601)(75.38043712,29.44730982)(75.40573017,29.76488213)
\curveto(75.43945425,30.09081162)(75.46474731,30.40420535)(75.48160934,30.70506333)
\curveto(75.5069024,31.01427848)(75.52376444,31.31931504)(75.53219546,31.62017302)
\curveto(75.54905749,31.92103101)(75.56591953,32.23860332)(75.58278157,32.57288997)
\lineto(79.35144692,32.57288997)
\lineto(79.35144692,31.69538752)
\lineto(76.49333159,31.69538752)
\curveto(76.48490057,31.57838719)(76.47225404,31.42377962)(76.455392,31.2315648)
\curveto(76.44696098,31.04770714)(76.43431446,30.85131374)(76.41745242,30.64238458)
\curveto(76.40059038,30.43345543)(76.38372834,30.23288344)(76.36686631,30.04066862)
\curveto(76.35000427,29.8484538)(76.33735774,29.69384622)(76.32892672,29.5768459)
\closepath
}
}
{
\newrgbcolor{curcolor}{0 0 0}
\pscustom[linestyle=none,fillstyle=solid,fillcolor=curcolor]
{
\newpath
\moveto(81.24842709,30.96831406)
\curveto(81.67840904,31.13545739)(82.09574447,31.34438654)(82.50043337,31.59510153)
\curveto(82.90512226,31.85417368)(83.28030259,32.18010316)(83.62597436,32.57288997)
\lineto(84.35947299,32.57288997)
\lineto(84.35947299,25.69076363)
\lineto(85.83911677,25.69076363)
\lineto(85.83911677,24.81326118)
\lineto(81.62782293,24.81326118)
\lineto(81.62782293,25.69076363)
\lineto(83.32245769,25.69076363)
\lineto(83.32245769,31.1312788)
\curveto(83.22971648,31.04770714)(83.11589773,30.9599569)(82.98100143,30.86802807)
\curveto(82.85453615,30.78445641)(82.71120883,30.70088475)(82.55101948,30.61731308)
\curveto(82.39926114,30.53374142)(82.23907179,30.45434834)(82.07045141,30.37913385)
\curveto(81.90183104,30.30391935)(81.73742617,30.24124061)(81.57723682,30.19109761)
\lineto(81.24842709,30.96831406)
\closepath
}
}
{
\newrgbcolor{curcolor}{0 0 0}
\pscustom[linestyle=none,fillstyle=solid,fillcolor=curcolor]
{
\newpath
\moveto(93.89495322,30.96831406)
\curveto(94.32493517,31.13545739)(94.7422706,31.34438654)(95.14695949,31.59510153)
\curveto(95.55164839,31.85417368)(95.92682872,32.18010316)(96.27250049,32.57288997)
\lineto(97.00599912,32.57288997)
\lineto(97.00599912,25.69076363)
\lineto(98.4856429,25.69076363)
\lineto(98.4856429,24.81326118)
\lineto(94.27434906,24.81326118)
\lineto(94.27434906,25.69076363)
\lineto(95.96898382,25.69076363)
\lineto(95.96898382,31.1312788)
\curveto(95.87624261,31.04770714)(95.76242386,30.9599569)(95.62752756,30.86802807)
\curveto(95.50106228,30.78445641)(95.35773496,30.70088475)(95.19754561,30.61731308)
\curveto(95.04578727,30.53374142)(94.88559791,30.45434834)(94.71697754,30.37913385)
\curveto(94.54835717,30.30391935)(94.3839523,30.24124061)(94.22376295,30.19109761)
\lineto(93.89495322,30.96831406)
\closepath
}
}
{
\newrgbcolor{curcolor}{0 0 0}
\pscustom[linestyle=none,fillstyle=solid,fillcolor=curcolor]
{
\newpath
\moveto(99.7250036,27.49591152)
\curveto(99.86833092,27.83019817)(100.06224435,28.21462781)(100.30674389,28.64920045)
\curveto(100.55967445,29.08377309)(100.84211358,29.53088148)(101.15406127,29.99052562)
\curveto(101.46600896,30.45852693)(101.7990342,30.91399248)(102.15313699,31.35692229)
\curveto(102.50723977,31.80820926)(102.86555807,32.21353182)(103.22809187,32.57288997)
\lineto(104.23981412,32.57288997)
\lineto(104.23981412,27.64634051)
\lineto(105.16301066,27.64634051)
\lineto(105.16301066,26.79390956)
\lineto(104.23981412,26.79390956)
\lineto(104.23981412,24.81326118)
\lineto(103.22809187,24.81326118)
\lineto(103.22809187,26.79390956)
\lineto(99.7250036,26.79390956)
\lineto(99.7250036,27.49591152)
\closepath
\moveto(103.22809187,31.34438654)
\curveto(103.00045437,31.10202872)(102.76860135,30.8345994)(102.53253283,30.54209859)
\curveto(102.30489532,30.24959777)(102.08147333,29.94038262)(101.86226684,29.61445314)
\curveto(101.64306036,29.29688083)(101.4365004,28.97095135)(101.24258697,28.6366647)
\curveto(101.05710456,28.30237805)(100.88848418,27.97226999)(100.73672585,27.64634051)
\lineto(103.22809187,27.64634051)
\lineto(103.22809187,31.34438654)
\closepath
}
}
{
\newrgbcolor{curcolor}{0 0 0}
\pscustom[linestyle=none,fillstyle=solid,fillcolor=curcolor]
{
\newpath
\moveto(117.22780033,30.59224159)
\curveto(117.22780033,30.32481227)(117.17299871,30.06574012)(117.06339547,29.81502513)
\curveto(116.96222324,29.56431015)(116.82311144,29.31777374)(116.64606004,29.07541593)
\curveto(116.47743967,28.83305811)(116.28352624,28.59487887)(116.06431975,28.36087822)
\curveto(115.84511327,28.12687757)(115.62169127,27.8970555)(115.39405377,27.67141201)
\curveto(115.26758849,27.54605452)(115.12004566,27.39562553)(114.95142528,27.22012504)
\curveto(114.78280491,27.04462455)(114.62261556,26.86494547)(114.47085722,26.68108782)
\curveto(114.31909888,26.49723016)(114.1926336,26.31755109)(114.09146138,26.1420506)
\curveto(113.99028915,25.96655011)(113.93970304,25.81612112)(113.93970304,25.69076363)
\lineto(117.54396353,25.69076363)
\lineto(117.54396353,24.81326118)
\lineto(112.80151552,24.81326118)
\curveto(112.7930845,24.85504701)(112.78886899,24.89683284)(112.78886899,24.93861867)
\lineto(112.78886899,25.07651191)
\curveto(112.78886899,25.42751289)(112.84788612,25.75344237)(112.96592038,26.05430035)
\curveto(113.08395464,26.35515834)(113.23571298,26.63930199)(113.42119539,26.9067313)
\curveto(113.6066778,27.17416062)(113.81323776,27.42487561)(114.04087527,27.65887626)
\curveto(114.27694379,27.90123408)(114.5087968,28.13523473)(114.73643431,28.36087822)
\curveto(114.92191672,28.54473587)(115.09896811,28.72441495)(115.26758849,28.89991544)
\curveto(115.44463988,29.07541593)(115.59639821,29.25091642)(115.7228635,29.4264169)
\curveto(115.85775979,29.60191739)(115.96314753,29.78159647)(116.0390267,29.96545412)
\curveto(116.12333688,30.15766894)(116.16549198,30.35406235)(116.16549198,30.55463434)
\curveto(116.16549198,30.78027782)(116.12755239,30.97249265)(116.05167322,31.1312788)
\curveto(115.98422507,31.29006496)(115.88726836,31.41960104)(115.76080308,31.51988703)
\curveto(115.64276882,31.62853019)(115.50365701,31.70792327)(115.34346765,31.75806627)
\curveto(115.1832783,31.80820926)(115.01044242,31.83328076)(114.82496,31.83328076)
\curveto(114.60575352,31.83328076)(114.40340907,31.80403068)(114.21792666,31.74553052)
\curveto(114.04087527,31.68703035)(113.88068591,31.61599444)(113.73735859,31.53242278)
\curveto(113.60246229,31.44885112)(113.48442803,31.36527946)(113.38325581,31.28170779)
\curveto(113.28208358,31.2064933)(113.20620441,31.14381455)(113.1556183,31.09367156)
\lineto(112.63711065,31.82074501)
\curveto(112.7045588,31.89595951)(112.80573103,31.98788834)(112.94062733,32.0965315)
\curveto(113.07552362,32.20517466)(113.23571298,32.30546065)(113.42119539,32.39738948)
\curveto(113.61510882,32.49767547)(113.8300998,32.58124713)(114.06616832,32.64810446)
\curveto(114.30223685,32.71496179)(114.55516741,32.74839046)(114.82496,32.74839046)
\curveto(115.64276882,32.74839046)(116.24558665,32.56035422)(116.63341351,32.18428174)
\curveto(117.02967139,31.81656643)(117.22780033,31.28588638)(117.22780033,30.59224159)
\closepath
}
}
{
\newrgbcolor{curcolor}{0 0 0}
\pscustom[linestyle=none,fillstyle=solid,fillcolor=curcolor]
{
\newpath
\moveto(122.13465417,28.83723669)
\curveto(122.13465417,28.61995037)(122.06720602,28.43191413)(121.93230972,28.27312797)
\curveto(121.80584444,28.11434182)(121.63722407,28.03494874)(121.4264486,28.03494874)
\curveto(121.20724212,28.03494874)(121.03019072,28.11434182)(120.89529442,28.27312797)
\curveto(120.76039812,28.43191413)(120.69294997,28.61995037)(120.69294997,28.83723669)
\curveto(120.69294997,29.05452301)(120.76039812,29.24673783)(120.89529442,29.41388116)
\curveto(121.03019072,29.58102448)(121.20724212,29.66459614)(121.4264486,29.66459614)
\curveto(121.63722407,29.66459614)(121.80584444,29.58102448)(121.93230972,29.41388116)
\curveto(122.06720602,29.24673783)(122.13465417,29.05452301)(122.13465417,28.83723669)
\closepath
\moveto(118.82126382,28.69934345)
\curveto(118.82126382,30.00306137)(119.04468582,31.00174273)(119.49152981,31.69538752)
\curveto(119.94680482,32.39738948)(120.58334673,32.74839046)(121.40115555,32.74839046)
\curveto(122.22739538,32.74839046)(122.86393729,32.39738948)(123.31078128,31.69538752)
\curveto(123.75762527,31.00174273)(123.98104727,30.00306137)(123.98104727,28.69934345)
\curveto(123.98104727,27.39562553)(123.75762527,26.39276558)(123.31078128,25.69076363)
\curveto(122.86393729,24.99711883)(122.22739538,24.65029644)(121.40115555,24.65029644)
\curveto(120.58334673,24.65029644)(119.94680482,24.99711883)(119.49152981,25.69076363)
\curveto(119.04468582,26.39276558)(118.82126382,27.39562553)(118.82126382,28.69934345)
\closepath
\moveto(122.91873891,28.69934345)
\curveto(122.91873891,29.12555892)(122.89344585,29.5267029)(122.84285974,29.90277538)
\curveto(122.79227363,30.28720502)(122.70796344,30.62149167)(122.58992918,30.90563532)
\curveto(122.47189492,31.18977897)(122.31592107,31.41542245)(122.12200764,31.58256578)
\curveto(121.92809421,31.7497091)(121.68781018,31.83328076)(121.40115555,31.83328076)
\curveto(121.11450091,31.83328076)(120.87421688,31.7497091)(120.68030345,31.58256578)
\curveto(120.48639002,31.41542245)(120.33041617,31.18977897)(120.21238191,30.90563532)
\curveto(120.09434765,30.62149167)(120.01003746,30.28720502)(119.95945135,29.90277538)
\curveto(119.90886524,29.5267029)(119.88357218,29.12555892)(119.88357218,28.69934345)
\curveto(119.88357218,28.27312797)(119.90886524,27.86780541)(119.95945135,27.48337577)
\curveto(120.01003746,27.10730329)(120.09434765,26.77719523)(120.21238191,26.49305158)
\curveto(120.33041617,26.20890793)(120.48639002,25.98326444)(120.68030345,25.81612112)
\curveto(120.87421688,25.6489778)(121.11450091,25.56540613)(121.40115555,25.56540613)
\curveto(121.68781018,25.56540613)(121.92809421,25.6489778)(122.12200764,25.81612112)
\curveto(122.31592107,25.98326444)(122.47189492,26.20890793)(122.58992918,26.49305158)
\curveto(122.70796344,26.77719523)(122.79227363,27.10730329)(122.84285974,27.48337577)
\curveto(122.89344585,27.86780541)(122.91873891,28.27312797)(122.91873891,28.69934345)
\closepath
}
}
{
\newrgbcolor{curcolor}{0 0 0}
\pscustom[linestyle=none,fillstyle=solid,fillcolor=curcolor]
{
\newpath
\moveto(131.83453543,30.96831406)
\curveto(132.26451739,31.13545739)(132.68185281,31.34438654)(133.08654171,31.59510153)
\curveto(133.49123061,31.85417368)(133.86641094,32.18010316)(134.21208271,32.57288997)
\lineto(134.94558133,32.57288997)
\lineto(134.94558133,25.69076363)
\lineto(136.42522511,25.69076363)
\lineto(136.42522511,24.81326118)
\lineto(132.21393128,24.81326118)
\lineto(132.21393128,25.69076363)
\lineto(133.90856603,25.69076363)
\lineto(133.90856603,31.1312788)
\curveto(133.81582483,31.04770714)(133.70200608,30.9599569)(133.56710978,30.86802807)
\curveto(133.4406445,30.78445641)(133.29731718,30.70088475)(133.13712782,30.61731308)
\curveto(132.98536949,30.53374142)(132.82518013,30.45434834)(132.65655976,30.37913385)
\curveto(132.48793938,30.30391935)(132.32353452,30.24124061)(132.16334516,30.19109761)
\lineto(131.83453543,30.96831406)
\closepath
}
}
{
\newrgbcolor{curcolor}{0 0 0}
\pscustom[linestyle=none,fillstyle=solid,fillcolor=curcolor]
{
\newpath
\moveto(141.10444145,28.83723669)
\curveto(141.10444145,28.61995037)(141.0369933,28.43191413)(140.902097,28.27312797)
\curveto(140.77563172,28.11434182)(140.60701135,28.03494874)(140.39623588,28.03494874)
\curveto(140.17702939,28.03494874)(139.999978,28.11434182)(139.8650817,28.27312797)
\curveto(139.7301854,28.43191413)(139.66273725,28.61995037)(139.66273725,28.83723669)
\curveto(139.66273725,29.05452301)(139.7301854,29.24673783)(139.8650817,29.41388116)
\curveto(139.999978,29.58102448)(140.17702939,29.66459614)(140.39623588,29.66459614)
\curveto(140.60701135,29.66459614)(140.77563172,29.58102448)(140.902097,29.41388116)
\curveto(141.0369933,29.24673783)(141.10444145,29.05452301)(141.10444145,28.83723669)
\closepath
\moveto(137.7910511,28.69934345)
\curveto(137.7910511,30.00306137)(138.0144731,31.00174273)(138.46131709,31.69538752)
\curveto(138.9165921,32.39738948)(139.55313401,32.74839046)(140.37094282,32.74839046)
\curveto(141.19718265,32.74839046)(141.83372457,32.39738948)(142.28056856,31.69538752)
\curveto(142.72741255,31.00174273)(142.95083454,30.00306137)(142.95083454,28.69934345)
\curveto(142.95083454,27.39562553)(142.72741255,26.39276558)(142.28056856,25.69076363)
\curveto(141.83372457,24.99711883)(141.19718265,24.65029644)(140.37094282,24.65029644)
\curveto(139.55313401,24.65029644)(138.9165921,24.99711883)(138.46131709,25.69076363)
\curveto(138.0144731,26.39276558)(137.7910511,27.39562553)(137.7910511,28.69934345)
\closepath
\moveto(141.88852619,28.69934345)
\curveto(141.88852619,29.12555892)(141.86323313,29.5267029)(141.81264702,29.90277538)
\curveto(141.76206091,30.28720502)(141.67775072,30.62149167)(141.55971646,30.90563532)
\curveto(141.4416822,31.18977897)(141.28570835,31.41542245)(141.09179492,31.58256578)
\curveto(140.89788149,31.7497091)(140.65759746,31.83328076)(140.37094282,31.83328076)
\curveto(140.08428819,31.83328076)(139.84400415,31.7497091)(139.65009072,31.58256578)
\curveto(139.45617729,31.41542245)(139.30020345,31.18977897)(139.18216919,30.90563532)
\curveto(139.06413492,30.62149167)(138.97982474,30.28720502)(138.92923862,29.90277538)
\curveto(138.87865251,29.5267029)(138.85335946,29.12555892)(138.85335946,28.69934345)
\curveto(138.85335946,28.27312797)(138.87865251,27.86780541)(138.92923862,27.48337577)
\curveto(138.97982474,27.10730329)(139.06413492,26.77719523)(139.18216919,26.49305158)
\curveto(139.30020345,26.20890793)(139.45617729,25.98326444)(139.65009072,25.81612112)
\curveto(139.84400415,25.6489778)(140.08428819,25.56540613)(140.37094282,25.56540613)
\curveto(140.65759746,25.56540613)(140.89788149,25.6489778)(141.09179492,25.81612112)
\curveto(141.28570835,25.98326444)(141.4416822,26.20890793)(141.55971646,26.49305158)
\curveto(141.67775072,26.77719523)(141.76206091,27.10730329)(141.81264702,27.48337577)
\curveto(141.86323313,27.86780541)(141.88852619,28.27312797)(141.88852619,28.69934345)
\closepath
}
}
{
\newrgbcolor{curcolor}{0 0 0}
\pscustom[linestyle=none,fillstyle=solid,fillcolor=curcolor]
{
\newpath
\moveto(152.53690472,25.56540613)
\curveto(153.20295519,25.56540613)(153.67509224,25.69494221)(153.95331586,25.95401436)
\curveto(154.23997049,26.22144368)(154.38329781,26.57662324)(154.38329781,27.01955305)
\curveto(154.38329781,27.3036967)(154.32428068,27.54187593)(154.20624642,27.73409076)
\curveto(154.08821216,27.92630558)(153.93223831,28.08091315)(153.73832488,28.19791348)
\curveto(153.54441145,28.3149138)(153.32098945,28.39848547)(153.06805889,28.44862846)
\curveto(152.81512833,28.49877146)(152.54955124,28.52384296)(152.27132763,28.52384296)
\lineto(152.00575054,28.52384296)
\lineto(152.00575054,29.36373816)
\lineto(152.37249985,29.36373816)
\curveto(152.55798226,29.36373816)(152.74768018,29.38045249)(152.94159361,29.41388116)
\curveto(153.14393806,29.45566699)(153.32520496,29.52252432)(153.48539432,29.61445314)
\curveto(153.65401469,29.71473914)(153.78891099,29.8484538)(153.89008322,30.01559712)
\curveto(153.99125544,30.18274044)(154.04184155,30.39584818)(154.04184155,30.65492033)
\curveto(154.04184155,31.08113581)(153.90694525,31.38199379)(153.63715266,31.55749428)
\curveto(153.37579108,31.74135193)(153.06805889,31.83328076)(152.71395611,31.83328076)
\curveto(152.3514223,31.83328076)(152.04369012,31.77895918)(151.79075956,31.67031602)
\curveto(151.537829,31.57003003)(151.32705353,31.46556545)(151.15843316,31.35692229)
\lineto(150.75374426,32.14667449)
\curveto(150.93079565,32.27203199)(151.19637274,32.40156806)(151.55047553,32.53528272)
\curveto(151.91300933,32.67735455)(152.31348272,32.74839046)(152.75189569,32.74839046)
\curveto(153.16501561,32.74839046)(153.51911839,32.69824746)(153.81420405,32.59796147)
\curveto(154.1092897,32.49767547)(154.34957374,32.35560365)(154.53505615,32.17174599)
\curveto(154.72896958,31.98788834)(154.87229689,31.77060202)(154.9650381,31.51988703)
\curveto(155.05777931,31.27752921)(155.10414991,31.01009989)(155.10414991,30.71759908)
\curveto(155.10414991,30.30809794)(154.99454667,29.96127554)(154.77534018,29.67713189)
\curveto(154.56456471,29.39298824)(154.2905566,29.17570192)(153.95331586,29.02527293)
\curveto(154.35800475,28.9082726)(154.70789203,28.67845053)(155.00297768,28.33580672)
\curveto(155.29806334,28.00152007)(155.44560617,27.55441168)(155.44560617,26.99448155)
\curveto(155.44560617,26.6601949)(155.38658904,26.34680117)(155.26855477,26.05430035)
\curveto(155.15895153,25.7701567)(154.98611565,25.5236203)(154.75004712,25.31469115)
\curveto(154.52240962,25.10576199)(154.22310846,24.94279725)(153.85214363,24.82579693)
\curveto(153.48960983,24.7087966)(153.05541237,24.65029644)(152.54955124,24.65029644)
\curveto(152.35563781,24.65029644)(152.15329337,24.66701077)(151.9425179,24.70043943)
\curveto(151.74017345,24.72551093)(151.55047553,24.76311818)(151.37342414,24.81326118)
\curveto(151.19637274,24.85504701)(151.03618339,24.89683284)(150.89285607,24.93861867)
\curveto(150.75795977,24.98876167)(150.66100306,25.02636892)(150.60198593,25.05144041)
\lineto(150.80433037,25.94147861)
\curveto(150.93922667,25.87462128)(151.15421765,25.7952282)(151.4493033,25.70329938)
\curveto(151.74438896,25.61137055)(152.10692276,25.56540613)(152.53690472,25.56540613)
\closepath
}
}
{
\newrgbcolor{curcolor}{0 0 0}
\pscustom[linestyle=none,fillstyle=solid,fillcolor=curcolor]
{
\newpath
\moveto(158.1393176,24.81326118)
\curveto(158.18147269,25.40661998)(158.28686042,26.03340744)(158.4554808,26.69362357)
\curveto(158.63253219,27.36219686)(158.84330766,28.00569866)(159.0878072,28.62412895)
\curveto(159.33230674,29.25091642)(159.60209934,29.82756088)(159.897185,30.35406235)
\curveto(160.19227065,30.88892098)(160.4831408,31.32767221)(160.76979543,31.67031602)
\lineto(156.97583702,31.67031602)
\lineto(156.97583702,32.57288997)
\lineto(161.93327601,32.57288997)
\lineto(161.93327601,31.70792327)
\curveto(161.68034545,31.41542245)(161.40633734,31.02263564)(161.11125169,30.52956284)
\curveto(160.81616603,30.03649004)(160.53372691,29.48073848)(160.26393431,28.86230819)
\curveto(160.00257273,28.25223506)(159.77493522,27.59619751)(159.58102179,26.89419555)
\curveto(159.38710836,26.20055076)(159.26485859,25.50690597)(159.21427248,24.81326118)
\lineto(158.1393176,24.81326118)
\closepath
}
}
{
\newrgbcolor{curcolor}{0 0 0}
\pscustom[linestyle=none,fillstyle=solid,fillcolor=curcolor]
{
\newpath
\moveto(180.80189873,29.46402415)
\curveto(180.80189873,27.91794841)(180.42250288,26.75212373)(179.6637112,25.96655011)
\curveto(178.90491952,25.18933366)(177.76251649,24.79654685)(176.2365021,24.78818968)
\lineto(176.19856252,25.66569213)
\curveto(177.14283661,25.66569213)(177.90162829,25.84954978)(178.47493757,26.21726509)
\curveto(179.05667786,26.59333757)(179.44028921,27.23683937)(179.62577162,28.14777048)
\curveto(179.42342717,28.05584165)(179.20000517,27.98062716)(178.95550563,27.92212699)
\curveto(178.71100609,27.871984)(178.45386002,27.8469125)(178.18406742,27.8469125)
\curveto(177.73722343,27.8469125)(177.3620431,27.90959124)(177.05852643,28.03494874)
\curveto(176.75500975,28.1686634)(176.51051021,28.34416389)(176.3250278,28.56145021)
\curveto(176.13954539,28.78709369)(176.00464909,29.04198726)(175.9203389,29.32613091)
\curveto(175.83602871,29.61027456)(175.79387362,29.91113254)(175.79387362,30.22870486)
\curveto(175.79387362,30.51284851)(175.84024422,30.80117074)(175.93298543,31.09367156)
\curveto(176.02572663,31.39452954)(176.16905395,31.66613744)(176.36296738,31.90849526)
\curveto(176.55688081,32.15085308)(176.80559586,32.35142506)(177.10911254,32.51021122)
\curveto(177.41262921,32.66899738)(177.77516301,32.74839046)(178.19671395,32.74839046)
\curveto(179.05667786,32.74839046)(179.7058663,32.45588964)(180.14427927,31.87088801)
\curveto(180.58269224,31.28588638)(180.80189873,30.48359843)(180.80189873,29.46402415)
\closepath
\moveto(178.29788617,28.69934345)
\curveto(178.56767877,28.69934345)(178.81639382,28.72441495)(179.04403133,28.77455794)
\curveto(179.28009985,28.82470094)(179.50773736,28.89573685)(179.72694384,28.98766568)
\curveto(179.73537486,29.07123734)(179.73959037,29.15063042)(179.73959037,29.22584492)
\lineto(179.73959037,29.46402415)
\curveto(179.73959037,29.78995363)(179.71429731,30.09916878)(179.6637112,30.3916696)
\curveto(179.62155611,30.68417041)(179.54146143,30.93906398)(179.42342717,31.1563503)
\curveto(179.31382393,31.37363662)(179.15785008,31.54495853)(178.95550563,31.67031602)
\curveto(178.7615922,31.80403068)(178.51287715,31.87088801)(178.20936048,31.87088801)
\curveto(177.95642992,31.87088801)(177.74565445,31.81656643)(177.57703408,31.70792327)
\curveto(177.4084137,31.60763728)(177.26930189,31.4781012)(177.15969865,31.31931504)
\curveto(177.05009541,31.16888605)(176.97000073,30.99756414)(176.91941462,30.80534932)
\curveto(176.87725952,30.6131345)(176.85618198,30.42927685)(176.85618198,30.25377636)
\curveto(176.85618198,29.74398922)(176.97000073,29.35538099)(177.19763823,29.08795167)
\curveto(177.43370676,28.82887952)(177.80045607,28.69934345)(178.29788617,28.69934345)
\closepath
}
}
{
\newrgbcolor{curcolor}{0 0 0}
\pscustom[linestyle=none,fillstyle=solid,fillcolor=curcolor]
{
\newpath
\moveto(190.47648693,25.56540613)
\curveto(191.14253741,25.56540613)(191.61467446,25.69494221)(191.89289807,25.95401436)
\curveto(192.17955271,26.22144368)(192.32288003,26.57662324)(192.32288003,27.01955305)
\curveto(192.32288003,27.3036967)(192.2638629,27.54187593)(192.14582863,27.73409076)
\curveto(192.02779437,27.92630558)(191.87182053,28.08091315)(191.6779071,28.19791348)
\curveto(191.48399367,28.3149138)(191.26057167,28.39848547)(191.00764111,28.44862846)
\curveto(190.75471055,28.49877146)(190.48913346,28.52384296)(190.21090984,28.52384296)
\lineto(189.94533275,28.52384296)
\lineto(189.94533275,29.36373816)
\lineto(190.31208207,29.36373816)
\curveto(190.49756448,29.36373816)(190.6872624,29.38045249)(190.88117583,29.41388116)
\curveto(191.08352028,29.45566699)(191.26478718,29.52252432)(191.42497654,29.61445314)
\curveto(191.59359691,29.71473914)(191.72849321,29.8484538)(191.82966543,30.01559712)
\curveto(191.93083766,30.18274044)(191.98142377,30.39584818)(191.98142377,30.65492033)
\curveto(191.98142377,31.08113581)(191.84652747,31.38199379)(191.57673487,31.55749428)
\curveto(191.31537329,31.74135193)(191.00764111,31.83328076)(190.65353833,31.83328076)
\curveto(190.29100452,31.83328076)(189.98327234,31.77895918)(189.73034178,31.67031602)
\curveto(189.47741122,31.57003003)(189.26663575,31.46556545)(189.09801538,31.35692229)
\lineto(188.69332648,32.14667449)
\curveto(188.87037787,32.27203199)(189.13595496,32.40156806)(189.49005775,32.53528272)
\curveto(189.85259155,32.67735455)(190.25306494,32.74839046)(190.69147791,32.74839046)
\curveto(191.10459783,32.74839046)(191.45870061,32.69824746)(191.75378627,32.59796147)
\curveto(192.04887192,32.49767547)(192.28915595,32.35560365)(192.47463836,32.17174599)
\curveto(192.66855179,31.98788834)(192.81187911,31.77060202)(192.90462032,31.51988703)
\curveto(192.99736152,31.27752921)(193.04373213,31.01009989)(193.04373213,30.71759908)
\curveto(193.04373213,30.30809794)(192.93412888,29.96127554)(192.7149224,29.67713189)
\curveto(192.50414693,29.39298824)(192.23013882,29.17570192)(191.89289807,29.02527293)
\curveto(192.29758697,28.9082726)(192.64747425,28.67845053)(192.9425599,28.33580672)
\curveto(193.23764556,28.00152007)(193.38518838,27.55441168)(193.38518838,26.99448155)
\curveto(193.38518838,26.6601949)(193.32617125,26.34680117)(193.20813699,26.05430035)
\curveto(193.09853375,25.7701567)(192.92569786,25.5236203)(192.68962934,25.31469115)
\curveto(192.46199184,25.10576199)(192.16269067,24.94279725)(191.79172585,24.82579693)
\curveto(191.42919205,24.7087966)(190.99499458,24.65029644)(190.48913346,24.65029644)
\curveto(190.29522003,24.65029644)(190.09287558,24.66701077)(189.88210011,24.70043943)
\curveto(189.67975567,24.72551093)(189.49005775,24.76311818)(189.31300635,24.81326118)
\curveto(189.13595496,24.85504701)(188.9757656,24.89683284)(188.83243829,24.93861867)
\curveto(188.69754199,24.98876167)(188.60058527,25.02636892)(188.54156814,25.05144041)
\lineto(188.74391259,25.94147861)
\curveto(188.87880889,25.87462128)(189.09379987,25.7952282)(189.38888552,25.70329938)
\curveto(189.68397118,25.61137055)(190.04650498,25.56540613)(190.47648693,25.56540613)
\closepath
}
}
{
\newrgbcolor{curcolor}{0 0 0}
\pscustom[linestyle=none,fillstyle=solid,fillcolor=curcolor]
{
\newpath
\moveto(196.79975958,25.56540613)
\curveto(197.46581005,25.56540613)(197.9379471,25.69494221)(198.21617072,25.95401436)
\curveto(198.50282535,26.22144368)(198.64615267,26.57662324)(198.64615267,27.01955305)
\curveto(198.64615267,27.3036967)(198.58713554,27.54187593)(198.46910128,27.73409076)
\curveto(198.35106702,27.92630558)(198.19509317,28.08091315)(198.00117974,28.19791348)
\curveto(197.80726631,28.3149138)(197.58384431,28.39848547)(197.33091375,28.44862846)
\curveto(197.07798319,28.49877146)(196.8124061,28.52384296)(196.53418249,28.52384296)
\lineto(196.2686054,28.52384296)
\lineto(196.2686054,29.36373816)
\lineto(196.63535471,29.36373816)
\curveto(196.82083712,29.36373816)(197.01053504,29.38045249)(197.20444847,29.41388116)
\curveto(197.40679292,29.45566699)(197.58805982,29.52252432)(197.74824918,29.61445314)
\curveto(197.91686955,29.71473914)(198.05176585,29.8484538)(198.15293808,30.01559712)
\curveto(198.2541103,30.18274044)(198.30469641,30.39584818)(198.30469641,30.65492033)
\curveto(198.30469641,31.08113581)(198.16980011,31.38199379)(197.90000752,31.55749428)
\curveto(197.63864594,31.74135193)(197.33091375,31.83328076)(196.97681097,31.83328076)
\curveto(196.61427716,31.83328076)(196.30654498,31.77895918)(196.05361442,31.67031602)
\curveto(195.80068386,31.57003003)(195.58990839,31.46556545)(195.42128802,31.35692229)
\lineto(195.01659912,32.14667449)
\curveto(195.19365051,32.27203199)(195.4592276,32.40156806)(195.81333039,32.53528272)
\curveto(196.17586419,32.67735455)(196.57633758,32.74839046)(197.01475055,32.74839046)
\curveto(197.42787047,32.74839046)(197.78197325,32.69824746)(198.07705891,32.59796147)
\curveto(198.37214456,32.49767547)(198.6124286,32.35560365)(198.79791101,32.17174599)
\curveto(198.99182444,31.98788834)(199.13515175,31.77060202)(199.22789296,31.51988703)
\curveto(199.32063417,31.27752921)(199.36700477,31.01009989)(199.36700477,30.71759908)
\curveto(199.36700477,30.30809794)(199.25740153,29.96127554)(199.03819504,29.67713189)
\curveto(198.82741957,29.39298824)(198.55341146,29.17570192)(198.21617072,29.02527293)
\curveto(198.62085961,28.9082726)(198.97074689,28.67845053)(199.26583254,28.33580672)
\curveto(199.5609182,28.00152007)(199.70846103,27.55441168)(199.70846103,26.99448155)
\curveto(199.70846103,26.6601949)(199.6494439,26.34680117)(199.53140963,26.05430035)
\curveto(199.42180639,25.7701567)(199.24897051,25.5236203)(199.01290198,25.31469115)
\curveto(198.78526448,25.10576199)(198.48596331,24.94279725)(198.11499849,24.82579693)
\curveto(197.75246469,24.7087966)(197.31826723,24.65029644)(196.8124061,24.65029644)
\curveto(196.61849267,24.65029644)(196.41614823,24.66701077)(196.20537276,24.70043943)
\curveto(196.00302831,24.72551093)(195.81333039,24.76311818)(195.636279,24.81326118)
\curveto(195.4592276,24.85504701)(195.29903825,24.89683284)(195.15571093,24.93861867)
\curveto(195.02081463,24.98876167)(194.92385792,25.02636892)(194.86484078,25.05144041)
\lineto(195.06718523,25.94147861)
\curveto(195.20208153,25.87462128)(195.41707251,25.7952282)(195.71215816,25.70329938)
\curveto(196.00724382,25.61137055)(196.36977762,25.56540613)(196.79975958,25.56540613)
\closepath
}
}
{
\newrgbcolor{curcolor}{0 0 0}
\pscustom[linestyle=none,fillstyle=solid,fillcolor=curcolor]
{
\newpath
\moveto(213.77139542,27.92212699)
\curveto(213.77139542,28.67427195)(213.87256765,29.33866666)(214.07491209,29.91531113)
\curveto(214.28568756,30.50031276)(214.58077322,30.98920698)(214.96016906,31.38199379)
\curveto(215.34799592,31.7747806)(215.81591746,32.07563858)(216.36393367,32.28456774)
\curveto(216.91194989,32.49349689)(217.53162976,32.60214005)(218.22297329,32.61049722)
\lineto(218.31149899,31.73299477)
\curveto(217.864655,31.7246376)(217.45575059,31.6744946)(217.08478577,31.58256578)
\curveto(216.72225197,31.49899412)(216.39765775,31.35692229)(216.11100311,31.1563503)
\curveto(215.82434847,30.96413548)(215.58406444,30.7134205)(215.39015101,30.40420535)
\curveto(215.19623758,30.0949902)(215.04869475,29.71473914)(214.94752253,29.26345216)
\curveto(215.14986698,29.35538099)(215.36907346,29.43059549)(215.60514199,29.48909565)
\curveto(215.84964153,29.54759581)(216.1067876,29.5768459)(216.3765802,29.5768459)
\curveto(216.81499317,29.5768459)(217.18595799,29.50998857)(217.48947467,29.37627391)
\curveto(217.80142236,29.24255925)(218.05013741,29.06288018)(218.23561982,28.83723669)
\curveto(218.42110223,28.61995037)(218.55599853,28.3650568)(218.64030872,28.07255598)
\curveto(218.72461891,27.78005517)(218.766774,27.47919719)(218.766774,27.16998204)
\curveto(218.766774,26.88583839)(218.7204034,26.59333757)(218.62766219,26.29247959)
\curveto(218.53492098,25.99997877)(218.39159367,25.72837087)(218.19768024,25.47765589)
\curveto(218.00376681,25.23529807)(217.75505176,25.03472608)(217.45153508,24.87593992)
\curveto(217.14801841,24.72551093)(216.78548461,24.65029644)(216.36393367,24.65029644)
\curveto(215.49553874,24.65029644)(214.84635031,24.93861867)(214.41636835,25.51526314)
\curveto(213.9863864,26.0919076)(213.77139542,26.89419555)(213.77139542,27.92212699)
\closepath
\moveto(216.26276145,28.72441495)
\curveto(215.99296885,28.72441495)(215.7442538,28.69934345)(215.51661629,28.64920045)
\curveto(215.29740981,28.59905745)(215.07398781,28.52384296)(214.84635031,28.42355696)
\curveto(214.83791929,28.3399853)(214.83370378,28.25641364)(214.83370378,28.17284198)
\lineto(214.83370378,27.92212699)
\curveto(214.83370378,27.59619751)(214.85478132,27.28698236)(214.89693642,26.99448155)
\curveto(214.94752253,26.7103379)(215.02761721,26.45544433)(215.13722045,26.22980084)
\curveto(215.25525471,26.01251452)(215.41122856,25.83701403)(215.60514199,25.70329938)
\curveto(215.79905542,25.57794188)(216.04777047,25.51526314)(216.35128714,25.51526314)
\curveto(216.6042177,25.51526314)(216.81499317,25.56540613)(216.98361354,25.66569213)
\curveto(217.15223392,25.77433529)(217.29134573,25.90804995)(217.40094897,26.0668361)
\curveto(217.51055221,26.22562226)(217.58643138,26.40112275)(217.62858648,26.59333757)
\curveto(217.67917259,26.79390956)(217.70446564,26.9819458)(217.70446564,27.15744629)
\curveto(217.70446564,27.66723343)(217.58643138,28.05584165)(217.35036286,28.32327097)
\curveto(217.12272535,28.59070029)(216.76019155,28.72441495)(216.26276145,28.72441495)
\closepath
}
}
{
\newrgbcolor{curcolor}{0 0 0}
\pscustom[linestyle=none,fillstyle=solid,fillcolor=curcolor]
{
\newpath
\moveto(236.15574527,25.61554913)
\curveto(236.15574527,25.36483415)(236.07143508,25.14336924)(235.90281471,24.95115442)
\curveto(235.73419434,24.7589396)(235.51077234,24.66283219)(235.23254872,24.66283219)
\curveto(234.94589409,24.66283219)(234.71825658,24.7589396)(234.54963621,24.95115442)
\curveto(234.38101583,25.14336924)(234.29670565,25.36483415)(234.29670565,25.61554913)
\curveto(234.29670565,25.87462128)(234.38101583,26.10026477)(234.54963621,26.29247959)
\curveto(234.71825658,26.48469441)(234.94589409,26.58080182)(235.23254872,26.58080182)
\curveto(235.51077234,26.58080182)(235.73419434,26.48469441)(235.90281471,26.29247959)
\curveto(236.07143508,26.10026477)(236.15574527,25.87462128)(236.15574527,25.61554913)
\closepath
}
}
{
\newrgbcolor{curcolor}{0 0 0}
\pscustom[linestyle=none,fillstyle=solid,fillcolor=curcolor]
{
\newpath
\moveto(242.47901791,25.61554913)
\curveto(242.47901791,25.36483415)(242.39470773,25.14336924)(242.22608735,24.95115442)
\curveto(242.05746698,24.7589396)(241.83404498,24.66283219)(241.55582137,24.66283219)
\curveto(241.26916673,24.66283219)(241.04152923,24.7589396)(240.87290885,24.95115442)
\curveto(240.70428848,25.14336924)(240.61997829,25.36483415)(240.61997829,25.61554913)
\curveto(240.61997829,25.87462128)(240.70428848,26.10026477)(240.87290885,26.29247959)
\curveto(241.04152923,26.48469441)(241.26916673,26.58080182)(241.55582137,26.58080182)
\curveto(241.83404498,26.58080182)(242.05746698,26.48469441)(242.22608735,26.29247959)
\curveto(242.39470773,26.10026477)(242.47901791,25.87462128)(242.47901791,25.61554913)
\closepath
}
}
{
\newrgbcolor{curcolor}{0 0 0}
\pscustom[linestyle=none,fillstyle=solid,fillcolor=curcolor]
{
\newpath
\moveto(248.80227523,25.61554913)
\curveto(248.80227523,25.36483415)(248.71796504,25.14336924)(248.54934467,24.95115442)
\curveto(248.3807243,24.7589396)(248.1573023,24.66283219)(247.87907868,24.66283219)
\curveto(247.59242405,24.66283219)(247.36478654,24.7589396)(247.19616617,24.95115442)
\curveto(247.02754579,25.14336924)(246.94323561,25.36483415)(246.94323561,25.61554913)
\curveto(246.94323561,25.87462128)(247.02754579,26.10026477)(247.19616617,26.29247959)
\curveto(247.36478654,26.48469441)(247.59242405,26.58080182)(247.87907868,26.58080182)
\curveto(248.1573023,26.58080182)(248.3807243,26.48469441)(248.54934467,26.29247959)
\curveto(248.71796504,26.10026477)(248.80227523,25.87462128)(248.80227523,25.61554913)
\closepath
}
}
{
\newrgbcolor{curcolor}{0 0 0}
\pscustom[linestyle=none,fillstyle=solid,fillcolor=curcolor]
{
\newpath
\moveto(266.02684164,29.5768459)
\curveto(267.15659814,29.53506006)(267.97862247,29.28852366)(268.49291461,28.83723669)
\curveto(269.00720675,28.38594972)(269.26435282,27.78005517)(269.26435282,27.01955305)
\curveto(269.26435282,26.67690923)(269.20955119,26.35933692)(269.09994795,26.0668361)
\curveto(268.99034471,25.77433529)(268.81750883,25.5236203)(268.5814403,25.31469115)
\curveto(268.3538028,25.10576199)(268.05871714,24.94279725)(267.69618334,24.82579693)
\curveto(267.34208055,24.7087966)(266.92052962,24.65029644)(266.43153053,24.65029644)
\curveto(266.22918609,24.65029644)(266.02684164,24.66701077)(265.82449719,24.70043943)
\curveto(265.63058376,24.72551093)(265.44510135,24.7589396)(265.26804995,24.80072543)
\curveto(265.09099856,24.84251126)(264.93502472,24.88429709)(264.80012842,24.92608292)
\curveto(264.66523212,24.97622592)(264.5682754,25.01801175)(264.50925827,25.05144041)
\lineto(264.71160272,25.94147861)
\curveto(264.84649902,25.87462128)(265.05305898,25.7952282)(265.33128259,25.70329938)
\curveto(265.61793723,25.61137055)(265.97625552,25.56540613)(266.40623748,25.56540613)
\curveto(266.74347823,25.56540613)(267.02591735,25.60301338)(267.25355486,25.67822788)
\curveto(267.48119236,25.75344237)(267.66245926,25.85372837)(267.79735556,25.97908586)
\curveto(267.94068288,26.11280052)(268.04185511,26.26322951)(268.10087224,26.43037283)
\curveto(268.16832039,26.59751616)(268.20204446,26.77301665)(268.20204446,26.9568743)
\curveto(268.20204446,27.24101795)(268.15145835,27.49173294)(268.05028612,27.70901926)
\curveto(267.95754492,27.93466274)(267.78892454,28.12269898)(267.544425,28.27312797)
\curveto(267.30835648,28.42355696)(266.97954675,28.53637871)(266.55799581,28.6115932)
\curveto(266.1448759,28.69516486)(265.62215274,28.7369507)(264.98982634,28.7369507)
\curveto(265.04041245,29.10466601)(265.07835203,29.44730982)(265.10364509,29.76488213)
\curveto(265.13736916,30.09081162)(265.16266222,30.40420535)(265.17952426,30.70506333)
\curveto(265.20481731,31.01427848)(265.22167935,31.31931504)(265.23011037,31.62017302)
\curveto(265.24697241,31.92103101)(265.26383444,32.23860332)(265.28069648,32.57288997)
\lineto(269.04936184,32.57288997)
\lineto(269.04936184,31.69538752)
\lineto(266.1912465,31.69538752)
\curveto(266.18281548,31.57838719)(266.17016895,31.42377962)(266.15330692,31.2315648)
\curveto(266.1448759,31.04770714)(266.13222937,30.85131374)(266.11536733,30.64238458)
\curveto(266.0985053,30.43345543)(266.08164326,30.23288344)(266.06478122,30.04066862)
\curveto(266.04791918,29.8484538)(266.03527266,29.69384622)(266.02684164,29.5768459)
\closepath
}
}
{
\newrgbcolor{curcolor}{0 0 0}
\pscustom[linestyle=none,fillstyle=solid,fillcolor=curcolor]
{
\newpath
\moveto(275.30940184,30.59224159)
\curveto(275.30940184,30.32481227)(275.25460022,30.06574012)(275.14499698,29.81502513)
\curveto(275.04382475,29.56431015)(274.90471294,29.31777374)(274.72766155,29.07541593)
\curveto(274.55904118,28.83305811)(274.36512775,28.59487887)(274.14592126,28.36087822)
\curveto(273.92671478,28.12687757)(273.70329278,27.8970555)(273.47565528,27.67141201)
\curveto(273.34919,27.54605452)(273.20164717,27.39562553)(273.03302679,27.22012504)
\curveto(272.86440642,27.04462455)(272.70421707,26.86494547)(272.55245873,26.68108782)
\curveto(272.40070039,26.49723016)(272.27423511,26.31755109)(272.17306289,26.1420506)
\curveto(272.07189066,25.96655011)(272.02130455,25.81612112)(272.02130455,25.69076363)
\lineto(275.62556504,25.69076363)
\lineto(275.62556504,24.81326118)
\lineto(270.88311703,24.81326118)
\curveto(270.87468601,24.85504701)(270.8704705,24.89683284)(270.8704705,24.93861867)
\lineto(270.8704705,25.07651191)
\curveto(270.8704705,25.42751289)(270.92948763,25.75344237)(271.04752189,26.05430035)
\curveto(271.16555615,26.35515834)(271.31731449,26.63930199)(271.5027969,26.9067313)
\curveto(271.68827931,27.17416062)(271.89483927,27.42487561)(272.12247678,27.65887626)
\curveto(272.3585453,27.90123408)(272.59039831,28.13523473)(272.81803582,28.36087822)
\curveto(273.00351823,28.54473587)(273.18056962,28.72441495)(273.34919,28.89991544)
\curveto(273.52624139,29.07541593)(273.67799972,29.25091642)(273.80446501,29.4264169)
\curveto(273.9393613,29.60191739)(274.04474904,29.78159647)(274.12062821,29.96545412)
\curveto(274.20493839,30.15766894)(274.24709349,30.35406235)(274.24709349,30.55463434)
\curveto(274.24709349,30.78027782)(274.2091539,30.97249265)(274.13327473,31.1312788)
\curveto(274.06582658,31.29006496)(273.96886987,31.41960104)(273.84240459,31.51988703)
\curveto(273.72437033,31.62853019)(273.58525852,31.70792327)(273.42506916,31.75806627)
\curveto(273.26487981,31.80820926)(273.09204393,31.83328076)(272.90656151,31.83328076)
\curveto(272.68735503,31.83328076)(272.48501058,31.80403068)(272.29952817,31.74553052)
\curveto(272.12247678,31.68703035)(271.96228742,31.61599444)(271.8189601,31.53242278)
\curveto(271.6840638,31.44885112)(271.56602954,31.36527946)(271.46485732,31.28170779)
\curveto(271.36368509,31.2064933)(271.28780592,31.14381455)(271.23721981,31.09367156)
\lineto(270.71871216,31.82074501)
\curveto(270.78616031,31.89595951)(270.88733254,31.98788834)(271.02222884,32.0965315)
\curveto(271.15712513,32.20517466)(271.31731449,32.30546065)(271.5027969,32.39738948)
\curveto(271.69671033,32.49767547)(271.91170131,32.58124713)(272.14776983,32.64810446)
\curveto(272.38383835,32.71496179)(272.63676892,32.74839046)(272.90656151,32.74839046)
\curveto(273.72437033,32.74839046)(274.32718816,32.56035422)(274.71501502,32.18428174)
\curveto(275.1112729,31.81656643)(275.30940184,31.28588638)(275.30940184,30.59224159)
\closepath
}
}
{
\newrgbcolor{curcolor}{0 0 0}
\pscustom[linewidth=0.56097835,linecolor=curcolor]
{
\newpath
\moveto(52.53490942,35.05890672)
\lineto(277.23287695,35.05890672)
\lineto(277.23287695,22.33978765)
\lineto(52.53490942,22.33978765)
\closepath
}
}
{
\newrgbcolor{curcolor}{0 0 0}
\pscustom[linewidth=0.51800942,linecolor=curcolor]
{
\newpath
\moveto(71.388227,22.35389633)
\lineto(71.388227,34.95329133)
}
}
{
\newrgbcolor{curcolor}{0 0 0}
\pscustom[linewidth=0.5154072,linecolor=curcolor]
{
\newpath
\moveto(90.26845,22.44820733)
\lineto(90.26845,34.92133333)
}
}
{
\newrgbcolor{curcolor}{0 0 0}
\pscustom[linewidth=0.51930571,linecolor=curcolor]
{
\newpath
\moveto(109.10774,22.44820633)
\lineto(109.10774,35.11073533)
}
}
{
\newrgbcolor{curcolor}{0 0 0}
\pscustom[linewidth=0.52573889,linecolor=curcolor]
{
\newpath
\moveto(128.03631,22.26963533)
\lineto(128.03631,35.24783733)
}
}
{
\newrgbcolor{curcolor}{0 0 0}
\pscustom[linewidth=0.51670998,linecolor=curcolor]
{
\newpath
\moveto(146.87559,22.44820633)
\lineto(146.87559,34.98446633)
}
}
{
\newrgbcolor{curcolor}{0 0 0}
\pscustom[linewidth=0.52059871,linecolor=curcolor]
{
\newpath
\moveto(165.6256,22.44820733)
\lineto(165.6256,35.17387133)
}
}
{
\newrgbcolor{curcolor}{0 0 0}
\pscustom[linewidth=0.51670998,linecolor=curcolor]
{
\newpath
\moveto(184.55417,22.35892133)
\lineto(184.55417,34.89518133)
}
}
{
\newrgbcolor{curcolor}{0 0 0}
\pscustom[linewidth=0.51670998,linecolor=curcolor]
{
\newpath
\moveto(203.66131,22.44820733)
\lineto(203.66131,34.98446733)
}
}
{
\newrgbcolor{curcolor}{0 0 0}
\pscustom[linewidth=0.52059871,linecolor=curcolor]
{
\newpath
\moveto(222.58988,22.44820633)
\lineto(222.58988,35.17387033)
}
}
{
\newrgbcolor{curcolor}{0 0 0}
\pscustom[linewidth=0.52573889,linecolor=curcolor]
{
\newpath
\moveto(258.24019,35.09998133)
\lineto(258.20469,22.12182733)
}
}
{
\newrgbcolor{curcolor}{0 0 0}
\pscustom[linestyle=none,fillstyle=solid,fillcolor=curcolor]
{
\newpath
\moveto(62.73325152,8.90903606)
\curveto(63.16323347,9.07617939)(63.5805689,9.28510854)(63.9852578,9.53582353)
\curveto(64.3899467,9.79489568)(64.76512703,10.12082516)(65.11079879,10.51361197)
\lineto(65.84429742,10.51361197)
\lineto(65.84429742,3.63148563)
\lineto(67.3239412,3.63148563)
\lineto(67.3239412,2.75398318)
\lineto(63.11264736,2.75398318)
\lineto(63.11264736,3.63148563)
\lineto(64.80728212,3.63148563)
\lineto(64.80728212,9.0720008)
\curveto(64.71454092,8.98842914)(64.60072216,8.9006789)(64.46582586,8.80875007)
\curveto(64.33936058,8.72517841)(64.19603327,8.64160675)(64.03584391,8.55803508)
\curveto(63.88408557,8.47446342)(63.72389622,8.39507034)(63.55527584,8.31985585)
\curveto(63.38665547,8.24464135)(63.22225061,8.18196261)(63.06206125,8.13181961)
\lineto(62.73325152,8.90903606)
\closepath
}
}
{
\newrgbcolor{curcolor}{0 0 0}
\pscustom[linestyle=none,fillstyle=solid,fillcolor=curcolor]
{
\newpath
\moveto(76.78354322,7.5175679)
\curveto(77.91329973,7.47578206)(78.73532405,7.22924566)(79.24961619,6.77795869)
\curveto(79.76390833,6.32667172)(80.0210544,5.72077717)(80.0210544,4.96027505)
\curveto(80.0210544,4.61763123)(79.96625278,4.30005892)(79.85664954,4.0075581)
\curveto(79.74704629,3.71505729)(79.57421041,3.4643423)(79.33814189,3.25541315)
\curveto(79.11050438,3.04648399)(78.81541873,2.88351925)(78.45288492,2.76651893)
\curveto(78.09878214,2.6495186)(77.6772312,2.59101844)(77.18823212,2.59101844)
\curveto(76.98588767,2.59101844)(76.78354322,2.60773277)(76.58119877,2.64116143)
\curveto(76.38728534,2.66623293)(76.20180293,2.6996616)(76.02475154,2.74144743)
\curveto(75.84770015,2.78323326)(75.6917263,2.82501909)(75.55683,2.86680492)
\curveto(75.4219337,2.91694792)(75.32497699,2.95873375)(75.26595986,2.99216241)
\lineto(75.4683043,3.88220061)
\curveto(75.6032006,3.81534328)(75.80976056,3.7359502)(76.08798418,3.64402138)
\curveto(76.37463881,3.55209255)(76.73295711,3.50612813)(77.16293906,3.50612813)
\curveto(77.50017981,3.50612813)(77.78261894,3.54373538)(78.01025644,3.61894988)
\curveto(78.23789395,3.69416437)(78.41916085,3.79445037)(78.55405715,3.91980786)
\curveto(78.69738446,4.05352252)(78.79855669,4.20395151)(78.85757382,4.37109483)
\curveto(78.92502197,4.53823816)(78.95874604,4.71373865)(78.95874604,4.8975963)
\curveto(78.95874604,5.18173995)(78.90815993,5.43245494)(78.80698771,5.64974126)
\curveto(78.7142465,5.87538474)(78.54562613,6.06342098)(78.30112659,6.21384997)
\curveto(78.06505806,6.36427896)(77.73624833,6.47710071)(77.3146974,6.5523152)
\curveto(76.90157748,6.63588686)(76.37885432,6.6776727)(75.74652792,6.6776727)
\curveto(75.79711403,7.04538801)(75.83505362,7.38803182)(75.86034667,7.70560413)
\curveto(75.89407075,8.03153362)(75.9193638,8.34492735)(75.93622584,8.64578533)
\curveto(75.9615189,8.95500048)(75.97838094,9.26003704)(75.98681195,9.56089502)
\curveto(76.00367399,9.86175301)(76.02053603,10.17932532)(76.03739807,10.51361197)
\lineto(79.80606342,10.51361197)
\lineto(79.80606342,9.63610952)
\lineto(76.94794809,9.63610952)
\curveto(76.93951707,9.51910919)(76.92687054,9.36450162)(76.9100085,9.1722868)
\curveto(76.90157748,8.98842914)(76.88893095,8.79203574)(76.87206892,8.58310658)
\curveto(76.85520688,8.37417743)(76.83834484,8.17360544)(76.8214828,7.98139062)
\curveto(76.80462077,7.7891758)(76.79197424,7.63456822)(76.78354322,7.5175679)
\closepath
}
}
{
\newrgbcolor{curcolor}{0 0 0}
\pscustom[linestyle=none,fillstyle=solid,fillcolor=curcolor]
{
\newpath
\moveto(81.70304359,8.90903606)
\curveto(82.13302554,9.07617939)(82.55036097,9.28510854)(82.95504986,9.53582353)
\curveto(83.35973876,9.79489568)(83.73491909,10.12082516)(84.08059086,10.51361197)
\lineto(84.81408949,10.51361197)
\lineto(84.81408949,3.63148563)
\lineto(86.29373327,3.63148563)
\lineto(86.29373327,2.75398318)
\lineto(82.08243943,2.75398318)
\lineto(82.08243943,3.63148563)
\lineto(83.77707419,3.63148563)
\lineto(83.77707419,9.0720008)
\curveto(83.68433298,8.98842914)(83.57051423,8.9006789)(83.43561793,8.80875007)
\curveto(83.30915265,8.72517841)(83.16582533,8.64160675)(83.00563598,8.55803508)
\curveto(82.85387764,8.47446342)(82.69368828,8.39507034)(82.52506791,8.31985585)
\curveto(82.35644754,8.24464135)(82.19204267,8.18196261)(82.03185332,8.13181961)
\lineto(81.70304359,8.90903606)
\closepath
}
}
{
\newrgbcolor{curcolor}{0 0 0}
\pscustom[linestyle=none,fillstyle=solid,fillcolor=curcolor]
{
\newpath
\moveto(94.34956972,8.90903606)
\curveto(94.77955167,9.07617939)(95.19688709,9.28510854)(95.60157599,9.53582353)
\curveto(96.00626489,9.79489568)(96.38144522,10.12082516)(96.72711699,10.51361197)
\lineto(97.46061561,10.51361197)
\lineto(97.46061561,3.63148563)
\lineto(98.9402594,3.63148563)
\lineto(98.9402594,2.75398318)
\lineto(94.72896556,2.75398318)
\lineto(94.72896556,3.63148563)
\lineto(96.42360032,3.63148563)
\lineto(96.42360032,9.0720008)
\curveto(96.33085911,8.98842914)(96.21704036,8.9006789)(96.08214406,8.80875007)
\curveto(95.95567878,8.72517841)(95.81235146,8.64160675)(95.6521621,8.55803508)
\curveto(95.50040377,8.47446342)(95.34021441,8.39507034)(95.17159404,8.31985585)
\curveto(95.00297366,8.24464135)(94.8385688,8.18196261)(94.67837945,8.13181961)
\lineto(94.34956972,8.90903606)
\closepath
}
}
{
\newrgbcolor{curcolor}{0 0 0}
\pscustom[linestyle=none,fillstyle=solid,fillcolor=curcolor]
{
\newpath
\moveto(100.1796201,5.43663352)
\curveto(100.32294742,5.77092017)(100.51686085,6.15534981)(100.76136039,6.58992245)
\curveto(101.01429095,7.02449509)(101.29673008,7.47160348)(101.60867777,7.93124762)
\curveto(101.92062546,8.39924893)(102.2536507,8.85471448)(102.60775349,9.29764429)
\curveto(102.96185627,9.74893126)(103.32017457,10.15425382)(103.68270837,10.51361197)
\lineto(104.69443061,10.51361197)
\lineto(104.69443061,5.58706251)
\lineto(105.61762716,5.58706251)
\lineto(105.61762716,4.73463156)
\lineto(104.69443061,4.73463156)
\lineto(104.69443061,2.75398318)
\lineto(103.68270837,2.75398318)
\lineto(103.68270837,4.73463156)
\lineto(100.1796201,4.73463156)
\lineto(100.1796201,5.43663352)
\closepath
\moveto(103.68270837,9.28510854)
\curveto(103.45507087,9.04275072)(103.22321785,8.7753214)(102.98714933,8.48282059)
\curveto(102.75951182,8.19031977)(102.53608983,7.88110462)(102.31688334,7.55517514)
\curveto(102.09767686,7.23760283)(101.8911169,6.91167335)(101.69720347,6.5773867)
\curveto(101.51172106,6.24310005)(101.34310068,5.91299199)(101.19134235,5.58706251)
\lineto(103.68270837,5.58706251)
\lineto(103.68270837,9.28510854)
\closepath
}
}
{
\newrgbcolor{curcolor}{0 0 0}
\pscustom[linestyle=none,fillstyle=solid,fillcolor=curcolor]
{
\newpath
\moveto(117.68241683,8.53296359)
\curveto(117.68241683,8.26553427)(117.62761521,8.00646212)(117.51801197,7.75574713)
\curveto(117.41683974,7.50503215)(117.27772793,7.25849574)(117.10067654,7.01613793)
\curveto(116.93205617,6.77378011)(116.73814274,6.53560087)(116.51893625,6.30160022)
\curveto(116.29972976,6.06759957)(116.07630777,5.8377775)(115.84867026,5.61213401)
\curveto(115.72220498,5.48677652)(115.57466216,5.33634753)(115.40604178,5.16084704)
\curveto(115.23742141,4.98534655)(115.07723205,4.80566747)(114.92547372,4.62180982)
\curveto(114.77371538,4.43795216)(114.6472501,4.25827309)(114.54607788,4.0827726)
\curveto(114.44490565,3.90727211)(114.39431954,3.75684312)(114.39431954,3.63148563)
\lineto(117.99858003,3.63148563)
\lineto(117.99858003,2.75398318)
\lineto(113.25613202,2.75398318)
\curveto(113.247701,2.79576901)(113.24348549,2.83755484)(113.24348549,2.87934067)
\lineto(113.24348549,3.01723391)
\curveto(113.24348549,3.36823489)(113.30250262,3.69416437)(113.42053688,3.99502235)
\curveto(113.53857114,4.29588034)(113.69032948,4.58002399)(113.87581189,4.8474533)
\curveto(114.0612943,5.11488262)(114.26785426,5.36559761)(114.49549176,5.59959826)
\curveto(114.73156029,5.84195608)(114.9634133,6.07595673)(115.19105081,6.30160022)
\curveto(115.37653322,6.48545787)(115.55358461,6.66513695)(115.72220498,6.84063744)
\curveto(115.89925638,7.01613793)(116.05101471,7.19163842)(116.17747999,7.3671389)
\curveto(116.31237629,7.54263939)(116.41776403,7.72231847)(116.49364319,7.90617612)
\curveto(116.57795338,8.09839094)(116.62010848,8.29478435)(116.62010848,8.49535634)
\curveto(116.62010848,8.72099982)(116.58216889,8.91321465)(116.50628972,9.0720008)
\curveto(116.43884157,9.23078696)(116.34188486,9.36032304)(116.21541958,9.46060903)
\curveto(116.09738532,9.56925219)(115.95827351,9.64864527)(115.79808415,9.69878827)
\curveto(115.6378948,9.74893126)(115.46505891,9.77400276)(115.2795765,9.77400276)
\curveto(115.06037002,9.77400276)(114.85802557,9.74475268)(114.67254316,9.68625252)
\curveto(114.49549176,9.62775235)(114.33530241,9.55671644)(114.19197509,9.47314478)
\curveto(114.05707879,9.38957312)(113.93904453,9.30600146)(113.83787231,9.22242979)
\curveto(113.73670008,9.1472153)(113.66082091,9.08453655)(113.6102348,9.03439356)
\lineto(113.09172715,9.76146701)
\curveto(113.1591753,9.83668151)(113.26034753,9.92861034)(113.39524382,10.0372535)
\curveto(113.53014012,10.14589666)(113.69032948,10.24618265)(113.87581189,10.33811148)
\curveto(114.06972532,10.43839747)(114.2847163,10.52196913)(114.52078482,10.58882646)
\curveto(114.75685334,10.65568379)(115.0097839,10.68911246)(115.2795765,10.68911246)
\curveto(116.09738532,10.68911246)(116.70020315,10.50107622)(117.08803001,10.12500374)
\curveto(117.48428789,9.75728843)(117.68241683,9.22660838)(117.68241683,8.53296359)
\closepath
}
}
{
\newrgbcolor{curcolor}{0 0 0}
\pscustom[linestyle=none,fillstyle=solid,fillcolor=curcolor]
{
\newpath
\moveto(122.58927067,6.77795869)
\curveto(122.58927067,6.56067237)(122.52182252,6.37263613)(122.38692622,6.21384997)
\curveto(122.26046094,6.05506382)(122.09184057,5.97567074)(121.8810651,5.97567074)
\curveto(121.66185861,5.97567074)(121.48480722,6.05506382)(121.34991092,6.21384997)
\curveto(121.21501462,6.37263613)(121.14756647,6.56067237)(121.14756647,6.77795869)
\curveto(121.14756647,6.99524501)(121.21501462,7.18745983)(121.34991092,7.35460316)
\curveto(121.48480722,7.52174648)(121.66185861,7.60531814)(121.8810651,7.60531814)
\curveto(122.09184057,7.60531814)(122.26046094,7.52174648)(122.38692622,7.35460316)
\curveto(122.52182252,7.18745983)(122.58927067,6.99524501)(122.58927067,6.77795869)
\closepath
\moveto(119.27588032,6.64006545)
\curveto(119.27588032,7.94378337)(119.49930232,8.94246473)(119.94614631,9.63610952)
\curveto(120.40142132,10.33811148)(121.03796323,10.68911246)(121.85577204,10.68911246)
\curveto(122.68201188,10.68911246)(123.31855379,10.33811148)(123.76539778,9.63610952)
\curveto(124.21224177,8.94246473)(124.43566376,7.94378337)(124.43566376,6.64006545)
\curveto(124.43566376,5.33634753)(124.21224177,4.33348758)(123.76539778,3.63148563)
\curveto(123.31855379,2.93784083)(122.68201188,2.59101844)(121.85577204,2.59101844)
\curveto(121.03796323,2.59101844)(120.40142132,2.93784083)(119.94614631,3.63148563)
\curveto(119.49930232,4.33348758)(119.27588032,5.33634753)(119.27588032,6.64006545)
\closepath
\moveto(123.37335541,6.64006545)
\curveto(123.37335541,7.06628092)(123.34806235,7.4674249)(123.29747624,7.84349738)
\curveto(123.24689013,8.22792702)(123.16257994,8.56221367)(123.04454568,8.84635732)
\curveto(122.92651142,9.13050097)(122.77053757,9.35614445)(122.57662414,9.52328778)
\curveto(122.38271071,9.6904311)(122.14242668,9.77400276)(121.85577204,9.77400276)
\curveto(121.56911741,9.77400276)(121.32883338,9.6904311)(121.13491995,9.52328778)
\curveto(120.94100652,9.35614445)(120.78503267,9.13050097)(120.66699841,8.84635732)
\curveto(120.54896415,8.56221367)(120.46465396,8.22792702)(120.41406785,7.84349738)
\curveto(120.36348173,7.4674249)(120.33818868,7.06628092)(120.33818868,6.64006545)
\curveto(120.33818868,6.21384997)(120.36348173,5.80852741)(120.41406785,5.42409777)
\curveto(120.46465396,5.04802529)(120.54896415,4.71791723)(120.66699841,4.43377358)
\curveto(120.78503267,4.14962993)(120.94100652,3.92398644)(121.13491995,3.75684312)
\curveto(121.32883338,3.5896998)(121.56911741,3.50612813)(121.85577204,3.50612813)
\curveto(122.14242668,3.50612813)(122.38271071,3.5896998)(122.57662414,3.75684312)
\curveto(122.77053757,3.92398644)(122.92651142,4.14962993)(123.04454568,4.43377358)
\curveto(123.16257994,4.71791723)(123.24689013,5.04802529)(123.29747624,5.42409777)
\curveto(123.34806235,5.80852741)(123.37335541,6.21384997)(123.37335541,6.64006545)
\closepath
}
}
{
\newrgbcolor{curcolor}{0 0 0}
\pscustom[linestyle=none,fillstyle=solid,fillcolor=curcolor]
{
\newpath
\moveto(132.28915193,8.90903606)
\curveto(132.71913389,9.07617939)(133.13646931,9.28510854)(133.54115821,9.53582353)
\curveto(133.94584711,9.79489568)(134.32102744,10.12082516)(134.66669921,10.51361197)
\lineto(135.40019783,10.51361197)
\lineto(135.40019783,3.63148563)
\lineto(136.87984161,3.63148563)
\lineto(136.87984161,2.75398318)
\lineto(132.66854777,2.75398318)
\lineto(132.66854777,3.63148563)
\lineto(134.36318253,3.63148563)
\lineto(134.36318253,9.0720008)
\curveto(134.27044133,8.98842914)(134.15662257,8.9006789)(134.02172627,8.80875007)
\curveto(133.89526099,8.72517841)(133.75193368,8.64160675)(133.59174432,8.55803508)
\curveto(133.43998598,8.47446342)(133.27979663,8.39507034)(133.11117626,8.31985585)
\curveto(132.94255588,8.24464135)(132.77815102,8.18196261)(132.61796166,8.13181961)
\lineto(132.28915193,8.90903606)
\closepath
}
}
{
\newrgbcolor{curcolor}{0 0 0}
\pscustom[linestyle=none,fillstyle=solid,fillcolor=curcolor]
{
\newpath
\moveto(141.55905795,6.77795869)
\curveto(141.55905795,6.56067237)(141.4916098,6.37263613)(141.3567135,6.21384997)
\curveto(141.23024822,6.05506382)(141.06162784,5.97567074)(140.85085238,5.97567074)
\curveto(140.63164589,5.97567074)(140.4545945,6.05506382)(140.3196982,6.21384997)
\curveto(140.1848019,6.37263613)(140.11735375,6.56067237)(140.11735375,6.77795869)
\curveto(140.11735375,6.99524501)(140.1848019,7.18745983)(140.3196982,7.35460316)
\curveto(140.4545945,7.52174648)(140.63164589,7.60531814)(140.85085238,7.60531814)
\curveto(141.06162784,7.60531814)(141.23024822,7.52174648)(141.3567135,7.35460316)
\curveto(141.4916098,7.18745983)(141.55905795,6.99524501)(141.55905795,6.77795869)
\closepath
\moveto(138.2456676,6.64006545)
\curveto(138.2456676,7.94378337)(138.46908959,8.94246473)(138.91593359,9.63610952)
\curveto(139.3712086,10.33811148)(140.00775051,10.68911246)(140.82555932,10.68911246)
\curveto(141.65179915,10.68911246)(142.28834106,10.33811148)(142.73518506,9.63610952)
\curveto(143.18202905,8.94246473)(143.40545104,7.94378337)(143.40545104,6.64006545)
\curveto(143.40545104,5.33634753)(143.18202905,4.33348758)(142.73518506,3.63148563)
\curveto(142.28834106,2.93784083)(141.65179915,2.59101844)(140.82555932,2.59101844)
\curveto(140.00775051,2.59101844)(139.3712086,2.93784083)(138.91593359,3.63148563)
\curveto(138.46908959,4.33348758)(138.2456676,5.33634753)(138.2456676,6.64006545)
\closepath
\moveto(142.34314269,6.64006545)
\curveto(142.34314269,7.06628092)(142.31784963,7.4674249)(142.26726352,7.84349738)
\curveto(142.21667741,8.22792702)(142.13236722,8.56221367)(142.01433296,8.84635732)
\curveto(141.89629869,9.13050097)(141.74032485,9.35614445)(141.54641142,9.52328778)
\curveto(141.35249799,9.6904311)(141.11221396,9.77400276)(140.82555932,9.77400276)
\curveto(140.53890468,9.77400276)(140.29862065,9.6904311)(140.10470722,9.52328778)
\curveto(139.91079379,9.35614445)(139.75481995,9.13050097)(139.63678568,8.84635732)
\curveto(139.51875142,8.56221367)(139.43444124,8.22792702)(139.38385512,7.84349738)
\curveto(139.33326901,7.4674249)(139.30797596,7.06628092)(139.30797596,6.64006545)
\curveto(139.30797596,6.21384997)(139.33326901,5.80852741)(139.38385512,5.42409777)
\curveto(139.43444124,5.04802529)(139.51875142,4.71791723)(139.63678568,4.43377358)
\curveto(139.75481995,4.14962993)(139.91079379,3.92398644)(140.10470722,3.75684312)
\curveto(140.29862065,3.5896998)(140.53890468,3.50612813)(140.82555932,3.50612813)
\curveto(141.11221396,3.50612813)(141.35249799,3.5896998)(141.54641142,3.75684312)
\curveto(141.74032485,3.92398644)(141.89629869,4.14962993)(142.01433296,4.43377358)
\curveto(142.13236722,4.71791723)(142.21667741,5.04802529)(142.26726352,5.42409777)
\curveto(142.31784963,5.80852741)(142.34314269,6.21384997)(142.34314269,6.64006545)
\closepath
}
}
{
\newrgbcolor{curcolor}{0 0 0}
\pscustom[linestyle=none,fillstyle=solid,fillcolor=curcolor]
{
\newpath
\moveto(152.99152121,3.50612813)
\curveto(153.65757169,3.50612813)(154.12970874,3.63566421)(154.40793236,3.89473636)
\curveto(154.69458699,4.16216568)(154.83791431,4.51734524)(154.83791431,4.96027505)
\curveto(154.83791431,5.2444187)(154.77889718,5.48259793)(154.66086292,5.67481276)
\curveto(154.54282865,5.86702758)(154.38685481,6.02163515)(154.19294138,6.13863548)
\curveto(153.99902795,6.2556358)(153.77560595,6.33920747)(153.52267539,6.38935046)
\curveto(153.26974483,6.43949346)(153.00416774,6.46456496)(152.72594413,6.46456496)
\lineto(152.46036704,6.46456496)
\lineto(152.46036704,7.30446016)
\lineto(152.82711635,7.30446016)
\curveto(153.01259876,7.30446016)(153.20229668,7.32117449)(153.39621011,7.35460316)
\curveto(153.59855456,7.39638899)(153.77982146,7.46324632)(153.94001082,7.55517514)
\curveto(154.10863119,7.65546114)(154.24352749,7.7891758)(154.34469972,7.95631912)
\curveto(154.44587194,8.12346244)(154.49645805,8.33657018)(154.49645805,8.59564233)
\curveto(154.49645805,9.02185781)(154.36156175,9.32271579)(154.09176915,9.49821628)
\curveto(153.83040757,9.68207393)(153.52267539,9.77400276)(153.16857261,9.77400276)
\curveto(152.8060388,9.77400276)(152.49830662,9.71968118)(152.24537606,9.61103802)
\curveto(151.9924455,9.51075203)(151.78167003,9.40628745)(151.61304966,9.29764429)
\lineto(151.20836076,10.08739649)
\curveto(151.38541215,10.21275399)(151.65098924,10.34229006)(152.00509203,10.47600472)
\curveto(152.36762583,10.61807655)(152.76809922,10.68911246)(153.20651219,10.68911246)
\curveto(153.61963211,10.68911246)(153.97373489,10.63896946)(154.26882055,10.53868347)
\curveto(154.5639062,10.43839747)(154.80419023,10.29632565)(154.98967265,10.11246799)
\curveto(155.18358608,9.92861034)(155.32691339,9.71132402)(155.4196546,9.46060903)
\curveto(155.5123958,9.21825121)(155.55876641,8.95082189)(155.55876641,8.65832108)
\curveto(155.55876641,8.24881994)(155.44916316,7.90199754)(155.22995668,7.61785389)
\curveto(155.01918121,7.33371024)(154.7451731,7.11642392)(154.40793236,6.96599493)
\curveto(154.81262125,6.8489946)(155.16250853,6.61917253)(155.45759418,6.27652872)
\curveto(155.75267984,5.94224207)(155.90022266,5.49513368)(155.90022266,4.93520355)
\curveto(155.90022266,4.6009169)(155.84120553,4.28752317)(155.72317127,3.99502235)
\curveto(155.61356803,3.7108787)(155.44073215,3.4643423)(155.20466362,3.25541315)
\curveto(154.97702612,3.04648399)(154.67772495,2.88351925)(154.30676013,2.76651893)
\curveto(153.94422633,2.6495186)(153.51002886,2.59101844)(153.00416774,2.59101844)
\curveto(152.81025431,2.59101844)(152.60790986,2.60773277)(152.3971344,2.64116143)
\curveto(152.19478995,2.66623293)(152.00509203,2.70384018)(151.82804063,2.75398318)
\curveto(151.65098924,2.79576901)(151.49079989,2.83755484)(151.34747257,2.87934067)
\curveto(151.21257627,2.92948367)(151.11561955,2.96709092)(151.05660242,2.99216241)
\lineto(151.25894687,3.88220061)
\curveto(151.39384317,3.81534328)(151.60883415,3.7359502)(151.9039198,3.64402138)
\curveto(152.19900546,3.55209255)(152.56153926,3.50612813)(152.99152121,3.50612813)
\closepath
}
}
{
\newrgbcolor{curcolor}{0 0 0}
\pscustom[linestyle=none,fillstyle=solid,fillcolor=curcolor]
{
\newpath
\moveto(158.5939341,2.75398318)
\curveto(158.63608919,3.34734198)(158.74147692,3.97412944)(158.9100973,4.63434557)
\curveto(159.08714869,5.30291886)(159.29792416,5.94642066)(159.5424237,6.56485095)
\curveto(159.78692324,7.19163842)(160.05671584,7.76828288)(160.35180149,8.29478435)
\curveto(160.64688715,8.82964298)(160.93775729,9.26839421)(161.22441193,9.61103802)
\lineto(157.43045352,9.61103802)
\lineto(157.43045352,10.51361197)
\lineto(162.38789251,10.51361197)
\lineto(162.38789251,9.64864527)
\curveto(162.13496195,9.35614445)(161.86095384,8.96335764)(161.56586819,8.47028484)
\curveto(161.27078253,7.97721204)(160.98834341,7.42146048)(160.71855081,6.80303019)
\curveto(160.45718923,6.19295706)(160.22955172,5.53691951)(160.03563829,4.83491755)
\curveto(159.84172486,4.14127276)(159.71947509,3.44762797)(159.66888898,2.75398318)
\lineto(158.5939341,2.75398318)
\closepath
}
}
{
\newrgbcolor{curcolor}{0 0 0}
\pscustom[linestyle=none,fillstyle=solid,fillcolor=curcolor]
{
\newpath
\moveto(181.25651522,7.40474615)
\curveto(181.25651522,5.85867041)(180.87711938,4.69284573)(180.1183277,3.90727211)
\curveto(179.35953602,3.13005566)(178.21713298,2.73726885)(176.6911186,2.72891168)
\lineto(176.65317902,3.60641413)
\curveto(177.59745311,3.60641413)(178.35624479,3.79027178)(178.92955406,4.15798709)
\curveto(179.51129435,4.53405957)(179.89490571,5.17756137)(180.08038812,6.08849248)
\curveto(179.87804367,5.99656365)(179.65462167,5.92134916)(179.41012213,5.86284899)
\curveto(179.16562259,5.812706)(178.90847652,5.7876345)(178.63868392,5.7876345)
\curveto(178.19183993,5.7876345)(177.8166596,5.85031324)(177.51314292,5.97567074)
\curveto(177.20962625,6.1093854)(176.96512671,6.28488589)(176.7796443,6.50217221)
\curveto(176.59416189,6.72781569)(176.45926559,6.98270926)(176.3749554,7.26685291)
\curveto(176.29064521,7.55099656)(176.24849012,7.85185454)(176.24849012,8.16942686)
\curveto(176.24849012,8.45357051)(176.29486072,8.74189274)(176.38760193,9.03439356)
\curveto(176.48034313,9.33525154)(176.62367045,9.60685944)(176.81758388,9.84921726)
\curveto(177.01149731,10.09157508)(177.26021236,10.29214706)(177.56372904,10.45093322)
\curveto(177.86724571,10.60971938)(178.22977951,10.68911246)(178.65133045,10.68911246)
\curveto(179.51129435,10.68911246)(180.16048279,10.39661164)(180.59889577,9.81161001)
\curveto(181.03730874,9.22660838)(181.25651522,8.42432043)(181.25651522,7.40474615)
\closepath
\moveto(178.75250267,6.64006545)
\curveto(179.02229527,6.64006545)(179.27101032,6.66513695)(179.49864783,6.71527994)
\curveto(179.73471635,6.76542294)(179.96235385,6.83645885)(180.18156034,6.92838768)
\curveto(180.18999136,7.01195934)(180.19420687,7.09135242)(180.19420687,7.16656692)
\lineto(180.19420687,7.40474615)
\curveto(180.19420687,7.73067563)(180.16891381,8.03989078)(180.1183277,8.3323916)
\curveto(180.07617261,8.62489241)(179.99607793,8.87978598)(179.87804367,9.0970723)
\curveto(179.76844042,9.31435862)(179.61246658,9.48568053)(179.41012213,9.61103802)
\curveto(179.2162087,9.74475268)(178.96749365,9.81161001)(178.66397698,9.81161001)
\curveto(178.41104641,9.81161001)(178.20027095,9.75728843)(178.03165057,9.64864527)
\curveto(177.8630302,9.54835928)(177.72391839,9.4188232)(177.61431515,9.26003704)
\curveto(177.50471191,9.10960805)(177.42461723,8.93828614)(177.37403112,8.74607132)
\curveto(177.33187602,8.5538565)(177.31079848,8.36999885)(177.31079848,8.19449836)
\curveto(177.31079848,7.68471122)(177.42461723,7.29610299)(177.65225473,7.02867367)
\curveto(177.88832326,6.76960152)(178.25507257,6.64006545)(178.75250267,6.64006545)
\closepath
}
}
{
\newrgbcolor{curcolor}{0 0 0}
\pscustom[linestyle=none,fillstyle=solid,fillcolor=curcolor]
{
\newpath
\moveto(190.93110343,3.50612813)
\curveto(191.59715391,3.50612813)(192.06929096,3.63566421)(192.34751457,3.89473636)
\curveto(192.63416921,4.16216568)(192.77749653,4.51734524)(192.77749653,4.96027505)
\curveto(192.77749653,5.2444187)(192.71847939,5.48259793)(192.60044513,5.67481276)
\curveto(192.48241087,5.86702758)(192.32643703,6.02163515)(192.1325236,6.13863548)
\curveto(191.93861017,6.2556358)(191.71518817,6.33920747)(191.46225761,6.38935046)
\curveto(191.20932705,6.43949346)(190.94374996,6.46456496)(190.66552634,6.46456496)
\lineto(190.39994925,6.46456496)
\lineto(190.39994925,7.30446016)
\lineto(190.76669857,7.30446016)
\curveto(190.95218098,7.30446016)(191.1418789,7.32117449)(191.33579233,7.35460316)
\curveto(191.53813678,7.39638899)(191.71940368,7.46324632)(191.87959303,7.55517514)
\curveto(192.04821341,7.65546114)(192.18310971,7.7891758)(192.28428193,7.95631912)
\curveto(192.38545416,8.12346244)(192.43604027,8.33657018)(192.43604027,8.59564233)
\curveto(192.43604027,9.02185781)(192.30114397,9.32271579)(192.03135137,9.49821628)
\curveto(191.76998979,9.68207393)(191.46225761,9.77400276)(191.10815482,9.77400276)
\curveto(190.74562102,9.77400276)(190.43788884,9.71968118)(190.18495828,9.61103802)
\curveto(189.93202772,9.51075203)(189.72125225,9.40628745)(189.55263187,9.29764429)
\lineto(189.14794298,10.08739649)
\curveto(189.32499437,10.21275399)(189.59057146,10.34229006)(189.94467424,10.47600472)
\curveto(190.30720805,10.61807655)(190.70768144,10.68911246)(191.14609441,10.68911246)
\curveto(191.55921432,10.68911246)(191.91331711,10.63896946)(192.20840276,10.53868347)
\curveto(192.50348842,10.43839747)(192.74377245,10.29632565)(192.92925486,10.11246799)
\curveto(193.12316829,9.92861034)(193.26649561,9.71132402)(193.35923682,9.46060903)
\curveto(193.45197802,9.21825121)(193.49834862,8.95082189)(193.49834862,8.65832108)
\curveto(193.49834862,8.24881994)(193.38874538,7.90199754)(193.1695389,7.61785389)
\curveto(192.95876343,7.33371024)(192.68475532,7.11642392)(192.34751457,6.96599493)
\curveto(192.75220347,6.8489946)(193.10209075,6.61917253)(193.3971764,6.27652872)
\curveto(193.69226205,5.94224207)(193.83980488,5.49513368)(193.83980488,4.93520355)
\curveto(193.83980488,4.6009169)(193.78078775,4.28752317)(193.66275349,3.99502235)
\curveto(193.55315025,3.7108787)(193.38031436,3.4643423)(193.14424584,3.25541315)
\curveto(192.91660833,3.04648399)(192.61730717,2.88351925)(192.24634235,2.76651893)
\curveto(191.88380854,2.6495186)(191.44961108,2.59101844)(190.94374996,2.59101844)
\curveto(190.74983653,2.59101844)(190.54749208,2.60773277)(190.33671661,2.64116143)
\curveto(190.13437216,2.66623293)(189.94467424,2.70384018)(189.76762285,2.75398318)
\curveto(189.59057146,2.79576901)(189.4303821,2.83755484)(189.28705479,2.87934067)
\curveto(189.15215849,2.92948367)(189.05520177,2.96709092)(188.99618464,2.99216241)
\lineto(189.19852909,3.88220061)
\curveto(189.33342539,3.81534328)(189.54841637,3.7359502)(189.84350202,3.64402138)
\curveto(190.13858767,3.55209255)(190.50112148,3.50612813)(190.93110343,3.50612813)
\closepath
}
}
{
\newrgbcolor{curcolor}{0 0 0}
\pscustom[linestyle=none,fillstyle=solid,fillcolor=curcolor]
{
\newpath
\moveto(197.25437607,3.50612813)
\curveto(197.92042655,3.50612813)(198.3925636,3.63566421)(198.67078722,3.89473636)
\curveto(198.95744185,4.16216568)(199.10076917,4.51734524)(199.10076917,4.96027505)
\curveto(199.10076917,5.2444187)(199.04175204,5.48259793)(198.92371778,5.67481276)
\curveto(198.80568351,5.86702758)(198.64970967,6.02163515)(198.45579624,6.13863548)
\curveto(198.26188281,6.2556358)(198.03846081,6.33920747)(197.78553025,6.38935046)
\curveto(197.53259969,6.43949346)(197.2670226,6.46456496)(196.98879899,6.46456496)
\lineto(196.7232219,6.46456496)
\lineto(196.7232219,7.30446016)
\lineto(197.08997121,7.30446016)
\curveto(197.27545362,7.30446016)(197.46515154,7.32117449)(197.65906497,7.35460316)
\curveto(197.86140942,7.39638899)(198.04267632,7.46324632)(198.20286568,7.55517514)
\curveto(198.37148605,7.65546114)(198.50638235,7.7891758)(198.60755457,7.95631912)
\curveto(198.7087268,8.12346244)(198.75931291,8.33657018)(198.75931291,8.59564233)
\curveto(198.75931291,9.02185781)(198.62441661,9.32271579)(198.35462401,9.49821628)
\curveto(198.09326243,9.68207393)(197.78553025,9.77400276)(197.43142747,9.77400276)
\curveto(197.06889366,9.77400276)(196.76116148,9.71968118)(196.50823092,9.61103802)
\curveto(196.25530036,9.51075203)(196.04452489,9.40628745)(195.87590452,9.29764429)
\lineto(195.47121562,10.08739649)
\curveto(195.64826701,10.21275399)(195.9138441,10.34229006)(196.26794689,10.47600472)
\curveto(196.63048069,10.61807655)(197.03095408,10.68911246)(197.46936705,10.68911246)
\curveto(197.88248697,10.68911246)(198.23658975,10.63896946)(198.53167541,10.53868347)
\curveto(198.82676106,10.43839747)(199.06704509,10.29632565)(199.25252751,10.11246799)
\curveto(199.44644094,9.92861034)(199.58976825,9.71132402)(199.68250946,9.46060903)
\curveto(199.77525066,9.21825121)(199.82162127,8.95082189)(199.82162127,8.65832108)
\curveto(199.82162127,8.24881994)(199.71201802,7.90199754)(199.49281154,7.61785389)
\curveto(199.28203607,7.33371024)(199.00802796,7.11642392)(198.67078722,6.96599493)
\curveto(199.07547611,6.8489946)(199.42536339,6.61917253)(199.72044904,6.27652872)
\curveto(200.0155347,5.94224207)(200.16307752,5.49513368)(200.16307752,4.93520355)
\curveto(200.16307752,4.6009169)(200.10406039,4.28752317)(199.98602613,3.99502235)
\curveto(199.87642289,3.7108787)(199.70358701,3.4643423)(199.46751848,3.25541315)
\curveto(199.23988098,3.04648399)(198.94057981,2.88351925)(198.56961499,2.76651893)
\curveto(198.20708119,2.6495186)(197.77288372,2.59101844)(197.2670226,2.59101844)
\curveto(197.07310917,2.59101844)(196.87076472,2.60773277)(196.65998926,2.64116143)
\curveto(196.45764481,2.66623293)(196.26794689,2.70384018)(196.09089549,2.75398318)
\curveto(195.9138441,2.79576901)(195.75365475,2.83755484)(195.61032743,2.87934067)
\curveto(195.47543113,2.92948367)(195.37847441,2.96709092)(195.31945728,2.99216241)
\lineto(195.52180173,3.88220061)
\curveto(195.65669803,3.81534328)(195.87168901,3.7359502)(196.16677466,3.64402138)
\curveto(196.46186032,3.55209255)(196.82439412,3.50612813)(197.25437607,3.50612813)
\closepath
}
}
{
\newrgbcolor{curcolor}{0 0 0}
\pscustom[linestyle=none,fillstyle=solid,fillcolor=curcolor]
{
\newpath
\moveto(214.22601192,5.86284899)
\curveto(214.22601192,6.61499395)(214.32718414,7.27938866)(214.52952859,7.85603313)
\curveto(214.74030406,8.44103476)(215.03538971,8.92992898)(215.41478556,9.32271579)
\curveto(215.80261242,9.7155026)(216.27053395,10.01636058)(216.81855017,10.22528974)
\curveto(217.36656638,10.43421889)(217.98624626,10.54286205)(218.67758979,10.55121922)
\lineto(218.76611549,9.67371677)
\curveto(218.3192715,9.6653596)(217.91036709,9.6152166)(217.53940227,9.52328778)
\curveto(217.17686846,9.43971612)(216.85227424,9.29764429)(216.56561961,9.0970723)
\curveto(216.27896497,8.90485748)(216.03868094,8.6541425)(215.84476751,8.34492735)
\curveto(215.65085408,8.0357122)(215.50331125,7.65546114)(215.40213903,7.20417416)
\curveto(215.60448348,7.29610299)(215.82368996,7.37131749)(216.05975849,7.42981765)
\curveto(216.30425803,7.48831781)(216.5614041,7.5175679)(216.8311967,7.5175679)
\curveto(217.26960967,7.5175679)(217.64057449,7.45071057)(217.94409117,7.31699591)
\curveto(218.25603886,7.18328125)(218.50475391,7.00360218)(218.69023632,6.77795869)
\curveto(218.87571873,6.56067237)(219.01061503,6.3057788)(219.09492522,6.01327798)
\curveto(219.1792354,5.72077717)(219.2213905,5.41991919)(219.2213905,5.11070404)
\curveto(219.2213905,4.82656039)(219.17501989,4.53405957)(219.08227869,4.23320159)
\curveto(218.98953748,3.94070077)(218.84621017,3.66909287)(218.65229674,3.41837789)
\curveto(218.45838331,3.17602007)(218.20966825,2.97544808)(217.90615158,2.81666192)
\curveto(217.60263491,2.66623293)(217.2401011,2.59101844)(216.81855017,2.59101844)
\curveto(215.95015524,2.59101844)(215.3009668,2.87934067)(214.87098485,3.45598514)
\curveto(214.4410029,4.0326296)(214.22601192,4.83491755)(214.22601192,5.86284899)
\closepath
\moveto(216.71737794,6.66513695)
\curveto(216.44758535,6.66513695)(216.19887029,6.64006545)(215.97123279,6.58992245)
\curveto(215.7520263,6.53977945)(215.52860431,6.46456496)(215.3009668,6.36427896)
\curveto(215.29253579,6.2807073)(215.28832028,6.19713564)(215.28832028,6.11356398)
\lineto(215.28832028,5.86284899)
\curveto(215.28832028,5.53691951)(215.30939782,5.22770436)(215.35155292,4.93520355)
\curveto(215.40213903,4.6510599)(215.48223371,4.39616633)(215.59183695,4.17052284)
\curveto(215.70987121,3.95323652)(215.86584506,3.77773603)(216.05975849,3.64402138)
\curveto(216.25367192,3.51866388)(216.50238697,3.45598514)(216.80590364,3.45598514)
\curveto(217.0588342,3.45598514)(217.26960967,3.50612813)(217.43823004,3.60641413)
\curveto(217.60685042,3.71505729)(217.74596223,3.84877195)(217.85556547,4.0075581)
\curveto(217.96516871,4.16634426)(218.04104788,4.34184475)(218.08320297,4.53405957)
\curveto(218.13378909,4.73463156)(218.15908214,4.9226678)(218.15908214,5.09816829)
\curveto(218.15908214,5.60795543)(218.04104788,5.99656365)(217.80497936,6.26399297)
\curveto(217.57734185,6.53142229)(217.21480805,6.66513695)(216.71737794,6.66513695)
\closepath
}
}
{
\newrgbcolor{curcolor}{0 0 0}
\pscustom[linewidth=0.48927385,linecolor=curcolor]
{
\newpath
\moveto(52.4990589,13.03547626)
\lineto(222.46796149,13.03547626)
\lineto(222.46796149,0.24465232)
\lineto(52.4990589,0.24465232)
\closepath
}
}
{
\newrgbcolor{curcolor}{0 0 0}
\pscustom[linewidth=0.51800942,linecolor=curcolor]
{
\newpath
\moveto(71.388227,0.29461833)
\lineto(71.388227,12.89401333)
}
}
{
\newrgbcolor{curcolor}{0 0 0}
\pscustom[linewidth=0.5154072,linecolor=curcolor]
{
\newpath
\moveto(90.26845,0.38892933)
\lineto(90.26845,12.86205533)
}
}
{
\newrgbcolor{curcolor}{0 0 0}
\pscustom[linewidth=0.51930571,linecolor=curcolor]
{
\newpath
\moveto(109.10774,0.38892833)
\lineto(109.10774,13.05145733)
}
}
{
\newrgbcolor{curcolor}{0 0 0}
\pscustom[linewidth=0.52573889,linecolor=curcolor]
{
\newpath
\moveto(128.03631,0.21035733)
\lineto(128.03631,13.18855933)
}
}
{
\newrgbcolor{curcolor}{0 0 0}
\pscustom[linewidth=0.51670998,linecolor=curcolor]
{
\newpath
\moveto(146.87559,0.38892833)
\lineto(146.87559,12.92518833)
}
}
{
\newrgbcolor{curcolor}{0 0 0}
\pscustom[linewidth=0.52059871,linecolor=curcolor]
{
\newpath
\moveto(165.6256,0.38892933)
\lineto(165.6256,13.11459333)
}
}
{
\newrgbcolor{curcolor}{0 0 0}
\pscustom[linewidth=0.51670998,linecolor=curcolor]
{
\newpath
\moveto(184.55417,0.29964333)
\lineto(184.55417,12.83590333)
}
}
{
\newrgbcolor{curcolor}{0 0 0}
\pscustom[linewidth=0.51670998,linecolor=curcolor]
{
\newpath
\moveto(203.66131,0.38892933)
\lineto(203.66131,12.92518933)
}
}
{
\newrgbcolor{curcolor}{0 0 0}
\pscustom[linestyle=none,fillstyle=solid,fillcolor=curcolor]
{
\newpath
\moveto(4.73422218,7.79842477)
\curveto(4.73422218,7.31996613)(4.60831201,6.89606856)(4.35649168,6.52673207)
\curveto(4.10467134,6.16578959)(3.76471389,5.88459022)(3.33661932,5.68313395)
\curveto(3.46252949,5.48167768)(3.60103067,5.25084237)(3.75212287,4.99062803)
\curveto(3.91160909,4.73880769)(4.0710953,4.46600233)(4.23058151,4.17221194)
\curveto(4.39006772,3.88681556)(4.54535693,3.58882816)(4.69644913,3.27824975)
\curveto(4.85593534,2.97606534)(5.00283054,2.67807795)(5.13713472,2.38428755)
\lineto(4.02912524,2.38428755)
\curveto(3.77730491,2.98865636)(3.50030254,3.56784313)(3.19811814,4.12184787)
\curveto(2.90432775,4.67585261)(2.61893137,5.1249322)(2.341929,5.46908666)
\curveto(2.29156493,5.46069265)(2.21601883,5.45649565)(2.11529069,5.45649565)
\lineto(1.92642544,5.45649565)
\lineto(1.04505427,5.45649565)
\lineto(1.04505427,2.38428755)
\lineto(-0.00000012,2.38428755)
\lineto(-0.00000012,10.06480778)
\curveto(0.12591004,10.09838383)(0.26860823,10.12776287)(0.42809445,10.1529449)
\curveto(0.59597467,10.17812693)(0.76385489,10.19491496)(0.93173512,10.20330897)
\curveto(1.10800935,10.22009699)(1.28008658,10.23268801)(1.44796681,10.24108202)
\curveto(1.61584703,10.24947603)(1.76693923,10.25367304)(1.90124341,10.25367304)
\curveto(2.84137266,10.25367304)(3.5464696,10.04801976)(4.01653423,9.63671321)
\curveto(4.49499286,9.22540667)(4.73422218,8.61264385)(4.73422218,7.79842477)
\closepath
\moveto(1.98938053,9.35971084)
\curveto(1.79631827,9.35971084)(1.60745302,9.35551384)(1.42278477,9.34711983)
\curveto(1.24651054,9.34711983)(1.12060037,9.33872582)(1.04505427,9.32193779)
\lineto(1.04505427,6.31268479)
\lineto(1.71237816,6.31268479)
\curveto(2.01456256,6.31268479)(2.28317092,6.33366981)(2.51820323,6.37563987)
\curveto(2.76162956,6.41760993)(2.96728283,6.49315603)(3.13516305,6.60227817)
\curveto(3.31143729,6.71979433)(3.44574147,6.87508354)(3.53807559,7.06814579)
\curveto(3.63040971,7.26960206)(3.67657677,7.52981641)(3.67657677,7.84878883)
\curveto(3.67657677,8.15097324)(3.63040971,8.39859656)(3.53807559,8.59165882)
\curveto(3.44574147,8.78472108)(3.3240283,8.93581328)(3.1729361,9.04493543)
\curveto(3.0218439,9.16245158)(2.84137266,9.24219469)(2.63152238,9.28416474)
\curveto(2.43006611,9.33452881)(2.21601883,9.35971084)(1.98938053,9.35971084)
\closepath
}
}
{
\newrgbcolor{curcolor}{0 0 0}
\pscustom[linestyle=none,fillstyle=solid,fillcolor=curcolor]
{
\newpath
\moveto(8.86407568,8.36502052)
\curveto(9.65311273,8.36502052)(10.26167854,8.11739719)(10.68977311,7.62215053)
\curveto(11.11786768,7.13529788)(11.33191497,6.39242789)(11.33191497,5.39354056)
\lineto(11.33191497,5.02840108)
\lineto(7.1516974,5.02840108)
\curveto(7.19366746,4.42403227)(7.39092672,3.96236166)(7.74347519,3.64338923)
\curveto(8.10441767,3.33281082)(8.60805834,3.17752161)(9.2543972,3.17752161)
\curveto(9.62373369,3.17752161)(9.93850911,3.20690065)(10.19872346,3.26565873)
\curveto(10.45893781,3.32441681)(10.65619707,3.38737189)(10.79050125,3.45452398)
\lineto(10.92900243,2.57315281)
\curveto(10.80309227,2.50600072)(10.57645396,2.43465162)(10.24908753,2.35910552)
\curveto(9.92172109,2.28355942)(9.5523846,2.24578637)(9.14107805,2.24578637)
\curveto(8.63743738,2.24578637)(8.19255479,2.32133247)(7.80643027,2.47242467)
\curveto(7.42869977,2.63191088)(7.11392435,2.84595817)(6.86210402,3.11456653)
\curveto(6.61028368,3.38317489)(6.42141843,3.70214731)(6.29550826,4.0714838)
\curveto(6.16959809,4.4492143)(6.10664301,4.85632385)(6.10664301,5.29281243)
\curveto(6.10664301,5.81324112)(6.18638612,6.26651773)(6.34587233,6.65264224)
\curveto(6.50535854,7.03876675)(6.71520882,7.35773918)(6.97542317,7.60955951)
\curveto(7.23563751,7.86137985)(7.52942791,8.0502451)(7.85679434,8.17615527)
\curveto(8.18416078,8.30206544)(8.51992122,8.36502052)(8.86407568,8.36502052)
\closepath
\moveto(10.27426956,5.8719992)
\curveto(10.27426956,6.36724586)(10.14416239,6.75756738)(9.88394804,7.04296376)
\curveto(9.62373369,7.33675415)(9.27957924,7.48364935)(8.85148467,7.48364935)
\curveto(8.60805834,7.48364935)(8.38561705,7.43748228)(8.18416078,7.34514816)
\curveto(7.99109852,7.25281404)(7.8232183,7.13110088)(7.68052011,6.98000868)
\curveto(7.53782192,6.82891647)(7.42450277,6.65683925)(7.34056265,6.46377699)
\curveto(7.25662254,6.27071473)(7.20206147,6.07345547)(7.17687944,5.8719992)
\lineto(10.27426956,5.8719992)
\closepath
}
}
{
\newrgbcolor{curcolor}{0 0 0}
\pscustom[linestyle=none,fillstyle=solid,fillcolor=curcolor]
{
\newpath
\moveto(16.31795761,3.8448455)
\curveto(16.31795761,4.05469578)(16.22982049,4.22677301)(16.05354626,4.36107719)
\curveto(15.88566603,4.49538137)(15.67161875,4.61289752)(15.4114044,4.71362566)
\curveto(15.15958407,4.81435379)(14.8825817,4.91088492)(14.5803973,5.00321904)
\curveto(14.27821289,5.10394718)(13.99701352,5.22566034)(13.73679917,5.36835853)
\curveto(13.48497884,5.51105672)(13.27093155,5.68733095)(13.09465732,5.89718123)
\curveto(12.92677709,6.10703151)(12.84283698,6.37983688)(12.84283698,6.71559732)
\curveto(12.84283698,7.18566195)(13.03170223,7.57598347)(13.40943274,7.88656188)
\curveto(13.79555725,8.20553431)(14.39572905,8.36502052)(15.20994813,8.36502052)
\curveto(15.52892056,8.36502052)(15.856287,8.33983849)(16.19204744,8.28947442)
\curveto(16.5362019,8.24750436)(16.82999229,8.18874629)(17.07341862,8.11320018)
\lineto(16.88455336,7.18146494)
\curveto(16.81740128,7.21504099)(16.72506715,7.24861703)(16.607551,7.28219308)
\curveto(16.49003484,7.32416313)(16.35573066,7.35773918)(16.20463846,7.38292121)
\curveto(16.05354626,7.41649726)(15.88986304,7.44167929)(15.71358881,7.45846731)
\curveto(15.54570858,7.47525534)(15.38202536,7.48364935)(15.22253915,7.48364935)
\curveto(14.31598594,7.48364935)(13.86270934,7.23602602)(13.86270934,6.74077936)
\curveto(13.86270934,6.56450512)(13.94664945,6.41341292)(14.11452968,6.28750275)
\curveto(14.29080391,6.1699866)(14.5090482,6.06086445)(14.76926255,5.96013632)
\curveto(15.02947689,5.85940818)(15.31067627,5.75448304)(15.61286067,5.6453609)
\curveto(15.91504507,5.54463276)(16.19624445,5.4187226)(16.45645879,5.26763039)
\curveto(16.71667314,5.11653819)(16.93072043,4.93186995)(17.09860065,4.71362566)
\curveto(17.27487488,4.50377538)(17.363012,4.23516702)(17.363012,3.90780058)
\curveto(17.363012,3.37897788)(17.15735873,2.96767133)(16.74605218,2.67388094)
\curveto(16.33474563,2.38848456)(15.68420977,2.24578637)(14.79444458,2.24578637)
\curveto(14.39153204,2.24578637)(14.02219555,2.27936241)(13.68643511,2.3465145)
\curveto(13.35067466,2.41366659)(13.03170223,2.51439473)(12.72951783,2.64869891)
\lineto(12.9309741,3.59302516)
\curveto(13.22476449,3.45872099)(13.5311459,3.34540183)(13.85011832,3.25306771)
\curveto(14.17748476,3.1691276)(14.52583622,3.12715754)(14.89517272,3.12715754)
\curveto(15.84369598,3.12715754)(16.31795761,3.36638686)(16.31795761,3.8448455)
\closepath
}
}
{
\newrgbcolor{curcolor}{0 0 0}
\pscustom[linestyle=none,fillstyle=solid,fillcolor=curcolor]
{
\newpath
\moveto(23.59556435,2.54797077)
\curveto(23.36892605,2.48921269)(23.06674165,2.42625761)(22.68901114,2.35910552)
\curveto(22.31967465,2.29195343)(21.88318607,2.25837739)(21.3795454,2.25837739)
\curveto(20.93466281,2.25837739)(20.56532631,2.32133247)(20.27153592,2.44724264)
\curveto(19.97774553,2.58154682)(19.73851621,2.76621506)(19.55384797,3.00124738)
\curveto(19.36917972,3.2446737)(19.23907255,3.53007008)(19.16352645,3.85743652)
\curveto(19.08798035,4.18480295)(19.0502073,4.54574543)(19.0502073,4.94026396)
\lineto(19.0502073,8.22651934)
\lineto(20.08267067,8.22651934)
\lineto(20.08267067,5.16690226)
\curveto(20.08267067,4.4450173)(20.18759581,3.93298262)(20.39744609,3.63079821)
\curveto(20.61569038,3.32861381)(20.97663286,3.17752161)(21.48027353,3.17752161)
\curveto(21.58939568,3.17752161)(21.69851782,3.18171862)(21.80763997,3.19011263)
\curveto(21.92515612,3.19850664)(22.03427827,3.20690065)(22.1350064,3.21529466)
\curveto(22.23573454,3.23208268)(22.32387166,3.2446737)(22.39941776,3.25306771)
\curveto(22.47496386,3.26985573)(22.52532792,3.28244675)(22.55050996,3.29084076)
\lineto(22.55050996,8.22651934)
\lineto(23.59556435,8.22651934)
\lineto(23.59556435,2.54797077)
\closepath
}
}
{
\newrgbcolor{curcolor}{0 0 0}
\pscustom[linestyle=none,fillstyle=solid,fillcolor=curcolor]
{
\newpath
\moveto(28.74529116,2.24578637)
\curveto(28.37595467,2.24578637)(28.06957326,2.29615044)(27.82614694,2.39687857)
\curveto(27.58272061,2.49760671)(27.38965836,2.64869891)(27.24696017,2.85015518)
\curveto(27.10426198,3.05161144)(27.00353384,3.29923477)(26.94477576,3.59302516)
\curveto(26.88601769,3.89520957)(26.85663865,4.24356103)(26.85663865,4.63807956)
\lineto(26.85663865,10.16553592)
\lineto(25.1694424,10.16553592)
\lineto(25.1694424,11.04690709)
\lineto(27.88910202,11.04690709)
\lineto(27.88910202,4.63807956)
\curveto(27.88910202,4.34428917)(27.90589004,4.10505985)(27.93946609,3.9203916)
\curveto(27.98143614,3.73572335)(28.04019422,3.58463115)(28.11574032,3.467115)
\curveto(28.19968044,3.35799285)(28.30040857,3.28244675)(28.41792473,3.24047669)
\curveto(28.53544088,3.19850664)(28.67394207,3.17752161)(28.83342828,3.17752161)
\curveto(29.0768546,3.17752161)(29.30349291,3.20690065)(29.51334318,3.26565873)
\curveto(29.72319346,3.32441681)(29.88687668,3.38737189)(30.00439284,3.45452398)
\lineto(30.15548504,2.57315281)
\curveto(30.10512097,2.54797077)(30.03377188,2.51439473)(29.94143775,2.47242467)
\curveto(29.84910363,2.43884863)(29.73998149,2.40527258)(29.61407132,2.37169654)
\curveto(29.48816115,2.33812049)(29.34965997,2.30874145)(29.19856777,2.28355942)
\curveto(29.05586958,2.25837739)(28.90477737,2.24578637)(28.74529116,2.24578637)
\closepath
}
}
{
\newrgbcolor{curcolor}{0 0 0}
\pscustom[linestyle=none,fillstyle=solid,fillcolor=curcolor]
{
\newpath
\moveto(33.84465391,8.22651934)
\lineto(36.31249319,8.22651934)
\lineto(36.31249319,7.35773918)
\lineto(33.84465391,7.35773918)
\lineto(33.84465391,4.63807956)
\curveto(33.84465391,4.34428917)(33.86563893,4.10505985)(33.90760899,3.9203916)
\curveto(33.94957905,3.73572335)(34.01673114,3.58463115)(34.10906526,3.467115)
\curveto(34.20979339,3.35799285)(34.33570356,3.28244675)(34.48679576,3.24047669)
\curveto(34.63788796,3.19850664)(34.82255621,3.17752161)(35.0408005,3.17752161)
\curveto(35.3429849,3.17752161)(35.58641123,3.20270364)(35.77107947,3.25306771)
\curveto(35.95574772,3.30343178)(36.13202195,3.37058387)(36.29990218,3.45452398)
\lineto(36.45099438,2.57315281)
\curveto(36.33347822,2.52278874)(36.14461297,2.45563665)(35.88439862,2.37169654)
\curveto(35.63257829,2.28775643)(35.31780287,2.24578637)(34.94007237,2.24578637)
\curveto(34.50358378,2.24578637)(34.14683831,2.29615044)(33.86983594,2.39687857)
\curveto(33.59283357,2.49760671)(33.37458928,2.64869891)(33.21510307,2.85015518)
\curveto(33.06401087,3.05161144)(32.95908573,3.29923477)(32.90032765,3.59302516)
\curveto(32.84156957,3.89520957)(32.81219053,4.24356103)(32.81219053,4.63807956)
\lineto(32.81219053,7.35773918)
\lineto(31.57827089,7.35773918)
\lineto(31.57827089,8.22651934)
\lineto(32.81219053,8.22651934)
\lineto(32.81219053,9.86335152)
\lineto(33.84465391,10.03962575)
\lineto(33.84465391,8.22651934)
\closepath
}
}
{
\newrgbcolor{curcolor}{0 0 0}
\pscustom[linestyle=none,fillstyle=solid,fillcolor=curcolor]
{
\newpath
\moveto(41.16003179,3.19011263)
\curveto(41.16003179,2.93829229)(41.07609168,2.715851)(40.90821145,2.52278874)
\curveto(40.74033123,2.32972648)(40.51788993,2.23319535)(40.24088756,2.23319535)
\curveto(39.95549118,2.23319535)(39.72885288,2.32972648)(39.56097266,2.52278874)
\curveto(39.39309244,2.715851)(39.30915232,2.93829229)(39.30915232,3.19011263)
\curveto(39.30915232,3.45032697)(39.39309244,3.67696528)(39.56097266,3.87002753)
\curveto(39.72885288,4.06308979)(39.95549118,4.15962092)(40.24088756,4.15962092)
\curveto(40.51788993,4.15962092)(40.74033123,4.06308979)(40.90821145,3.87002753)
\curveto(41.07609168,3.67696528)(41.16003179,3.45032697)(41.16003179,3.19011263)
\closepath
\moveto(41.16003179,7.25701104)
\curveto(41.16003179,7.00519071)(41.07609168,6.78274941)(40.90821145,6.58968716)
\curveto(40.74033123,6.3966249)(40.51788993,6.30009377)(40.24088756,6.30009377)
\curveto(39.95549118,6.30009377)(39.72885288,6.3966249)(39.56097266,6.58968716)
\curveto(39.39309244,6.78274941)(39.30915232,7.00519071)(39.30915232,7.25701104)
\curveto(39.30915232,7.51722539)(39.39309244,7.74386369)(39.56097266,7.93692595)
\curveto(39.72885288,8.12998821)(39.95549118,8.22651934)(40.24088756,8.22651934)
\curveto(40.51788993,8.22651934)(40.74033123,8.12998821)(40.90821145,7.93692595)
\curveto(41.07609168,7.74386369)(41.16003179,7.51722539)(41.16003179,7.25701104)
\closepath
}
}
\end{pspicture}
}
\caption{Fisher-Yates algorithm in action: In this figure a 9 out of 52 shuffle has been completed to ilustrade how the algorithm works. First 1 is swpaed with 1. Then 2 is swaped with 51. 3 with 14. 4 with 20 and so on until the first 9 numbers has completed a full permutation. Resulting in 1, 51, 14, 20, 10, 37, 9, 33, 6.}
\end{figure}


%%%%%%%%%%%%%%%%%%%%% SHUFFLE NETWOKS %%%%%%%%%%%%%%%%%%%%%%
%%%%%%%%%%%%%%%%%%%%%%%%%%%%%%%%%%%%%%%%%%%%%%%%%%%%%%%%%%%%
\section{Shuffle Networks}
Shuffling networks or permutation networks has a lot of resemblance to sorting networks. The idea behind this type of networks is that they consist of a number of input wires and equally many output wires. These wires go straight through the entire network. On each pair of wires there is placed a conditional swap gate. Such that if the condition of the gate is satisfied the input on the two wires are swapped. By placing these swap gates correctly on the input wires it is possible to get a complete uniform random permutation of the input on the output wires.

Applying such a shuffle network in the setting of a poker game is simple. The input to the shuffle algorithm is the $deck$ that we want to shuffle and the output is the shuffled $deck$. The more interesting part is how to place the swap gates to ensure that the right number of possible permutations is satisfied. There are many different shuffle algorithms that can be implemented using shuffle networks. The first and only I have looked into and implemeted builds on ideas from \citeA{psi} where they introduces conditional swap. The algorithm is a combination of the well known $bubble\text{-}sort$ algorithm and the conditional swap.

\bigskip

\paragraph{Conditional Swap:}
\begin{algorithm}
\caption{\textbf{\textit{Conditional swap}} \newline
    $deck$ is initialised to hold $n$ cards $c$. \newline
    $seeds$ is initialised to hold $\frac{n^2}{2}$ random $bit$ values where $bit_i\in[0,1]$ for $i\in [1,\frac{n^2}{2}]$.
}
\label{con_swap_alg}

\begin{algorithmic}[1]
\Function{ConditionalSwap}{bit, card1, card2}
\If{bit equal 1}
\State $tmp = card1$
\State $card1 = card2$
\State $card2 = tmp$
\EndIf
\EndFunction
\State
\Function{Shuffle}{deck, seeds}
\State $index = 0$
\For{i=1 to n}
\For{j=n-1 to i}
\State $index = index + 1$
\State $bit = seeds[index]$
\State \Call{ConditionalSwap}{$bit,~deck[j],~deck[j + 1]$}
\EndFor
\EndFor
\EndFunction
\end{algorithmic}
\end{algorithm}


The conditional swap algorithm takes ones again two inputs where the first input is an array, here a $deck$ of $n$ cards $c_i$ for $i=1,\dots,n$. The second input is differint from the other algorithm. This time an array $seeds$ of size $l=\frac{n^2}{2}$ bits $b_j$ where $j=1,\dots, l$. The algorithm creates $n-1$ layers of conditional swap gates where each layer is decreasing by one in size. Each layer is constructed such that each swap gate overlap with one input wire. Then each layer is made such that the first layer contains $n-1$ swap gates. The second layer $n-2$ and so on to the last layer containing only one gate. The layers are stacked in a way such that the first input wire is only represented in the first layer. This resembles what is done in the $Fisher\text{-}Yates$ algorithm, where output $c_1$ is determined by the first round in the for-loop. It can be seen in such a way that the first input $c_1$ has $n$ places to go. The second layer determines which output $c_2$ will have. Continuing this way until reaching the last layer and the two last outputs $c_{n-1}$ and $c_n$ is determined. Resulting in a shuffle algorithm with a perfect shuffle and $n!$ different permutations as wanted.

This algorithm requires much more randomness as input compared to the $Fisher\text{-}Yates$ algorithm. But this algorithm does not have the same problems of how the randomness should be chosen. This algorithm uses one bit of randomness at a time and therefore do not suffer from the problem of choosing randomness outside of the correct interval. Therefore this algorithm is more robust in terms of the input seeds. But if the inner for-loop is not running fewer and fewer rounds it will suffer the same problems as encountered by $Fisher\text{-}Yates$ because it will produce $n^n$ distinct possible sequences of swaps which is not compatible with the $n!$ possible permutations. This resulting in a skew of the probability of the different permutations such that this is no longer uniform.

\bigskip
Once again it is possible to do some optimization to the algorithm since we do not need a complete permutation of the $n$ inputs, but it is enough to only have $m$ out of $n$. This can be done as in the case for the $Fisher\text{-}Yates$ where we let the outer loop run for $m$ iterations and then we are done. This yields $n$ possible values for $c_1$, $n-1$ possible values for $c_2$ and so one until $n-m$ values for $c_{n-m}$. This is exactly the amount of permutation we require for our algorithm as this gives us $\frac{n!}{(n-m)!}$.

\begin{figure}
\label{con_swap_fig}
\centering
\scalebox{1.5}{%LaTeX with PSTricks extensions
%%Creator: inkscape 0.91
%%Please note this file requires PSTricks extensions
\psset{xunit=.5pt,yunit=.5pt,runit=.5pt}
\begin{pspicture}(277.51337787,256.36721033)
{
\newrgbcolor{curcolor}{0 0 0}
\pscustom[linestyle=none,fillstyle=solid,fillcolor=curcolor]
{
\newpath
\moveto(8.54929742,246.17650195)
\curveto(9.07812013,246.17650195)(9.48103266,246.27723009)(9.75803503,246.47868635)
\curveto(10.0350374,246.68014262)(10.17353859,246.98232703)(10.17353859,247.38523956)
\curveto(10.17353859,247.62866589)(10.12317452,247.83431916)(10.02244639,248.00219938)
\curveto(9.92171825,248.17847362)(9.78741407,248.32956582)(9.61953385,248.45547599)
\curveto(9.45165363,248.58978017)(9.26278837,248.70309932)(9.05293809,248.79543344)
\curveto(8.84308781,248.89616157)(8.62904053,248.9884957)(8.41079624,249.07243581)
\curveto(8.1589759,249.17316394)(7.91554958,249.28648309)(7.68051727,249.41239326)
\curveto(7.45387896,249.53830343)(7.24822569,249.68519863)(7.06355744,249.85307885)
\curveto(6.88728321,250.02095907)(6.74458502,250.21821834)(6.63546287,250.44485664)
\curveto(6.53473474,250.67149494)(6.48437067,250.9401033)(6.48437067,251.25068171)
\curveto(6.48437067,251.93059662)(6.69841796,252.45941932)(7.12651253,252.83714982)
\curveto(7.5546071,253.22327434)(8.15058189,253.41633659)(8.91443691,253.41633659)
\curveto(9.12428719,253.41633659)(9.32994046,253.39954857)(9.53139673,253.36597253)
\curveto(9.74124701,253.34079049)(9.93430927,253.30721445)(10.1105835,253.26524439)
\curveto(10.28685774,253.22327434)(10.44214694,253.17291027)(10.57645112,253.11415219)
\curveto(10.71914931,253.05539411)(10.83246846,252.99663604)(10.91640858,252.93787796)
\lineto(10.58904214,252.0690978)
\curveto(10.42116192,252.16982593)(10.19032661,252.27055407)(9.89653622,252.3712822)
\curveto(9.60274583,252.47201034)(9.27537939,252.5223744)(8.91443691,252.5223744)
\curveto(8.53670641,252.5223744)(8.20933997,252.42584328)(7.9323376,252.23278102)
\curveto(7.65533523,252.04811277)(7.51683405,251.7669134)(7.51683405,251.38918289)
\curveto(7.51683405,251.1709386)(7.5546071,250.98627036)(7.6301532,250.83517816)
\curveto(7.71409331,250.68408596)(7.82321546,250.54978178)(7.95751964,250.43226562)
\curveto(8.10021783,250.32314347)(8.26390104,250.22241534)(8.44856929,250.13008122)
\curveto(8.63323754,250.03774709)(8.8346938,249.94541297)(9.05293809,249.85307885)
\curveto(9.37191052,249.71877467)(9.6615039,249.58447049)(9.92171825,249.45016631)
\curveto(10.19032661,249.31586213)(10.41696491,249.15637592)(10.60163316,248.97170767)
\curveto(10.79469541,248.78703943)(10.94159061,248.56879514)(11.04231874,248.3169748)
\curveto(11.15144089,248.06515447)(11.20600196,247.75877306)(11.20600196,247.39783058)
\curveto(11.20600196,246.71791567)(10.97516665,246.19328997)(10.51349604,245.82395348)
\curveto(10.06021944,245.463011)(9.40548656,245.28253976)(8.54929742,245.28253976)
\curveto(8.27229505,245.28253976)(8.01208071,245.30352479)(7.76865438,245.34549485)
\curveto(7.52522806,245.37907089)(7.30698377,245.42523795)(7.11392151,245.48399603)
\curveto(6.92925327,245.54275411)(6.76557005,245.60151219)(6.62287186,245.66027026)
\curveto(6.48017367,245.71902834)(6.37105152,245.77358942)(6.29550542,245.82395348)
\lineto(6.61028084,246.68014262)
\curveto(6.77816106,246.5878085)(7.02578439,246.47868635)(7.35315083,246.35277619)
\curveto(7.68051727,246.23526003)(8.0792328,246.17650195)(8.54929742,246.17650195)
\closepath
}
}
{
\newrgbcolor{curcolor}{0 0 0}
\pscustom[linestyle=none,fillstyle=solid,fillcolor=curcolor]
{
\newpath
\moveto(15.15958123,251.42695594)
\curveto(15.94861828,251.42695594)(16.55718409,251.17933262)(16.98527866,250.68408596)
\curveto(17.41337323,250.19723331)(17.62742051,249.45436332)(17.62742051,248.45547599)
\lineto(17.62742051,248.0903365)
\lineto(13.44720295,248.0903365)
\curveto(13.489173,247.4859677)(13.68643227,247.02429708)(14.03898073,246.70532466)
\curveto(14.39992322,246.39474624)(14.90356389,246.23945704)(15.54990275,246.23945704)
\curveto(15.91923924,246.23945704)(16.23401466,246.26883608)(16.494229,246.32759415)
\curveto(16.75444335,246.38635223)(16.95170261,246.44930732)(17.08600679,246.5164594)
\lineto(17.22450798,245.63508823)
\curveto(17.09859781,245.56793614)(16.87195951,245.49658705)(16.54459307,245.42104095)
\curveto(16.21722664,245.34549485)(15.84789014,245.3077218)(15.4365836,245.3077218)
\curveto(14.93294293,245.3077218)(14.48806033,245.3832679)(14.10193582,245.5343601)
\curveto(13.72420532,245.69384631)(13.4094299,245.90789359)(13.15760956,246.17650195)
\curveto(12.90578923,246.44511031)(12.71692397,246.76408273)(12.59101381,247.13341923)
\curveto(12.46510364,247.51114973)(12.40214855,247.91825927)(12.40214855,248.35474785)
\curveto(12.40214855,248.87517655)(12.48189166,249.32845315)(12.64137787,249.71457766)
\curveto(12.80086409,250.10070218)(13.01071437,250.4196746)(13.27092871,250.67149494)
\curveto(13.53114306,250.92331527)(13.82493345,251.11218053)(14.15229989,251.23809069)
\curveto(14.47966632,251.36400086)(14.81542677,251.42695594)(15.15958123,251.42695594)
\closepath
\moveto(16.56977511,248.93393462)
\curveto(16.56977511,249.42918128)(16.43966793,249.8195028)(16.17945359,250.10489918)
\curveto(15.91923924,250.39868958)(15.57508478,250.54558477)(15.14699021,250.54558477)
\curveto(14.90356389,250.54558477)(14.68112259,250.49941771)(14.47966632,250.40708359)
\curveto(14.28660406,250.31474946)(14.11872384,250.1930363)(13.97602565,250.0419441)
\curveto(13.83332746,249.8908519)(13.72000831,249.71877467)(13.6360682,249.52571241)
\curveto(13.55212809,249.33265016)(13.49756701,249.13539089)(13.47238498,248.93393462)
\lineto(16.56977511,248.93393462)
\closepath
}
}
{
\newrgbcolor{curcolor}{0 0 0}
\pscustom[linestyle=none,fillstyle=solid,fillcolor=curcolor]
{
\newpath
\moveto(21.45508961,251.42695594)
\curveto(22.24412666,251.42695594)(22.85269247,251.17933262)(23.28078704,250.68408596)
\curveto(23.70888161,250.19723331)(23.9229289,249.45436332)(23.9229289,248.45547599)
\lineto(23.9229289,248.0903365)
\lineto(19.74271133,248.0903365)
\curveto(19.78468139,247.4859677)(19.98194065,247.02429708)(20.33448912,246.70532466)
\curveto(20.6954316,246.39474624)(21.19907227,246.23945704)(21.84541113,246.23945704)
\curveto(22.21474762,246.23945704)(22.52952304,246.26883608)(22.78973739,246.32759415)
\curveto(23.04995174,246.38635223)(23.247211,246.44930732)(23.38151518,246.5164594)
\lineto(23.52001636,245.63508823)
\curveto(23.39410619,245.56793614)(23.16746789,245.49658705)(22.84010146,245.42104095)
\curveto(22.51273502,245.34549485)(22.14339853,245.3077218)(21.73209198,245.3077218)
\curveto(21.22845131,245.3077218)(20.78356872,245.3832679)(20.3974442,245.5343601)
\curveto(20.0197137,245.69384631)(19.70493828,245.90789359)(19.45311795,246.17650195)
\curveto(19.20129761,246.44511031)(19.01243236,246.76408273)(18.88652219,247.13341923)
\curveto(18.76061202,247.51114973)(18.69765694,247.91825927)(18.69765694,248.35474785)
\curveto(18.69765694,248.87517655)(18.77740005,249.32845315)(18.93688626,249.71457766)
\curveto(19.09637247,250.10070218)(19.30622275,250.4196746)(19.5664371,250.67149494)
\curveto(19.82665144,250.92331527)(20.12044183,251.11218053)(20.44780827,251.23809069)
\curveto(20.77517471,251.36400086)(21.11093515,251.42695594)(21.45508961,251.42695594)
\closepath
\moveto(22.86528349,248.93393462)
\curveto(22.86528349,249.42918128)(22.73517632,249.8195028)(22.47496197,250.10489918)
\curveto(22.21474762,250.39868958)(21.87059317,250.54558477)(21.4424986,250.54558477)
\curveto(21.19907227,250.54558477)(20.97663097,250.49941771)(20.77517471,250.40708359)
\curveto(20.58211245,250.31474946)(20.41423223,250.1930363)(20.27153404,250.0419441)
\curveto(20.12883585,249.8908519)(20.01551669,249.71877467)(19.93157658,249.52571241)
\curveto(19.84763647,249.33265016)(19.7930754,249.13539089)(19.76789336,248.93393462)
\lineto(22.86528349,248.93393462)
\closepath
}
}
{
\newrgbcolor{curcolor}{0 0 0}
\pscustom[linestyle=none,fillstyle=solid,fillcolor=curcolor]
{
\newpath
\moveto(28.8460155,250.05453512)
\curveto(28.72849935,250.16365726)(28.55642212,250.2643854)(28.32978381,250.35671952)
\curveto(28.10314551,250.45744765)(27.8681132,250.50781172)(27.62468688,250.50781172)
\curveto(27.34768451,250.50781172)(27.10845519,250.45325065)(26.90699892,250.3441285)
\curveto(26.71393666,250.23500636)(26.55445045,250.08391416)(26.42854028,249.8908519)
\curveto(26.30263011,249.70618365)(26.21029599,249.47954535)(26.15153791,249.21093699)
\curveto(26.09277984,248.95072265)(26.0634008,248.66952327)(26.0634008,248.36733887)
\curveto(26.0634008,247.68742396)(26.22288701,247.16279827)(26.54185943,246.79346177)
\curveto(26.86083186,246.42412528)(27.27633541,246.23945704)(27.78837009,246.23945704)
\curveto(28.04858444,246.23945704)(28.26682873,246.25204805)(28.44310297,246.27723009)
\curveto(28.62777121,246.30241212)(28.76207539,246.32759415)(28.8460155,246.35277619)
\lineto(28.8460155,250.05453512)
\closepath
\moveto(28.8460155,253.99552337)
\lineto(29.89106989,254.1717976)
\lineto(29.89106989,245.6099062)
\curveto(29.66443159,245.54275411)(29.37483821,245.47560202)(29.02228974,245.40844993)
\curveto(28.66974127,245.34129784)(28.25843472,245.3077218)(27.78837009,245.3077218)
\curveto(27.36866953,245.3077218)(26.98674203,245.37907089)(26.64258757,245.52176908)
\curveto(26.29843311,245.66446727)(26.00464272,245.86592354)(25.76121639,246.12613789)
\curveto(25.51779007,246.39474624)(25.32892482,246.71791567)(25.19462064,247.09564618)
\curveto(25.06031646,247.47337668)(24.99316437,247.89727424)(24.99316437,248.36733887)
\curveto(24.99316437,248.82061547)(25.04772544,249.23192202)(25.15684759,249.60125851)
\curveto(25.27436375,249.97898902)(25.44224397,250.30215845)(25.66048826,250.5707668)
\curveto(25.87873255,250.83937516)(26.1389469,251.04922544)(26.4411313,251.20031764)
\curveto(26.75170971,251.35140984)(27.10425818,251.42695594)(27.49877671,251.42695594)
\curveto(27.80935512,251.42695594)(28.08216048,251.38918289)(28.3171928,251.31363679)
\curveto(28.55222511,251.23809069)(28.72849935,251.15834759)(28.8460155,251.07440748)
\lineto(28.8460155,253.99552337)
\closepath
}
}
{
\newrgbcolor{curcolor}{0 0 0}
\pscustom[linestyle=none,fillstyle=solid,fillcolor=curcolor]
{
\newpath
\moveto(35.20447992,246.90678092)
\curveto(35.20447992,247.1166312)(35.11634281,247.28870843)(34.94006857,247.42301261)
\curveto(34.77218835,247.55731679)(34.55814106,247.67483295)(34.29792672,247.77556108)
\curveto(34.04610638,247.87628922)(33.76910401,247.97282034)(33.46691961,248.06515447)
\curveto(33.16473521,248.1658826)(32.88353583,248.28759576)(32.62332149,248.43029395)
\curveto(32.37150115,248.57299214)(32.15745387,248.74926638)(31.98117963,248.95911666)
\curveto(31.81329941,249.16896694)(31.7293593,249.4417723)(31.7293593,249.77753275)
\curveto(31.7293593,250.24759737)(31.91822455,250.63791889)(32.29595505,250.94849731)
\curveto(32.68207956,251.26746973)(33.28225136,251.42695594)(34.09647045,251.42695594)
\curveto(34.41544287,251.42695594)(34.74280931,251.40177391)(35.07856976,251.35140984)
\curveto(35.42272421,251.30943979)(35.71651461,251.25068171)(35.95994093,251.17513561)
\lineto(35.77107568,250.24340037)
\curveto(35.70392359,250.27697641)(35.61158947,250.31055246)(35.49407331,250.3441285)
\curveto(35.37655715,250.38609856)(35.24225297,250.4196746)(35.09116077,250.44485664)
\curveto(34.94006857,250.47843268)(34.77638535,250.50361472)(34.60011112,250.52040274)
\curveto(34.4322309,250.53719076)(34.26854768,250.54558477)(34.10906147,250.54558477)
\curveto(33.20250826,250.54558477)(32.74923165,250.29796144)(32.74923165,249.80271478)
\curveto(32.74923165,249.62644055)(32.83317177,249.47534835)(33.00105199,249.34943818)
\curveto(33.17732622,249.23192202)(33.39557052,249.12279988)(33.65578486,249.02207174)
\curveto(33.91599921,248.92134361)(34.19719858,248.81641847)(34.49938299,248.70729632)
\curveto(34.80156739,248.60656819)(35.08276676,248.48065802)(35.34298111,248.32956582)
\curveto(35.60319546,248.17847362)(35.81724274,247.99380537)(35.98512296,247.77556108)
\curveto(36.1613972,247.5657108)(36.24953432,247.29710244)(36.24953432,246.96973601)
\curveto(36.24953432,246.4409133)(36.04388104,246.02960676)(35.63257449,245.73581637)
\curveto(35.22126795,245.45041999)(34.57073208,245.3077218)(33.6809669,245.3077218)
\curveto(33.27805436,245.3077218)(32.90871787,245.34129784)(32.57295742,245.40844993)
\curveto(32.23719697,245.47560202)(31.91822455,245.57633015)(31.61604015,245.71063433)
\lineto(31.81749641,246.65496059)
\curveto(32.1112868,246.52065641)(32.41766821,246.40733726)(32.73664064,246.31500314)
\curveto(33.06400707,246.23106302)(33.41235854,246.18909297)(33.78169503,246.18909297)
\curveto(34.73021829,246.18909297)(35.20447992,246.42832229)(35.20447992,246.90678092)
\closepath
}
}
{
\newrgbcolor{curcolor}{0 0 0}
\pscustom[linestyle=none,fillstyle=solid,fillcolor=curcolor]
{
\newpath
\moveto(41.16003181,246.25204805)
\curveto(41.16003181,246.00022772)(41.0760917,245.77778642)(40.90821147,245.58472416)
\curveto(40.74033125,245.39166191)(40.51788995,245.29513078)(40.24088759,245.29513078)
\curveto(39.95549121,245.29513078)(39.7288529,245.39166191)(39.56097268,245.58472416)
\curveto(39.39309246,245.77778642)(39.30915234,246.00022772)(39.30915234,246.25204805)
\curveto(39.30915234,246.5122624)(39.39309246,246.7389007)(39.56097268,246.93196296)
\curveto(39.7288529,247.12502522)(39.95549121,247.22155634)(40.24088759,247.22155634)
\curveto(40.51788995,247.22155634)(40.74033125,247.12502522)(40.90821147,246.93196296)
\curveto(41.0760917,246.7389007)(41.16003181,246.5122624)(41.16003181,246.25204805)
\closepath
\moveto(41.16003181,250.31894647)
\curveto(41.16003181,250.06712613)(41.0760917,249.84468484)(40.90821147,249.65162258)
\curveto(40.74033125,249.45856032)(40.51788995,249.36202919)(40.24088759,249.36202919)
\curveto(39.95549121,249.36202919)(39.7288529,249.45856032)(39.56097268,249.65162258)
\curveto(39.39309246,249.84468484)(39.30915234,250.06712613)(39.30915234,250.31894647)
\curveto(39.30915234,250.57916082)(39.39309246,250.80579912)(39.56097268,250.99886137)
\curveto(39.7288529,251.19192363)(39.95549121,251.28845476)(40.24088759,251.28845476)
\curveto(40.51788995,251.28845476)(40.74033125,251.19192363)(40.90821147,250.99886137)
\curveto(41.0760917,250.80579912)(41.16003181,250.57916082)(41.16003181,250.31894647)
\closepath
}
}
{
\newrgbcolor{curcolor}{0 0 0}
\pscustom[linestyle=none,fillstyle=solid,fillcolor=curcolor]
{
\newpath
\moveto(17.69037607,229.18181275)
\curveto(17.69037607,228.45992779)(17.59804195,227.84716497)(17.41337371,227.3435243)
\curveto(17.23709947,226.83988363)(16.98947614,226.42857708)(16.67050372,226.10960466)
\curveto(16.3599253,225.79902625)(15.98639181,225.57238794)(15.54990322,225.42968975)
\curveto(15.12180865,225.28699156)(14.65594103,225.21564247)(14.15230036,225.21564247)
\curveto(13.63187167,225.21564247)(13.10724597,225.27859755)(12.57842327,225.40450772)
\lineto(12.57842327,232.95911778)
\curveto(13.10724597,233.08502795)(13.63187167,233.14798303)(14.15230036,233.14798303)
\curveto(14.65594103,233.14798303)(15.12180865,233.07663394)(15.54990322,232.93393575)
\curveto(15.98639181,232.79123756)(16.3599253,232.56040225)(16.67050372,232.24142983)
\curveto(16.98947614,231.9224574)(17.23709947,231.51115085)(17.41337371,231.00751018)
\curveto(17.59804195,230.50386951)(17.69037607,229.8953037)(17.69037607,229.18181275)
\closepath
\moveto(13.61088664,226.14737771)
\curveto(13.82073692,226.12219568)(14.03478421,226.10960466)(14.2530285,226.10960466)
\curveto(14.630759,226.10960466)(14.96651945,226.16416573)(15.26030984,226.27328788)
\curveto(15.55410023,226.39080403)(15.80172356,226.57127527)(16.00317983,226.8147016)
\curveto(16.2046361,227.06652193)(16.3599253,227.38549436)(16.46904745,227.77161887)
\curveto(16.57816959,228.1661374)(16.63273067,228.63620202)(16.63273067,229.18181275)
\curveto(16.63273067,230.23945816)(16.42707739,231.01590419)(16.01577084,231.51115085)
\curveto(15.61285831,232.00639751)(15.01268651,232.25402084)(14.21525545,232.25402084)
\curveto(14.1061333,232.25402084)(13.99701116,232.24982384)(13.88788901,232.24142983)
\curveto(13.78716088,232.24142983)(13.69482675,232.23303581)(13.61088664,232.21624779)
\lineto(13.61088664,226.14737771)
\closepath
}
}
{
\newrgbcolor{curcolor}{0 0 0}
\pscustom[linestyle=none,fillstyle=solid,fillcolor=curcolor]
{
\newpath
\moveto(21.45509009,231.25933052)
\curveto(22.24412714,231.25933052)(22.85269295,231.01170719)(23.28078752,230.51646053)
\curveto(23.70888209,230.02960788)(23.92292938,229.28673789)(23.92292938,228.28785056)
\lineto(23.92292938,227.92271107)
\lineto(19.74271181,227.92271107)
\curveto(19.78468186,227.31834227)(19.98194113,226.85667165)(20.3344896,226.53769923)
\curveto(20.69543208,226.22712082)(21.19907275,226.07183161)(21.84541161,226.07183161)
\curveto(22.2147481,226.07183161)(22.52952352,226.10121065)(22.78973787,226.15996873)
\curveto(23.04995221,226.2187268)(23.24721148,226.28168189)(23.38151565,226.34883398)
\lineto(23.52001684,225.4674628)
\curveto(23.39410667,225.40031071)(23.16746837,225.32896162)(22.84010193,225.25341552)
\curveto(22.5127355,225.17786942)(22.14339901,225.14009637)(21.73209246,225.14009637)
\curveto(21.22845179,225.14009637)(20.78356919,225.21564247)(20.39744468,225.36673467)
\curveto(20.01971418,225.52622088)(19.70493876,225.74026817)(19.45311842,226.00887652)
\curveto(19.20129809,226.27748488)(19.01243284,226.59645731)(18.88652267,226.9657938)
\curveto(18.7606125,227.3435243)(18.69765742,227.75063384)(18.69765742,228.18712243)
\curveto(18.69765742,228.70755112)(18.77740052,229.16082772)(18.93688674,229.54695224)
\curveto(19.09637295,229.93307675)(19.30622323,230.25204918)(19.56643757,230.50386951)
\curveto(19.82665192,230.75568985)(20.12044231,230.9445551)(20.44780875,231.07046527)
\curveto(20.77517518,231.19637543)(21.11093563,231.25933052)(21.45509009,231.25933052)
\closepath
\moveto(22.86528397,228.7663092)
\curveto(22.86528397,229.26155586)(22.73517679,229.65187738)(22.47496245,229.93727376)
\curveto(22.2147481,230.23106415)(21.87059364,230.37795934)(21.44249907,230.37795934)
\curveto(21.19907275,230.37795934)(20.97663145,230.33179228)(20.77517518,230.23945816)
\curveto(20.58211293,230.14712404)(20.4142327,230.02541087)(20.27153451,229.87431867)
\curveto(20.12883632,229.72322647)(20.01551717,229.55114924)(19.93157706,229.35808699)
\curveto(19.84763695,229.16502473)(19.79307588,228.96776547)(19.76789384,228.7663092)
\lineto(22.86528397,228.7663092)
\closepath
}
}
{
\newrgbcolor{curcolor}{0 0 0}
\pscustom[linestyle=none,fillstyle=solid,fillcolor=curcolor]
{
\newpath
\moveto(25.10648495,228.18712243)
\curveto(25.10648495,228.71594513)(25.19042506,229.17341874)(25.35830529,229.55954325)
\curveto(25.52618551,229.94566777)(25.75702082,230.26464019)(26.05081121,230.51646053)
\curveto(26.3446016,230.76828086)(26.68455905,230.95294911)(27.07068357,231.07046527)
\curveto(27.46520209,231.19637543)(27.88490265,231.25933052)(28.32978525,231.25933052)
\curveto(28.61518163,231.25933052)(28.896381,231.23834549)(29.17338337,231.19637543)
\curveto(29.45877975,231.16279939)(29.76096415,231.0956473)(30.07993658,230.99491917)
\lineto(29.84070726,230.10095697)
\curveto(29.56370489,230.20168511)(29.30768755,230.26464019)(29.07265523,230.28982223)
\curveto(28.84601693,230.32339827)(28.61518163,230.34018629)(28.38014931,230.34018629)
\curveto(28.07796491,230.34018629)(27.79256853,230.29821624)(27.52396017,230.21427613)
\curveto(27.25535181,230.13873003)(27.0203195,230.01281986)(26.81886323,229.83654562)
\curveto(26.62580098,229.6686654)(26.47051177,229.4462241)(26.35299561,229.16922173)
\curveto(26.23547946,228.90061338)(26.17672138,228.57324694)(26.17672138,228.18712243)
\curveto(26.17672138,227.81778593)(26.23128245,227.49881351)(26.3404046,227.23020515)
\curveto(26.44952674,226.9699908)(26.60061894,226.75174651)(26.7936812,226.57547228)
\curveto(26.99513747,226.40759206)(27.23436679,226.28168189)(27.51136916,226.19774178)
\curveto(27.78837152,226.11380166)(28.09475293,226.07183161)(28.43051338,226.07183161)
\curveto(28.69912174,226.07183161)(28.95513908,226.08442263)(29.1985654,226.10960466)
\curveto(29.45038574,226.1431807)(29.7231911,226.21033279)(30.01698149,226.31106093)
\lineto(30.16807369,225.44228077)
\curveto(29.8742833,225.33315862)(29.57629591,225.25761252)(29.2741115,225.21564247)
\curveto(28.9719271,225.1652784)(28.64456066,225.14009637)(28.29201219,225.14009637)
\curveto(27.82194757,225.14009637)(27.38965599,225.20305145)(26.99513747,225.32896162)
\curveto(26.60901295,225.4632658)(26.27325251,225.65632806)(25.98785613,225.90814839)
\curveto(25.71085376,226.15996873)(25.49260947,226.47474415)(25.33312325,226.85247465)
\curveto(25.18203105,227.23859916)(25.10648495,227.68348176)(25.10648495,228.18712243)
\closepath
}
}
{
\newrgbcolor{curcolor}{0 0 0}
\pscustom[linestyle=none,fillstyle=solid,fillcolor=curcolor]
{
\newpath
\moveto(33.78169455,228.48930683)
\curveto(34.00833285,228.3214266)(34.2643502,228.10737932)(34.54974658,227.84716497)
\curveto(34.83514296,227.59534464)(35.11634233,227.32253927)(35.3933447,227.02874888)
\curveto(35.67874108,226.73495849)(35.94734944,226.43277409)(36.19916977,226.12219568)
\curveto(36.45099011,225.82001127)(36.65244638,225.5388119)(36.80353858,225.27859755)
\lineto(35.56961893,225.27859755)
\curveto(35.41013272,225.5388119)(35.21287346,225.80322325)(34.97784115,226.07183161)
\curveto(34.74280883,226.34883398)(34.4909885,226.61324533)(34.22238014,226.86506567)
\curveto(33.96216579,227.116886)(33.69775444,227.34772131)(33.42914608,227.55757159)
\curveto(33.16893174,227.76742187)(32.93389942,227.9436961)(32.72404914,228.08639429)
\lineto(32.72404914,225.27859755)
\lineto(31.67899475,225.27859755)
\lineto(31.67899475,233.82789794)
\lineto(32.72404914,234.00417217)
\lineto(32.72404914,228.67817208)
\curveto(33.18571976,229.08108462)(33.64739037,229.47980015)(34.10906099,229.87431867)
\curveto(34.5707316,230.27723121)(34.98623516,230.69273476)(35.35557165,231.12082933)
\lineto(36.57690028,231.12082933)
\curveto(36.21595779,230.69273476)(35.77946921,230.25204918)(35.26743453,229.79877257)
\curveto(34.76379386,229.34549597)(34.2685472,228.90900739)(33.78169455,228.48930683)
\closepath
}
}
{
\newrgbcolor{curcolor}{0 0 0}
\pscustom[linestyle=none,fillstyle=solid,fillcolor=curcolor]
{
\newpath
\moveto(41.16003133,226.08442263)
\curveto(41.16003133,225.83260229)(41.07609122,225.61016099)(40.908211,225.41709874)
\curveto(40.74033077,225.22403648)(40.51788948,225.12750535)(40.24088711,225.12750535)
\curveto(39.95549073,225.12750535)(39.72885243,225.22403648)(39.5609722,225.41709874)
\curveto(39.39309198,225.61016099)(39.30915187,225.83260229)(39.30915187,226.08442263)
\curveto(39.30915187,226.34463697)(39.39309198,226.57127527)(39.5609722,226.76433753)
\curveto(39.72885243,226.95739979)(39.95549073,227.05393092)(40.24088711,227.05393092)
\curveto(40.51788948,227.05393092)(40.74033077,226.95739979)(40.908211,226.76433753)
\curveto(41.07609122,226.57127527)(41.16003133,226.34463697)(41.16003133,226.08442263)
\closepath
\moveto(41.16003133,230.15132104)
\curveto(41.16003133,229.89950071)(41.07609122,229.67705941)(40.908211,229.48399715)
\curveto(40.74033077,229.2909349)(40.51788948,229.19440377)(40.24088711,229.19440377)
\curveto(39.95549073,229.19440377)(39.72885243,229.2909349)(39.5609722,229.48399715)
\curveto(39.39309198,229.67705941)(39.30915187,229.89950071)(39.30915187,230.15132104)
\curveto(39.30915187,230.41153539)(39.39309198,230.63817369)(39.5609722,230.83123595)
\curveto(39.72885243,231.0242982)(39.95549073,231.12082933)(40.24088711,231.12082933)
\curveto(40.51788948,231.12082933)(40.74033077,231.0242982)(40.908211,230.83123595)
\curveto(41.07609122,230.63817369)(41.16003133,230.41153539)(41.16003133,230.15132104)
\closepath
}
}
{
\newrgbcolor{curcolor}{0 0 0}
\pscustom[linestyle=none,fillstyle=solid,fillcolor=curcolor]
{
\newpath
\moveto(64.36246136,249.86567069)
\curveto(64.36246136,249.6474264)(64.29530927,249.45856114)(64.16100509,249.29907493)
\curveto(64.03509493,249.13958872)(63.8672147,249.05984561)(63.65736442,249.05984561)
\curveto(63.43912013,249.05984561)(63.2628459,249.13958872)(63.12854172,249.29907493)
\curveto(62.99423754,249.45856114)(62.92708545,249.6474264)(62.92708545,249.86567069)
\curveto(62.92708545,250.08391498)(62.99423754,250.27697723)(63.12854172,250.44485746)
\curveto(63.2628459,250.61273768)(63.43912013,250.69667779)(63.65736442,250.69667779)
\curveto(63.8672147,250.69667779)(64.03509493,250.61273768)(64.16100509,250.44485746)
\curveto(64.29530927,250.27697723)(64.36246136,250.08391498)(64.36246136,249.86567069)
\closepath
\moveto(61.06361497,249.7271695)
\curveto(61.06361497,251.03663525)(61.28605626,252.03971958)(61.73093886,252.73642251)
\curveto(62.18421546,253.44151945)(62.8179633,253.79406792)(63.63218239,253.79406792)
\curveto(64.45479549,253.79406792)(65.08854333,253.44151945)(65.53342592,252.73642251)
\curveto(65.97830851,252.03971958)(66.20074981,251.03663525)(66.20074981,249.7271695)
\curveto(66.20074981,248.41770376)(65.97830851,247.41042242)(65.53342592,246.70532548)
\curveto(65.08854333,246.00862255)(64.45479549,245.66027108)(63.63218239,245.66027108)
\curveto(62.8179633,245.66027108)(62.18421546,246.00862255)(61.73093886,246.70532548)
\curveto(61.28605626,247.41042242)(61.06361497,248.41770376)(61.06361497,249.7271695)
\closepath
\moveto(65.1431044,249.7271695)
\curveto(65.1431044,250.15526407)(65.11792237,250.55817661)(65.0675583,250.93590711)
\curveto(65.01719423,251.32203163)(64.93325412,251.65779207)(64.81573797,251.94318845)
\curveto(64.69822181,252.22858483)(64.5429326,252.45522313)(64.34987035,252.62310336)
\curveto(64.15680809,252.79098358)(63.91757877,252.87492369)(63.63218239,252.87492369)
\curveto(63.34678601,252.87492369)(63.10755669,252.79098358)(62.91449443,252.62310336)
\curveto(62.72143218,252.45522313)(62.56614297,252.22858483)(62.44862681,251.94318845)
\curveto(62.33111066,251.65779207)(62.24717054,251.32203163)(62.19680648,250.93590711)
\curveto(62.14644241,250.55817661)(62.12126038,250.15526407)(62.12126038,249.7271695)
\curveto(62.12126038,249.29907493)(62.14644241,248.89196539)(62.19680648,248.50584087)
\curveto(62.24717054,248.12811037)(62.33111066,247.79654693)(62.44862681,247.51115055)
\curveto(62.56614297,247.22575417)(62.72143218,246.99911587)(62.91449443,246.83123564)
\curveto(63.10755669,246.66335542)(63.34678601,246.57941531)(63.63218239,246.57941531)
\curveto(63.91757877,246.57941531)(64.15680809,246.66335542)(64.34987035,246.83123564)
\curveto(64.5429326,246.99911587)(64.69822181,247.22575417)(64.81573797,247.51115055)
\curveto(64.93325412,247.79654693)(65.01719423,248.12811037)(65.0675583,248.50584087)
\curveto(65.11792237,248.89196539)(65.1431044,249.29907493)(65.1431044,249.7271695)
\closepath
}
}
{
\newrgbcolor{curcolor}{0 0 0}
\pscustom[linestyle=none,fillstyle=solid,fillcolor=curcolor]
{
\newpath
\moveto(83.24898652,249.86567069)
\curveto(83.24898652,249.6474264)(83.18183443,249.45856114)(83.04753025,249.29907493)
\curveto(82.92162008,249.13958872)(82.75373986,249.05984561)(82.54388958,249.05984561)
\curveto(82.32564529,249.05984561)(82.14937105,249.13958872)(82.01506687,249.29907493)
\curveto(81.88076269,249.45856114)(81.8136106,249.6474264)(81.8136106,249.86567069)
\curveto(81.8136106,250.08391498)(81.88076269,250.27697723)(82.01506687,250.44485746)
\curveto(82.14937105,250.61273768)(82.32564529,250.69667779)(82.54388958,250.69667779)
\curveto(82.75373986,250.69667779)(82.92162008,250.61273768)(83.04753025,250.44485746)
\curveto(83.18183443,250.27697723)(83.24898652,250.08391498)(83.24898652,249.86567069)
\closepath
\moveto(79.95014012,249.7271695)
\curveto(79.95014012,251.03663525)(80.17258142,252.03971958)(80.61746401,252.73642251)
\curveto(81.07074062,253.44151945)(81.70448846,253.79406792)(82.51870754,253.79406792)
\curveto(83.34132064,253.79406792)(83.97506848,253.44151945)(84.41995108,252.73642251)
\curveto(84.86483367,252.03971958)(85.08727496,251.03663525)(85.08727496,249.7271695)
\curveto(85.08727496,248.41770376)(84.86483367,247.41042242)(84.41995108,246.70532548)
\curveto(83.97506848,246.00862255)(83.34132064,245.66027108)(82.51870754,245.66027108)
\curveto(81.70448846,245.66027108)(81.07074062,246.00862255)(80.61746401,246.70532548)
\curveto(80.17258142,247.41042242)(79.95014012,248.41770376)(79.95014012,249.7271695)
\closepath
\moveto(84.02962956,249.7271695)
\curveto(84.02962956,250.15526407)(84.00444752,250.55817661)(83.95408346,250.93590711)
\curveto(83.90371939,251.32203163)(83.81977928,251.65779207)(83.70226312,251.94318845)
\curveto(83.58474696,252.22858483)(83.42945776,252.45522313)(83.2363955,252.62310336)
\curveto(83.04333324,252.79098358)(82.80410392,252.87492369)(82.51870754,252.87492369)
\curveto(82.23331116,252.87492369)(81.99408184,252.79098358)(81.80101959,252.62310336)
\curveto(81.60795733,252.45522313)(81.45266812,252.22858483)(81.33515197,251.94318845)
\curveto(81.21763581,251.65779207)(81.1336957,251.32203163)(81.08333163,250.93590711)
\curveto(81.03296756,250.55817661)(81.00778553,250.15526407)(81.00778553,249.7271695)
\curveto(81.00778553,249.29907493)(81.03296756,248.89196539)(81.08333163,248.50584087)
\curveto(81.1336957,248.12811037)(81.21763581,247.79654693)(81.33515197,247.51115055)
\curveto(81.45266812,247.22575417)(81.60795733,246.99911587)(81.80101959,246.83123564)
\curveto(81.99408184,246.66335542)(82.23331116,246.57941531)(82.51870754,246.57941531)
\curveto(82.80410392,246.57941531)(83.04333324,246.66335542)(83.2363955,246.83123564)
\curveto(83.42945776,246.99911587)(83.58474696,247.22575417)(83.70226312,247.51115055)
\curveto(83.81977928,247.79654693)(83.90371939,248.12811037)(83.95408346,248.50584087)
\curveto(84.00444752,248.89196539)(84.02962956,249.29907493)(84.02962956,249.7271695)
\closepath
}
}
{
\newrgbcolor{curcolor}{0 0 0}
\pscustom[linestyle=none,fillstyle=solid,fillcolor=curcolor]
{
\newpath
\moveto(99.20180381,252.00614354)
\curveto(99.62989838,252.17402376)(100.04540193,252.38387404)(100.44831447,252.63569437)
\curveto(100.85122701,252.89590872)(101.2247605,253.22327516)(101.56891496,253.61779368)
\lineto(102.29919393,253.61779368)
\lineto(102.29919393,246.70532548)
\lineto(103.7723429,246.70532548)
\lineto(103.7723429,245.8239543)
\lineto(99.57953431,245.8239543)
\lineto(99.57953431,246.70532548)
\lineto(101.26673056,246.70532548)
\lineto(101.26673056,252.16982675)
\curveto(101.17439644,252.08588664)(101.06107729,251.99774953)(100.92677311,251.9054154)
\curveto(100.80086294,251.82147529)(100.65816475,251.73753518)(100.49867854,251.65359507)
\curveto(100.34758634,251.56965496)(100.18810012,251.48991185)(100.0202199,251.41436575)
\curveto(99.85233968,251.33881965)(99.68865646,251.27586456)(99.52917025,251.2255005)
\lineto(99.20180381,252.00614354)
\closepath
}
}
{
\newrgbcolor{curcolor}{0 0 0}
\pscustom[linestyle=none,fillstyle=solid,fillcolor=curcolor]
{
\newpath
\moveto(118.08833182,252.00614354)
\curveto(118.51642639,252.17402376)(118.93192995,252.38387404)(119.33484248,252.63569437)
\curveto(119.73775502,252.89590872)(120.11128852,253.22327516)(120.45544298,253.61779368)
\lineto(121.18572195,253.61779368)
\lineto(121.18572195,246.70532548)
\lineto(122.65887091,246.70532548)
\lineto(122.65887091,245.8239543)
\lineto(118.46606233,245.8239543)
\lineto(118.46606233,246.70532548)
\lineto(120.15325857,246.70532548)
\lineto(120.15325857,252.16982675)
\curveto(120.06092445,252.08588664)(119.9476053,251.99774953)(119.81330112,251.9054154)
\curveto(119.68739095,251.82147529)(119.54469276,251.73753518)(119.38520655,251.65359507)
\curveto(119.23411435,251.56965496)(119.07462814,251.48991185)(118.90674791,251.41436575)
\curveto(118.73886769,251.33881965)(118.57518447,251.27586456)(118.41569826,251.2255005)
\lineto(118.08833182,252.00614354)
\closepath
}
}
{
\newrgbcolor{curcolor}{0 0 0}
\pscustom[linestyle=none,fillstyle=solid,fillcolor=curcolor]
{
\newpath
\moveto(139.90855912,249.86567069)
\curveto(139.90855912,249.6474264)(139.84140703,249.45856114)(139.70710285,249.29907493)
\curveto(139.58119268,249.13958872)(139.41331246,249.05984561)(139.20346218,249.05984561)
\curveto(138.98521789,249.05984561)(138.80894365,249.13958872)(138.67463947,249.29907493)
\curveto(138.5403353,249.45856114)(138.47318321,249.6474264)(138.47318321,249.86567069)
\curveto(138.47318321,250.08391498)(138.5403353,250.27697723)(138.67463947,250.44485746)
\curveto(138.80894365,250.61273768)(138.98521789,250.69667779)(139.20346218,250.69667779)
\curveto(139.41331246,250.69667779)(139.58119268,250.61273768)(139.70710285,250.44485746)
\curveto(139.84140703,250.27697723)(139.90855912,250.08391498)(139.90855912,249.86567069)
\closepath
\moveto(136.60971272,249.7271695)
\curveto(136.60971272,251.03663525)(136.83215402,252.03971958)(137.27703661,252.73642251)
\curveto(137.73031322,253.44151945)(138.36406106,253.79406792)(139.17828014,253.79406792)
\curveto(140.00089324,253.79406792)(140.63464108,253.44151945)(141.07952368,252.73642251)
\curveto(141.52440627,252.03971958)(141.74684757,251.03663525)(141.74684757,249.7271695)
\curveto(141.74684757,248.41770376)(141.52440627,247.41042242)(141.07952368,246.70532548)
\curveto(140.63464108,246.00862255)(140.00089324,245.66027108)(139.17828014,245.66027108)
\curveto(138.36406106,245.66027108)(137.73031322,246.00862255)(137.27703661,246.70532548)
\curveto(136.83215402,247.41042242)(136.60971272,248.41770376)(136.60971272,249.7271695)
\closepath
\moveto(140.68920216,249.7271695)
\curveto(140.68920216,250.15526407)(140.66402012,250.55817661)(140.61365606,250.93590711)
\curveto(140.56329199,251.32203163)(140.47935188,251.65779207)(140.36183572,251.94318845)
\curveto(140.24431956,252.22858483)(140.08903036,252.45522313)(139.8959681,252.62310336)
\curveto(139.70290584,252.79098358)(139.46367652,252.87492369)(139.17828014,252.87492369)
\curveto(138.89288376,252.87492369)(138.65365445,252.79098358)(138.46059219,252.62310336)
\curveto(138.26752993,252.45522313)(138.11224073,252.22858483)(137.99472457,251.94318845)
\curveto(137.87720841,251.65779207)(137.7932683,251.32203163)(137.74290423,250.93590711)
\curveto(137.69254017,250.55817661)(137.66735813,250.15526407)(137.66735813,249.7271695)
\curveto(137.66735813,249.29907493)(137.69254017,248.89196539)(137.74290423,248.50584087)
\curveto(137.7932683,248.12811037)(137.87720841,247.79654693)(137.99472457,247.51115055)
\curveto(138.11224073,247.22575417)(138.26752993,246.99911587)(138.46059219,246.83123564)
\curveto(138.65365445,246.66335542)(138.89288376,246.57941531)(139.17828014,246.57941531)
\curveto(139.46367652,246.57941531)(139.70290584,246.66335542)(139.8959681,246.83123564)
\curveto(140.08903036,246.99911587)(140.24431956,247.22575417)(140.36183572,247.51115055)
\curveto(140.47935188,247.79654693)(140.56329199,248.12811037)(140.61365606,248.50584087)
\curveto(140.66402012,248.89196539)(140.68920216,249.29907493)(140.68920216,249.7271695)
\closepath
}
}
{
\newrgbcolor{curcolor}{0 0 0}
\pscustom[linestyle=none,fillstyle=solid,fillcolor=curcolor]
{
\newpath
\moveto(155.86138023,252.00614354)
\curveto(156.2894748,252.17402376)(156.70497835,252.38387404)(157.10789089,252.63569437)
\curveto(157.51080342,252.89590872)(157.88433692,253.22327516)(158.22849138,253.61779368)
\lineto(158.95877035,253.61779368)
\lineto(158.95877035,246.70532548)
\lineto(160.43191931,246.70532548)
\lineto(160.43191931,245.8239543)
\lineto(156.23911073,245.8239543)
\lineto(156.23911073,246.70532548)
\lineto(157.92630698,246.70532548)
\lineto(157.92630698,252.16982675)
\curveto(157.83397285,252.08588664)(157.7206537,251.99774953)(157.58634952,251.9054154)
\curveto(157.46043936,251.82147529)(157.31774116,251.73753518)(157.15825495,251.65359507)
\curveto(157.00716275,251.56965496)(156.84767654,251.48991185)(156.67979632,251.41436575)
\curveto(156.51191609,251.33881965)(156.34823287,251.27586456)(156.18874666,251.2255005)
\lineto(155.86138023,252.00614354)
\closepath
}
}
{
\newrgbcolor{curcolor}{0 0 0}
\pscustom[linestyle=none,fillstyle=solid,fillcolor=curcolor]
{
\newpath
\moveto(177.68161515,249.86567069)
\curveto(177.68161515,249.6474264)(177.61446306,249.45856114)(177.48015888,249.29907493)
\curveto(177.35424871,249.13958872)(177.18636849,249.05984561)(176.97651821,249.05984561)
\curveto(176.75827392,249.05984561)(176.58199968,249.13958872)(176.4476955,249.29907493)
\curveto(176.31339133,249.45856114)(176.24623924,249.6474264)(176.24623924,249.86567069)
\curveto(176.24623924,250.08391498)(176.31339133,250.27697723)(176.4476955,250.44485746)
\curveto(176.58199968,250.61273768)(176.75827392,250.69667779)(176.97651821,250.69667779)
\curveto(177.18636849,250.69667779)(177.35424871,250.61273768)(177.48015888,250.44485746)
\curveto(177.61446306,250.27697723)(177.68161515,250.08391498)(177.68161515,249.86567069)
\closepath
\moveto(174.38276875,249.7271695)
\curveto(174.38276875,251.03663525)(174.60521005,252.03971958)(175.05009264,252.73642251)
\curveto(175.50336925,253.44151945)(176.13711709,253.79406792)(176.95133618,253.79406792)
\curveto(177.77394927,253.79406792)(178.40769711,253.44151945)(178.85257971,252.73642251)
\curveto(179.2974623,252.03971958)(179.5199036,251.03663525)(179.5199036,249.7271695)
\curveto(179.5199036,248.41770376)(179.2974623,247.41042242)(178.85257971,246.70532548)
\curveto(178.40769711,246.00862255)(177.77394927,245.66027108)(176.95133618,245.66027108)
\curveto(176.13711709,245.66027108)(175.50336925,246.00862255)(175.05009264,246.70532548)
\curveto(174.60521005,247.41042242)(174.38276875,248.41770376)(174.38276875,249.7271695)
\closepath
\moveto(178.46225819,249.7271695)
\curveto(178.46225819,250.15526407)(178.43707615,250.55817661)(178.38671209,250.93590711)
\curveto(178.33634802,251.32203163)(178.25240791,251.65779207)(178.13489175,251.94318845)
\curveto(178.01737559,252.22858483)(177.86208639,252.45522313)(177.66902413,252.62310336)
\curveto(177.47596187,252.79098358)(177.23673256,252.87492369)(176.95133618,252.87492369)
\curveto(176.6659398,252.87492369)(176.42671048,252.79098358)(176.23364822,252.62310336)
\curveto(176.04058596,252.45522313)(175.88529676,252.22858483)(175.7677806,251.94318845)
\curveto(175.65026444,251.65779207)(175.56632433,251.32203163)(175.51596026,250.93590711)
\curveto(175.4655962,250.55817661)(175.44041416,250.15526407)(175.44041416,249.7271695)
\curveto(175.44041416,249.29907493)(175.4655962,248.89196539)(175.51596026,248.50584087)
\curveto(175.56632433,248.12811037)(175.65026444,247.79654693)(175.7677806,247.51115055)
\curveto(175.88529676,247.22575417)(176.04058596,246.99911587)(176.23364822,246.83123564)
\curveto(176.42671048,246.66335542)(176.6659398,246.57941531)(176.95133618,246.57941531)
\curveto(177.23673256,246.57941531)(177.47596187,246.66335542)(177.66902413,246.83123564)
\curveto(177.86208639,246.99911587)(178.01737559,247.22575417)(178.13489175,247.51115055)
\curveto(178.25240791,247.79654693)(178.33634802,248.12811037)(178.38671209,248.50584087)
\curveto(178.43707615,248.89196539)(178.46225819,249.29907493)(178.46225819,249.7271695)
\closepath
}
}
{
\newrgbcolor{curcolor}{0 0 0}
\pscustom[linestyle=none,fillstyle=solid,fillcolor=curcolor]
{
\newpath
\moveto(196.56814316,249.86567069)
\curveto(196.56814316,249.6474264)(196.50099107,249.45856114)(196.36668689,249.29907493)
\curveto(196.24077673,249.13958872)(196.0728965,249.05984561)(195.86304622,249.05984561)
\curveto(195.64480193,249.05984561)(195.4685277,249.13958872)(195.33422352,249.29907493)
\curveto(195.19991934,249.45856114)(195.13276725,249.6474264)(195.13276725,249.86567069)
\curveto(195.13276725,250.08391498)(195.19991934,250.27697723)(195.33422352,250.44485746)
\curveto(195.4685277,250.61273768)(195.64480193,250.69667779)(195.86304622,250.69667779)
\curveto(196.0728965,250.69667779)(196.24077673,250.61273768)(196.36668689,250.44485746)
\curveto(196.50099107,250.27697723)(196.56814316,250.08391498)(196.56814316,249.86567069)
\closepath
\moveto(193.26929677,249.7271695)
\curveto(193.26929677,251.03663525)(193.49173807,252.03971958)(193.93662066,252.73642251)
\curveto(194.38989726,253.44151945)(195.02364511,253.79406792)(195.83786419,253.79406792)
\curveto(196.66047729,253.79406792)(197.29422513,253.44151945)(197.73910772,252.73642251)
\curveto(198.18399031,252.03971958)(198.40643161,251.03663525)(198.40643161,249.7271695)
\curveto(198.40643161,248.41770376)(198.18399031,247.41042242)(197.73910772,246.70532548)
\curveto(197.29422513,246.00862255)(196.66047729,245.66027108)(195.83786419,245.66027108)
\curveto(195.02364511,245.66027108)(194.38989726,246.00862255)(193.93662066,246.70532548)
\curveto(193.49173807,247.41042242)(193.26929677,248.41770376)(193.26929677,249.7271695)
\closepath
\moveto(197.3487862,249.7271695)
\curveto(197.3487862,250.15526407)(197.32360417,250.55817661)(197.2732401,250.93590711)
\curveto(197.22287603,251.32203163)(197.13893592,251.65779207)(197.02141977,251.94318845)
\curveto(196.90390361,252.22858483)(196.7486144,252.45522313)(196.55555215,252.62310336)
\curveto(196.36248989,252.79098358)(196.12326057,252.87492369)(195.83786419,252.87492369)
\curveto(195.55246781,252.87492369)(195.31323849,252.79098358)(195.12017623,252.62310336)
\curveto(194.92711398,252.45522313)(194.77182477,252.22858483)(194.65430861,251.94318845)
\curveto(194.53679246,251.65779207)(194.45285235,251.32203163)(194.40248828,250.93590711)
\curveto(194.35212421,250.55817661)(194.32694218,250.15526407)(194.32694218,249.7271695)
\curveto(194.32694218,249.29907493)(194.35212421,248.89196539)(194.40248828,248.50584087)
\curveto(194.45285235,248.12811037)(194.53679246,247.79654693)(194.65430861,247.51115055)
\curveto(194.77182477,247.22575417)(194.92711398,246.99911587)(195.12017623,246.83123564)
\curveto(195.31323849,246.66335542)(195.55246781,246.57941531)(195.83786419,246.57941531)
\curveto(196.12326057,246.57941531)(196.36248989,246.66335542)(196.55555215,246.83123564)
\curveto(196.7486144,246.99911587)(196.90390361,247.22575417)(197.02141977,247.51115055)
\curveto(197.13893592,247.79654693)(197.22287603,248.12811037)(197.2732401,248.50584087)
\curveto(197.32360417,248.89196539)(197.3487862,249.29907493)(197.3487862,249.7271695)
\closepath
}
}
{
\newrgbcolor{curcolor}{0 0 0}
\pscustom[linestyle=none,fillstyle=solid,fillcolor=curcolor]
{
\newpath
\moveto(209.36060285,246.62977938)
\curveto(209.36060285,246.37795904)(209.27666274,246.15551774)(209.10878251,245.96245549)
\curveto(208.94090229,245.76939323)(208.71846099,245.6728621)(208.44145862,245.6728621)
\curveto(208.15606224,245.6728621)(207.92942394,245.76939323)(207.76154372,245.96245549)
\curveto(207.5936635,246.15551774)(207.50972338,246.37795904)(207.50972338,246.62977938)
\curveto(207.50972338,246.88999372)(207.5936635,247.11663202)(207.76154372,247.30969428)
\curveto(207.92942394,247.50275654)(208.15606224,247.59928767)(208.44145862,247.59928767)
\curveto(208.71846099,247.59928767)(208.94090229,247.50275654)(209.10878251,247.30969428)
\curveto(209.27666274,247.11663202)(209.36060285,246.88999372)(209.36060285,246.62977938)
\closepath
}
}
{
\newrgbcolor{curcolor}{0 0 0}
\pscustom[linestyle=none,fillstyle=solid,fillcolor=curcolor]
{
\newpath
\moveto(215.65611982,246.62977938)
\curveto(215.65611982,246.37795904)(215.57217971,246.15551774)(215.40429948,245.96245549)
\curveto(215.23641926,245.76939323)(215.01397796,245.6728621)(214.73697559,245.6728621)
\curveto(214.45157921,245.6728621)(214.22494091,245.76939323)(214.05706069,245.96245549)
\curveto(213.88918046,246.15551774)(213.80524035,246.37795904)(213.80524035,246.62977938)
\curveto(213.80524035,246.88999372)(213.88918046,247.11663202)(214.05706069,247.30969428)
\curveto(214.22494091,247.50275654)(214.45157921,247.59928767)(214.73697559,247.59928767)
\curveto(215.01397796,247.59928767)(215.23641926,247.50275654)(215.40429948,247.30969428)
\curveto(215.57217971,247.11663202)(215.65611982,246.88999372)(215.65611982,246.62977938)
\closepath
}
}
{
\newrgbcolor{curcolor}{0 0 0}
\pscustom[linestyle=none,fillstyle=solid,fillcolor=curcolor]
{
\newpath
\moveto(221.95162153,246.62977938)
\curveto(221.95162153,246.37795904)(221.86768141,246.15551774)(221.69980119,245.96245549)
\curveto(221.53192097,245.76939323)(221.30947967,245.6728621)(221.0324773,245.6728621)
\curveto(220.74708092,245.6728621)(220.52044262,245.76939323)(220.3525624,245.96245549)
\curveto(220.18468217,246.15551774)(220.10074206,246.37795904)(220.10074206,246.62977938)
\curveto(220.10074206,246.88999372)(220.18468217,247.11663202)(220.3525624,247.30969428)
\curveto(220.52044262,247.50275654)(220.74708092,247.59928767)(221.0324773,247.59928767)
\curveto(221.30947967,247.59928767)(221.53192097,247.50275654)(221.69980119,247.30969428)
\curveto(221.86768141,247.11663202)(221.95162153,246.88999372)(221.95162153,246.62977938)
\closepath
}
}
{
\newrgbcolor{curcolor}{0 0 0}
\pscustom[linewidth=0.4940055,linecolor=curcolor]
{
\newpath
\moveto(52.50142601,256.12021514)
\lineto(225.83779076,256.12021514)
\lineto(225.83779076,243.33412333)
\lineto(52.50142601,243.33412333)
\closepath
}
}
{
\newrgbcolor{curcolor}{0 0 0}
\pscustom[linewidth=0.51800942,linecolor=curcolor]
{
\newpath
\moveto(71.388229,243.38171933)
\lineto(71.388229,255.98111433)
}
}
{
\newrgbcolor{curcolor}{0 0 0}
\pscustom[linewidth=0.5154072,linecolor=curcolor]
{
\newpath
\moveto(90.268452,243.47603033)
\lineto(90.268452,255.94915633)
}
}
{
\newrgbcolor{curcolor}{0 0 0}
\pscustom[linewidth=0.51930571,linecolor=curcolor]
{
\newpath
\moveto(109.107742,243.47602933)
\lineto(109.107742,256.13855833)
}
}
{
\newrgbcolor{curcolor}{0 0 0}
\pscustom[linewidth=0.52573889,linecolor=curcolor]
{
\newpath
\moveto(128.036312,243.29745833)
\lineto(128.036312,256.27566033)
}
}
{
\newrgbcolor{curcolor}{0 0 0}
\pscustom[linewidth=0.51670998,linecolor=curcolor]
{
\newpath
\moveto(146.875592,243.47602933)
\lineto(146.875592,256.01228933)
}
}
{
\newrgbcolor{curcolor}{0 0 0}
\pscustom[linewidth=0.52059871,linecolor=curcolor]
{
\newpath
\moveto(165.625602,243.47603033)
\lineto(165.625602,256.20169433)
}
}
{
\newrgbcolor{curcolor}{0 0 0}
\pscustom[linewidth=0.51670998,linecolor=curcolor]
{
\newpath
\moveto(184.554172,243.38674433)
\lineto(184.554172,255.92300433)
}
}
{
\newrgbcolor{curcolor}{0 0 0}
\pscustom[linewidth=0.51670998,linecolor=curcolor]
{
\newpath
\moveto(203.661312,243.47603033)
\lineto(203.661312,256.01229033)
}
}
{
\newrgbcolor{curcolor}{0 0 0}
\pscustom[linestyle=none,fillstyle=solid,fillcolor=curcolor]
{
\newpath
\moveto(62.27863702,231.83480506)
\curveto(62.70861898,232.00194839)(63.1259544,232.21087754)(63.5306433,232.46159253)
\curveto(63.9353322,232.72066468)(64.31051253,233.04659416)(64.6561843,233.43938097)
\lineto(65.38968292,233.43938097)
\lineto(65.38968292,226.55725463)
\lineto(66.8693267,226.55725463)
\lineto(66.8693267,225.67975218)
\lineto(62.65803286,225.67975218)
\lineto(62.65803286,226.55725463)
\lineto(64.35266762,226.55725463)
\lineto(64.35266762,231.9977698)
\curveto(64.25992642,231.91419814)(64.14610766,231.8264479)(64.01121136,231.73451907)
\curveto(63.88474608,231.65094741)(63.74141877,231.56737575)(63.58122941,231.48380408)
\curveto(63.42947107,231.40023242)(63.26928172,231.32083934)(63.10066135,231.24562485)
\curveto(62.93204097,231.17041035)(62.76763611,231.10773161)(62.60744675,231.05758861)
\lineto(62.27863702,231.83480506)
\closepath
}
}
{
\newrgbcolor{curcolor}{0 0 0}
\pscustom[linestyle=none,fillstyle=solid,fillcolor=curcolor]
{
\newpath
\moveto(85.61148126,231.45873259)
\curveto(85.61148126,231.19130327)(85.55667964,230.93223112)(85.4470764,230.68151613)
\curveto(85.34590418,230.43080115)(85.20679237,230.18426474)(85.02974097,229.94190693)
\curveto(84.8611206,229.69954911)(84.66720717,229.46136987)(84.44800068,229.22736922)
\curveto(84.2287942,228.99336857)(84.0053722,228.7635465)(83.7777347,228.53790301)
\curveto(83.65126942,228.41254552)(83.50372659,228.26211653)(83.33510622,228.08661604)
\curveto(83.16648584,227.91111555)(83.00629649,227.73143647)(82.85453815,227.54757882)
\curveto(82.70277981,227.36372116)(82.57631453,227.18404209)(82.47514231,227.0085416)
\curveto(82.37397009,226.83304111)(82.32338397,226.68261212)(82.32338397,226.55725463)
\lineto(85.92764447,226.55725463)
\lineto(85.92764447,225.67975218)
\lineto(81.18519645,225.67975218)
\curveto(81.17676543,225.72153801)(81.17254992,225.76332384)(81.17254992,225.80510967)
\lineto(81.17254992,225.94300291)
\curveto(81.17254992,226.29400389)(81.23156705,226.61993337)(81.34960131,226.92079135)
\curveto(81.46763558,227.22164934)(81.61939391,227.50579299)(81.80487632,227.7732223)
\curveto(81.99035873,228.04065162)(82.19691869,228.29136661)(82.4245562,228.52536726)
\curveto(82.66062472,228.76772508)(82.89247773,229.00172573)(83.12011524,229.22736922)
\curveto(83.30559765,229.41122687)(83.48264904,229.59090595)(83.65126942,229.76640644)
\curveto(83.82832081,229.94190693)(83.98007915,230.11740742)(84.10654443,230.2929079)
\curveto(84.24144073,230.46840839)(84.34682846,230.64808747)(84.42270763,230.83194512)
\curveto(84.50701782,231.02415994)(84.54917291,231.22055335)(84.54917291,231.42112534)
\curveto(84.54917291,231.64676882)(84.51123332,231.83898365)(84.43535416,231.9977698)
\curveto(84.36790601,232.15655596)(84.27094929,232.28609204)(84.14448401,232.38637803)
\curveto(84.02644975,232.49502119)(83.88733794,232.57441427)(83.72714859,232.62455727)
\curveto(83.56695923,232.67470026)(83.39412335,232.69977176)(83.20864094,232.69977176)
\curveto(82.98943445,232.69977176)(82.78709,232.67052168)(82.60160759,232.61202152)
\curveto(82.4245562,232.55352135)(82.26436684,232.48248544)(82.12103952,232.39891378)
\curveto(81.98614323,232.31534212)(81.86810896,232.23177046)(81.76693674,232.14819879)
\curveto(81.66576451,232.0729843)(81.58988535,232.01030555)(81.53929923,231.96016256)
\lineto(81.02079158,232.68723601)
\curveto(81.08823973,232.76245051)(81.18941196,232.85437934)(81.32430826,232.9630225)
\curveto(81.45920456,233.07166566)(81.61939391,233.17195165)(81.80487632,233.26388048)
\curveto(81.99878975,233.36416647)(82.21378073,233.44773813)(82.44984925,233.51459546)
\curveto(82.68591778,233.58145279)(82.93884834,233.61488146)(83.20864094,233.61488146)
\curveto(84.02644975,233.61488146)(84.62926759,233.42684522)(85.01709445,233.05077274)
\curveto(85.41335233,232.68305743)(85.61148126,232.15237738)(85.61148126,231.45873259)
\closepath
}
}
{
\newrgbcolor{curcolor}{0 0 0}
\pscustom[linestyle=none,fillstyle=solid,fillcolor=curcolor]
{
\newpath
\moveto(101.95079454,226.43189713)
\curveto(102.61684502,226.43189713)(103.08898206,226.56143321)(103.36720568,226.82050536)
\curveto(103.65386032,227.08793468)(103.79718763,227.44311424)(103.79718763,227.88604405)
\curveto(103.79718763,228.1701877)(103.7381705,228.40836693)(103.62013624,228.60058176)
\curveto(103.50210198,228.79279658)(103.34612813,228.94740415)(103.1522147,229.06440448)
\curveto(102.95830127,229.1814048)(102.73487928,229.26497647)(102.48194872,229.31511946)
\curveto(102.22901816,229.36526246)(101.96344107,229.39033396)(101.68521745,229.39033396)
\lineto(101.41964036,229.39033396)
\lineto(101.41964036,230.23022916)
\lineto(101.78638967,230.23022916)
\curveto(101.97187209,230.23022916)(102.16157001,230.24694349)(102.35548344,230.28037216)
\curveto(102.55782789,230.32215799)(102.73909479,230.38901532)(102.89928414,230.48094414)
\curveto(103.06790452,230.58123014)(103.20280082,230.7149448)(103.30397304,230.88208812)
\curveto(103.40514526,231.04923144)(103.45573138,231.26233918)(103.45573138,231.52141133)
\curveto(103.45573138,231.94762681)(103.32083508,232.24848479)(103.05104248,232.42398528)
\curveto(102.7896809,232.60784293)(102.48194872,232.69977176)(102.12784593,232.69977176)
\curveto(101.76531213,232.69977176)(101.45757995,232.64545018)(101.20464938,232.53680702)
\curveto(100.95171882,232.43652103)(100.74094336,232.33205645)(100.57232298,232.22341329)
\lineto(100.16763409,233.01316549)
\curveto(100.34468548,233.13852299)(100.61026257,233.26805906)(100.96436535,233.40177372)
\curveto(101.32689916,233.54384555)(101.72737254,233.61488146)(102.16578552,233.61488146)
\curveto(102.57890543,233.61488146)(102.93300822,233.56473846)(103.22809387,233.46445247)
\curveto(103.52317953,233.36416647)(103.76346356,233.22209465)(103.94894597,233.03823699)
\curveto(104.1428594,232.85437934)(104.28618672,232.63709302)(104.37892792,232.38637803)
\curveto(104.47166913,232.14402021)(104.51803973,231.87659089)(104.51803973,231.58409008)
\curveto(104.51803973,231.17458894)(104.40843649,230.82776654)(104.18923,230.54362289)
\curveto(103.97845454,230.25947924)(103.70444643,230.04219292)(103.36720568,229.89176393)
\curveto(103.77189458,229.7747636)(104.12178185,229.54494153)(104.41686751,229.20229772)
\curveto(104.71195316,228.86801107)(104.85949599,228.42090268)(104.85949599,227.86097255)
\curveto(104.85949599,227.5266859)(104.80047886,227.21329217)(104.6824446,226.92079135)
\curveto(104.57284135,226.6366477)(104.40000547,226.3901113)(104.16393695,226.18118215)
\curveto(103.93629944,225.97225299)(103.63699828,225.80928825)(103.26603346,225.69228793)
\curveto(102.90349965,225.5752876)(102.46930219,225.51678744)(101.96344107,225.51678744)
\curveto(101.76952764,225.51678744)(101.56718319,225.53350177)(101.35640772,225.56693043)
\curveto(101.15406327,225.59200193)(100.96436535,225.62960918)(100.78731396,225.67975218)
\curveto(100.61026257,225.72153801)(100.45007321,225.76332384)(100.30674589,225.80510967)
\curveto(100.17184959,225.85525267)(100.07489288,225.89285992)(100.01587575,225.91793141)
\lineto(100.2182202,226.80796961)
\curveto(100.3531165,226.74111228)(100.56810747,226.6617192)(100.86319313,226.56979038)
\curveto(101.15827878,226.47786155)(101.52081259,226.43189713)(101.95079454,226.43189713)
\closepath
}
}
{
\newrgbcolor{curcolor}{0 0 0}
\pscustom[linestyle=none,fillstyle=solid,fillcolor=curcolor]
{
\newpath
\moveto(118.69480054,228.36240252)
\curveto(118.83812786,228.69668917)(119.03204129,229.08111881)(119.27654083,229.51569145)
\curveto(119.52947139,229.95026409)(119.81191052,230.39737248)(120.12385821,230.85701662)
\curveto(120.4358059,231.32501793)(120.76883114,231.78048348)(121.12293393,232.22341329)
\curveto(121.47703671,232.67470026)(121.83535501,233.08002282)(122.19788881,233.43938097)
\lineto(123.20961106,233.43938097)
\lineto(123.20961106,228.51283151)
\lineto(124.1328076,228.51283151)
\lineto(124.1328076,227.66040056)
\lineto(123.20961106,227.66040056)
\lineto(123.20961106,225.67975218)
\lineto(122.19788881,225.67975218)
\lineto(122.19788881,227.66040056)
\lineto(118.69480054,227.66040056)
\lineto(118.69480054,228.36240252)
\closepath
\moveto(122.19788881,232.21087754)
\curveto(121.97025131,231.96851972)(121.73839829,231.7010904)(121.50232977,231.40858959)
\curveto(121.27469226,231.11608877)(121.05127027,230.80687362)(120.83206378,230.48094414)
\curveto(120.6128573,230.16337183)(120.40629734,229.83744235)(120.21238391,229.5031557)
\curveto(120.0269015,229.16886905)(119.85828112,228.83876099)(119.70652279,228.51283151)
\lineto(122.19788881,228.51283151)
\lineto(122.19788881,232.21087754)
\closepath
}
}
{
\newrgbcolor{curcolor}{0 0 0}
\pscustom[linestyle=none,fillstyle=solid,fillcolor=curcolor]
{
\newpath
\moveto(139.56156703,230.4433369)
\curveto(140.69132353,230.40155106)(141.51334786,230.15501466)(142.02764,229.70372769)
\curveto(142.54193214,229.25244072)(142.79907821,228.64654617)(142.79907821,227.88604405)
\curveto(142.79907821,227.54340023)(142.74427658,227.22582792)(142.63467334,226.9333271)
\curveto(142.5250701,226.64082629)(142.35223422,226.3901113)(142.11616569,226.18118215)
\curveto(141.88852819,225.97225299)(141.59344253,225.80928825)(141.23090873,225.69228793)
\curveto(140.87680594,225.5752876)(140.45525501,225.51678744)(139.96625592,225.51678744)
\curveto(139.76391148,225.51678744)(139.56156703,225.53350177)(139.35922258,225.56693043)
\curveto(139.16530915,225.59200193)(138.97982674,225.6254306)(138.80277534,225.66721643)
\curveto(138.62572395,225.70900226)(138.46975011,225.75078809)(138.33485381,225.79257392)
\curveto(138.19995751,225.84271692)(138.10300079,225.88450275)(138.04398366,225.91793141)
\lineto(138.24632811,226.80796961)
\curveto(138.38122441,226.74111228)(138.58778437,226.6617192)(138.86600798,226.56979038)
\curveto(139.15266262,226.47786155)(139.51098091,226.43189713)(139.94096287,226.43189713)
\curveto(140.27820362,226.43189713)(140.56064274,226.46950438)(140.78828025,226.54471888)
\curveto(141.01591775,226.61993337)(141.19718465,226.72021937)(141.33208095,226.84557686)
\curveto(141.47540827,226.97929152)(141.5765805,227.12972051)(141.63559763,227.29686383)
\curveto(141.70304578,227.46400716)(141.73676985,227.63950765)(141.73676985,227.8233653)
\curveto(141.73676985,228.10750895)(141.68618374,228.35822394)(141.58501151,228.57551026)
\curveto(141.49227031,228.80115374)(141.32364993,228.98918998)(141.07915039,229.13961897)
\curveto(140.84308187,229.29004796)(140.51427214,229.40286971)(140.0927212,229.4780842)
\curveto(139.67960129,229.56165586)(139.15687813,229.6034417)(138.52455173,229.6034417)
\curveto(138.57513784,229.97115701)(138.61307742,230.31380082)(138.63837048,230.63137313)
\curveto(138.67209455,230.95730262)(138.69738761,231.27069635)(138.71424965,231.57155433)
\curveto(138.7395427,231.88076948)(138.75640474,232.18580604)(138.76483576,232.48666402)
\curveto(138.7816978,232.78752201)(138.79855984,233.10509432)(138.81542187,233.43938097)
\lineto(142.58408723,233.43938097)
\lineto(142.58408723,232.56187852)
\lineto(139.72597189,232.56187852)
\curveto(139.71754087,232.44487819)(139.70489434,232.29027062)(139.68803231,232.0980558)
\curveto(139.67960129,231.91419814)(139.66695476,231.71780474)(139.65009272,231.50887558)
\curveto(139.63323069,231.29994643)(139.61636865,231.09937444)(139.59950661,230.90715962)
\curveto(139.58264457,230.7149448)(139.56999805,230.56033722)(139.56156703,230.4433369)
\closepath
}
}
{
\newrgbcolor{curcolor}{0 0 0}
\pscustom[linestyle=none,fillstyle=solid,fillcolor=curcolor]
{
\newpath
\moveto(156.86202027,228.78861799)
\curveto(156.86202027,229.54076295)(156.96319249,230.20515766)(157.16553694,230.78180213)
\curveto(157.37631241,231.36680376)(157.67139806,231.85569798)(158.0507939,232.24848479)
\curveto(158.43862076,232.6412716)(158.9065423,232.94212958)(159.45455851,233.15105874)
\curveto(160.00257473,233.35998789)(160.6222546,233.46863105)(161.31359814,233.47698822)
\lineto(161.40212383,232.59948577)
\curveto(160.95527984,232.5911286)(160.54637544,232.5409856)(160.17541061,232.44905678)
\curveto(159.81287681,232.36548512)(159.48828259,232.22341329)(159.20162795,232.0228413)
\curveto(158.91497332,231.83062648)(158.67468928,231.5799115)(158.48077585,231.27069635)
\curveto(158.28686242,230.9614812)(158.1393196,230.58123014)(158.03814737,230.12994316)
\curveto(158.24049182,230.22187199)(158.45969831,230.29708649)(158.69576683,230.35558665)
\curveto(158.94026637,230.41408681)(159.19741244,230.4433369)(159.46720504,230.4433369)
\curveto(159.90561801,230.4433369)(160.27658284,230.37647957)(160.58009951,230.24276491)
\curveto(160.8920472,230.10905025)(161.14076225,229.92937118)(161.32624466,229.70372769)
\curveto(161.51172708,229.48644137)(161.64662338,229.2315478)(161.73093356,228.93904698)
\curveto(161.81524375,228.64654617)(161.85739884,228.34568819)(161.85739884,228.03647304)
\curveto(161.85739884,227.75232939)(161.81102824,227.45982857)(161.71828703,227.15897059)
\curveto(161.62554583,226.86646977)(161.48221851,226.59486187)(161.28830508,226.34414689)
\curveto(161.09439165,226.10178907)(160.8456766,225.90121708)(160.54215993,225.74243092)
\curveto(160.23864325,225.59200193)(159.87610945,225.51678744)(159.45455851,225.51678744)
\curveto(158.58616359,225.51678744)(157.93697515,225.80510967)(157.5069932,226.38175414)
\curveto(157.07701124,226.9583986)(156.86202027,227.76068655)(156.86202027,228.78861799)
\closepath
\moveto(159.35338629,229.59090595)
\curveto(159.08359369,229.59090595)(158.83487864,229.56583445)(158.60724114,229.51569145)
\curveto(158.38803465,229.46554845)(158.16461265,229.39033396)(157.93697515,229.29004796)
\curveto(157.92854413,229.2064763)(157.92432862,229.12290464)(157.92432862,229.03933298)
\lineto(157.92432862,228.78861799)
\curveto(157.92432862,228.46268851)(157.94540617,228.15347336)(157.98756126,227.86097255)
\curveto(158.03814737,227.5768289)(158.11824205,227.32193533)(158.22784529,227.09629184)
\curveto(158.34587956,226.87900552)(158.5018534,226.70350503)(158.69576683,226.56979038)
\curveto(158.88968026,226.44443288)(159.13839531,226.38175414)(159.44191199,226.38175414)
\curveto(159.69484255,226.38175414)(159.90561801,226.43189713)(160.07423839,226.53218313)
\curveto(160.24285876,226.64082629)(160.38197057,226.77454095)(160.49157381,226.9333271)
\curveto(160.60117706,227.09211326)(160.67705623,227.26761375)(160.71921132,227.45982857)
\curveto(160.76979743,227.66040056)(160.79509049,227.8484368)(160.79509049,228.02393729)
\curveto(160.79509049,228.53372443)(160.67705623,228.92233265)(160.4409877,229.18976197)
\curveto(160.2133502,229.45719129)(159.85081639,229.59090595)(159.35338629,229.59090595)
\closepath
}
}
{
\newrgbcolor{curcolor}{0 0 0}
\pscustom[linestyle=none,fillstyle=solid,fillcolor=curcolor]
{
\newpath
\moveto(177.10911454,225.67975218)
\curveto(177.15126963,226.27311098)(177.25665736,226.89989844)(177.42527774,227.56011457)
\curveto(177.60232913,228.22868786)(177.8131046,228.87218966)(178.05760414,229.49061995)
\curveto(178.30210368,230.11740742)(178.57189628,230.69405188)(178.86698194,231.22055335)
\curveto(179.16206759,231.75541198)(179.45293773,232.19416321)(179.73959237,232.53680702)
\lineto(175.94563396,232.53680702)
\lineto(175.94563396,233.43938097)
\lineto(180.90307295,233.43938097)
\lineto(180.90307295,232.57441427)
\curveto(180.65014239,232.28191345)(180.37613428,231.88912664)(180.08104863,231.39605384)
\curveto(179.78596297,230.90298104)(179.50352385,230.34722948)(179.23373125,229.72879919)
\curveto(178.97236967,229.11872606)(178.74473216,228.46268851)(178.55081873,227.76068655)
\curveto(178.3569053,227.06704176)(178.23465553,226.37339697)(178.18406942,225.67975218)
\lineto(177.10911454,225.67975218)
\closepath
}
}
{
\newrgbcolor{curcolor}{0 0 0}
\pscustom[linestyle=none,fillstyle=solid,fillcolor=curcolor]
{
\newpath
\moveto(199.79698872,227.69800781)
\curveto(199.79698872,227.04614885)(199.58621325,226.51964738)(199.16466232,226.1185034)
\curveto(198.7515424,225.71735943)(198.119216,225.51678744)(197.26768311,225.51678744)
\curveto(196.77868403,225.51678744)(196.37399513,225.57946618)(196.05361642,225.70482368)
\curveto(195.73323771,225.83853834)(195.47609164,226.00568166)(195.28217821,226.20625365)
\curveto(195.0966958,226.4151828)(194.9617995,226.64500487)(194.87748931,226.89571986)
\curveto(194.80161014,227.14643484)(194.76367056,227.39297124)(194.76367056,227.63532906)
\curveto(194.76367056,228.07825887)(194.88592033,228.47104568)(195.13041987,228.81368949)
\curveto(195.37491942,229.15633331)(195.66578956,229.43211979)(196.00303031,229.64104894)
\curveto(195.28639372,230.04219292)(194.92807543,230.65644463)(194.92807543,231.48380408)
\curveto(194.92807543,231.76794773)(194.98287705,232.03955563)(195.09248029,232.29862779)
\curveto(195.20208353,232.55769994)(195.35805738,232.78334342)(195.56040183,232.97555825)
\curveto(195.76274628,233.16777307)(196.00724582,233.32238064)(196.29390045,233.43938097)
\curveto(196.58898611,233.55638129)(196.91779584,233.61488146)(197.28032964,233.61488146)
\curveto(197.70188058,233.61488146)(198.06019887,233.55220271)(198.35528453,233.42684522)
\curveto(198.6588012,233.30148773)(198.90330074,233.13852299)(199.08878315,232.937951)
\curveto(199.27426556,232.74573618)(199.40916186,232.52844986)(199.49347205,232.28609204)
\curveto(199.57778224,232.05209138)(199.61993733,231.82226931)(199.61993733,231.59662583)
\curveto(199.61993733,231.15369602)(199.50611858,230.76926638)(199.27848107,230.4433369)
\curveto(199.05084357,230.12576458)(198.78948199,229.87087101)(198.49439633,229.67865619)
\curveto(199.36279126,229.26915505)(199.79698872,228.60893892)(199.79698872,227.69800781)
\closepath
\moveto(195.7753928,227.62279331)
\curveto(195.7753928,227.48907865)(195.80068586,227.34700683)(195.85127197,227.19657784)
\curveto(195.90185808,227.05450601)(195.98616827,226.92079135)(196.10420253,226.79543386)
\curveto(196.2222368,226.67007637)(196.37821064,226.56561179)(196.57212407,226.48204013)
\curveto(196.7660375,226.40682564)(197.00210602,226.36921839)(197.28032964,226.36921839)
\curveto(197.54169122,226.36921839)(197.76511322,226.40264705)(197.95059563,226.46950438)
\curveto(198.14450906,226.54471888)(198.3004829,226.64082629)(198.41851717,226.75782661)
\curveto(198.54498245,226.88318411)(198.63772365,227.02107735)(198.69674078,227.17150634)
\curveto(198.75575791,227.32193533)(198.78526648,227.47236432)(198.78526648,227.62279331)
\curveto(198.78526648,228.09915178)(198.6124306,228.46268851)(198.26675883,228.7134035)
\curveto(197.92108706,228.97247565)(197.44473451,229.16886905)(196.83770116,229.30258371)
\curveto(196.50046041,229.11872606)(196.23909883,228.88890399)(196.05361642,228.6131175)
\curveto(195.86813401,228.33733102)(195.7753928,228.00722296)(195.7753928,227.62279331)
\closepath
\moveto(198.60821509,231.59662583)
\curveto(198.60821509,231.70526899)(198.58292203,231.8264479)(198.53233592,231.96016256)
\curveto(198.48174981,232.10223438)(198.40165513,232.22759187)(198.29205188,232.33623503)
\curveto(198.19087966,232.45323536)(198.05598336,232.54934277)(197.88736299,232.62455727)
\curveto(197.71874261,232.70812893)(197.51639816,232.74991476)(197.28032964,232.74991476)
\curveto(197.0358301,232.74991476)(196.82927014,232.71230751)(196.66064977,232.63709302)
\curveto(196.50046041,232.56187852)(196.36556411,232.46577111)(196.25596087,232.34877078)
\curveto(196.14635763,232.24012762)(196.06626295,232.11477013)(196.01567684,231.97269831)
\curveto(195.97352174,231.83898365)(195.9524442,231.70526899)(195.9524442,231.57155433)
\curveto(195.9524442,231.22891052)(196.07469397,230.90715962)(196.31919351,230.60630164)
\curveto(196.57212407,230.31380082)(196.98524399,230.10069308)(197.55855326,229.96697842)
\curveto(197.87893197,230.15083608)(198.13186253,230.3681224)(198.31734494,230.61883739)
\curveto(198.51125837,230.86955237)(198.60821509,231.19548185)(198.60821509,231.59662583)
\closepath
}
}
{
\newrgbcolor{curcolor}{0 0 0}
\pscustom[linestyle=none,fillstyle=solid,fillcolor=curcolor]
{
\newpath
\moveto(218.74148294,230.33051515)
\curveto(218.74148294,228.78443941)(218.3620871,227.61861473)(217.60329542,226.83304111)
\curveto(216.84450374,226.05582466)(215.7021007,225.66303785)(214.17608632,225.65468068)
\lineto(214.13814673,226.53218313)
\curveto(215.08242083,226.53218313)(215.84121251,226.71604078)(216.41452178,227.08375609)
\curveto(216.99626207,227.45982857)(217.37987342,228.10333037)(217.56535583,229.01426148)
\curveto(217.36301139,228.92233265)(217.13958939,228.84711816)(216.89508985,228.78861799)
\curveto(216.65059031,228.738475)(216.39344424,228.7134035)(216.12365164,228.7134035)
\curveto(215.67680765,228.7134035)(215.30162731,228.77608224)(214.99811064,228.90143974)
\curveto(214.69459397,229.0351544)(214.45009443,229.21065489)(214.26461202,229.42794121)
\curveto(214.0791296,229.65358469)(213.9442333,229.90847826)(213.85992312,230.19262191)
\curveto(213.77561293,230.47676556)(213.73345784,230.77762354)(213.73345784,231.09519586)
\curveto(213.73345784,231.37933951)(213.77982844,231.66766174)(213.87256965,231.96016256)
\curveto(213.96531085,232.26102054)(214.10863817,232.53262844)(214.3025516,232.77498626)
\curveto(214.49646503,233.01734408)(214.74518008,233.21791606)(215.04869675,233.37670222)
\curveto(215.35221343,233.53548838)(215.71474723,233.61488146)(216.13629817,233.61488146)
\curveto(216.99626207,233.61488146)(217.64545051,233.32238064)(218.08386348,232.73737901)
\curveto(218.52227646,232.15237738)(218.74148294,231.35008943)(218.74148294,230.33051515)
\closepath
\moveto(216.23747039,229.56583445)
\curveto(216.50726299,229.56583445)(216.75597804,229.59090595)(216.98361554,229.64104894)
\curveto(217.21968407,229.69119194)(217.44732157,229.76222785)(217.66652806,229.85415668)
\curveto(217.67495908,229.93772834)(217.67917459,230.01712142)(217.67917459,230.09233592)
\lineto(217.67917459,230.33051515)
\curveto(217.67917459,230.65644463)(217.65388153,230.96565978)(217.60329542,231.2581606)
\curveto(217.56114033,231.55066141)(217.48104565,231.80555498)(217.36301139,232.0228413)
\curveto(217.25340814,232.24012762)(217.0974343,232.41144953)(216.89508985,232.53680702)
\curveto(216.70117642,232.67052168)(216.45246137,232.73737901)(216.14894469,232.73737901)
\curveto(215.89601413,232.73737901)(215.68523867,232.68305743)(215.51661829,232.57441427)
\curveto(215.34799792,232.47412828)(215.20888611,232.3445922)(215.09928287,232.18580604)
\curveto(214.98967962,232.03537705)(214.90958495,231.86405514)(214.85899883,231.67184032)
\curveto(214.81684374,231.4796255)(214.79576619,231.29576785)(214.79576619,231.12026736)
\curveto(214.79576619,230.61048022)(214.90958495,230.22187199)(215.13722245,229.95444267)
\curveto(215.37329097,229.69537052)(215.74004029,229.56583445)(216.23747039,229.56583445)
\closepath
}
}
{
\newrgbcolor{curcolor}{0 0 0}
\pscustom[linestyle=none,fillstyle=solid,fillcolor=curcolor]
{
\newpath
\moveto(236.15574727,226.48204013)
\curveto(236.15574727,226.23132515)(236.07143708,226.00986024)(235.90281671,225.81764542)
\curveto(235.73419634,225.6254306)(235.51077434,225.52932319)(235.23255072,225.52932319)
\curveto(234.94589609,225.52932319)(234.71825858,225.6254306)(234.54963821,225.81764542)
\curveto(234.38101783,226.00986024)(234.29670765,226.23132515)(234.29670765,226.48204013)
\curveto(234.29670765,226.74111228)(234.38101783,226.96675577)(234.54963821,227.15897059)
\curveto(234.71825858,227.35118541)(234.94589609,227.44729282)(235.23255072,227.44729282)
\curveto(235.51077434,227.44729282)(235.73419634,227.35118541)(235.90281671,227.15897059)
\curveto(236.07143708,226.96675577)(236.15574727,226.74111228)(236.15574727,226.48204013)
\closepath
}
}
{
\newrgbcolor{curcolor}{0 0 0}
\pscustom[linestyle=none,fillstyle=solid,fillcolor=curcolor]
{
\newpath
\moveto(242.47901991,226.48204013)
\curveto(242.47901991,226.23132515)(242.39470973,226.00986024)(242.22608935,225.81764542)
\curveto(242.05746898,225.6254306)(241.83404698,225.52932319)(241.55582337,225.52932319)
\curveto(241.26916873,225.52932319)(241.04153123,225.6254306)(240.87291085,225.81764542)
\curveto(240.70429048,226.00986024)(240.61998029,226.23132515)(240.61998029,226.48204013)
\curveto(240.61998029,226.74111228)(240.70429048,226.96675577)(240.87291085,227.15897059)
\curveto(241.04153123,227.35118541)(241.26916873,227.44729282)(241.55582337,227.44729282)
\curveto(241.83404698,227.44729282)(242.05746898,227.35118541)(242.22608935,227.15897059)
\curveto(242.39470973,226.96675577)(242.47901991,226.74111228)(242.47901991,226.48204013)
\closepath
}
}
{
\newrgbcolor{curcolor}{0 0 0}
\pscustom[linestyle=none,fillstyle=solid,fillcolor=curcolor]
{
\newpath
\moveto(248.80227723,226.48204013)
\curveto(248.80227723,226.23132515)(248.71796704,226.00986024)(248.54934667,225.81764542)
\curveto(248.3807263,225.6254306)(248.1573043,225.52932319)(247.87908068,225.52932319)
\curveto(247.59242605,225.52932319)(247.36478854,225.6254306)(247.19616817,225.81764542)
\curveto(247.02754779,226.00986024)(246.94323761,226.23132515)(246.94323761,226.48204013)
\curveto(246.94323761,226.74111228)(247.02754779,226.96675577)(247.19616817,227.15897059)
\curveto(247.36478854,227.35118541)(247.59242605,227.44729282)(247.87908068,227.44729282)
\curveto(248.1573043,227.44729282)(248.3807263,227.35118541)(248.54934667,227.15897059)
\curveto(248.71796704,226.96675577)(248.80227723,226.74111228)(248.80227723,226.48204013)
\closepath
}
}
{
\newrgbcolor{curcolor}{0 0 0}
\pscustom[linestyle=none,fillstyle=solid,fillcolor=curcolor]
{
\newpath
\moveto(266.02684364,230.4433369)
\curveto(267.15660014,230.40155106)(267.97862447,230.15501466)(268.49291661,229.70372769)
\curveto(269.00720875,229.25244072)(269.26435482,228.64654617)(269.26435482,227.88604405)
\curveto(269.26435482,227.54340023)(269.20955319,227.22582792)(269.09994995,226.9333271)
\curveto(268.99034671,226.64082629)(268.81751083,226.3901113)(268.5814423,226.18118215)
\curveto(268.3538048,225.97225299)(268.05871914,225.80928825)(267.69618534,225.69228793)
\curveto(267.34208255,225.5752876)(266.92053162,225.51678744)(266.43153253,225.51678744)
\curveto(266.22918809,225.51678744)(266.02684364,225.53350177)(265.82449919,225.56693043)
\curveto(265.63058576,225.59200193)(265.44510335,225.6254306)(265.26805195,225.66721643)
\curveto(265.09100056,225.70900226)(264.93502672,225.75078809)(264.80013042,225.79257392)
\curveto(264.66523412,225.84271692)(264.5682774,225.88450275)(264.50926027,225.91793141)
\lineto(264.71160472,226.80796961)
\curveto(264.84650102,226.74111228)(265.05306098,226.6617192)(265.33128459,226.56979038)
\curveto(265.61793923,226.47786155)(265.97625752,226.43189713)(266.40623948,226.43189713)
\curveto(266.74348023,226.43189713)(267.02591935,226.46950438)(267.25355686,226.54471888)
\curveto(267.48119436,226.61993337)(267.66246126,226.72021937)(267.79735756,226.84557686)
\curveto(267.94068488,226.97929152)(268.04185711,227.12972051)(268.10087424,227.29686383)
\curveto(268.16832239,227.46400716)(268.20204646,227.63950765)(268.20204646,227.8233653)
\curveto(268.20204646,228.10750895)(268.15146035,228.35822394)(268.05028812,228.57551026)
\curveto(267.95754692,228.80115374)(267.78892654,228.98918998)(267.544427,229.13961897)
\curveto(267.30835848,229.29004796)(266.97954875,229.40286971)(266.55799781,229.4780842)
\curveto(266.1448779,229.56165586)(265.62215474,229.6034417)(264.98982834,229.6034417)
\curveto(265.04041445,229.97115701)(265.07835403,230.31380082)(265.10364709,230.63137313)
\curveto(265.13737116,230.95730262)(265.16266422,231.27069635)(265.17952626,231.57155433)
\curveto(265.20481931,231.88076948)(265.22168135,232.18580604)(265.23011237,232.48666402)
\curveto(265.24697441,232.78752201)(265.26383644,233.10509432)(265.28069848,233.43938097)
\lineto(269.04936384,233.43938097)
\lineto(269.04936384,232.56187852)
\lineto(266.1912485,232.56187852)
\curveto(266.18281748,232.44487819)(266.17017095,232.29027062)(266.15330892,232.0980558)
\curveto(266.1448779,231.91419814)(266.13223137,231.71780474)(266.11536933,231.50887558)
\curveto(266.0985073,231.29994643)(266.08164526,231.09937444)(266.06478322,230.90715962)
\curveto(266.04792118,230.7149448)(266.03527466,230.56033722)(266.02684364,230.4433369)
\closepath
}
}
{
\newrgbcolor{curcolor}{0 0 0}
\pscustom[linestyle=none,fillstyle=solid,fillcolor=curcolor]
{
\newpath
\moveto(275.30940384,231.45873259)
\curveto(275.30940384,231.19130327)(275.25460222,230.93223112)(275.14499898,230.68151613)
\curveto(275.04382675,230.43080115)(274.90471494,230.18426474)(274.72766355,229.94190693)
\curveto(274.55904318,229.69954911)(274.36512975,229.46136987)(274.14592326,229.22736922)
\curveto(273.92671678,228.99336857)(273.70329478,228.7635465)(273.47565728,228.53790301)
\curveto(273.349192,228.41254552)(273.20164917,228.26211653)(273.03302879,228.08661604)
\curveto(272.86440842,227.91111555)(272.70421907,227.73143647)(272.55246073,227.54757882)
\curveto(272.40070239,227.36372116)(272.27423711,227.18404209)(272.17306489,227.0085416)
\curveto(272.07189266,226.83304111)(272.02130655,226.68261212)(272.02130655,226.55725463)
\lineto(275.62556704,226.55725463)
\lineto(275.62556704,225.67975218)
\lineto(270.88311903,225.67975218)
\curveto(270.87468801,225.72153801)(270.8704725,225.76332384)(270.8704725,225.80510967)
\lineto(270.8704725,225.94300291)
\curveto(270.8704725,226.29400389)(270.92948963,226.61993337)(271.04752389,226.92079135)
\curveto(271.16555815,227.22164934)(271.31731649,227.50579299)(271.5027989,227.7732223)
\curveto(271.68828131,228.04065162)(271.89484127,228.29136661)(272.12247878,228.52536726)
\curveto(272.3585473,228.76772508)(272.59040031,229.00172573)(272.81803782,229.22736922)
\curveto(273.00352023,229.41122687)(273.18057162,229.59090595)(273.349192,229.76640644)
\curveto(273.52624339,229.94190693)(273.67800172,230.11740742)(273.80446701,230.2929079)
\curveto(273.9393633,230.46840839)(274.04475104,230.64808747)(274.12063021,230.83194512)
\curveto(274.20494039,231.02415994)(274.24709549,231.22055335)(274.24709549,231.42112534)
\curveto(274.24709549,231.64676882)(274.2091559,231.83898365)(274.13327673,231.9977698)
\curveto(274.06582858,232.15655596)(273.96887187,232.28609204)(273.84240659,232.38637803)
\curveto(273.72437233,232.49502119)(273.58526052,232.57441427)(273.42507116,232.62455727)
\curveto(273.26488181,232.67470026)(273.09204593,232.69977176)(272.90656351,232.69977176)
\curveto(272.68735703,232.69977176)(272.48501258,232.67052168)(272.29953017,232.61202152)
\curveto(272.12247878,232.55352135)(271.96228942,232.48248544)(271.8189621,232.39891378)
\curveto(271.6840658,232.31534212)(271.56603154,232.23177046)(271.46485932,232.14819879)
\curveto(271.36368709,232.0729843)(271.28780792,232.01030555)(271.23722181,231.96016256)
\lineto(270.71871416,232.68723601)
\curveto(270.78616231,232.76245051)(270.88733454,232.85437934)(271.02223084,232.9630225)
\curveto(271.15712713,233.07166566)(271.31731649,233.17195165)(271.5027989,233.26388048)
\curveto(271.69671233,233.36416647)(271.91170331,233.44773813)(272.14777183,233.51459546)
\curveto(272.38384035,233.58145279)(272.63677092,233.61488146)(272.90656351,233.61488146)
\curveto(273.72437233,233.61488146)(274.32719016,233.42684522)(274.71501702,233.05077274)
\curveto(275.1112749,232.68305743)(275.30940384,232.15237738)(275.30940384,231.45873259)
\closepath
}
}
{
\newrgbcolor{curcolor}{0 0 0}
\pscustom[linewidth=0.56097835,linecolor=curcolor]
{
\newpath
\moveto(52.53491142,235.92539772)
\lineto(277.23287895,235.92539772)
\lineto(277.23287895,223.20627865)
\lineto(52.53491142,223.20627865)
\closepath
}
}
{
\newrgbcolor{curcolor}{0 0 0}
\pscustom[linewidth=0.51800942,linecolor=curcolor]
{
\newpath
\moveto(71.388229,223.22038733)
\lineto(71.388229,235.81978233)
}
}
{
\newrgbcolor{curcolor}{0 0 0}
\pscustom[linewidth=0.5154072,linecolor=curcolor]
{
\newpath
\moveto(90.268452,223.31469833)
\lineto(90.268452,235.78782433)
}
}
{
\newrgbcolor{curcolor}{0 0 0}
\pscustom[linewidth=0.51930571,linecolor=curcolor]
{
\newpath
\moveto(109.107742,223.31469733)
\lineto(109.107742,235.97722633)
}
}
{
\newrgbcolor{curcolor}{0 0 0}
\pscustom[linewidth=0.52573889,linecolor=curcolor]
{
\newpath
\moveto(128.036312,223.13612633)
\lineto(128.036312,236.11432833)
}
}
{
\newrgbcolor{curcolor}{0 0 0}
\pscustom[linewidth=0.51670998,linecolor=curcolor]
{
\newpath
\moveto(146.875592,223.31469733)
\lineto(146.875592,235.85095733)
}
}
{
\newrgbcolor{curcolor}{0 0 0}
\pscustom[linewidth=0.52059871,linecolor=curcolor]
{
\newpath
\moveto(165.625602,223.31469833)
\lineto(165.625602,236.04036233)
}
}
{
\newrgbcolor{curcolor}{0 0 0}
\pscustom[linewidth=0.51670998,linecolor=curcolor]
{
\newpath
\moveto(184.554172,223.22541233)
\lineto(184.554172,235.76167233)
}
}
{
\newrgbcolor{curcolor}{0 0 0}
\pscustom[linewidth=0.51670998,linecolor=curcolor]
{
\newpath
\moveto(203.661312,223.31469833)
\lineto(203.661312,235.85095833)
}
}
{
\newrgbcolor{curcolor}{0 0 0}
\pscustom[linewidth=0.52059871,linecolor=curcolor]
{
\newpath
\moveto(222.589882,223.31469733)
\lineto(222.589882,236.04036133)
}
}
{
\newrgbcolor{curcolor}{0 0 0}
\pscustom[linewidth=0.52573889,linecolor=curcolor]
{
\newpath
\moveto(258.240192,235.96647233)
\lineto(258.204692,222.98831833)
}
}
{
\newrgbcolor{curcolor}{0 0 0}
\pscustom[linestyle=none,fillstyle=solid,fillcolor=curcolor]
{
\newpath
\moveto(62.27863502,30.96831506)
\curveto(62.70861698,31.13545839)(63.1259524,31.34438754)(63.5306413,31.59510253)
\curveto(63.9353302,31.85417468)(64.31051053,32.18010416)(64.6561823,32.57289097)
\lineto(65.38968092,32.57289097)
\lineto(65.38968092,25.69076463)
\lineto(66.8693247,25.69076463)
\lineto(66.8693247,24.81326218)
\lineto(62.65803086,24.81326218)
\lineto(62.65803086,25.69076463)
\lineto(64.35266562,25.69076463)
\lineto(64.35266562,31.1312798)
\curveto(64.25992442,31.04770814)(64.14610566,30.9599579)(64.01120936,30.86802907)
\curveto(63.88474408,30.78445741)(63.74141677,30.70088575)(63.58122741,30.61731408)
\curveto(63.42946907,30.53374242)(63.26927972,30.45434934)(63.10065935,30.37913485)
\curveto(62.93203897,30.30392035)(62.76763411,30.24124161)(62.60744475,30.19109861)
\lineto(62.27863502,30.96831506)
\closepath
}
}
{
\newrgbcolor{curcolor}{0 0 0}
\pscustom[linestyle=none,fillstyle=solid,fillcolor=curcolor]
{
\newpath
\moveto(76.32892672,29.5768469)
\curveto(77.45868323,29.53506106)(78.28070755,29.28852466)(78.79499969,28.83723769)
\curveto(79.30929183,28.38595072)(79.5664379,27.78005617)(79.5664379,27.01955405)
\curveto(79.5664379,26.67691023)(79.51163628,26.35933792)(79.40203304,26.0668371)
\curveto(79.29242979,25.77433629)(79.11959391,25.5236213)(78.88352539,25.31469215)
\curveto(78.65588788,25.10576299)(78.36080223,24.94279825)(77.99826842,24.82579793)
\curveto(77.64416564,24.7087976)(77.2226147,24.65029744)(76.73361562,24.65029744)
\curveto(76.53127117,24.65029744)(76.32892672,24.66701177)(76.12658227,24.70044043)
\curveto(75.93266884,24.72551193)(75.74718643,24.7589406)(75.57013504,24.80072643)
\curveto(75.39308365,24.84251226)(75.2371098,24.88429809)(75.1022135,24.92608392)
\curveto(74.9673172,24.97622692)(74.87036049,25.01801275)(74.81134336,25.05144141)
\lineto(75.01368781,25.94147961)
\curveto(75.1485841,25.87462228)(75.35514406,25.7952292)(75.63336768,25.70330038)
\curveto(75.92002232,25.61137155)(76.27834061,25.56540713)(76.70832256,25.56540713)
\curveto(77.04556331,25.56540713)(77.32800244,25.60301438)(77.55563994,25.67822888)
\curveto(77.78327745,25.75344337)(77.96454435,25.85372937)(78.09944065,25.97908686)
\curveto(78.24276797,26.11280152)(78.34394019,26.26323051)(78.40295732,26.43037383)
\curveto(78.47040547,26.59751716)(78.50412955,26.77301765)(78.50412955,26.9568753)
\curveto(78.50412955,27.24101895)(78.45354343,27.49173394)(78.35237121,27.70902026)
\curveto(78.25963,27.93466374)(78.09100963,28.12269998)(77.84651009,28.27312897)
\curveto(77.61044156,28.42355796)(77.28163183,28.53637971)(76.8600809,28.6115942)
\curveto(76.44696098,28.69516586)(75.92423782,28.7369517)(75.29191142,28.7369517)
\curveto(75.34249753,29.10466701)(75.38043712,29.44731082)(75.40573017,29.76488313)
\curveto(75.43945425,30.09081262)(75.46474731,30.40420635)(75.48160934,30.70506433)
\curveto(75.5069024,31.01427948)(75.52376444,31.31931604)(75.53219546,31.62017402)
\curveto(75.54905749,31.92103201)(75.56591953,32.23860432)(75.58278157,32.57289097)
\lineto(79.35144692,32.57289097)
\lineto(79.35144692,31.69538852)
\lineto(76.49333159,31.69538852)
\curveto(76.48490057,31.57838819)(76.47225404,31.42378062)(76.455392,31.2315658)
\curveto(76.44696098,31.04770814)(76.43431446,30.85131474)(76.41745242,30.64238558)
\curveto(76.40059038,30.43345643)(76.38372834,30.23288444)(76.36686631,30.04066962)
\curveto(76.35000427,29.8484548)(76.33735774,29.69384722)(76.32892672,29.5768469)
\closepath
}
}
{
\newrgbcolor{curcolor}{0 0 0}
\pscustom[linestyle=none,fillstyle=solid,fillcolor=curcolor]
{
\newpath
\moveto(81.24842709,30.96831506)
\curveto(81.67840904,31.13545839)(82.09574447,31.34438754)(82.50043337,31.59510253)
\curveto(82.90512226,31.85417468)(83.28030259,32.18010416)(83.62597436,32.57289097)
\lineto(84.35947299,32.57289097)
\lineto(84.35947299,25.69076463)
\lineto(85.83911677,25.69076463)
\lineto(85.83911677,24.81326218)
\lineto(81.62782293,24.81326218)
\lineto(81.62782293,25.69076463)
\lineto(83.32245769,25.69076463)
\lineto(83.32245769,31.1312798)
\curveto(83.22971648,31.04770814)(83.11589773,30.9599579)(82.98100143,30.86802907)
\curveto(82.85453615,30.78445741)(82.71120883,30.70088575)(82.55101948,30.61731408)
\curveto(82.39926114,30.53374242)(82.23907179,30.45434934)(82.07045141,30.37913485)
\curveto(81.90183104,30.30392035)(81.73742617,30.24124161)(81.57723682,30.19109861)
\lineto(81.24842709,30.96831506)
\closepath
}
}
{
\newrgbcolor{curcolor}{0 0 0}
\pscustom[linestyle=none,fillstyle=solid,fillcolor=curcolor]
{
\newpath
\moveto(93.89495322,30.96831506)
\curveto(94.32493517,31.13545839)(94.7422706,31.34438754)(95.14695949,31.59510253)
\curveto(95.55164839,31.85417468)(95.92682872,32.18010416)(96.27250049,32.57289097)
\lineto(97.00599912,32.57289097)
\lineto(97.00599912,25.69076463)
\lineto(98.4856429,25.69076463)
\lineto(98.4856429,24.81326218)
\lineto(94.27434906,24.81326218)
\lineto(94.27434906,25.69076463)
\lineto(95.96898382,25.69076463)
\lineto(95.96898382,31.1312798)
\curveto(95.87624261,31.04770814)(95.76242386,30.9599579)(95.62752756,30.86802907)
\curveto(95.50106228,30.78445741)(95.35773496,30.70088575)(95.19754561,30.61731408)
\curveto(95.04578727,30.53374242)(94.88559791,30.45434934)(94.71697754,30.37913485)
\curveto(94.54835717,30.30392035)(94.3839523,30.24124161)(94.22376295,30.19109861)
\lineto(93.89495322,30.96831506)
\closepath
}
}
{
\newrgbcolor{curcolor}{0 0 0}
\pscustom[linestyle=none,fillstyle=solid,fillcolor=curcolor]
{
\newpath
\moveto(99.7250036,27.49591252)
\curveto(99.86833092,27.83019917)(100.06224435,28.21462881)(100.30674389,28.64920145)
\curveto(100.55967445,29.08377409)(100.84211358,29.53088248)(101.15406127,29.99052662)
\curveto(101.46600896,30.45852793)(101.7990342,30.91399348)(102.15313699,31.35692329)
\curveto(102.50723977,31.80821026)(102.86555807,32.21353282)(103.22809187,32.57289097)
\lineto(104.23981412,32.57289097)
\lineto(104.23981412,27.64634151)
\lineto(105.16301066,27.64634151)
\lineto(105.16301066,26.79391056)
\lineto(104.23981412,26.79391056)
\lineto(104.23981412,24.81326218)
\lineto(103.22809187,24.81326218)
\lineto(103.22809187,26.79391056)
\lineto(99.7250036,26.79391056)
\lineto(99.7250036,27.49591252)
\closepath
\moveto(103.22809187,31.34438754)
\curveto(103.00045437,31.10202972)(102.76860135,30.8346004)(102.53253283,30.54209959)
\curveto(102.30489532,30.24959877)(102.08147333,29.94038362)(101.86226684,29.61445414)
\curveto(101.64306036,29.29688183)(101.4365004,28.97095235)(101.24258697,28.6366657)
\curveto(101.05710456,28.30237905)(100.88848418,27.97227099)(100.73672585,27.64634151)
\lineto(103.22809187,27.64634151)
\lineto(103.22809187,31.34438754)
\closepath
}
}
{
\newrgbcolor{curcolor}{0 0 0}
\pscustom[linestyle=none,fillstyle=solid,fillcolor=curcolor]
{
\newpath
\moveto(117.22780033,30.59224259)
\curveto(117.22780033,30.32481327)(117.17299871,30.06574112)(117.06339547,29.81502613)
\curveto(116.96222324,29.56431115)(116.82311144,29.31777474)(116.64606004,29.07541693)
\curveto(116.47743967,28.83305911)(116.28352624,28.59487987)(116.06431975,28.36087922)
\curveto(115.84511327,28.12687857)(115.62169127,27.8970565)(115.39405377,27.67141301)
\curveto(115.26758849,27.54605552)(115.12004566,27.39562653)(114.95142528,27.22012604)
\curveto(114.78280491,27.04462555)(114.62261556,26.86494647)(114.47085722,26.68108882)
\curveto(114.31909888,26.49723116)(114.1926336,26.31755209)(114.09146138,26.1420516)
\curveto(113.99028915,25.96655111)(113.93970304,25.81612212)(113.93970304,25.69076463)
\lineto(117.54396353,25.69076463)
\lineto(117.54396353,24.81326218)
\lineto(112.80151552,24.81326218)
\curveto(112.7930845,24.85504801)(112.78886899,24.89683384)(112.78886899,24.93861967)
\lineto(112.78886899,25.07651291)
\curveto(112.78886899,25.42751389)(112.84788612,25.75344337)(112.96592038,26.05430135)
\curveto(113.08395464,26.35515934)(113.23571298,26.63930299)(113.42119539,26.9067323)
\curveto(113.6066778,27.17416162)(113.81323776,27.42487661)(114.04087527,27.65887726)
\curveto(114.27694379,27.90123508)(114.5087968,28.13523573)(114.73643431,28.36087922)
\curveto(114.92191672,28.54473687)(115.09896811,28.72441595)(115.26758849,28.89991644)
\curveto(115.44463988,29.07541693)(115.59639821,29.25091742)(115.7228635,29.4264179)
\curveto(115.85775979,29.60191839)(115.96314753,29.78159747)(116.0390267,29.96545512)
\curveto(116.12333688,30.15766994)(116.16549198,30.35406335)(116.16549198,30.55463534)
\curveto(116.16549198,30.78027882)(116.12755239,30.97249365)(116.05167322,31.1312798)
\curveto(115.98422507,31.29006596)(115.88726836,31.41960204)(115.76080308,31.51988803)
\curveto(115.64276882,31.62853119)(115.50365701,31.70792427)(115.34346765,31.75806727)
\curveto(115.1832783,31.80821026)(115.01044242,31.83328176)(114.82496,31.83328176)
\curveto(114.60575352,31.83328176)(114.40340907,31.80403168)(114.21792666,31.74553152)
\curveto(114.04087527,31.68703135)(113.88068591,31.61599544)(113.73735859,31.53242378)
\curveto(113.60246229,31.44885212)(113.48442803,31.36528046)(113.38325581,31.28170879)
\curveto(113.28208358,31.2064943)(113.20620441,31.14381555)(113.1556183,31.09367256)
\lineto(112.63711065,31.82074601)
\curveto(112.7045588,31.89596051)(112.80573103,31.98788934)(112.94062733,32.0965325)
\curveto(113.07552362,32.20517566)(113.23571298,32.30546165)(113.42119539,32.39739048)
\curveto(113.61510882,32.49767647)(113.8300998,32.58124813)(114.06616832,32.64810546)
\curveto(114.30223685,32.71496279)(114.55516741,32.74839146)(114.82496,32.74839146)
\curveto(115.64276882,32.74839146)(116.24558665,32.56035522)(116.63341351,32.18428274)
\curveto(117.02967139,31.81656743)(117.22780033,31.28588738)(117.22780033,30.59224259)
\closepath
}
}
{
\newrgbcolor{curcolor}{0 0 0}
\pscustom[linestyle=none,fillstyle=solid,fillcolor=curcolor]
{
\newpath
\moveto(122.13465417,28.83723769)
\curveto(122.13465417,28.61995137)(122.06720602,28.43191513)(121.93230972,28.27312897)
\curveto(121.80584444,28.11434282)(121.63722407,28.03494974)(121.4264486,28.03494974)
\curveto(121.20724212,28.03494974)(121.03019072,28.11434282)(120.89529442,28.27312897)
\curveto(120.76039812,28.43191513)(120.69294997,28.61995137)(120.69294997,28.83723769)
\curveto(120.69294997,29.05452401)(120.76039812,29.24673883)(120.89529442,29.41388216)
\curveto(121.03019072,29.58102548)(121.20724212,29.66459714)(121.4264486,29.66459714)
\curveto(121.63722407,29.66459714)(121.80584444,29.58102548)(121.93230972,29.41388216)
\curveto(122.06720602,29.24673883)(122.13465417,29.05452401)(122.13465417,28.83723769)
\closepath
\moveto(118.82126382,28.69934445)
\curveto(118.82126382,30.00306237)(119.04468582,31.00174373)(119.49152981,31.69538852)
\curveto(119.94680482,32.39739048)(120.58334673,32.74839146)(121.40115555,32.74839146)
\curveto(122.22739538,32.74839146)(122.86393729,32.39739048)(123.31078128,31.69538852)
\curveto(123.75762527,31.00174373)(123.98104727,30.00306237)(123.98104727,28.69934445)
\curveto(123.98104727,27.39562653)(123.75762527,26.39276658)(123.31078128,25.69076463)
\curveto(122.86393729,24.99711983)(122.22739538,24.65029744)(121.40115555,24.65029744)
\curveto(120.58334673,24.65029744)(119.94680482,24.99711983)(119.49152981,25.69076463)
\curveto(119.04468582,26.39276658)(118.82126382,27.39562653)(118.82126382,28.69934445)
\closepath
\moveto(122.91873891,28.69934445)
\curveto(122.91873891,29.12555992)(122.89344585,29.5267039)(122.84285974,29.90277638)
\curveto(122.79227363,30.28720602)(122.70796344,30.62149267)(122.58992918,30.90563632)
\curveto(122.47189492,31.18977997)(122.31592107,31.41542345)(122.12200764,31.58256678)
\curveto(121.92809421,31.7497101)(121.68781018,31.83328176)(121.40115555,31.83328176)
\curveto(121.11450091,31.83328176)(120.87421688,31.7497101)(120.68030345,31.58256678)
\curveto(120.48639002,31.41542345)(120.33041617,31.18977997)(120.21238191,30.90563632)
\curveto(120.09434765,30.62149267)(120.01003746,30.28720602)(119.95945135,29.90277638)
\curveto(119.90886524,29.5267039)(119.88357218,29.12555992)(119.88357218,28.69934445)
\curveto(119.88357218,28.27312897)(119.90886524,27.86780641)(119.95945135,27.48337677)
\curveto(120.01003746,27.10730429)(120.09434765,26.77719623)(120.21238191,26.49305258)
\curveto(120.33041617,26.20890893)(120.48639002,25.98326544)(120.68030345,25.81612212)
\curveto(120.87421688,25.6489788)(121.11450091,25.56540713)(121.40115555,25.56540713)
\curveto(121.68781018,25.56540713)(121.92809421,25.6489788)(122.12200764,25.81612212)
\curveto(122.31592107,25.98326544)(122.47189492,26.20890893)(122.58992918,26.49305258)
\curveto(122.70796344,26.77719623)(122.79227363,27.10730429)(122.84285974,27.48337677)
\curveto(122.89344585,27.86780641)(122.91873891,28.27312897)(122.91873891,28.69934445)
\closepath
}
}
{
\newrgbcolor{curcolor}{0 0 0}
\pscustom[linestyle=none,fillstyle=solid,fillcolor=curcolor]
{
\newpath
\moveto(131.83453543,30.96831506)
\curveto(132.26451739,31.13545839)(132.68185281,31.34438754)(133.08654171,31.59510253)
\curveto(133.49123061,31.85417468)(133.86641094,32.18010416)(134.21208271,32.57289097)
\lineto(134.94558133,32.57289097)
\lineto(134.94558133,25.69076463)
\lineto(136.42522511,25.69076463)
\lineto(136.42522511,24.81326218)
\lineto(132.21393128,24.81326218)
\lineto(132.21393128,25.69076463)
\lineto(133.90856603,25.69076463)
\lineto(133.90856603,31.1312798)
\curveto(133.81582483,31.04770814)(133.70200608,30.9599579)(133.56710978,30.86802907)
\curveto(133.4406445,30.78445741)(133.29731718,30.70088575)(133.13712782,30.61731408)
\curveto(132.98536949,30.53374242)(132.82518013,30.45434934)(132.65655976,30.37913485)
\curveto(132.48793938,30.30392035)(132.32353452,30.24124161)(132.16334516,30.19109861)
\lineto(131.83453543,30.96831506)
\closepath
}
}
{
\newrgbcolor{curcolor}{0 0 0}
\pscustom[linestyle=none,fillstyle=solid,fillcolor=curcolor]
{
\newpath
\moveto(141.10444145,28.83723769)
\curveto(141.10444145,28.61995137)(141.0369933,28.43191513)(140.902097,28.27312897)
\curveto(140.77563172,28.11434282)(140.60701135,28.03494974)(140.39623588,28.03494974)
\curveto(140.17702939,28.03494974)(139.999978,28.11434282)(139.8650817,28.27312897)
\curveto(139.7301854,28.43191513)(139.66273725,28.61995137)(139.66273725,28.83723769)
\curveto(139.66273725,29.05452401)(139.7301854,29.24673883)(139.8650817,29.41388216)
\curveto(139.999978,29.58102548)(140.17702939,29.66459714)(140.39623588,29.66459714)
\curveto(140.60701135,29.66459714)(140.77563172,29.58102548)(140.902097,29.41388216)
\curveto(141.0369933,29.24673883)(141.10444145,29.05452401)(141.10444145,28.83723769)
\closepath
\moveto(137.7910511,28.69934445)
\curveto(137.7910511,30.00306237)(138.0144731,31.00174373)(138.46131709,31.69538852)
\curveto(138.9165921,32.39739048)(139.55313401,32.74839146)(140.37094282,32.74839146)
\curveto(141.19718265,32.74839146)(141.83372457,32.39739048)(142.28056856,31.69538852)
\curveto(142.72741255,31.00174373)(142.95083454,30.00306237)(142.95083454,28.69934445)
\curveto(142.95083454,27.39562653)(142.72741255,26.39276658)(142.28056856,25.69076463)
\curveto(141.83372457,24.99711983)(141.19718265,24.65029744)(140.37094282,24.65029744)
\curveto(139.55313401,24.65029744)(138.9165921,24.99711983)(138.46131709,25.69076463)
\curveto(138.0144731,26.39276658)(137.7910511,27.39562653)(137.7910511,28.69934445)
\closepath
\moveto(141.88852619,28.69934445)
\curveto(141.88852619,29.12555992)(141.86323313,29.5267039)(141.81264702,29.90277638)
\curveto(141.76206091,30.28720602)(141.67775072,30.62149267)(141.55971646,30.90563632)
\curveto(141.4416822,31.18977997)(141.28570835,31.41542345)(141.09179492,31.58256678)
\curveto(140.89788149,31.7497101)(140.65759746,31.83328176)(140.37094282,31.83328176)
\curveto(140.08428819,31.83328176)(139.84400415,31.7497101)(139.65009072,31.58256678)
\curveto(139.45617729,31.41542345)(139.30020345,31.18977997)(139.18216919,30.90563632)
\curveto(139.06413492,30.62149267)(138.97982474,30.28720602)(138.92923862,29.90277638)
\curveto(138.87865251,29.5267039)(138.85335946,29.12555992)(138.85335946,28.69934445)
\curveto(138.85335946,28.27312897)(138.87865251,27.86780641)(138.92923862,27.48337677)
\curveto(138.97982474,27.10730429)(139.06413492,26.77719623)(139.18216919,26.49305258)
\curveto(139.30020345,26.20890893)(139.45617729,25.98326544)(139.65009072,25.81612212)
\curveto(139.84400415,25.6489788)(140.08428819,25.56540713)(140.37094282,25.56540713)
\curveto(140.65759746,25.56540713)(140.89788149,25.6489788)(141.09179492,25.81612212)
\curveto(141.28570835,25.98326544)(141.4416822,26.20890893)(141.55971646,26.49305258)
\curveto(141.67775072,26.77719623)(141.76206091,27.10730429)(141.81264702,27.48337677)
\curveto(141.86323313,27.86780641)(141.88852619,28.27312897)(141.88852619,28.69934445)
\closepath
}
}
{
\newrgbcolor{curcolor}{0 0 0}
\pscustom[linestyle=none,fillstyle=solid,fillcolor=curcolor]
{
\newpath
\moveto(152.53690472,25.56540713)
\curveto(153.20295519,25.56540713)(153.67509224,25.69494321)(153.95331586,25.95401536)
\curveto(154.23997049,26.22144468)(154.38329781,26.57662424)(154.38329781,27.01955405)
\curveto(154.38329781,27.3036977)(154.32428068,27.54187693)(154.20624642,27.73409176)
\curveto(154.08821216,27.92630658)(153.93223831,28.08091415)(153.73832488,28.19791448)
\curveto(153.54441145,28.3149148)(153.32098945,28.39848647)(153.06805889,28.44862946)
\curveto(152.81512833,28.49877246)(152.54955124,28.52384396)(152.27132763,28.52384396)
\lineto(152.00575054,28.52384396)
\lineto(152.00575054,29.36373916)
\lineto(152.37249985,29.36373916)
\curveto(152.55798226,29.36373916)(152.74768018,29.38045349)(152.94159361,29.41388216)
\curveto(153.14393806,29.45566799)(153.32520496,29.52252532)(153.48539432,29.61445414)
\curveto(153.65401469,29.71474014)(153.78891099,29.8484548)(153.89008322,30.01559812)
\curveto(153.99125544,30.18274144)(154.04184155,30.39584918)(154.04184155,30.65492133)
\curveto(154.04184155,31.08113681)(153.90694525,31.38199479)(153.63715266,31.55749528)
\curveto(153.37579108,31.74135293)(153.06805889,31.83328176)(152.71395611,31.83328176)
\curveto(152.3514223,31.83328176)(152.04369012,31.77896018)(151.79075956,31.67031702)
\curveto(151.537829,31.57003103)(151.32705353,31.46556645)(151.15843316,31.35692329)
\lineto(150.75374426,32.14667549)
\curveto(150.93079565,32.27203299)(151.19637274,32.40156906)(151.55047553,32.53528372)
\curveto(151.91300933,32.67735555)(152.31348272,32.74839146)(152.75189569,32.74839146)
\curveto(153.16501561,32.74839146)(153.51911839,32.69824846)(153.81420405,32.59796247)
\curveto(154.1092897,32.49767647)(154.34957374,32.35560465)(154.53505615,32.17174699)
\curveto(154.72896958,31.98788934)(154.87229689,31.77060302)(154.9650381,31.51988803)
\curveto(155.05777931,31.27753021)(155.10414991,31.01010089)(155.10414991,30.71760008)
\curveto(155.10414991,30.30809894)(154.99454667,29.96127654)(154.77534018,29.67713289)
\curveto(154.56456471,29.39298924)(154.2905566,29.17570292)(153.95331586,29.02527393)
\curveto(154.35800475,28.9082736)(154.70789203,28.67845153)(155.00297768,28.33580772)
\curveto(155.29806334,28.00152107)(155.44560617,27.55441268)(155.44560617,26.99448255)
\curveto(155.44560617,26.6601959)(155.38658904,26.34680217)(155.26855477,26.05430135)
\curveto(155.15895153,25.7701577)(154.98611565,25.5236213)(154.75004712,25.31469215)
\curveto(154.52240962,25.10576299)(154.22310846,24.94279825)(153.85214363,24.82579793)
\curveto(153.48960983,24.7087976)(153.05541237,24.65029744)(152.54955124,24.65029744)
\curveto(152.35563781,24.65029744)(152.15329337,24.66701177)(151.9425179,24.70044043)
\curveto(151.74017345,24.72551193)(151.55047553,24.76311918)(151.37342414,24.81326218)
\curveto(151.19637274,24.85504801)(151.03618339,24.89683384)(150.89285607,24.93861967)
\curveto(150.75795977,24.98876267)(150.66100306,25.02636992)(150.60198593,25.05144141)
\lineto(150.80433037,25.94147961)
\curveto(150.93922667,25.87462228)(151.15421765,25.7952292)(151.4493033,25.70330038)
\curveto(151.74438896,25.61137155)(152.10692276,25.56540713)(152.53690472,25.56540713)
\closepath
}
}
{
\newrgbcolor{curcolor}{0 0 0}
\pscustom[linestyle=none,fillstyle=solid,fillcolor=curcolor]
{
\newpath
\moveto(158.1393176,24.81326218)
\curveto(158.18147269,25.40662098)(158.28686042,26.03340844)(158.4554808,26.69362457)
\curveto(158.63253219,27.36219786)(158.84330766,28.00569966)(159.0878072,28.62412995)
\curveto(159.33230674,29.25091742)(159.60209934,29.82756188)(159.897185,30.35406335)
\curveto(160.19227065,30.88892198)(160.4831408,31.32767321)(160.76979543,31.67031702)
\lineto(156.97583702,31.67031702)
\lineto(156.97583702,32.57289097)
\lineto(161.93327601,32.57289097)
\lineto(161.93327601,31.70792427)
\curveto(161.68034545,31.41542345)(161.40633734,31.02263664)(161.11125169,30.52956384)
\curveto(160.81616603,30.03649104)(160.53372691,29.48073948)(160.26393431,28.86230919)
\curveto(160.00257273,28.25223606)(159.77493522,27.59619851)(159.58102179,26.89419655)
\curveto(159.38710836,26.20055176)(159.26485859,25.50690697)(159.21427248,24.81326218)
\lineto(158.1393176,24.81326218)
\closepath
}
}
{
\newrgbcolor{curcolor}{0 0 0}
\pscustom[linestyle=none,fillstyle=solid,fillcolor=curcolor]
{
\newpath
\moveto(180.80189873,29.46402515)
\curveto(180.80189873,27.91794941)(180.42250288,26.75212473)(179.6637112,25.96655111)
\curveto(178.90491952,25.18933466)(177.76251649,24.79654785)(176.2365021,24.78819068)
\lineto(176.19856252,25.66569313)
\curveto(177.14283661,25.66569313)(177.90162829,25.84955078)(178.47493757,26.21726609)
\curveto(179.05667786,26.59333857)(179.44028921,27.23684037)(179.62577162,28.14777148)
\curveto(179.42342717,28.05584265)(179.20000517,27.98062816)(178.95550563,27.92212799)
\curveto(178.71100609,27.871985)(178.45386002,27.8469135)(178.18406742,27.8469135)
\curveto(177.73722343,27.8469135)(177.3620431,27.90959224)(177.05852643,28.03494974)
\curveto(176.75500975,28.1686644)(176.51051021,28.34416489)(176.3250278,28.56145121)
\curveto(176.13954539,28.78709469)(176.00464909,29.04198826)(175.9203389,29.32613191)
\curveto(175.83602871,29.61027556)(175.79387362,29.91113354)(175.79387362,30.22870586)
\curveto(175.79387362,30.51284951)(175.84024422,30.80117174)(175.93298543,31.09367256)
\curveto(176.02572663,31.39453054)(176.16905395,31.66613844)(176.36296738,31.90849626)
\curveto(176.55688081,32.15085408)(176.80559586,32.35142606)(177.10911254,32.51021222)
\curveto(177.41262921,32.66899838)(177.77516301,32.74839146)(178.19671395,32.74839146)
\curveto(179.05667786,32.74839146)(179.7058663,32.45589064)(180.14427927,31.87088901)
\curveto(180.58269224,31.28588738)(180.80189873,30.48359943)(180.80189873,29.46402515)
\closepath
\moveto(178.29788617,28.69934445)
\curveto(178.56767877,28.69934445)(178.81639382,28.72441595)(179.04403133,28.77455894)
\curveto(179.28009985,28.82470194)(179.50773736,28.89573785)(179.72694384,28.98766668)
\curveto(179.73537486,29.07123834)(179.73959037,29.15063142)(179.73959037,29.22584592)
\lineto(179.73959037,29.46402515)
\curveto(179.73959037,29.78995463)(179.71429731,30.09916978)(179.6637112,30.3916706)
\curveto(179.62155611,30.68417141)(179.54146143,30.93906498)(179.42342717,31.1563513)
\curveto(179.31382393,31.37363762)(179.15785008,31.54495953)(178.95550563,31.67031702)
\curveto(178.7615922,31.80403168)(178.51287715,31.87088901)(178.20936048,31.87088901)
\curveto(177.95642992,31.87088901)(177.74565445,31.81656743)(177.57703408,31.70792427)
\curveto(177.4084137,31.60763828)(177.26930189,31.4781022)(177.15969865,31.31931604)
\curveto(177.05009541,31.16888705)(176.97000073,30.99756514)(176.91941462,30.80535032)
\curveto(176.87725952,30.6131355)(176.85618198,30.42927785)(176.85618198,30.25377736)
\curveto(176.85618198,29.74399022)(176.97000073,29.35538199)(177.19763823,29.08795267)
\curveto(177.43370676,28.82888052)(177.80045607,28.69934445)(178.29788617,28.69934445)
\closepath
}
}
{
\newrgbcolor{curcolor}{0 0 0}
\pscustom[linestyle=none,fillstyle=solid,fillcolor=curcolor]
{
\newpath
\moveto(190.47648693,25.56540713)
\curveto(191.14253741,25.56540713)(191.61467446,25.69494321)(191.89289807,25.95401536)
\curveto(192.17955271,26.22144468)(192.32288003,26.57662424)(192.32288003,27.01955405)
\curveto(192.32288003,27.3036977)(192.2638629,27.54187693)(192.14582863,27.73409176)
\curveto(192.02779437,27.92630658)(191.87182053,28.08091415)(191.6779071,28.19791448)
\curveto(191.48399367,28.3149148)(191.26057167,28.39848647)(191.00764111,28.44862946)
\curveto(190.75471055,28.49877246)(190.48913346,28.52384396)(190.21090984,28.52384396)
\lineto(189.94533275,28.52384396)
\lineto(189.94533275,29.36373916)
\lineto(190.31208207,29.36373916)
\curveto(190.49756448,29.36373916)(190.6872624,29.38045349)(190.88117583,29.41388216)
\curveto(191.08352028,29.45566799)(191.26478718,29.52252532)(191.42497654,29.61445414)
\curveto(191.59359691,29.71474014)(191.72849321,29.8484548)(191.82966543,30.01559812)
\curveto(191.93083766,30.18274144)(191.98142377,30.39584918)(191.98142377,30.65492133)
\curveto(191.98142377,31.08113681)(191.84652747,31.38199479)(191.57673487,31.55749528)
\curveto(191.31537329,31.74135293)(191.00764111,31.83328176)(190.65353833,31.83328176)
\curveto(190.29100452,31.83328176)(189.98327234,31.77896018)(189.73034178,31.67031702)
\curveto(189.47741122,31.57003103)(189.26663575,31.46556645)(189.09801538,31.35692329)
\lineto(188.69332648,32.14667549)
\curveto(188.87037787,32.27203299)(189.13595496,32.40156906)(189.49005775,32.53528372)
\curveto(189.85259155,32.67735555)(190.25306494,32.74839146)(190.69147791,32.74839146)
\curveto(191.10459783,32.74839146)(191.45870061,32.69824846)(191.75378627,32.59796247)
\curveto(192.04887192,32.49767647)(192.28915595,32.35560465)(192.47463836,32.17174699)
\curveto(192.66855179,31.98788934)(192.81187911,31.77060302)(192.90462032,31.51988803)
\curveto(192.99736152,31.27753021)(193.04373213,31.01010089)(193.04373213,30.71760008)
\curveto(193.04373213,30.30809894)(192.93412888,29.96127654)(192.7149224,29.67713289)
\curveto(192.50414693,29.39298924)(192.23013882,29.17570292)(191.89289807,29.02527393)
\curveto(192.29758697,28.9082736)(192.64747425,28.67845153)(192.9425599,28.33580772)
\curveto(193.23764556,28.00152107)(193.38518838,27.55441268)(193.38518838,26.99448255)
\curveto(193.38518838,26.6601959)(193.32617125,26.34680217)(193.20813699,26.05430135)
\curveto(193.09853375,25.7701577)(192.92569786,25.5236213)(192.68962934,25.31469215)
\curveto(192.46199184,25.10576299)(192.16269067,24.94279825)(191.79172585,24.82579793)
\curveto(191.42919205,24.7087976)(190.99499458,24.65029744)(190.48913346,24.65029744)
\curveto(190.29522003,24.65029744)(190.09287558,24.66701177)(189.88210011,24.70044043)
\curveto(189.67975567,24.72551193)(189.49005775,24.76311918)(189.31300635,24.81326218)
\curveto(189.13595496,24.85504801)(188.9757656,24.89683384)(188.83243829,24.93861967)
\curveto(188.69754199,24.98876267)(188.60058527,25.02636992)(188.54156814,25.05144141)
\lineto(188.74391259,25.94147961)
\curveto(188.87880889,25.87462228)(189.09379987,25.7952292)(189.38888552,25.70330038)
\curveto(189.68397118,25.61137155)(190.04650498,25.56540713)(190.47648693,25.56540713)
\closepath
}
}
{
\newrgbcolor{curcolor}{0 0 0}
\pscustom[linestyle=none,fillstyle=solid,fillcolor=curcolor]
{
\newpath
\moveto(196.79975958,25.56540713)
\curveto(197.46581005,25.56540713)(197.9379471,25.69494321)(198.21617072,25.95401536)
\curveto(198.50282535,26.22144468)(198.64615267,26.57662424)(198.64615267,27.01955405)
\curveto(198.64615267,27.3036977)(198.58713554,27.54187693)(198.46910128,27.73409176)
\curveto(198.35106702,27.92630658)(198.19509317,28.08091415)(198.00117974,28.19791448)
\curveto(197.80726631,28.3149148)(197.58384431,28.39848647)(197.33091375,28.44862946)
\curveto(197.07798319,28.49877246)(196.8124061,28.52384396)(196.53418249,28.52384396)
\lineto(196.2686054,28.52384396)
\lineto(196.2686054,29.36373916)
\lineto(196.63535471,29.36373916)
\curveto(196.82083712,29.36373916)(197.01053504,29.38045349)(197.20444847,29.41388216)
\curveto(197.40679292,29.45566799)(197.58805982,29.52252532)(197.74824918,29.61445414)
\curveto(197.91686955,29.71474014)(198.05176585,29.8484548)(198.15293808,30.01559812)
\curveto(198.2541103,30.18274144)(198.30469641,30.39584918)(198.30469641,30.65492133)
\curveto(198.30469641,31.08113681)(198.16980011,31.38199479)(197.90000752,31.55749528)
\curveto(197.63864594,31.74135293)(197.33091375,31.83328176)(196.97681097,31.83328176)
\curveto(196.61427716,31.83328176)(196.30654498,31.77896018)(196.05361442,31.67031702)
\curveto(195.80068386,31.57003103)(195.58990839,31.46556645)(195.42128802,31.35692329)
\lineto(195.01659912,32.14667549)
\curveto(195.19365051,32.27203299)(195.4592276,32.40156906)(195.81333039,32.53528372)
\curveto(196.17586419,32.67735555)(196.57633758,32.74839146)(197.01475055,32.74839146)
\curveto(197.42787047,32.74839146)(197.78197325,32.69824846)(198.07705891,32.59796247)
\curveto(198.37214456,32.49767647)(198.6124286,32.35560465)(198.79791101,32.17174699)
\curveto(198.99182444,31.98788934)(199.13515175,31.77060302)(199.22789296,31.51988803)
\curveto(199.32063417,31.27753021)(199.36700477,31.01010089)(199.36700477,30.71760008)
\curveto(199.36700477,30.30809894)(199.25740153,29.96127654)(199.03819504,29.67713289)
\curveto(198.82741957,29.39298924)(198.55341146,29.17570292)(198.21617072,29.02527393)
\curveto(198.62085961,28.9082736)(198.97074689,28.67845153)(199.26583254,28.33580772)
\curveto(199.5609182,28.00152107)(199.70846103,27.55441268)(199.70846103,26.99448255)
\curveto(199.70846103,26.6601959)(199.6494439,26.34680217)(199.53140963,26.05430135)
\curveto(199.42180639,25.7701577)(199.24897051,25.5236213)(199.01290198,25.31469215)
\curveto(198.78526448,25.10576299)(198.48596331,24.94279825)(198.11499849,24.82579793)
\curveto(197.75246469,24.7087976)(197.31826723,24.65029744)(196.8124061,24.65029744)
\curveto(196.61849267,24.65029744)(196.41614823,24.66701177)(196.20537276,24.70044043)
\curveto(196.00302831,24.72551193)(195.81333039,24.76311918)(195.636279,24.81326218)
\curveto(195.4592276,24.85504801)(195.29903825,24.89683384)(195.15571093,24.93861967)
\curveto(195.02081463,24.98876267)(194.92385792,25.02636992)(194.86484078,25.05144141)
\lineto(195.06718523,25.94147961)
\curveto(195.20208153,25.87462228)(195.41707251,25.7952292)(195.71215816,25.70330038)
\curveto(196.00724382,25.61137155)(196.36977762,25.56540713)(196.79975958,25.56540713)
\closepath
}
}
{
\newrgbcolor{curcolor}{0 0 0}
\pscustom[linestyle=none,fillstyle=solid,fillcolor=curcolor]
{
\newpath
\moveto(213.77139542,27.92212799)
\curveto(213.77139542,28.67427295)(213.87256765,29.33866766)(214.07491209,29.91531213)
\curveto(214.28568756,30.50031376)(214.58077322,30.98920798)(214.96016906,31.38199479)
\curveto(215.34799592,31.7747816)(215.81591746,32.07563958)(216.36393367,32.28456874)
\curveto(216.91194989,32.49349789)(217.53162976,32.60214105)(218.22297329,32.61049822)
\lineto(218.31149899,31.73299577)
\curveto(217.864655,31.7246386)(217.45575059,31.6744956)(217.08478577,31.58256678)
\curveto(216.72225197,31.49899512)(216.39765775,31.35692329)(216.11100311,31.1563513)
\curveto(215.82434847,30.96413648)(215.58406444,30.7134215)(215.39015101,30.40420635)
\curveto(215.19623758,30.0949912)(215.04869475,29.71474014)(214.94752253,29.26345316)
\curveto(215.14986698,29.35538199)(215.36907346,29.43059649)(215.60514199,29.48909665)
\curveto(215.84964153,29.54759681)(216.1067876,29.5768469)(216.3765802,29.5768469)
\curveto(216.81499317,29.5768469)(217.18595799,29.50998957)(217.48947467,29.37627491)
\curveto(217.80142236,29.24256025)(218.05013741,29.06288118)(218.23561982,28.83723769)
\curveto(218.42110223,28.61995137)(218.55599853,28.3650578)(218.64030872,28.07255698)
\curveto(218.72461891,27.78005617)(218.766774,27.47919819)(218.766774,27.16998304)
\curveto(218.766774,26.88583939)(218.7204034,26.59333857)(218.62766219,26.29248059)
\curveto(218.53492098,25.99997977)(218.39159367,25.72837187)(218.19768024,25.47765689)
\curveto(218.00376681,25.23529907)(217.75505176,25.03472708)(217.45153508,24.87594092)
\curveto(217.14801841,24.72551193)(216.78548461,24.65029744)(216.36393367,24.65029744)
\curveto(215.49553874,24.65029744)(214.84635031,24.93861967)(214.41636835,25.51526414)
\curveto(213.9863864,26.0919086)(213.77139542,26.89419655)(213.77139542,27.92212799)
\closepath
\moveto(216.26276145,28.72441595)
\curveto(215.99296885,28.72441595)(215.7442538,28.69934445)(215.51661629,28.64920145)
\curveto(215.29740981,28.59905845)(215.07398781,28.52384396)(214.84635031,28.42355796)
\curveto(214.83791929,28.3399863)(214.83370378,28.25641464)(214.83370378,28.17284298)
\lineto(214.83370378,27.92212799)
\curveto(214.83370378,27.59619851)(214.85478132,27.28698336)(214.89693642,26.99448255)
\curveto(214.94752253,26.7103389)(215.02761721,26.45544533)(215.13722045,26.22980184)
\curveto(215.25525471,26.01251552)(215.41122856,25.83701503)(215.60514199,25.70330038)
\curveto(215.79905542,25.57794288)(216.04777047,25.51526414)(216.35128714,25.51526414)
\curveto(216.6042177,25.51526414)(216.81499317,25.56540713)(216.98361354,25.66569313)
\curveto(217.15223392,25.77433629)(217.29134573,25.90805095)(217.40094897,26.0668371)
\curveto(217.51055221,26.22562326)(217.58643138,26.40112375)(217.62858648,26.59333857)
\curveto(217.67917259,26.79391056)(217.70446564,26.9819468)(217.70446564,27.15744729)
\curveto(217.70446564,27.66723443)(217.58643138,28.05584265)(217.35036286,28.32327197)
\curveto(217.12272535,28.59070129)(216.76019155,28.72441595)(216.26276145,28.72441595)
\closepath
}
}
{
\newrgbcolor{curcolor}{0 0 0}
\pscustom[linestyle=none,fillstyle=solid,fillcolor=curcolor]
{
\newpath
\moveto(236.15574527,25.61555013)
\curveto(236.15574527,25.36483515)(236.07143508,25.14337024)(235.90281471,24.95115542)
\curveto(235.73419434,24.7589406)(235.51077234,24.66283319)(235.23254872,24.66283319)
\curveto(234.94589409,24.66283319)(234.71825658,24.7589406)(234.54963621,24.95115542)
\curveto(234.38101583,25.14337024)(234.29670565,25.36483515)(234.29670565,25.61555013)
\curveto(234.29670565,25.87462228)(234.38101583,26.10026577)(234.54963621,26.29248059)
\curveto(234.71825658,26.48469541)(234.94589409,26.58080282)(235.23254872,26.58080282)
\curveto(235.51077234,26.58080282)(235.73419434,26.48469541)(235.90281471,26.29248059)
\curveto(236.07143508,26.10026577)(236.15574527,25.87462228)(236.15574527,25.61555013)
\closepath
}
}
{
\newrgbcolor{curcolor}{0 0 0}
\pscustom[linestyle=none,fillstyle=solid,fillcolor=curcolor]
{
\newpath
\moveto(242.47901791,25.61555013)
\curveto(242.47901791,25.36483515)(242.39470773,25.14337024)(242.22608735,24.95115542)
\curveto(242.05746698,24.7589406)(241.83404498,24.66283319)(241.55582137,24.66283319)
\curveto(241.26916673,24.66283319)(241.04152923,24.7589406)(240.87290885,24.95115542)
\curveto(240.70428848,25.14337024)(240.61997829,25.36483515)(240.61997829,25.61555013)
\curveto(240.61997829,25.87462228)(240.70428848,26.10026577)(240.87290885,26.29248059)
\curveto(241.04152923,26.48469541)(241.26916673,26.58080282)(241.55582137,26.58080282)
\curveto(241.83404498,26.58080282)(242.05746698,26.48469541)(242.22608735,26.29248059)
\curveto(242.39470773,26.10026577)(242.47901791,25.87462228)(242.47901791,25.61555013)
\closepath
}
}
{
\newrgbcolor{curcolor}{0 0 0}
\pscustom[linestyle=none,fillstyle=solid,fillcolor=curcolor]
{
\newpath
\moveto(248.80227523,25.61555013)
\curveto(248.80227523,25.36483515)(248.71796504,25.14337024)(248.54934467,24.95115542)
\curveto(248.3807243,24.7589406)(248.1573023,24.66283319)(247.87907868,24.66283319)
\curveto(247.59242405,24.66283319)(247.36478654,24.7589406)(247.19616617,24.95115542)
\curveto(247.02754579,25.14337024)(246.94323561,25.36483515)(246.94323561,25.61555013)
\curveto(246.94323561,25.87462228)(247.02754579,26.10026577)(247.19616617,26.29248059)
\curveto(247.36478654,26.48469541)(247.59242405,26.58080282)(247.87907868,26.58080282)
\curveto(248.1573023,26.58080282)(248.3807243,26.48469541)(248.54934467,26.29248059)
\curveto(248.71796504,26.10026577)(248.80227523,25.87462228)(248.80227523,25.61555013)
\closepath
}
}
{
\newrgbcolor{curcolor}{0 0 0}
\pscustom[linestyle=none,fillstyle=solid,fillcolor=curcolor]
{
\newpath
\moveto(266.02684164,29.5768469)
\curveto(267.15659814,29.53506106)(267.97862247,29.28852466)(268.49291461,28.83723769)
\curveto(269.00720675,28.38595072)(269.26435282,27.78005617)(269.26435282,27.01955405)
\curveto(269.26435282,26.67691023)(269.20955119,26.35933792)(269.09994795,26.0668371)
\curveto(268.99034471,25.77433629)(268.81750883,25.5236213)(268.5814403,25.31469215)
\curveto(268.3538028,25.10576299)(268.05871714,24.94279825)(267.69618334,24.82579793)
\curveto(267.34208055,24.7087976)(266.92052962,24.65029744)(266.43153053,24.65029744)
\curveto(266.22918609,24.65029744)(266.02684164,24.66701177)(265.82449719,24.70044043)
\curveto(265.63058376,24.72551193)(265.44510135,24.7589406)(265.26804995,24.80072643)
\curveto(265.09099856,24.84251226)(264.93502472,24.88429809)(264.80012842,24.92608392)
\curveto(264.66523212,24.97622692)(264.5682754,25.01801275)(264.50925827,25.05144141)
\lineto(264.71160272,25.94147961)
\curveto(264.84649902,25.87462228)(265.05305898,25.7952292)(265.33128259,25.70330038)
\curveto(265.61793723,25.61137155)(265.97625552,25.56540713)(266.40623748,25.56540713)
\curveto(266.74347823,25.56540713)(267.02591735,25.60301438)(267.25355486,25.67822888)
\curveto(267.48119236,25.75344337)(267.66245926,25.85372937)(267.79735556,25.97908686)
\curveto(267.94068288,26.11280152)(268.04185511,26.26323051)(268.10087224,26.43037383)
\curveto(268.16832039,26.59751716)(268.20204446,26.77301765)(268.20204446,26.9568753)
\curveto(268.20204446,27.24101895)(268.15145835,27.49173394)(268.05028612,27.70902026)
\curveto(267.95754492,27.93466374)(267.78892454,28.12269998)(267.544425,28.27312897)
\curveto(267.30835648,28.42355796)(266.97954675,28.53637971)(266.55799581,28.6115942)
\curveto(266.1448759,28.69516586)(265.62215274,28.7369517)(264.98982634,28.7369517)
\curveto(265.04041245,29.10466701)(265.07835203,29.44731082)(265.10364509,29.76488313)
\curveto(265.13736916,30.09081262)(265.16266222,30.40420635)(265.17952426,30.70506433)
\curveto(265.20481731,31.01427948)(265.22167935,31.31931604)(265.23011037,31.62017402)
\curveto(265.24697241,31.92103201)(265.26383444,32.23860432)(265.28069648,32.57289097)
\lineto(269.04936184,32.57289097)
\lineto(269.04936184,31.69538852)
\lineto(266.1912465,31.69538852)
\curveto(266.18281548,31.57838819)(266.17016895,31.42378062)(266.15330692,31.2315658)
\curveto(266.1448759,31.04770814)(266.13222937,30.85131474)(266.11536733,30.64238558)
\curveto(266.0985053,30.43345643)(266.08164326,30.23288444)(266.06478122,30.04066962)
\curveto(266.04791918,29.8484548)(266.03527266,29.69384722)(266.02684164,29.5768469)
\closepath
}
}
{
\newrgbcolor{curcolor}{0 0 0}
\pscustom[linestyle=none,fillstyle=solid,fillcolor=curcolor]
{
\newpath
\moveto(275.30940184,30.59224259)
\curveto(275.30940184,30.32481327)(275.25460022,30.06574112)(275.14499698,29.81502613)
\curveto(275.04382475,29.56431115)(274.90471294,29.31777474)(274.72766155,29.07541693)
\curveto(274.55904118,28.83305911)(274.36512775,28.59487987)(274.14592126,28.36087922)
\curveto(273.92671478,28.12687857)(273.70329278,27.8970565)(273.47565528,27.67141301)
\curveto(273.34919,27.54605552)(273.20164717,27.39562653)(273.03302679,27.22012604)
\curveto(272.86440642,27.04462555)(272.70421707,26.86494647)(272.55245873,26.68108882)
\curveto(272.40070039,26.49723116)(272.27423511,26.31755209)(272.17306289,26.1420516)
\curveto(272.07189066,25.96655111)(272.02130455,25.81612212)(272.02130455,25.69076463)
\lineto(275.62556504,25.69076463)
\lineto(275.62556504,24.81326218)
\lineto(270.88311703,24.81326218)
\curveto(270.87468601,24.85504801)(270.8704705,24.89683384)(270.8704705,24.93861967)
\lineto(270.8704705,25.07651291)
\curveto(270.8704705,25.42751389)(270.92948763,25.75344337)(271.04752189,26.05430135)
\curveto(271.16555615,26.35515934)(271.31731449,26.63930299)(271.5027969,26.9067323)
\curveto(271.68827931,27.17416162)(271.89483927,27.42487661)(272.12247678,27.65887726)
\curveto(272.3585453,27.90123508)(272.59039831,28.13523573)(272.81803582,28.36087922)
\curveto(273.00351823,28.54473687)(273.18056962,28.72441595)(273.34919,28.89991644)
\curveto(273.52624139,29.07541693)(273.67799972,29.25091742)(273.80446501,29.4264179)
\curveto(273.9393613,29.60191839)(274.04474904,29.78159747)(274.12062821,29.96545512)
\curveto(274.20493839,30.15766994)(274.24709349,30.35406335)(274.24709349,30.55463534)
\curveto(274.24709349,30.78027882)(274.2091539,30.97249365)(274.13327473,31.1312798)
\curveto(274.06582658,31.29006596)(273.96886987,31.41960204)(273.84240459,31.51988803)
\curveto(273.72437033,31.62853119)(273.58525852,31.70792427)(273.42506916,31.75806727)
\curveto(273.26487981,31.80821026)(273.09204393,31.83328176)(272.90656151,31.83328176)
\curveto(272.68735503,31.83328176)(272.48501058,31.80403168)(272.29952817,31.74553152)
\curveto(272.12247678,31.68703135)(271.96228742,31.61599544)(271.8189601,31.53242378)
\curveto(271.6840638,31.44885212)(271.56602954,31.36528046)(271.46485732,31.28170879)
\curveto(271.36368509,31.2064943)(271.28780592,31.14381555)(271.23721981,31.09367256)
\lineto(270.71871216,31.82074601)
\curveto(270.78616031,31.89596051)(270.88733254,31.98788934)(271.02222884,32.0965325)
\curveto(271.15712513,32.20517566)(271.31731449,32.30546165)(271.5027969,32.39739048)
\curveto(271.69671033,32.49767647)(271.91170131,32.58124813)(272.14776983,32.64810546)
\curveto(272.38383835,32.71496279)(272.63676892,32.74839146)(272.90656151,32.74839146)
\curveto(273.72437033,32.74839146)(274.32718816,32.56035522)(274.71501502,32.18428274)
\curveto(275.1112729,31.81656743)(275.30940184,31.28588738)(275.30940184,30.59224259)
\closepath
}
}
{
\newrgbcolor{curcolor}{0 0 0}
\pscustom[linewidth=0.56097835,linecolor=curcolor]
{
\newpath
\moveto(52.53490942,35.05890473)
\lineto(277.23287695,35.05890473)
\lineto(277.23287695,22.33978566)
\lineto(52.53490942,22.33978566)
\closepath
}
}
{
\newrgbcolor{curcolor}{0 0 0}
\pscustom[linewidth=0.51800942,linecolor=curcolor]
{
\newpath
\moveto(71.388227,22.35389733)
\lineto(71.388227,34.95329233)
}
}
{
\newrgbcolor{curcolor}{0 0 0}
\pscustom[linewidth=0.5154072,linecolor=curcolor]
{
\newpath
\moveto(90.26845,22.44820833)
\lineto(90.26845,34.92133433)
}
}
{
\newrgbcolor{curcolor}{0 0 0}
\pscustom[linewidth=0.51930571,linecolor=curcolor]
{
\newpath
\moveto(109.10774,22.44820733)
\lineto(109.10774,35.11073633)
}
}
{
\newrgbcolor{curcolor}{0 0 0}
\pscustom[linewidth=0.52573889,linecolor=curcolor]
{
\newpath
\moveto(128.03631,22.26963633)
\lineto(128.03631,35.24783833)
}
}
{
\newrgbcolor{curcolor}{0 0 0}
\pscustom[linewidth=0.51670998,linecolor=curcolor]
{
\newpath
\moveto(146.87559,22.44820733)
\lineto(146.87559,34.98446733)
}
}
{
\newrgbcolor{curcolor}{0 0 0}
\pscustom[linewidth=0.52059871,linecolor=curcolor]
{
\newpath
\moveto(165.6256,22.44820833)
\lineto(165.6256,35.17387233)
}
}
{
\newrgbcolor{curcolor}{0 0 0}
\pscustom[linewidth=0.51670998,linecolor=curcolor]
{
\newpath
\moveto(184.55417,22.35892233)
\lineto(184.55417,34.89518233)
}
}
{
\newrgbcolor{curcolor}{0 0 0}
\pscustom[linewidth=0.51670998,linecolor=curcolor]
{
\newpath
\moveto(203.66131,22.44820833)
\lineto(203.66131,34.98446833)
}
}
{
\newrgbcolor{curcolor}{0 0 0}
\pscustom[linewidth=0.52059871,linecolor=curcolor]
{
\newpath
\moveto(222.58988,22.44820733)
\lineto(222.58988,35.17387133)
}
}
{
\newrgbcolor{curcolor}{0 0 0}
\pscustom[linewidth=0.52573889,linecolor=curcolor]
{
\newpath
\moveto(258.24019,35.09998233)
\lineto(258.20469,22.12182833)
}
}
{
\newrgbcolor{curcolor}{0 0 0}
\pscustom[linewidth=1,linecolor=curcolor]
{
\newpath
\moveto(61.071429,215.89285433)
\lineto(61.071429,44.10714433)
}
}
{
\newrgbcolor{curcolor}{0 0 0}
\pscustom[linewidth=1,linecolor=curcolor]
{
\newpath
\moveto(80.267863,215.89285433)
\lineto(80.267863,44.10714433)
}
}
{
\newrgbcolor{curcolor}{0 0 0}
\pscustom[linewidth=1,linecolor=curcolor]
{
\newpath
\moveto(99.46429,215.89285433)
\lineto(99.46429,44.10714433)
}
}
{
\newrgbcolor{curcolor}{0 0 0}
\pscustom[linewidth=1,linecolor=curcolor]
{
\newpath
\moveto(118.66072,215.89285433)
\lineto(118.66072,44.10714433)
}
}
{
\newrgbcolor{curcolor}{0 0 0}
\pscustom[linewidth=1,linecolor=curcolor]
{
\newpath
\moveto(137.85714,215.89285433)
\lineto(137.85714,44.10714433)
}
}
{
\newrgbcolor{curcolor}{0 0 0}
\pscustom[linewidth=1,linecolor=curcolor]
{
\newpath
\moveto(157.05357,215.89285433)
\lineto(157.05357,44.10714433)
}
}
{
\newrgbcolor{curcolor}{0 0 0}
\pscustom[linewidth=1,linecolor=curcolor]
{
\newpath
\moveto(176.25,215.89285433)
\lineto(176.25,44.10714433)
}
}
{
\newrgbcolor{curcolor}{0 0 0}
\pscustom[linewidth=1,linecolor=curcolor]
{
\newpath
\moveto(195.44643,215.89285433)
\lineto(195.44643,44.10714433)
}
}
{
\newrgbcolor{curcolor}{0 0 0}
\pscustom[linewidth=1,linecolor=curcolor]
{
\newpath
\moveto(214.64286,215.89285433)
\lineto(214.64286,44.10714433)
}
}
{
\newrgbcolor{curcolor}{0 0 0}
\pscustom[linewidth=1,linecolor=curcolor]
{
\newpath
\moveto(269.785712,215.89285433)
\lineto(269.785712,44.10714533)
}
}
{
\newrgbcolor{curcolor}{0 0 0}
\pscustom[linewidth=0.98769152,linecolor=curcolor]
{
\newpath
\moveto(195.41836,212.64285433)
\lineto(214.58065,212.64285433)
\lineto(214.58065,212.64285433)
}
}
{
\newrgbcolor{curcolor}{0 0 0}
\pscustom[linestyle=none,fillstyle=solid,fillcolor=curcolor]
{
\newpath
\moveto(196.38629769,212.64285433)
\curveto(196.38629769,212.09764861)(195.94381189,211.65516281)(195.39860617,211.65516281)
\curveto(194.85340045,211.65516281)(194.41091465,212.09764861)(194.41091465,212.64285433)
\curveto(194.41091465,213.18806005)(194.85340045,213.63054585)(195.39860617,213.63054585)
\curveto(195.94381189,213.63054585)(196.38629769,213.18806005)(196.38629769,212.64285433)
\closepath
}
}
{
\newrgbcolor{curcolor}{0 0 0}
\pscustom[linewidth=0.24692288,linecolor=curcolor]
{
\newpath
\moveto(196.38629769,212.64285433)
\curveto(196.38629769,212.09764861)(195.94381189,211.65516281)(195.39860617,211.65516281)
\curveto(194.85340045,211.65516281)(194.41091465,212.09764861)(194.41091465,212.64285433)
\curveto(194.41091465,213.18806005)(194.85340045,213.63054585)(195.39860617,213.63054585)
\curveto(195.94381189,213.63054585)(196.38629769,213.18806005)(196.38629769,212.64285433)
\closepath
}
}
{
\newrgbcolor{curcolor}{0 0 0}
\pscustom[linestyle=none,fillstyle=solid,fillcolor=curcolor]
{
\newpath
\moveto(215.54858769,212.64285433)
\curveto(215.54858769,212.09764861)(215.10610189,211.65516281)(214.56089617,211.65516281)
\curveto(214.01569045,211.65516281)(213.57320465,212.09764861)(213.57320465,212.64285433)
\curveto(213.57320465,213.18806005)(214.01569045,213.63054585)(214.56089617,213.63054585)
\curveto(215.10610189,213.63054585)(215.54858769,213.18806005)(215.54858769,212.64285433)
\closepath
}
}
{
\newrgbcolor{curcolor}{0 0 0}
\pscustom[linewidth=0.24692288,linecolor=curcolor]
{
\newpath
\moveto(215.54858769,212.64285433)
\curveto(215.54858769,212.09764861)(215.10610189,211.65516281)(214.56089617,211.65516281)
\curveto(214.01569045,211.65516281)(213.57320465,212.09764861)(213.57320465,212.64285433)
\curveto(213.57320465,213.18806005)(214.01569045,213.63054585)(214.56089617,213.63054585)
\curveto(215.10610189,213.63054585)(215.54858769,213.18806005)(215.54858769,212.64285433)
\closepath
}
}
{
\newrgbcolor{curcolor}{0 0 0}
\pscustom[linewidth=0.98769152,linecolor=curcolor]
{
\newpath
\moveto(176.26004,208.69641433)
\lineto(195.42233,208.69641433)
\lineto(195.42233,208.69641433)
}
}
{
\newrgbcolor{curcolor}{0 0 0}
\pscustom[linestyle=none,fillstyle=solid,fillcolor=curcolor]
{
\newpath
\moveto(177.22797769,208.69641433)
\curveto(177.22797769,208.15120861)(176.78549189,207.70872281)(176.24028617,207.70872281)
\curveto(175.69508045,207.70872281)(175.25259465,208.15120861)(175.25259465,208.69641433)
\curveto(175.25259465,209.24162005)(175.69508045,209.68410585)(176.24028617,209.68410585)
\curveto(176.78549189,209.68410585)(177.22797769,209.24162005)(177.22797769,208.69641433)
\closepath
}
}
{
\newrgbcolor{curcolor}{0 0 0}
\pscustom[linewidth=0.24692288,linecolor=curcolor]
{
\newpath
\moveto(177.22797769,208.69641433)
\curveto(177.22797769,208.15120861)(176.78549189,207.70872281)(176.24028617,207.70872281)
\curveto(175.69508045,207.70872281)(175.25259465,208.15120861)(175.25259465,208.69641433)
\curveto(175.25259465,209.24162005)(175.69508045,209.68410585)(176.24028617,209.68410585)
\curveto(176.78549189,209.68410585)(177.22797769,209.24162005)(177.22797769,208.69641433)
\closepath
}
}
{
\newrgbcolor{curcolor}{0 0 0}
\pscustom[linestyle=none,fillstyle=solid,fillcolor=curcolor]
{
\newpath
\moveto(196.39026769,208.69641433)
\curveto(196.39026769,208.15120861)(195.94778189,207.70872281)(195.40257617,207.70872281)
\curveto(194.85737045,207.70872281)(194.41488465,208.15120861)(194.41488465,208.69641433)
\curveto(194.41488465,209.24162005)(194.85737045,209.68410585)(195.40257617,209.68410585)
\curveto(195.94778189,209.68410585)(196.39026769,209.24162005)(196.39026769,208.69641433)
\closepath
}
}
{
\newrgbcolor{curcolor}{0 0 0}
\pscustom[linewidth=0.24692288,linecolor=curcolor]
{
\newpath
\moveto(196.39026769,208.69641433)
\curveto(196.39026769,208.15120861)(195.94778189,207.70872281)(195.40257617,207.70872281)
\curveto(194.85737045,207.70872281)(194.41488465,208.15120861)(194.41488465,208.69641433)
\curveto(194.41488465,209.24162005)(194.85737045,209.68410585)(195.40257617,209.68410585)
\curveto(195.94778189,209.68410585)(196.39026769,209.24162005)(196.39026769,208.69641433)
\closepath
}
}
{
\newrgbcolor{curcolor}{0 0 0}
\pscustom[linewidth=0.98769152,linecolor=curcolor]
{
\newpath
\moveto(157.09933,204.78571433)
\lineto(176.26162,204.78571433)
\lineto(176.26162,204.78571433)
}
}
{
\newrgbcolor{curcolor}{0 0 0}
\pscustom[linestyle=none,fillstyle=solid,fillcolor=curcolor]
{
\newpath
\moveto(158.06726769,204.78571433)
\curveto(158.06726769,204.24050861)(157.62478189,203.79802281)(157.07957617,203.79802281)
\curveto(156.53437045,203.79802281)(156.09188465,204.24050861)(156.09188465,204.78571433)
\curveto(156.09188465,205.33092005)(156.53437045,205.77340585)(157.07957617,205.77340585)
\curveto(157.62478189,205.77340585)(158.06726769,205.33092005)(158.06726769,204.78571433)
\closepath
}
}
{
\newrgbcolor{curcolor}{0 0 0}
\pscustom[linewidth=0.24692288,linecolor=curcolor]
{
\newpath
\moveto(158.06726769,204.78571433)
\curveto(158.06726769,204.24050861)(157.62478189,203.79802281)(157.07957617,203.79802281)
\curveto(156.53437045,203.79802281)(156.09188465,204.24050861)(156.09188465,204.78571433)
\curveto(156.09188465,205.33092005)(156.53437045,205.77340585)(157.07957617,205.77340585)
\curveto(157.62478189,205.77340585)(158.06726769,205.33092005)(158.06726769,204.78571433)
\closepath
}
}
{
\newrgbcolor{curcolor}{0 0 0}
\pscustom[linestyle=none,fillstyle=solid,fillcolor=curcolor]
{
\newpath
\moveto(177.22955769,204.78571433)
\curveto(177.22955769,204.24050861)(176.78707189,203.79802281)(176.24186617,203.79802281)
\curveto(175.69666045,203.79802281)(175.25417465,204.24050861)(175.25417465,204.78571433)
\curveto(175.25417465,205.33092005)(175.69666045,205.77340585)(176.24186617,205.77340585)
\curveto(176.78707189,205.77340585)(177.22955769,205.33092005)(177.22955769,204.78571433)
\closepath
}
}
{
\newrgbcolor{curcolor}{0 0 0}
\pscustom[linewidth=0.24692288,linecolor=curcolor]
{
\newpath
\moveto(177.22955769,204.78571433)
\curveto(177.22955769,204.24050861)(176.78707189,203.79802281)(176.24186617,203.79802281)
\curveto(175.69666045,203.79802281)(175.25417465,204.24050861)(175.25417465,204.78571433)
\curveto(175.25417465,205.33092005)(175.69666045,205.77340585)(176.24186617,205.77340585)
\curveto(176.78707189,205.77340585)(177.22955769,205.33092005)(177.22955769,204.78571433)
\closepath
}
}
{
\newrgbcolor{curcolor}{0 0 0}
\pscustom[linewidth=0.98769152,linecolor=curcolor]
{
\newpath
\moveto(137.9029,200.69643433)
\lineto(157.06519,200.69643433)
\lineto(157.06519,200.69643433)
}
}
{
\newrgbcolor{curcolor}{0 0 0}
\pscustom[linestyle=none,fillstyle=solid,fillcolor=curcolor]
{
\newpath
\moveto(138.87083769,200.69643433)
\curveto(138.87083769,200.15122861)(138.42835189,199.70874281)(137.88314617,199.70874281)
\curveto(137.33794045,199.70874281)(136.89545465,200.15122861)(136.89545465,200.69643433)
\curveto(136.89545465,201.24164005)(137.33794045,201.68412585)(137.88314617,201.68412585)
\curveto(138.42835189,201.68412585)(138.87083769,201.24164005)(138.87083769,200.69643433)
\closepath
}
}
{
\newrgbcolor{curcolor}{0 0 0}
\pscustom[linewidth=0.24692288,linecolor=curcolor]
{
\newpath
\moveto(138.87083769,200.69643433)
\curveto(138.87083769,200.15122861)(138.42835189,199.70874281)(137.88314617,199.70874281)
\curveto(137.33794045,199.70874281)(136.89545465,200.15122861)(136.89545465,200.69643433)
\curveto(136.89545465,201.24164005)(137.33794045,201.68412585)(137.88314617,201.68412585)
\curveto(138.42835189,201.68412585)(138.87083769,201.24164005)(138.87083769,200.69643433)
\closepath
}
}
{
\newrgbcolor{curcolor}{0 0 0}
\pscustom[linestyle=none,fillstyle=solid,fillcolor=curcolor]
{
\newpath
\moveto(158.03312769,200.69643433)
\curveto(158.03312769,200.15122861)(157.59064189,199.70874281)(157.04543617,199.70874281)
\curveto(156.50023045,199.70874281)(156.05774465,200.15122861)(156.05774465,200.69643433)
\curveto(156.05774465,201.24164005)(156.50023045,201.68412585)(157.04543617,201.68412585)
\curveto(157.59064189,201.68412585)(158.03312769,201.24164005)(158.03312769,200.69643433)
\closepath
}
}
{
\newrgbcolor{curcolor}{0 0 0}
\pscustom[linewidth=0.24692288,linecolor=curcolor]
{
\newpath
\moveto(158.03312769,200.69643433)
\curveto(158.03312769,200.15122861)(157.59064189,199.70874281)(157.04543617,199.70874281)
\curveto(156.50023045,199.70874281)(156.05774465,200.15122861)(156.05774465,200.69643433)
\curveto(156.05774465,201.24164005)(156.50023045,201.68412585)(157.04543617,201.68412585)
\curveto(157.59064189,201.68412585)(158.03312769,201.24164005)(158.03312769,200.69643433)
\closepath
}
}
{
\newrgbcolor{curcolor}{0 0 0}
\pscustom[linewidth=0.98769152,linecolor=curcolor]
{
\newpath
\moveto(118.70647,196.78571433)
\lineto(137.86876,196.78571433)
\lineto(137.86876,196.78571433)
}
}
{
\newrgbcolor{curcolor}{0 0 0}
\pscustom[linestyle=none,fillstyle=solid,fillcolor=curcolor]
{
\newpath
\moveto(119.67440769,196.78571433)
\curveto(119.67440769,196.24050861)(119.23192189,195.79802281)(118.68671617,195.79802281)
\curveto(118.14151045,195.79802281)(117.69902465,196.24050861)(117.69902465,196.78571433)
\curveto(117.69902465,197.33092005)(118.14151045,197.77340585)(118.68671617,197.77340585)
\curveto(119.23192189,197.77340585)(119.67440769,197.33092005)(119.67440769,196.78571433)
\closepath
}
}
{
\newrgbcolor{curcolor}{0 0 0}
\pscustom[linewidth=0.24692288,linecolor=curcolor]
{
\newpath
\moveto(119.67440769,196.78571433)
\curveto(119.67440769,196.24050861)(119.23192189,195.79802281)(118.68671617,195.79802281)
\curveto(118.14151045,195.79802281)(117.69902465,196.24050861)(117.69902465,196.78571433)
\curveto(117.69902465,197.33092005)(118.14151045,197.77340585)(118.68671617,197.77340585)
\curveto(119.23192189,197.77340585)(119.67440769,197.33092005)(119.67440769,196.78571433)
\closepath
}
}
{
\newrgbcolor{curcolor}{0 0 0}
\pscustom[linestyle=none,fillstyle=solid,fillcolor=curcolor]
{
\newpath
\moveto(138.83669769,196.78571433)
\curveto(138.83669769,196.24050861)(138.39421189,195.79802281)(137.84900617,195.79802281)
\curveto(137.30380045,195.79802281)(136.86131465,196.24050861)(136.86131465,196.78571433)
\curveto(136.86131465,197.33092005)(137.30380045,197.77340585)(137.84900617,197.77340585)
\curveto(138.39421189,197.77340585)(138.83669769,197.33092005)(138.83669769,196.78571433)
\closepath
}
}
{
\newrgbcolor{curcolor}{0 0 0}
\pscustom[linewidth=0.24692288,linecolor=curcolor]
{
\newpath
\moveto(138.83669769,196.78571433)
\curveto(138.83669769,196.24050861)(138.39421189,195.79802281)(137.84900617,195.79802281)
\curveto(137.30380045,195.79802281)(136.86131465,196.24050861)(136.86131465,196.78571433)
\curveto(136.86131465,197.33092005)(137.30380045,197.77340585)(137.84900617,197.77340585)
\curveto(138.39421189,197.77340585)(138.83669769,197.33092005)(138.83669769,196.78571433)
\closepath
}
}
{
\newrgbcolor{curcolor}{0 0 0}
\pscustom[linewidth=0.98769152,linecolor=curcolor]
{
\newpath
\moveto(99.61718,192.82142433)
\lineto(118.77947,192.82142433)
\lineto(118.77947,192.82142433)
}
}
{
\newrgbcolor{curcolor}{0 0 0}
\pscustom[linestyle=none,fillstyle=solid,fillcolor=curcolor]
{
\newpath
\moveto(100.58511769,192.82142433)
\curveto(100.58511769,192.27621861)(100.14263189,191.83373281)(99.59742617,191.83373281)
\curveto(99.05222045,191.83373281)(98.60973465,192.27621861)(98.60973465,192.82142433)
\curveto(98.60973465,193.36663005)(99.05222045,193.80911585)(99.59742617,193.80911585)
\curveto(100.14263189,193.80911585)(100.58511769,193.36663005)(100.58511769,192.82142433)
\closepath
}
}
{
\newrgbcolor{curcolor}{0 0 0}
\pscustom[linewidth=0.24692288,linecolor=curcolor]
{
\newpath
\moveto(100.58511769,192.82142433)
\curveto(100.58511769,192.27621861)(100.14263189,191.83373281)(99.59742617,191.83373281)
\curveto(99.05222045,191.83373281)(98.60973465,192.27621861)(98.60973465,192.82142433)
\curveto(98.60973465,193.36663005)(99.05222045,193.80911585)(99.59742617,193.80911585)
\curveto(100.14263189,193.80911585)(100.58511769,193.36663005)(100.58511769,192.82142433)
\closepath
}
}
{
\newrgbcolor{curcolor}{0 0 0}
\pscustom[linestyle=none,fillstyle=solid,fillcolor=curcolor]
{
\newpath
\moveto(119.74740769,192.82142433)
\curveto(119.74740769,192.27621861)(119.30492189,191.83373281)(118.75971617,191.83373281)
\curveto(118.21451045,191.83373281)(117.77202465,192.27621861)(117.77202465,192.82142433)
\curveto(117.77202465,193.36663005)(118.21451045,193.80911585)(118.75971617,193.80911585)
\curveto(119.30492189,193.80911585)(119.74740769,193.36663005)(119.74740769,192.82142433)
\closepath
}
}
{
\newrgbcolor{curcolor}{0 0 0}
\pscustom[linewidth=0.24692288,linecolor=curcolor]
{
\newpath
\moveto(119.74740769,192.82142433)
\curveto(119.74740769,192.27621861)(119.30492189,191.83373281)(118.75971617,191.83373281)
\curveto(118.21451045,191.83373281)(117.77202465,192.27621861)(117.77202465,192.82142433)
\curveto(117.77202465,193.36663005)(118.21451045,193.80911585)(118.75971617,193.80911585)
\curveto(119.30492189,193.80911585)(119.74740769,193.36663005)(119.74740769,192.82142433)
\closepath
}
}
{
\newrgbcolor{curcolor}{0 0 0}
\pscustom[linewidth=0.98769152,linecolor=curcolor]
{
\newpath
\moveto(80.402895,188.80357433)
\lineto(99.56519,188.80357433)
\lineto(99.56519,188.80357433)
}
}
{
\newrgbcolor{curcolor}{0 0 0}
\pscustom[linestyle=none,fillstyle=solid,fillcolor=curcolor]
{
\newpath
\moveto(81.37083269,188.80357433)
\curveto(81.37083269,188.25836861)(80.92834689,187.81588281)(80.38314117,187.81588281)
\curveto(79.83793545,187.81588281)(79.39544965,188.25836861)(79.39544965,188.80357433)
\curveto(79.39544965,189.34878005)(79.83793545,189.79126585)(80.38314117,189.79126585)
\curveto(80.92834689,189.79126585)(81.37083269,189.34878005)(81.37083269,188.80357433)
\closepath
}
}
{
\newrgbcolor{curcolor}{0 0 0}
\pscustom[linewidth=0.24692288,linecolor=curcolor]
{
\newpath
\moveto(81.37083269,188.80357433)
\curveto(81.37083269,188.25836861)(80.92834689,187.81588281)(80.38314117,187.81588281)
\curveto(79.83793545,187.81588281)(79.39544965,188.25836861)(79.39544965,188.80357433)
\curveto(79.39544965,189.34878005)(79.83793545,189.79126585)(80.38314117,189.79126585)
\curveto(80.92834689,189.79126585)(81.37083269,189.34878005)(81.37083269,188.80357433)
\closepath
}
}
{
\newrgbcolor{curcolor}{0 0 0}
\pscustom[linestyle=none,fillstyle=solid,fillcolor=curcolor]
{
\newpath
\moveto(100.53312769,188.80357433)
\curveto(100.53312769,188.25836861)(100.09064189,187.81588281)(99.54543617,187.81588281)
\curveto(99.00023045,187.81588281)(98.55774465,188.25836861)(98.55774465,188.80357433)
\curveto(98.55774465,189.34878005)(99.00023045,189.79126585)(99.54543617,189.79126585)
\curveto(100.09064189,189.79126585)(100.53312769,189.34878005)(100.53312769,188.80357433)
\closepath
}
}
{
\newrgbcolor{curcolor}{0 0 0}
\pscustom[linewidth=0.24692288,linecolor=curcolor]
{
\newpath
\moveto(100.53312769,188.80357433)
\curveto(100.53312769,188.25836861)(100.09064189,187.81588281)(99.54543617,187.81588281)
\curveto(99.00023045,187.81588281)(98.55774465,188.25836861)(98.55774465,188.80357433)
\curveto(98.55774465,189.34878005)(99.00023045,189.79126585)(99.54543617,189.79126585)
\curveto(100.09064189,189.79126585)(100.53312769,189.34878005)(100.53312769,188.80357433)
\closepath
}
}
{
\newrgbcolor{curcolor}{0 0 0}
\pscustom[linewidth=0.98769152,linecolor=curcolor]
{
\newpath
\moveto(61.11718,184.83928433)
\lineto(80.27947,184.83928433)
\lineto(80.27947,184.83928433)
}
}
{
\newrgbcolor{curcolor}{0 0 0}
\pscustom[linestyle=none,fillstyle=solid,fillcolor=curcolor]
{
\newpath
\moveto(62.08511769,184.83928433)
\curveto(62.08511769,184.29407861)(61.64263189,183.85159281)(61.09742617,183.85159281)
\curveto(60.55222045,183.85159281)(60.10973465,184.29407861)(60.10973465,184.83928433)
\curveto(60.10973465,185.38449005)(60.55222045,185.82697585)(61.09742617,185.82697585)
\curveto(61.64263189,185.82697585)(62.08511769,185.38449005)(62.08511769,184.83928433)
\closepath
}
}
{
\newrgbcolor{curcolor}{0 0 0}
\pscustom[linewidth=0.24692288,linecolor=curcolor]
{
\newpath
\moveto(62.08511769,184.83928433)
\curveto(62.08511769,184.29407861)(61.64263189,183.85159281)(61.09742617,183.85159281)
\curveto(60.55222045,183.85159281)(60.10973465,184.29407861)(60.10973465,184.83928433)
\curveto(60.10973465,185.38449005)(60.55222045,185.82697585)(61.09742617,185.82697585)
\curveto(61.64263189,185.82697585)(62.08511769,185.38449005)(62.08511769,184.83928433)
\closepath
}
}
{
\newrgbcolor{curcolor}{0 0 0}
\pscustom[linestyle=none,fillstyle=solid,fillcolor=curcolor]
{
\newpath
\moveto(81.24740769,184.83928433)
\curveto(81.24740769,184.29407861)(80.80492189,183.85159281)(80.25971617,183.85159281)
\curveto(79.71451045,183.85159281)(79.27202465,184.29407861)(79.27202465,184.83928433)
\curveto(79.27202465,185.38449005)(79.71451045,185.82697585)(80.25971617,185.82697585)
\curveto(80.80492189,185.82697585)(81.24740769,185.38449005)(81.24740769,184.83928433)
\closepath
}
}
{
\newrgbcolor{curcolor}{0 0 0}
\pscustom[linewidth=0.24692288,linecolor=curcolor]
{
\newpath
\moveto(81.24740769,184.83928433)
\curveto(81.24740769,184.29407861)(80.80492189,183.85159281)(80.25971617,183.85159281)
\curveto(79.71451045,183.85159281)(79.27202465,184.29407861)(79.27202465,184.83928433)
\curveto(79.27202465,185.38449005)(79.71451045,185.82697585)(80.25971617,185.82697585)
\curveto(80.80492189,185.82697585)(81.24740769,185.38449005)(81.24740769,184.83928433)
\closepath
}
}
{
\newrgbcolor{curcolor}{0 0 0}
\pscustom[linewidth=1,linecolor=curcolor,linestyle=dashed,dash=1 4]
{
\newpath
\moveto(214.64286,216.42856433)
\lineto(269.64286,216.42856433)
\lineto(269.64286,216.42856433)
\lineto(269.64286,216.42856433)
}
}
{
\newrgbcolor{curcolor}{0 0 0}
\pscustom[linestyle=none,fillstyle=solid,fillcolor=curcolor]
{
\newpath
\moveto(215.62286,216.42856433)
\curveto(215.62286,215.87656433)(215.17486,215.42856433)(214.62286,215.42856433)
\curveto(214.07086,215.42856433)(213.62286,215.87656433)(213.62286,216.42856433)
\curveto(213.62286,216.98056433)(214.07086,217.42856433)(214.62286,217.42856433)
\curveto(215.17486,217.42856433)(215.62286,216.98056433)(215.62286,216.42856433)
\closepath
}
}
{
\newrgbcolor{curcolor}{0 0 0}
\pscustom[linewidth=0.25,linecolor=curcolor]
{
\newpath
\moveto(215.62286,216.42856433)
\curveto(215.62286,215.87656433)(215.17486,215.42856433)(214.62286,215.42856433)
\curveto(214.07086,215.42856433)(213.62286,215.87656433)(213.62286,216.42856433)
\curveto(213.62286,216.98056433)(214.07086,217.42856433)(214.62286,217.42856433)
\curveto(215.17486,217.42856433)(215.62286,216.98056433)(215.62286,216.42856433)
\closepath
}
}
{
\newrgbcolor{curcolor}{0 0 0}
\pscustom[linestyle=none,fillstyle=solid,fillcolor=curcolor]
{
\newpath
\moveto(270.62286,216.42856433)
\curveto(270.62286,215.87656433)(270.17486,215.42856433)(269.62286,215.42856433)
\curveto(269.07086,215.42856433)(268.62286,215.87656433)(268.62286,216.42856433)
\curveto(268.62286,216.98056433)(269.07086,217.42856433)(269.62286,217.42856433)
\curveto(270.17486,217.42856433)(270.62286,216.98056433)(270.62286,216.42856433)
\closepath
}
}
{
\newrgbcolor{curcolor}{0 0 0}
\pscustom[linewidth=0.25,linecolor=curcolor]
{
\newpath
\moveto(270.62286,216.42856433)
\curveto(270.62286,215.87656433)(270.17486,215.42856433)(269.62286,215.42856433)
\curveto(269.07086,215.42856433)(268.62286,215.87656433)(268.62286,216.42856433)
\curveto(268.62286,216.98056433)(269.07086,217.42856433)(269.62286,217.42856433)
\curveto(270.17486,217.42856433)(270.62286,216.98056433)(270.62286,216.42856433)
\closepath
}
}
{
\newrgbcolor{curcolor}{0 0 0}
\pscustom[linewidth=0.98769152,linecolor=curcolor]
{
\newpath
\moveto(195.49451,177.53092433)
\lineto(214.6568,177.53092433)
\lineto(214.6568,177.53092433)
}
}
{
\newrgbcolor{curcolor}{0 0 0}
\pscustom[linestyle=none,fillstyle=solid,fillcolor=curcolor]
{
\newpath
\moveto(196.46244769,177.53092433)
\curveto(196.46244769,176.98571861)(196.01996189,176.54323281)(195.47475617,176.54323281)
\curveto(194.92955045,176.54323281)(194.48706465,176.98571861)(194.48706465,177.53092433)
\curveto(194.48706465,178.07613005)(194.92955045,178.51861585)(195.47475617,178.51861585)
\curveto(196.01996189,178.51861585)(196.46244769,178.07613005)(196.46244769,177.53092433)
\closepath
}
}
{
\newrgbcolor{curcolor}{0 0 0}
\pscustom[linewidth=0.24692288,linecolor=curcolor]
{
\newpath
\moveto(196.46244769,177.53092433)
\curveto(196.46244769,176.98571861)(196.01996189,176.54323281)(195.47475617,176.54323281)
\curveto(194.92955045,176.54323281)(194.48706465,176.98571861)(194.48706465,177.53092433)
\curveto(194.48706465,178.07613005)(194.92955045,178.51861585)(195.47475617,178.51861585)
\curveto(196.01996189,178.51861585)(196.46244769,178.07613005)(196.46244769,177.53092433)
\closepath
}
}
{
\newrgbcolor{curcolor}{0 0 0}
\pscustom[linestyle=none,fillstyle=solid,fillcolor=curcolor]
{
\newpath
\moveto(215.62473769,177.53092433)
\curveto(215.62473769,176.98571861)(215.18225189,176.54323281)(214.63704617,176.54323281)
\curveto(214.09184045,176.54323281)(213.64935465,176.98571861)(213.64935465,177.53092433)
\curveto(213.64935465,178.07613005)(214.09184045,178.51861585)(214.63704617,178.51861585)
\curveto(215.18225189,178.51861585)(215.62473769,178.07613005)(215.62473769,177.53092433)
\closepath
}
}
{
\newrgbcolor{curcolor}{0 0 0}
\pscustom[linewidth=0.24692288,linecolor=curcolor]
{
\newpath
\moveto(215.62473769,177.53092433)
\curveto(215.62473769,176.98571861)(215.18225189,176.54323281)(214.63704617,176.54323281)
\curveto(214.09184045,176.54323281)(213.64935465,176.98571861)(213.64935465,177.53092433)
\curveto(213.64935465,178.07613005)(214.09184045,178.51861585)(214.63704617,178.51861585)
\curveto(215.18225189,178.51861585)(215.62473769,178.07613005)(215.62473769,177.53092433)
\closepath
}
}
{
\newrgbcolor{curcolor}{0 0 0}
\pscustom[linewidth=0.98769152,linecolor=curcolor]
{
\newpath
\moveto(176.33619,173.58448433)
\lineto(195.49848,173.58448433)
\lineto(195.49848,173.58448433)
}
}
{
\newrgbcolor{curcolor}{0 0 0}
\pscustom[linestyle=none,fillstyle=solid,fillcolor=curcolor]
{
\newpath
\moveto(177.30412769,173.58448433)
\curveto(177.30412769,173.03927861)(176.86164189,172.59679281)(176.31643617,172.59679281)
\curveto(175.77123045,172.59679281)(175.32874465,173.03927861)(175.32874465,173.58448433)
\curveto(175.32874465,174.12969005)(175.77123045,174.57217585)(176.31643617,174.57217585)
\curveto(176.86164189,174.57217585)(177.30412769,174.12969005)(177.30412769,173.58448433)
\closepath
}
}
{
\newrgbcolor{curcolor}{0 0 0}
\pscustom[linewidth=0.24692288,linecolor=curcolor]
{
\newpath
\moveto(177.30412769,173.58448433)
\curveto(177.30412769,173.03927861)(176.86164189,172.59679281)(176.31643617,172.59679281)
\curveto(175.77123045,172.59679281)(175.32874465,173.03927861)(175.32874465,173.58448433)
\curveto(175.32874465,174.12969005)(175.77123045,174.57217585)(176.31643617,174.57217585)
\curveto(176.86164189,174.57217585)(177.30412769,174.12969005)(177.30412769,173.58448433)
\closepath
}
}
{
\newrgbcolor{curcolor}{0 0 0}
\pscustom[linestyle=none,fillstyle=solid,fillcolor=curcolor]
{
\newpath
\moveto(196.46641769,173.58448433)
\curveto(196.46641769,173.03927861)(196.02393189,172.59679281)(195.47872617,172.59679281)
\curveto(194.93352045,172.59679281)(194.49103465,173.03927861)(194.49103465,173.58448433)
\curveto(194.49103465,174.12969005)(194.93352045,174.57217585)(195.47872617,174.57217585)
\curveto(196.02393189,174.57217585)(196.46641769,174.12969005)(196.46641769,173.58448433)
\closepath
}
}
{
\newrgbcolor{curcolor}{0 0 0}
\pscustom[linewidth=0.24692288,linecolor=curcolor]
{
\newpath
\moveto(196.46641769,173.58448433)
\curveto(196.46641769,173.03927861)(196.02393189,172.59679281)(195.47872617,172.59679281)
\curveto(194.93352045,172.59679281)(194.49103465,173.03927861)(194.49103465,173.58448433)
\curveto(194.49103465,174.12969005)(194.93352045,174.57217585)(195.47872617,174.57217585)
\curveto(196.02393189,174.57217585)(196.46641769,174.12969005)(196.46641769,173.58448433)
\closepath
}
}
{
\newrgbcolor{curcolor}{0 0 0}
\pscustom[linewidth=0.98769152,linecolor=curcolor]
{
\newpath
\moveto(157.17548,169.67378433)
\lineto(176.33777,169.67378433)
\lineto(176.33777,169.67378433)
}
}
{
\newrgbcolor{curcolor}{0 0 0}
\pscustom[linestyle=none,fillstyle=solid,fillcolor=curcolor]
{
\newpath
\moveto(158.14341769,169.67378433)
\curveto(158.14341769,169.12857861)(157.70093189,168.68609281)(157.15572617,168.68609281)
\curveto(156.61052045,168.68609281)(156.16803465,169.12857861)(156.16803465,169.67378433)
\curveto(156.16803465,170.21899005)(156.61052045,170.66147585)(157.15572617,170.66147585)
\curveto(157.70093189,170.66147585)(158.14341769,170.21899005)(158.14341769,169.67378433)
\closepath
}
}
{
\newrgbcolor{curcolor}{0 0 0}
\pscustom[linewidth=0.24692288,linecolor=curcolor]
{
\newpath
\moveto(158.14341769,169.67378433)
\curveto(158.14341769,169.12857861)(157.70093189,168.68609281)(157.15572617,168.68609281)
\curveto(156.61052045,168.68609281)(156.16803465,169.12857861)(156.16803465,169.67378433)
\curveto(156.16803465,170.21899005)(156.61052045,170.66147585)(157.15572617,170.66147585)
\curveto(157.70093189,170.66147585)(158.14341769,170.21899005)(158.14341769,169.67378433)
\closepath
}
}
{
\newrgbcolor{curcolor}{0 0 0}
\pscustom[linestyle=none,fillstyle=solid,fillcolor=curcolor]
{
\newpath
\moveto(177.30570769,169.67378433)
\curveto(177.30570769,169.12857861)(176.86322189,168.68609281)(176.31801617,168.68609281)
\curveto(175.77281045,168.68609281)(175.33032465,169.12857861)(175.33032465,169.67378433)
\curveto(175.33032465,170.21899005)(175.77281045,170.66147585)(176.31801617,170.66147585)
\curveto(176.86322189,170.66147585)(177.30570769,170.21899005)(177.30570769,169.67378433)
\closepath
}
}
{
\newrgbcolor{curcolor}{0 0 0}
\pscustom[linewidth=0.24692288,linecolor=curcolor]
{
\newpath
\moveto(177.30570769,169.67378433)
\curveto(177.30570769,169.12857861)(176.86322189,168.68609281)(176.31801617,168.68609281)
\curveto(175.77281045,168.68609281)(175.33032465,169.12857861)(175.33032465,169.67378433)
\curveto(175.33032465,170.21899005)(175.77281045,170.66147585)(176.31801617,170.66147585)
\curveto(176.86322189,170.66147585)(177.30570769,170.21899005)(177.30570769,169.67378433)
\closepath
}
}
{
\newrgbcolor{curcolor}{0 0 0}
\pscustom[linewidth=0.98769152,linecolor=curcolor]
{
\newpath
\moveto(137.97905,165.58450433)
\lineto(157.14134,165.58450433)
\lineto(157.14134,165.58450433)
}
}
{
\newrgbcolor{curcolor}{0 0 0}
\pscustom[linestyle=none,fillstyle=solid,fillcolor=curcolor]
{
\newpath
\moveto(138.94698769,165.58450433)
\curveto(138.94698769,165.03929861)(138.50450189,164.59681281)(137.95929617,164.59681281)
\curveto(137.41409045,164.59681281)(136.97160465,165.03929861)(136.97160465,165.58450433)
\curveto(136.97160465,166.12971005)(137.41409045,166.57219585)(137.95929617,166.57219585)
\curveto(138.50450189,166.57219585)(138.94698769,166.12971005)(138.94698769,165.58450433)
\closepath
}
}
{
\newrgbcolor{curcolor}{0 0 0}
\pscustom[linewidth=0.24692288,linecolor=curcolor]
{
\newpath
\moveto(138.94698769,165.58450433)
\curveto(138.94698769,165.03929861)(138.50450189,164.59681281)(137.95929617,164.59681281)
\curveto(137.41409045,164.59681281)(136.97160465,165.03929861)(136.97160465,165.58450433)
\curveto(136.97160465,166.12971005)(137.41409045,166.57219585)(137.95929617,166.57219585)
\curveto(138.50450189,166.57219585)(138.94698769,166.12971005)(138.94698769,165.58450433)
\closepath
}
}
{
\newrgbcolor{curcolor}{0 0 0}
\pscustom[linestyle=none,fillstyle=solid,fillcolor=curcolor]
{
\newpath
\moveto(158.10927769,165.58450433)
\curveto(158.10927769,165.03929861)(157.66679189,164.59681281)(157.12158617,164.59681281)
\curveto(156.57638045,164.59681281)(156.13389465,165.03929861)(156.13389465,165.58450433)
\curveto(156.13389465,166.12971005)(156.57638045,166.57219585)(157.12158617,166.57219585)
\curveto(157.66679189,166.57219585)(158.10927769,166.12971005)(158.10927769,165.58450433)
\closepath
}
}
{
\newrgbcolor{curcolor}{0 0 0}
\pscustom[linewidth=0.24692288,linecolor=curcolor]
{
\newpath
\moveto(158.10927769,165.58450433)
\curveto(158.10927769,165.03929861)(157.66679189,164.59681281)(157.12158617,164.59681281)
\curveto(156.57638045,164.59681281)(156.13389465,165.03929861)(156.13389465,165.58450433)
\curveto(156.13389465,166.12971005)(156.57638045,166.57219585)(157.12158617,166.57219585)
\curveto(157.66679189,166.57219585)(158.10927769,166.12971005)(158.10927769,165.58450433)
\closepath
}
}
{
\newrgbcolor{curcolor}{0 0 0}
\pscustom[linewidth=0.98769152,linecolor=curcolor]
{
\newpath
\moveto(118.78262,161.67378433)
\lineto(137.94491,161.67378433)
\lineto(137.94491,161.67378433)
}
}
{
\newrgbcolor{curcolor}{0 0 0}
\pscustom[linestyle=none,fillstyle=solid,fillcolor=curcolor]
{
\newpath
\moveto(119.75055769,161.67378433)
\curveto(119.75055769,161.12857861)(119.30807189,160.68609281)(118.76286617,160.68609281)
\curveto(118.21766045,160.68609281)(117.77517465,161.12857861)(117.77517465,161.67378433)
\curveto(117.77517465,162.21899005)(118.21766045,162.66147585)(118.76286617,162.66147585)
\curveto(119.30807189,162.66147585)(119.75055769,162.21899005)(119.75055769,161.67378433)
\closepath
}
}
{
\newrgbcolor{curcolor}{0 0 0}
\pscustom[linewidth=0.24692288,linecolor=curcolor]
{
\newpath
\moveto(119.75055769,161.67378433)
\curveto(119.75055769,161.12857861)(119.30807189,160.68609281)(118.76286617,160.68609281)
\curveto(118.21766045,160.68609281)(117.77517465,161.12857861)(117.77517465,161.67378433)
\curveto(117.77517465,162.21899005)(118.21766045,162.66147585)(118.76286617,162.66147585)
\curveto(119.30807189,162.66147585)(119.75055769,162.21899005)(119.75055769,161.67378433)
\closepath
}
}
{
\newrgbcolor{curcolor}{0 0 0}
\pscustom[linestyle=none,fillstyle=solid,fillcolor=curcolor]
{
\newpath
\moveto(138.91284769,161.67378433)
\curveto(138.91284769,161.12857861)(138.47036189,160.68609281)(137.92515617,160.68609281)
\curveto(137.37995045,160.68609281)(136.93746465,161.12857861)(136.93746465,161.67378433)
\curveto(136.93746465,162.21899005)(137.37995045,162.66147585)(137.92515617,162.66147585)
\curveto(138.47036189,162.66147585)(138.91284769,162.21899005)(138.91284769,161.67378433)
\closepath
}
}
{
\newrgbcolor{curcolor}{0 0 0}
\pscustom[linewidth=0.24692288,linecolor=curcolor]
{
\newpath
\moveto(138.91284769,161.67378433)
\curveto(138.91284769,161.12857861)(138.47036189,160.68609281)(137.92515617,160.68609281)
\curveto(137.37995045,160.68609281)(136.93746465,161.12857861)(136.93746465,161.67378433)
\curveto(136.93746465,162.21899005)(137.37995045,162.66147585)(137.92515617,162.66147585)
\curveto(138.47036189,162.66147585)(138.91284769,162.21899005)(138.91284769,161.67378433)
\closepath
}
}
{
\newrgbcolor{curcolor}{0 0 0}
\pscustom[linewidth=0.98769152,linecolor=curcolor]
{
\newpath
\moveto(99.69333,157.70949433)
\lineto(118.85562,157.70949433)
\lineto(118.85562,157.70949433)
}
}
{
\newrgbcolor{curcolor}{0 0 0}
\pscustom[linestyle=none,fillstyle=solid,fillcolor=curcolor]
{
\newpath
\moveto(100.66126769,157.70949433)
\curveto(100.66126769,157.16428861)(100.21878189,156.72180281)(99.67357617,156.72180281)
\curveto(99.12837045,156.72180281)(98.68588465,157.16428861)(98.68588465,157.70949433)
\curveto(98.68588465,158.25470005)(99.12837045,158.69718585)(99.67357617,158.69718585)
\curveto(100.21878189,158.69718585)(100.66126769,158.25470005)(100.66126769,157.70949433)
\closepath
}
}
{
\newrgbcolor{curcolor}{0 0 0}
\pscustom[linewidth=0.24692288,linecolor=curcolor]
{
\newpath
\moveto(100.66126769,157.70949433)
\curveto(100.66126769,157.16428861)(100.21878189,156.72180281)(99.67357617,156.72180281)
\curveto(99.12837045,156.72180281)(98.68588465,157.16428861)(98.68588465,157.70949433)
\curveto(98.68588465,158.25470005)(99.12837045,158.69718585)(99.67357617,158.69718585)
\curveto(100.21878189,158.69718585)(100.66126769,158.25470005)(100.66126769,157.70949433)
\closepath
}
}
{
\newrgbcolor{curcolor}{0 0 0}
\pscustom[linestyle=none,fillstyle=solid,fillcolor=curcolor]
{
\newpath
\moveto(119.82355769,157.70949433)
\curveto(119.82355769,157.16428861)(119.38107189,156.72180281)(118.83586617,156.72180281)
\curveto(118.29066045,156.72180281)(117.84817465,157.16428861)(117.84817465,157.70949433)
\curveto(117.84817465,158.25470005)(118.29066045,158.69718585)(118.83586617,158.69718585)
\curveto(119.38107189,158.69718585)(119.82355769,158.25470005)(119.82355769,157.70949433)
\closepath
}
}
{
\newrgbcolor{curcolor}{0 0 0}
\pscustom[linewidth=0.24692288,linecolor=curcolor]
{
\newpath
\moveto(119.82355769,157.70949433)
\curveto(119.82355769,157.16428861)(119.38107189,156.72180281)(118.83586617,156.72180281)
\curveto(118.29066045,156.72180281)(117.84817465,157.16428861)(117.84817465,157.70949433)
\curveto(117.84817465,158.25470005)(118.29066045,158.69718585)(118.83586617,158.69718585)
\curveto(119.38107189,158.69718585)(119.82355769,158.25470005)(119.82355769,157.70949433)
\closepath
}
}
{
\newrgbcolor{curcolor}{0 0 0}
\pscustom[linewidth=0.98769152,linecolor=curcolor]
{
\newpath
\moveto(80.479049,153.69164433)
\lineto(99.64134,153.69164433)
\lineto(99.64134,153.69164433)
}
}
{
\newrgbcolor{curcolor}{0 0 0}
\pscustom[linestyle=none,fillstyle=solid,fillcolor=curcolor]
{
\newpath
\moveto(81.44698669,153.69164433)
\curveto(81.44698669,153.14643861)(81.00450089,152.70395281)(80.45929517,152.70395281)
\curveto(79.91408945,152.70395281)(79.47160365,153.14643861)(79.47160365,153.69164433)
\curveto(79.47160365,154.23685005)(79.91408945,154.67933585)(80.45929517,154.67933585)
\curveto(81.00450089,154.67933585)(81.44698669,154.23685005)(81.44698669,153.69164433)
\closepath
}
}
{
\newrgbcolor{curcolor}{0 0 0}
\pscustom[linewidth=0.24692288,linecolor=curcolor]
{
\newpath
\moveto(81.44698669,153.69164433)
\curveto(81.44698669,153.14643861)(81.00450089,152.70395281)(80.45929517,152.70395281)
\curveto(79.91408945,152.70395281)(79.47160365,153.14643861)(79.47160365,153.69164433)
\curveto(79.47160365,154.23685005)(79.91408945,154.67933585)(80.45929517,154.67933585)
\curveto(81.00450089,154.67933585)(81.44698669,154.23685005)(81.44698669,153.69164433)
\closepath
}
}
{
\newrgbcolor{curcolor}{0 0 0}
\pscustom[linestyle=none,fillstyle=solid,fillcolor=curcolor]
{
\newpath
\moveto(100.60927769,153.69164433)
\curveto(100.60927769,153.14643861)(100.16679189,152.70395281)(99.62158617,152.70395281)
\curveto(99.07638045,152.70395281)(98.63389465,153.14643861)(98.63389465,153.69164433)
\curveto(98.63389465,154.23685005)(99.07638045,154.67933585)(99.62158617,154.67933585)
\curveto(100.16679189,154.67933585)(100.60927769,154.23685005)(100.60927769,153.69164433)
\closepath
}
}
{
\newrgbcolor{curcolor}{0 0 0}
\pscustom[linewidth=0.24692288,linecolor=curcolor]
{
\newpath
\moveto(100.60927769,153.69164433)
\curveto(100.60927769,153.14643861)(100.16679189,152.70395281)(99.62158617,152.70395281)
\curveto(99.07638045,152.70395281)(98.63389465,153.14643861)(98.63389465,153.69164433)
\curveto(98.63389465,154.23685005)(99.07638045,154.67933585)(99.62158617,154.67933585)
\curveto(100.16679189,154.67933585)(100.60927769,154.23685005)(100.60927769,153.69164433)
\closepath
}
}
{
\newrgbcolor{curcolor}{0 0 0}
\pscustom[linewidth=1,linecolor=curcolor,linestyle=dashed,dash=1 4]
{
\newpath
\moveto(214.69858,181.50000433)
\lineto(269.69858,181.50000433)
\lineto(269.69858,181.50000433)
\lineto(269.69858,181.50000433)
}
}
{
\newrgbcolor{curcolor}{0 0 0}
\pscustom[linestyle=none,fillstyle=solid,fillcolor=curcolor]
{
\newpath
\moveto(215.67858,181.50000433)
\curveto(215.67858,180.94800433)(215.23058,180.50000433)(214.67858,180.50000433)
\curveto(214.12658,180.50000433)(213.67858,180.94800433)(213.67858,181.50000433)
\curveto(213.67858,182.05200433)(214.12658,182.50000433)(214.67858,182.50000433)
\curveto(215.23058,182.50000433)(215.67858,182.05200433)(215.67858,181.50000433)
\closepath
}
}
{
\newrgbcolor{curcolor}{0 0 0}
\pscustom[linewidth=0.25,linecolor=curcolor]
{
\newpath
\moveto(215.67858,181.50000433)
\curveto(215.67858,180.94800433)(215.23058,180.50000433)(214.67858,180.50000433)
\curveto(214.12658,180.50000433)(213.67858,180.94800433)(213.67858,181.50000433)
\curveto(213.67858,182.05200433)(214.12658,182.50000433)(214.67858,182.50000433)
\curveto(215.23058,182.50000433)(215.67858,182.05200433)(215.67858,181.50000433)
\closepath
}
}
{
\newrgbcolor{curcolor}{0 0 0}
\pscustom[linestyle=none,fillstyle=solid,fillcolor=curcolor]
{
\newpath
\moveto(270.67858,181.50000433)
\curveto(270.67858,180.94800433)(270.23058,180.50000433)(269.67858,180.50000433)
\curveto(269.12658,180.50000433)(268.67858,180.94800433)(268.67858,181.50000433)
\curveto(268.67858,182.05200433)(269.12658,182.50000433)(269.67858,182.50000433)
\curveto(270.23058,182.50000433)(270.67858,182.05200433)(270.67858,181.50000433)
\closepath
}
}
{
\newrgbcolor{curcolor}{0 0 0}
\pscustom[linewidth=0.25,linecolor=curcolor]
{
\newpath
\moveto(270.67858,181.50000433)
\curveto(270.67858,180.94800433)(270.23058,180.50000433)(269.67858,180.50000433)
\curveto(269.12658,180.50000433)(268.67858,180.94800433)(268.67858,181.50000433)
\curveto(268.67858,182.05200433)(269.12658,182.50000433)(269.67858,182.50000433)
\curveto(270.23058,182.50000433)(270.67858,182.05200433)(270.67858,181.50000433)
\closepath
}
}
{
\newrgbcolor{curcolor}{0 0 0}
\pscustom[linewidth=0.98769152,linecolor=curcolor]
{
\newpath
\moveto(195.31594,148.20949433)
\lineto(214.47823,148.20949433)
\lineto(214.47823,148.20949433)
}
}
{
\newrgbcolor{curcolor}{0 0 0}
\pscustom[linestyle=none,fillstyle=solid,fillcolor=curcolor]
{
\newpath
\moveto(196.28387769,148.20949433)
\curveto(196.28387769,147.66428861)(195.84139189,147.22180281)(195.29618617,147.22180281)
\curveto(194.75098045,147.22180281)(194.30849465,147.66428861)(194.30849465,148.20949433)
\curveto(194.30849465,148.75470005)(194.75098045,149.19718585)(195.29618617,149.19718585)
\curveto(195.84139189,149.19718585)(196.28387769,148.75470005)(196.28387769,148.20949433)
\closepath
}
}
{
\newrgbcolor{curcolor}{0 0 0}
\pscustom[linewidth=0.24692288,linecolor=curcolor]
{
\newpath
\moveto(196.28387769,148.20949433)
\curveto(196.28387769,147.66428861)(195.84139189,147.22180281)(195.29618617,147.22180281)
\curveto(194.75098045,147.22180281)(194.30849465,147.66428861)(194.30849465,148.20949433)
\curveto(194.30849465,148.75470005)(194.75098045,149.19718585)(195.29618617,149.19718585)
\curveto(195.84139189,149.19718585)(196.28387769,148.75470005)(196.28387769,148.20949433)
\closepath
}
}
{
\newrgbcolor{curcolor}{0 0 0}
\pscustom[linestyle=none,fillstyle=solid,fillcolor=curcolor]
{
\newpath
\moveto(215.44616769,148.20949433)
\curveto(215.44616769,147.66428861)(215.00368189,147.22180281)(214.45847617,147.22180281)
\curveto(213.91327045,147.22180281)(213.47078465,147.66428861)(213.47078465,148.20949433)
\curveto(213.47078465,148.75470005)(213.91327045,149.19718585)(214.45847617,149.19718585)
\curveto(215.00368189,149.19718585)(215.44616769,148.75470005)(215.44616769,148.20949433)
\closepath
}
}
{
\newrgbcolor{curcolor}{0 0 0}
\pscustom[linewidth=0.24692288,linecolor=curcolor]
{
\newpath
\moveto(215.44616769,148.20949433)
\curveto(215.44616769,147.66428861)(215.00368189,147.22180281)(214.45847617,147.22180281)
\curveto(213.91327045,147.22180281)(213.47078465,147.66428861)(213.47078465,148.20949433)
\curveto(213.47078465,148.75470005)(213.91327045,149.19718585)(214.45847617,149.19718585)
\curveto(215.00368189,149.19718585)(215.44616769,148.75470005)(215.44616769,148.20949433)
\closepath
}
}
{
\newrgbcolor{curcolor}{0 0 0}
\pscustom[linewidth=0.98769152,linecolor=curcolor]
{
\newpath
\moveto(176.15762,144.26305433)
\lineto(195.31991,144.26305433)
\lineto(195.31991,144.26305433)
}
}
{
\newrgbcolor{curcolor}{0 0 0}
\pscustom[linestyle=none,fillstyle=solid,fillcolor=curcolor]
{
\newpath
\moveto(177.12555769,144.26305433)
\curveto(177.12555769,143.71784861)(176.68307189,143.27536281)(176.13786617,143.27536281)
\curveto(175.59266045,143.27536281)(175.15017465,143.71784861)(175.15017465,144.26305433)
\curveto(175.15017465,144.80826005)(175.59266045,145.25074585)(176.13786617,145.25074585)
\curveto(176.68307189,145.25074585)(177.12555769,144.80826005)(177.12555769,144.26305433)
\closepath
}
}
{
\newrgbcolor{curcolor}{0 0 0}
\pscustom[linewidth=0.24692288,linecolor=curcolor]
{
\newpath
\moveto(177.12555769,144.26305433)
\curveto(177.12555769,143.71784861)(176.68307189,143.27536281)(176.13786617,143.27536281)
\curveto(175.59266045,143.27536281)(175.15017465,143.71784861)(175.15017465,144.26305433)
\curveto(175.15017465,144.80826005)(175.59266045,145.25074585)(176.13786617,145.25074585)
\curveto(176.68307189,145.25074585)(177.12555769,144.80826005)(177.12555769,144.26305433)
\closepath
}
}
{
\newrgbcolor{curcolor}{0 0 0}
\pscustom[linestyle=none,fillstyle=solid,fillcolor=curcolor]
{
\newpath
\moveto(196.28784769,144.26305433)
\curveto(196.28784769,143.71784861)(195.84536189,143.27536281)(195.30015617,143.27536281)
\curveto(194.75495045,143.27536281)(194.31246465,143.71784861)(194.31246465,144.26305433)
\curveto(194.31246465,144.80826005)(194.75495045,145.25074585)(195.30015617,145.25074585)
\curveto(195.84536189,145.25074585)(196.28784769,144.80826005)(196.28784769,144.26305433)
\closepath
}
}
{
\newrgbcolor{curcolor}{0 0 0}
\pscustom[linewidth=0.24692288,linecolor=curcolor]
{
\newpath
\moveto(196.28784769,144.26305433)
\curveto(196.28784769,143.71784861)(195.84536189,143.27536281)(195.30015617,143.27536281)
\curveto(194.75495045,143.27536281)(194.31246465,143.71784861)(194.31246465,144.26305433)
\curveto(194.31246465,144.80826005)(194.75495045,145.25074585)(195.30015617,145.25074585)
\curveto(195.84536189,145.25074585)(196.28784769,144.80826005)(196.28784769,144.26305433)
\closepath
}
}
{
\newrgbcolor{curcolor}{0 0 0}
\pscustom[linewidth=0.98769152,linecolor=curcolor]
{
\newpath
\moveto(156.99691,140.35235433)
\lineto(176.1592,140.35235433)
\lineto(176.1592,140.35235433)
}
}
{
\newrgbcolor{curcolor}{0 0 0}
\pscustom[linestyle=none,fillstyle=solid,fillcolor=curcolor]
{
\newpath
\moveto(157.96484769,140.35235433)
\curveto(157.96484769,139.80714861)(157.52236189,139.36466281)(156.97715617,139.36466281)
\curveto(156.43195045,139.36466281)(155.98946465,139.80714861)(155.98946465,140.35235433)
\curveto(155.98946465,140.89756005)(156.43195045,141.34004585)(156.97715617,141.34004585)
\curveto(157.52236189,141.34004585)(157.96484769,140.89756005)(157.96484769,140.35235433)
\closepath
}
}
{
\newrgbcolor{curcolor}{0 0 0}
\pscustom[linewidth=0.24692288,linecolor=curcolor]
{
\newpath
\moveto(157.96484769,140.35235433)
\curveto(157.96484769,139.80714861)(157.52236189,139.36466281)(156.97715617,139.36466281)
\curveto(156.43195045,139.36466281)(155.98946465,139.80714861)(155.98946465,140.35235433)
\curveto(155.98946465,140.89756005)(156.43195045,141.34004585)(156.97715617,141.34004585)
\curveto(157.52236189,141.34004585)(157.96484769,140.89756005)(157.96484769,140.35235433)
\closepath
}
}
{
\newrgbcolor{curcolor}{0 0 0}
\pscustom[linestyle=none,fillstyle=solid,fillcolor=curcolor]
{
\newpath
\moveto(177.12713769,140.35235433)
\curveto(177.12713769,139.80714861)(176.68465189,139.36466281)(176.13944617,139.36466281)
\curveto(175.59424045,139.36466281)(175.15175465,139.80714861)(175.15175465,140.35235433)
\curveto(175.15175465,140.89756005)(175.59424045,141.34004585)(176.13944617,141.34004585)
\curveto(176.68465189,141.34004585)(177.12713769,140.89756005)(177.12713769,140.35235433)
\closepath
}
}
{
\newrgbcolor{curcolor}{0 0 0}
\pscustom[linewidth=0.24692288,linecolor=curcolor]
{
\newpath
\moveto(177.12713769,140.35235433)
\curveto(177.12713769,139.80714861)(176.68465189,139.36466281)(176.13944617,139.36466281)
\curveto(175.59424045,139.36466281)(175.15175465,139.80714861)(175.15175465,140.35235433)
\curveto(175.15175465,140.89756005)(175.59424045,141.34004585)(176.13944617,141.34004585)
\curveto(176.68465189,141.34004585)(177.12713769,140.89756005)(177.12713769,140.35235433)
\closepath
}
}
{
\newrgbcolor{curcolor}{0 0 0}
\pscustom[linewidth=0.98769152,linecolor=curcolor]
{
\newpath
\moveto(137.80048,136.26307433)
\lineto(156.96277,136.26307433)
\lineto(156.96277,136.26307433)
}
}
{
\newrgbcolor{curcolor}{0 0 0}
\pscustom[linestyle=none,fillstyle=solid,fillcolor=curcolor]
{
\newpath
\moveto(138.76841769,136.26307433)
\curveto(138.76841769,135.71786861)(138.32593189,135.27538281)(137.78072617,135.27538281)
\curveto(137.23552045,135.27538281)(136.79303465,135.71786861)(136.79303465,136.26307433)
\curveto(136.79303465,136.80828005)(137.23552045,137.25076585)(137.78072617,137.25076585)
\curveto(138.32593189,137.25076585)(138.76841769,136.80828005)(138.76841769,136.26307433)
\closepath
}
}
{
\newrgbcolor{curcolor}{0 0 0}
\pscustom[linewidth=0.24692288,linecolor=curcolor]
{
\newpath
\moveto(138.76841769,136.26307433)
\curveto(138.76841769,135.71786861)(138.32593189,135.27538281)(137.78072617,135.27538281)
\curveto(137.23552045,135.27538281)(136.79303465,135.71786861)(136.79303465,136.26307433)
\curveto(136.79303465,136.80828005)(137.23552045,137.25076585)(137.78072617,137.25076585)
\curveto(138.32593189,137.25076585)(138.76841769,136.80828005)(138.76841769,136.26307433)
\closepath
}
}
{
\newrgbcolor{curcolor}{0 0 0}
\pscustom[linestyle=none,fillstyle=solid,fillcolor=curcolor]
{
\newpath
\moveto(157.93070769,136.26307433)
\curveto(157.93070769,135.71786861)(157.48822189,135.27538281)(156.94301617,135.27538281)
\curveto(156.39781045,135.27538281)(155.95532465,135.71786861)(155.95532465,136.26307433)
\curveto(155.95532465,136.80828005)(156.39781045,137.25076585)(156.94301617,137.25076585)
\curveto(157.48822189,137.25076585)(157.93070769,136.80828005)(157.93070769,136.26307433)
\closepath
}
}
{
\newrgbcolor{curcolor}{0 0 0}
\pscustom[linewidth=0.24692288,linecolor=curcolor]
{
\newpath
\moveto(157.93070769,136.26307433)
\curveto(157.93070769,135.71786861)(157.48822189,135.27538281)(156.94301617,135.27538281)
\curveto(156.39781045,135.27538281)(155.95532465,135.71786861)(155.95532465,136.26307433)
\curveto(155.95532465,136.80828005)(156.39781045,137.25076585)(156.94301617,137.25076585)
\curveto(157.48822189,137.25076585)(157.93070769,136.80828005)(157.93070769,136.26307433)
\closepath
}
}
{
\newrgbcolor{curcolor}{0 0 0}
\pscustom[linewidth=0.98769152,linecolor=curcolor]
{
\newpath
\moveto(118.60405,132.35235433)
\lineto(137.76634,132.35235433)
\lineto(137.76634,132.35235433)
}
}
{
\newrgbcolor{curcolor}{0 0 0}
\pscustom[linestyle=none,fillstyle=solid,fillcolor=curcolor]
{
\newpath
\moveto(119.57198769,132.35235433)
\curveto(119.57198769,131.80714861)(119.12950189,131.36466281)(118.58429617,131.36466281)
\curveto(118.03909045,131.36466281)(117.59660465,131.80714861)(117.59660465,132.35235433)
\curveto(117.59660465,132.89756005)(118.03909045,133.34004585)(118.58429617,133.34004585)
\curveto(119.12950189,133.34004585)(119.57198769,132.89756005)(119.57198769,132.35235433)
\closepath
}
}
{
\newrgbcolor{curcolor}{0 0 0}
\pscustom[linewidth=0.24692288,linecolor=curcolor]
{
\newpath
\moveto(119.57198769,132.35235433)
\curveto(119.57198769,131.80714861)(119.12950189,131.36466281)(118.58429617,131.36466281)
\curveto(118.03909045,131.36466281)(117.59660465,131.80714861)(117.59660465,132.35235433)
\curveto(117.59660465,132.89756005)(118.03909045,133.34004585)(118.58429617,133.34004585)
\curveto(119.12950189,133.34004585)(119.57198769,132.89756005)(119.57198769,132.35235433)
\closepath
}
}
{
\newrgbcolor{curcolor}{0 0 0}
\pscustom[linestyle=none,fillstyle=solid,fillcolor=curcolor]
{
\newpath
\moveto(138.73427769,132.35235433)
\curveto(138.73427769,131.80714861)(138.29179189,131.36466281)(137.74658617,131.36466281)
\curveto(137.20138045,131.36466281)(136.75889465,131.80714861)(136.75889465,132.35235433)
\curveto(136.75889465,132.89756005)(137.20138045,133.34004585)(137.74658617,133.34004585)
\curveto(138.29179189,133.34004585)(138.73427769,132.89756005)(138.73427769,132.35235433)
\closepath
}
}
{
\newrgbcolor{curcolor}{0 0 0}
\pscustom[linewidth=0.24692288,linecolor=curcolor]
{
\newpath
\moveto(138.73427769,132.35235433)
\curveto(138.73427769,131.80714861)(138.29179189,131.36466281)(137.74658617,131.36466281)
\curveto(137.20138045,131.36466281)(136.75889465,131.80714861)(136.75889465,132.35235433)
\curveto(136.75889465,132.89756005)(137.20138045,133.34004585)(137.74658617,133.34004585)
\curveto(138.29179189,133.34004585)(138.73427769,132.89756005)(138.73427769,132.35235433)
\closepath
}
}
{
\newrgbcolor{curcolor}{0 0 0}
\pscustom[linewidth=0.98769152,linecolor=curcolor]
{
\newpath
\moveto(99.51476,128.38806433)
\lineto(118.67705,128.38806433)
\lineto(118.67705,128.38806433)
}
}
{
\newrgbcolor{curcolor}{0 0 0}
\pscustom[linestyle=none,fillstyle=solid,fillcolor=curcolor]
{
\newpath
\moveto(100.48269769,128.38806433)
\curveto(100.48269769,127.84285861)(100.04021189,127.40037281)(99.49500617,127.40037281)
\curveto(98.94980045,127.40037281)(98.50731465,127.84285861)(98.50731465,128.38806433)
\curveto(98.50731465,128.93327005)(98.94980045,129.37575585)(99.49500617,129.37575585)
\curveto(100.04021189,129.37575585)(100.48269769,128.93327005)(100.48269769,128.38806433)
\closepath
}
}
{
\newrgbcolor{curcolor}{0 0 0}
\pscustom[linewidth=0.24692288,linecolor=curcolor]
{
\newpath
\moveto(100.48269769,128.38806433)
\curveto(100.48269769,127.84285861)(100.04021189,127.40037281)(99.49500617,127.40037281)
\curveto(98.94980045,127.40037281)(98.50731465,127.84285861)(98.50731465,128.38806433)
\curveto(98.50731465,128.93327005)(98.94980045,129.37575585)(99.49500617,129.37575585)
\curveto(100.04021189,129.37575585)(100.48269769,128.93327005)(100.48269769,128.38806433)
\closepath
}
}
{
\newrgbcolor{curcolor}{0 0 0}
\pscustom[linestyle=none,fillstyle=solid,fillcolor=curcolor]
{
\newpath
\moveto(119.64498769,128.38806433)
\curveto(119.64498769,127.84285861)(119.20250189,127.40037281)(118.65729617,127.40037281)
\curveto(118.11209045,127.40037281)(117.66960465,127.84285861)(117.66960465,128.38806433)
\curveto(117.66960465,128.93327005)(118.11209045,129.37575585)(118.65729617,129.37575585)
\curveto(119.20250189,129.37575585)(119.64498769,128.93327005)(119.64498769,128.38806433)
\closepath
}
}
{
\newrgbcolor{curcolor}{0 0 0}
\pscustom[linewidth=0.24692288,linecolor=curcolor]
{
\newpath
\moveto(119.64498769,128.38806433)
\curveto(119.64498769,127.84285861)(119.20250189,127.40037281)(118.65729617,127.40037281)
\curveto(118.11209045,127.40037281)(117.66960465,127.84285861)(117.66960465,128.38806433)
\curveto(117.66960465,128.93327005)(118.11209045,129.37575585)(118.65729617,129.37575585)
\curveto(119.20250189,129.37575585)(119.64498769,128.93327005)(119.64498769,128.38806433)
\closepath
}
}
{
\newrgbcolor{curcolor}{0 0 0}
\pscustom[linewidth=1,linecolor=curcolor,linestyle=dashed,dash=1 4]
{
\newpath
\moveto(214.48428,152.10714433)
\lineto(269.48428,152.10714433)
\lineto(269.48428,152.10714433)
\lineto(269.48428,152.10714433)
}
}
{
\newrgbcolor{curcolor}{0 0 0}
\pscustom[linestyle=none,fillstyle=solid,fillcolor=curcolor]
{
\newpath
\moveto(215.46428,152.10714433)
\curveto(215.46428,151.55514433)(215.01628,151.10714433)(214.46428,151.10714433)
\curveto(213.91228,151.10714433)(213.46428,151.55514433)(213.46428,152.10714433)
\curveto(213.46428,152.65914433)(213.91228,153.10714433)(214.46428,153.10714433)
\curveto(215.01628,153.10714433)(215.46428,152.65914433)(215.46428,152.10714433)
\closepath
}
}
{
\newrgbcolor{curcolor}{0 0 0}
\pscustom[linewidth=0.25,linecolor=curcolor]
{
\newpath
\moveto(215.46428,152.10714433)
\curveto(215.46428,151.55514433)(215.01628,151.10714433)(214.46428,151.10714433)
\curveto(213.91228,151.10714433)(213.46428,151.55514433)(213.46428,152.10714433)
\curveto(213.46428,152.65914433)(213.91228,153.10714433)(214.46428,153.10714433)
\curveto(215.01628,153.10714433)(215.46428,152.65914433)(215.46428,152.10714433)
\closepath
}
}
{
\newrgbcolor{curcolor}{0 0 0}
\pscustom[linestyle=none,fillstyle=solid,fillcolor=curcolor]
{
\newpath
\moveto(270.46428,152.10714433)
\curveto(270.46428,151.55514433)(270.01628,151.10714433)(269.46428,151.10714433)
\curveto(268.91228,151.10714433)(268.46428,151.55514433)(268.46428,152.10714433)
\curveto(268.46428,152.65914433)(268.91228,153.10714433)(269.46428,153.10714433)
\curveto(270.01628,153.10714433)(270.46428,152.65914433)(270.46428,152.10714433)
\closepath
}
}
{
\newrgbcolor{curcolor}{0 0 0}
\pscustom[linewidth=0.25,linecolor=curcolor]
{
\newpath
\moveto(270.46428,152.10714433)
\curveto(270.46428,151.55514433)(270.01628,151.10714433)(269.46428,151.10714433)
\curveto(268.91228,151.10714433)(268.46428,151.55514433)(268.46428,152.10714433)
\curveto(268.46428,152.65914433)(268.91228,153.10714433)(269.46428,153.10714433)
\curveto(270.01628,153.10714433)(270.46428,152.65914433)(270.46428,152.10714433)
\closepath
}
}
{
\newrgbcolor{curcolor}{0 0 0}
\pscustom[linewidth=0.98769152,linecolor=curcolor]
{
\newpath
\moveto(195.49451,121.92378433)
\lineto(214.6568,121.92378433)
\lineto(214.6568,121.92378433)
}
}
{
\newrgbcolor{curcolor}{0 0 0}
\pscustom[linestyle=none,fillstyle=solid,fillcolor=curcolor]
{
\newpath
\moveto(196.46244769,121.92378433)
\curveto(196.46244769,121.37857861)(196.01996189,120.93609281)(195.47475617,120.93609281)
\curveto(194.92955045,120.93609281)(194.48706465,121.37857861)(194.48706465,121.92378433)
\curveto(194.48706465,122.46899005)(194.92955045,122.91147585)(195.47475617,122.91147585)
\curveto(196.01996189,122.91147585)(196.46244769,122.46899005)(196.46244769,121.92378433)
\closepath
}
}
{
\newrgbcolor{curcolor}{0 0 0}
\pscustom[linewidth=0.24692288,linecolor=curcolor]
{
\newpath
\moveto(196.46244769,121.92378433)
\curveto(196.46244769,121.37857861)(196.01996189,120.93609281)(195.47475617,120.93609281)
\curveto(194.92955045,120.93609281)(194.48706465,121.37857861)(194.48706465,121.92378433)
\curveto(194.48706465,122.46899005)(194.92955045,122.91147585)(195.47475617,122.91147585)
\curveto(196.01996189,122.91147585)(196.46244769,122.46899005)(196.46244769,121.92378433)
\closepath
}
}
{
\newrgbcolor{curcolor}{0 0 0}
\pscustom[linestyle=none,fillstyle=solid,fillcolor=curcolor]
{
\newpath
\moveto(215.62473769,121.92378433)
\curveto(215.62473769,121.37857861)(215.18225189,120.93609281)(214.63704617,120.93609281)
\curveto(214.09184045,120.93609281)(213.64935465,121.37857861)(213.64935465,121.92378433)
\curveto(213.64935465,122.46899005)(214.09184045,122.91147585)(214.63704617,122.91147585)
\curveto(215.18225189,122.91147585)(215.62473769,122.46899005)(215.62473769,121.92378433)
\closepath
}
}
{
\newrgbcolor{curcolor}{0 0 0}
\pscustom[linewidth=0.24692288,linecolor=curcolor]
{
\newpath
\moveto(215.62473769,121.92378433)
\curveto(215.62473769,121.37857861)(215.18225189,120.93609281)(214.63704617,120.93609281)
\curveto(214.09184045,120.93609281)(213.64935465,121.37857861)(213.64935465,121.92378433)
\curveto(213.64935465,122.46899005)(214.09184045,122.91147585)(214.63704617,122.91147585)
\curveto(215.18225189,122.91147585)(215.62473769,122.46899005)(215.62473769,121.92378433)
\closepath
}
}
{
\newrgbcolor{curcolor}{0 0 0}
\pscustom[linewidth=0.98769152,linecolor=curcolor]
{
\newpath
\moveto(176.33619,117.97734433)
\lineto(195.49848,117.97734433)
\lineto(195.49848,117.97734433)
}
}
{
\newrgbcolor{curcolor}{0 0 0}
\pscustom[linestyle=none,fillstyle=solid,fillcolor=curcolor]
{
\newpath
\moveto(177.30412769,117.97734433)
\curveto(177.30412769,117.43213861)(176.86164189,116.98965281)(176.31643617,116.98965281)
\curveto(175.77123045,116.98965281)(175.32874465,117.43213861)(175.32874465,117.97734433)
\curveto(175.32874465,118.52255005)(175.77123045,118.96503585)(176.31643617,118.96503585)
\curveto(176.86164189,118.96503585)(177.30412769,118.52255005)(177.30412769,117.97734433)
\closepath
}
}
{
\newrgbcolor{curcolor}{0 0 0}
\pscustom[linewidth=0.24692288,linecolor=curcolor]
{
\newpath
\moveto(177.30412769,117.97734433)
\curveto(177.30412769,117.43213861)(176.86164189,116.98965281)(176.31643617,116.98965281)
\curveto(175.77123045,116.98965281)(175.32874465,117.43213861)(175.32874465,117.97734433)
\curveto(175.32874465,118.52255005)(175.77123045,118.96503585)(176.31643617,118.96503585)
\curveto(176.86164189,118.96503585)(177.30412769,118.52255005)(177.30412769,117.97734433)
\closepath
}
}
{
\newrgbcolor{curcolor}{0 0 0}
\pscustom[linestyle=none,fillstyle=solid,fillcolor=curcolor]
{
\newpath
\moveto(196.46641769,117.97734433)
\curveto(196.46641769,117.43213861)(196.02393189,116.98965281)(195.47872617,116.98965281)
\curveto(194.93352045,116.98965281)(194.49103465,117.43213861)(194.49103465,117.97734433)
\curveto(194.49103465,118.52255005)(194.93352045,118.96503585)(195.47872617,118.96503585)
\curveto(196.02393189,118.96503585)(196.46641769,118.52255005)(196.46641769,117.97734433)
\closepath
}
}
{
\newrgbcolor{curcolor}{0 0 0}
\pscustom[linewidth=0.24692288,linecolor=curcolor]
{
\newpath
\moveto(196.46641769,117.97734433)
\curveto(196.46641769,117.43213861)(196.02393189,116.98965281)(195.47872617,116.98965281)
\curveto(194.93352045,116.98965281)(194.49103465,117.43213861)(194.49103465,117.97734433)
\curveto(194.49103465,118.52255005)(194.93352045,118.96503585)(195.47872617,118.96503585)
\curveto(196.02393189,118.96503585)(196.46641769,118.52255005)(196.46641769,117.97734433)
\closepath
}
}
{
\newrgbcolor{curcolor}{0 0 0}
\pscustom[linewidth=0.98769152,linecolor=curcolor]
{
\newpath
\moveto(157.17548,114.06664433)
\lineto(176.33777,114.06664433)
\lineto(176.33777,114.06664433)
}
}
{
\newrgbcolor{curcolor}{0 0 0}
\pscustom[linestyle=none,fillstyle=solid,fillcolor=curcolor]
{
\newpath
\moveto(158.14341769,114.06664433)
\curveto(158.14341769,113.52143861)(157.70093189,113.07895281)(157.15572617,113.07895281)
\curveto(156.61052045,113.07895281)(156.16803465,113.52143861)(156.16803465,114.06664433)
\curveto(156.16803465,114.61185005)(156.61052045,115.05433585)(157.15572617,115.05433585)
\curveto(157.70093189,115.05433585)(158.14341769,114.61185005)(158.14341769,114.06664433)
\closepath
}
}
{
\newrgbcolor{curcolor}{0 0 0}
\pscustom[linewidth=0.24692288,linecolor=curcolor]
{
\newpath
\moveto(158.14341769,114.06664433)
\curveto(158.14341769,113.52143861)(157.70093189,113.07895281)(157.15572617,113.07895281)
\curveto(156.61052045,113.07895281)(156.16803465,113.52143861)(156.16803465,114.06664433)
\curveto(156.16803465,114.61185005)(156.61052045,115.05433585)(157.15572617,115.05433585)
\curveto(157.70093189,115.05433585)(158.14341769,114.61185005)(158.14341769,114.06664433)
\closepath
}
}
{
\newrgbcolor{curcolor}{0 0 0}
\pscustom[linestyle=none,fillstyle=solid,fillcolor=curcolor]
{
\newpath
\moveto(177.30570769,114.06664433)
\curveto(177.30570769,113.52143861)(176.86322189,113.07895281)(176.31801617,113.07895281)
\curveto(175.77281045,113.07895281)(175.33032465,113.52143861)(175.33032465,114.06664433)
\curveto(175.33032465,114.61185005)(175.77281045,115.05433585)(176.31801617,115.05433585)
\curveto(176.86322189,115.05433585)(177.30570769,114.61185005)(177.30570769,114.06664433)
\closepath
}
}
{
\newrgbcolor{curcolor}{0 0 0}
\pscustom[linewidth=0.24692288,linecolor=curcolor]
{
\newpath
\moveto(177.30570769,114.06664433)
\curveto(177.30570769,113.52143861)(176.86322189,113.07895281)(176.31801617,113.07895281)
\curveto(175.77281045,113.07895281)(175.33032465,113.52143861)(175.33032465,114.06664433)
\curveto(175.33032465,114.61185005)(175.77281045,115.05433585)(176.31801617,115.05433585)
\curveto(176.86322189,115.05433585)(177.30570769,114.61185005)(177.30570769,114.06664433)
\closepath
}
}
{
\newrgbcolor{curcolor}{0 0 0}
\pscustom[linewidth=0.98769152,linecolor=curcolor]
{
\newpath
\moveto(137.97905,109.97736433)
\lineto(157.14134,109.97736433)
\lineto(157.14134,109.97736433)
}
}
{
\newrgbcolor{curcolor}{0 0 0}
\pscustom[linestyle=none,fillstyle=solid,fillcolor=curcolor]
{
\newpath
\moveto(138.94698769,109.97736433)
\curveto(138.94698769,109.43215861)(138.50450189,108.98967281)(137.95929617,108.98967281)
\curveto(137.41409045,108.98967281)(136.97160465,109.43215861)(136.97160465,109.97736433)
\curveto(136.97160465,110.52257005)(137.41409045,110.96505585)(137.95929617,110.96505585)
\curveto(138.50450189,110.96505585)(138.94698769,110.52257005)(138.94698769,109.97736433)
\closepath
}
}
{
\newrgbcolor{curcolor}{0 0 0}
\pscustom[linewidth=0.24692288,linecolor=curcolor]
{
\newpath
\moveto(138.94698769,109.97736433)
\curveto(138.94698769,109.43215861)(138.50450189,108.98967281)(137.95929617,108.98967281)
\curveto(137.41409045,108.98967281)(136.97160465,109.43215861)(136.97160465,109.97736433)
\curveto(136.97160465,110.52257005)(137.41409045,110.96505585)(137.95929617,110.96505585)
\curveto(138.50450189,110.96505585)(138.94698769,110.52257005)(138.94698769,109.97736433)
\closepath
}
}
{
\newrgbcolor{curcolor}{0 0 0}
\pscustom[linestyle=none,fillstyle=solid,fillcolor=curcolor]
{
\newpath
\moveto(158.10927769,109.97736433)
\curveto(158.10927769,109.43215861)(157.66679189,108.98967281)(157.12158617,108.98967281)
\curveto(156.57638045,108.98967281)(156.13389465,109.43215861)(156.13389465,109.97736433)
\curveto(156.13389465,110.52257005)(156.57638045,110.96505585)(157.12158617,110.96505585)
\curveto(157.66679189,110.96505585)(158.10927769,110.52257005)(158.10927769,109.97736433)
\closepath
}
}
{
\newrgbcolor{curcolor}{0 0 0}
\pscustom[linewidth=0.24692288,linecolor=curcolor]
{
\newpath
\moveto(158.10927769,109.97736433)
\curveto(158.10927769,109.43215861)(157.66679189,108.98967281)(157.12158617,108.98967281)
\curveto(156.57638045,108.98967281)(156.13389465,109.43215861)(156.13389465,109.97736433)
\curveto(156.13389465,110.52257005)(156.57638045,110.96505585)(157.12158617,110.96505585)
\curveto(157.66679189,110.96505585)(158.10927769,110.52257005)(158.10927769,109.97736433)
\closepath
}
}
{
\newrgbcolor{curcolor}{0 0 0}
\pscustom[linewidth=0.98769152,linecolor=curcolor]
{
\newpath
\moveto(118.78262,106.06664433)
\lineto(137.94491,106.06664433)
\lineto(137.94491,106.06664433)
}
}
{
\newrgbcolor{curcolor}{0 0 0}
\pscustom[linestyle=none,fillstyle=solid,fillcolor=curcolor]
{
\newpath
\moveto(119.75055769,106.06664433)
\curveto(119.75055769,105.52143861)(119.30807189,105.07895281)(118.76286617,105.07895281)
\curveto(118.21766045,105.07895281)(117.77517465,105.52143861)(117.77517465,106.06664433)
\curveto(117.77517465,106.61185005)(118.21766045,107.05433585)(118.76286617,107.05433585)
\curveto(119.30807189,107.05433585)(119.75055769,106.61185005)(119.75055769,106.06664433)
\closepath
}
}
{
\newrgbcolor{curcolor}{0 0 0}
\pscustom[linewidth=0.24692288,linecolor=curcolor]
{
\newpath
\moveto(119.75055769,106.06664433)
\curveto(119.75055769,105.52143861)(119.30807189,105.07895281)(118.76286617,105.07895281)
\curveto(118.21766045,105.07895281)(117.77517465,105.52143861)(117.77517465,106.06664433)
\curveto(117.77517465,106.61185005)(118.21766045,107.05433585)(118.76286617,107.05433585)
\curveto(119.30807189,107.05433585)(119.75055769,106.61185005)(119.75055769,106.06664433)
\closepath
}
}
{
\newrgbcolor{curcolor}{0 0 0}
\pscustom[linestyle=none,fillstyle=solid,fillcolor=curcolor]
{
\newpath
\moveto(138.91284769,106.06664433)
\curveto(138.91284769,105.52143861)(138.47036189,105.07895281)(137.92515617,105.07895281)
\curveto(137.37995045,105.07895281)(136.93746465,105.52143861)(136.93746465,106.06664433)
\curveto(136.93746465,106.61185005)(137.37995045,107.05433585)(137.92515617,107.05433585)
\curveto(138.47036189,107.05433585)(138.91284769,106.61185005)(138.91284769,106.06664433)
\closepath
}
}
{
\newrgbcolor{curcolor}{0 0 0}
\pscustom[linewidth=0.24692288,linecolor=curcolor]
{
\newpath
\moveto(138.91284769,106.06664433)
\curveto(138.91284769,105.52143861)(138.47036189,105.07895281)(137.92515617,105.07895281)
\curveto(137.37995045,105.07895281)(136.93746465,105.52143861)(136.93746465,106.06664433)
\curveto(136.93746465,106.61185005)(137.37995045,107.05433585)(137.92515617,107.05433585)
\curveto(138.47036189,107.05433585)(138.91284769,106.61185005)(138.91284769,106.06664433)
\closepath
}
}
{
\newrgbcolor{curcolor}{0 0 0}
\pscustom[linewidth=1,linecolor=curcolor,linestyle=dashed,dash=1 4]
{
\newpath
\moveto(214.66286,126.17856433)
\lineto(269.66286,126.17856433)
\lineto(269.66286,126.17856433)
\lineto(269.66286,126.17856433)
}
}
{
\newrgbcolor{curcolor}{0 0 0}
\pscustom[linestyle=none,fillstyle=solid,fillcolor=curcolor]
{
\newpath
\moveto(215.64286,126.17856433)
\curveto(215.64286,125.62656433)(215.19486,125.17856433)(214.64286,125.17856433)
\curveto(214.09086,125.17856433)(213.64286,125.62656433)(213.64286,126.17856433)
\curveto(213.64286,126.73056433)(214.09086,127.17856433)(214.64286,127.17856433)
\curveto(215.19486,127.17856433)(215.64286,126.73056433)(215.64286,126.17856433)
\closepath
}
}
{
\newrgbcolor{curcolor}{0 0 0}
\pscustom[linewidth=0.25,linecolor=curcolor]
{
\newpath
\moveto(215.64286,126.17856433)
\curveto(215.64286,125.62656433)(215.19486,125.17856433)(214.64286,125.17856433)
\curveto(214.09086,125.17856433)(213.64286,125.62656433)(213.64286,126.17856433)
\curveto(213.64286,126.73056433)(214.09086,127.17856433)(214.64286,127.17856433)
\curveto(215.19486,127.17856433)(215.64286,126.73056433)(215.64286,126.17856433)
\closepath
}
}
{
\newrgbcolor{curcolor}{0 0 0}
\pscustom[linestyle=none,fillstyle=solid,fillcolor=curcolor]
{
\newpath
\moveto(270.64286,126.17856433)
\curveto(270.64286,125.62656433)(270.19486,125.17856433)(269.64286,125.17856433)
\curveto(269.09086,125.17856433)(268.64286,125.62656433)(268.64286,126.17856433)
\curveto(268.64286,126.73056433)(269.09086,127.17856433)(269.64286,127.17856433)
\curveto(270.19486,127.17856433)(270.64286,126.73056433)(270.64286,126.17856433)
\closepath
}
}
{
\newrgbcolor{curcolor}{0 0 0}
\pscustom[linewidth=0.25,linecolor=curcolor]
{
\newpath
\moveto(270.64286,126.17856433)
\curveto(270.64286,125.62656433)(270.19486,125.17856433)(269.64286,125.17856433)
\curveto(269.09086,125.17856433)(268.64286,125.62656433)(268.64286,126.17856433)
\curveto(268.64286,126.73056433)(269.09086,127.17856433)(269.64286,127.17856433)
\curveto(270.19486,127.17856433)(270.64286,126.73056433)(270.64286,126.17856433)
\closepath
}
}
{
\newrgbcolor{curcolor}{0 0 0}
\pscustom[linewidth=0.98769152,linecolor=curcolor]
{
\newpath
\moveto(195.570442,99.18711533)
\lineto(214.732732,99.18711533)
\lineto(214.732732,99.18711533)
}
}
{
\newrgbcolor{curcolor}{0 0 0}
\pscustom[linestyle=none,fillstyle=solid,fillcolor=curcolor]
{
\newpath
\moveto(196.53837969,99.18711533)
\curveto(196.53837969,98.64190961)(196.09589389,98.19942381)(195.55068817,98.19942381)
\curveto(195.00548245,98.19942381)(194.56299665,98.64190961)(194.56299665,99.18711533)
\curveto(194.56299665,99.73232105)(195.00548245,100.17480685)(195.55068817,100.17480685)
\curveto(196.09589389,100.17480685)(196.53837969,99.73232105)(196.53837969,99.18711533)
\closepath
}
}
{
\newrgbcolor{curcolor}{0 0 0}
\pscustom[linewidth=0.24692288,linecolor=curcolor]
{
\newpath
\moveto(196.53837969,99.18711533)
\curveto(196.53837969,98.64190961)(196.09589389,98.19942381)(195.55068817,98.19942381)
\curveto(195.00548245,98.19942381)(194.56299665,98.64190961)(194.56299665,99.18711533)
\curveto(194.56299665,99.73232105)(195.00548245,100.17480685)(195.55068817,100.17480685)
\curveto(196.09589389,100.17480685)(196.53837969,99.73232105)(196.53837969,99.18711533)
\closepath
}
}
{
\newrgbcolor{curcolor}{0 0 0}
\pscustom[linestyle=none,fillstyle=solid,fillcolor=curcolor]
{
\newpath
\moveto(215.70066969,99.18711533)
\curveto(215.70066969,98.64190961)(215.25818389,98.19942381)(214.71297817,98.19942381)
\curveto(214.16777245,98.19942381)(213.72528665,98.64190961)(213.72528665,99.18711533)
\curveto(213.72528665,99.73232105)(214.16777245,100.17480685)(214.71297817,100.17480685)
\curveto(215.25818389,100.17480685)(215.70066969,99.73232105)(215.70066969,99.18711533)
\closepath
}
}
{
\newrgbcolor{curcolor}{0 0 0}
\pscustom[linewidth=0.24692288,linecolor=curcolor]
{
\newpath
\moveto(215.70066969,99.18711533)
\curveto(215.70066969,98.64190961)(215.25818389,98.19942381)(214.71297817,98.19942381)
\curveto(214.16777245,98.19942381)(213.72528665,98.64190961)(213.72528665,99.18711533)
\curveto(213.72528665,99.73232105)(214.16777245,100.17480685)(214.71297817,100.17480685)
\curveto(215.25818389,100.17480685)(215.70066969,99.73232105)(215.70066969,99.18711533)
\closepath
}
}
{
\newrgbcolor{curcolor}{0 0 0}
\pscustom[linewidth=0.98769152,linecolor=curcolor]
{
\newpath
\moveto(176.412122,95.24067533)
\lineto(195.574412,95.24067533)
\lineto(195.574412,95.24067533)
}
}
{
\newrgbcolor{curcolor}{0 0 0}
\pscustom[linestyle=none,fillstyle=solid,fillcolor=curcolor]
{
\newpath
\moveto(177.38005969,95.24067533)
\curveto(177.38005969,94.69546961)(176.93757389,94.25298381)(176.39236817,94.25298381)
\curveto(175.84716245,94.25298381)(175.40467665,94.69546961)(175.40467665,95.24067533)
\curveto(175.40467665,95.78588105)(175.84716245,96.22836685)(176.39236817,96.22836685)
\curveto(176.93757389,96.22836685)(177.38005969,95.78588105)(177.38005969,95.24067533)
\closepath
}
}
{
\newrgbcolor{curcolor}{0 0 0}
\pscustom[linewidth=0.24692288,linecolor=curcolor]
{
\newpath
\moveto(177.38005969,95.24067533)
\curveto(177.38005969,94.69546961)(176.93757389,94.25298381)(176.39236817,94.25298381)
\curveto(175.84716245,94.25298381)(175.40467665,94.69546961)(175.40467665,95.24067533)
\curveto(175.40467665,95.78588105)(175.84716245,96.22836685)(176.39236817,96.22836685)
\curveto(176.93757389,96.22836685)(177.38005969,95.78588105)(177.38005969,95.24067533)
\closepath
}
}
{
\newrgbcolor{curcolor}{0 0 0}
\pscustom[linestyle=none,fillstyle=solid,fillcolor=curcolor]
{
\newpath
\moveto(196.54234969,95.24067533)
\curveto(196.54234969,94.69546961)(196.09986389,94.25298381)(195.55465817,94.25298381)
\curveto(195.00945245,94.25298381)(194.56696665,94.69546961)(194.56696665,95.24067533)
\curveto(194.56696665,95.78588105)(195.00945245,96.22836685)(195.55465817,96.22836685)
\curveto(196.09986389,96.22836685)(196.54234969,95.78588105)(196.54234969,95.24067533)
\closepath
}
}
{
\newrgbcolor{curcolor}{0 0 0}
\pscustom[linewidth=0.24692288,linecolor=curcolor]
{
\newpath
\moveto(196.54234969,95.24067533)
\curveto(196.54234969,94.69546961)(196.09986389,94.25298381)(195.55465817,94.25298381)
\curveto(195.00945245,94.25298381)(194.56696665,94.69546961)(194.56696665,95.24067533)
\curveto(194.56696665,95.78588105)(195.00945245,96.22836685)(195.55465817,96.22836685)
\curveto(196.09986389,96.22836685)(196.54234969,95.78588105)(196.54234969,95.24067533)
\closepath
}
}
{
\newrgbcolor{curcolor}{0 0 0}
\pscustom[linewidth=0.98769152,linecolor=curcolor]
{
\newpath
\moveto(157.251412,91.32997533)
\lineto(176.413702,91.32997533)
\lineto(176.413702,91.32997533)
}
}
{
\newrgbcolor{curcolor}{0 0 0}
\pscustom[linestyle=none,fillstyle=solid,fillcolor=curcolor]
{
\newpath
\moveto(158.21934969,91.32997533)
\curveto(158.21934969,90.78476961)(157.77686389,90.34228381)(157.23165817,90.34228381)
\curveto(156.68645245,90.34228381)(156.24396665,90.78476961)(156.24396665,91.32997533)
\curveto(156.24396665,91.87518105)(156.68645245,92.31766685)(157.23165817,92.31766685)
\curveto(157.77686389,92.31766685)(158.21934969,91.87518105)(158.21934969,91.32997533)
\closepath
}
}
{
\newrgbcolor{curcolor}{0 0 0}
\pscustom[linewidth=0.24692288,linecolor=curcolor]
{
\newpath
\moveto(158.21934969,91.32997533)
\curveto(158.21934969,90.78476961)(157.77686389,90.34228381)(157.23165817,90.34228381)
\curveto(156.68645245,90.34228381)(156.24396665,90.78476961)(156.24396665,91.32997533)
\curveto(156.24396665,91.87518105)(156.68645245,92.31766685)(157.23165817,92.31766685)
\curveto(157.77686389,92.31766685)(158.21934969,91.87518105)(158.21934969,91.32997533)
\closepath
}
}
{
\newrgbcolor{curcolor}{0 0 0}
\pscustom[linestyle=none,fillstyle=solid,fillcolor=curcolor]
{
\newpath
\moveto(177.38163969,91.32997533)
\curveto(177.38163969,90.78476961)(176.93915389,90.34228381)(176.39394817,90.34228381)
\curveto(175.84874245,90.34228381)(175.40625665,90.78476961)(175.40625665,91.32997533)
\curveto(175.40625665,91.87518105)(175.84874245,92.31766685)(176.39394817,92.31766685)
\curveto(176.93915389,92.31766685)(177.38163969,91.87518105)(177.38163969,91.32997533)
\closepath
}
}
{
\newrgbcolor{curcolor}{0 0 0}
\pscustom[linewidth=0.24692288,linecolor=curcolor]
{
\newpath
\moveto(177.38163969,91.32997533)
\curveto(177.38163969,90.78476961)(176.93915389,90.34228381)(176.39394817,90.34228381)
\curveto(175.84874245,90.34228381)(175.40625665,90.78476961)(175.40625665,91.32997533)
\curveto(175.40625665,91.87518105)(175.84874245,92.31766685)(176.39394817,92.31766685)
\curveto(176.93915389,92.31766685)(177.38163969,91.87518105)(177.38163969,91.32997533)
\closepath
}
}
{
\newrgbcolor{curcolor}{0 0 0}
\pscustom[linewidth=0.98769152,linecolor=curcolor]
{
\newpath
\moveto(138.054982,87.24069533)
\lineto(157.217272,87.24069533)
\lineto(157.217272,87.24069533)
}
}
{
\newrgbcolor{curcolor}{0 0 0}
\pscustom[linestyle=none,fillstyle=solid,fillcolor=curcolor]
{
\newpath
\moveto(139.02291969,87.24069533)
\curveto(139.02291969,86.69548961)(138.58043389,86.25300381)(138.03522817,86.25300381)
\curveto(137.49002245,86.25300381)(137.04753665,86.69548961)(137.04753665,87.24069533)
\curveto(137.04753665,87.78590105)(137.49002245,88.22838685)(138.03522817,88.22838685)
\curveto(138.58043389,88.22838685)(139.02291969,87.78590105)(139.02291969,87.24069533)
\closepath
}
}
{
\newrgbcolor{curcolor}{0 0 0}
\pscustom[linewidth=0.24692288,linecolor=curcolor]
{
\newpath
\moveto(139.02291969,87.24069533)
\curveto(139.02291969,86.69548961)(138.58043389,86.25300381)(138.03522817,86.25300381)
\curveto(137.49002245,86.25300381)(137.04753665,86.69548961)(137.04753665,87.24069533)
\curveto(137.04753665,87.78590105)(137.49002245,88.22838685)(138.03522817,88.22838685)
\curveto(138.58043389,88.22838685)(139.02291969,87.78590105)(139.02291969,87.24069533)
\closepath
}
}
{
\newrgbcolor{curcolor}{0 0 0}
\pscustom[linestyle=none,fillstyle=solid,fillcolor=curcolor]
{
\newpath
\moveto(158.18520969,87.24069533)
\curveto(158.18520969,86.69548961)(157.74272389,86.25300381)(157.19751817,86.25300381)
\curveto(156.65231245,86.25300381)(156.20982665,86.69548961)(156.20982665,87.24069533)
\curveto(156.20982665,87.78590105)(156.65231245,88.22838685)(157.19751817,88.22838685)
\curveto(157.74272389,88.22838685)(158.18520969,87.78590105)(158.18520969,87.24069533)
\closepath
}
}
{
\newrgbcolor{curcolor}{0 0 0}
\pscustom[linewidth=0.24692288,linecolor=curcolor]
{
\newpath
\moveto(158.18520969,87.24069533)
\curveto(158.18520969,86.69548961)(157.74272389,86.25300381)(157.19751817,86.25300381)
\curveto(156.65231245,86.25300381)(156.20982665,86.69548961)(156.20982665,87.24069533)
\curveto(156.20982665,87.78590105)(156.65231245,88.22838685)(157.19751817,88.22838685)
\curveto(157.74272389,88.22838685)(158.18520969,87.78590105)(158.18520969,87.24069533)
\closepath
}
}
{
\newrgbcolor{curcolor}{0 0 0}
\pscustom[linewidth=1,linecolor=curcolor,linestyle=dashed,dash=1 4]
{
\newpath
\moveto(214.738792,103.44189533)
\lineto(269.738792,103.44189533)
\lineto(269.738792,103.44189533)
\lineto(269.738792,103.44189533)
}
}
{
\newrgbcolor{curcolor}{0 0 0}
\pscustom[linestyle=none,fillstyle=solid,fillcolor=curcolor]
{
\newpath
\moveto(215.718792,103.44189533)
\curveto(215.718792,102.88989533)(215.270792,102.44189533)(214.718792,102.44189533)
\curveto(214.166792,102.44189533)(213.718792,102.88989533)(213.718792,103.44189533)
\curveto(213.718792,103.99389533)(214.166792,104.44189533)(214.718792,104.44189533)
\curveto(215.270792,104.44189533)(215.718792,103.99389533)(215.718792,103.44189533)
\closepath
}
}
{
\newrgbcolor{curcolor}{0 0 0}
\pscustom[linewidth=0.25,linecolor=curcolor]
{
\newpath
\moveto(215.718792,103.44189533)
\curveto(215.718792,102.88989533)(215.270792,102.44189533)(214.718792,102.44189533)
\curveto(214.166792,102.44189533)(213.718792,102.88989533)(213.718792,103.44189533)
\curveto(213.718792,103.99389533)(214.166792,104.44189533)(214.718792,104.44189533)
\curveto(215.270792,104.44189533)(215.718792,103.99389533)(215.718792,103.44189533)
\closepath
}
}
{
\newrgbcolor{curcolor}{0 0 0}
\pscustom[linestyle=none,fillstyle=solid,fillcolor=curcolor]
{
\newpath
\moveto(270.718792,103.44189533)
\curveto(270.718792,102.88989533)(270.270792,102.44189533)(269.718792,102.44189533)
\curveto(269.166792,102.44189533)(268.718792,102.88989533)(268.718792,103.44189533)
\curveto(268.718792,103.99389533)(269.166792,104.44189533)(269.718792,104.44189533)
\curveto(270.270792,104.44189533)(270.718792,103.99389533)(270.718792,103.44189533)
\closepath
}
}
{
\newrgbcolor{curcolor}{0 0 0}
\pscustom[linewidth=0.25,linecolor=curcolor]
{
\newpath
\moveto(270.718792,103.44189533)
\curveto(270.718792,102.88989533)(270.270792,102.44189533)(269.718792,102.44189533)
\curveto(269.166792,102.44189533)(268.718792,102.88989533)(268.718792,103.44189533)
\curveto(268.718792,103.99389533)(269.166792,104.44189533)(269.718792,104.44189533)
\curveto(270.270792,104.44189533)(270.718792,103.99389533)(270.718792,103.44189533)
\closepath
}
}
{
\newrgbcolor{curcolor}{0 0 0}
\pscustom[linewidth=0.98769152,linecolor=curcolor]
{
\newpath
\moveto(195.570442,79.54425533)
\lineto(214.732732,79.54425533)
\lineto(214.732732,79.54425533)
}
}
{
\newrgbcolor{curcolor}{0 0 0}
\pscustom[linestyle=none,fillstyle=solid,fillcolor=curcolor]
{
\newpath
\moveto(196.53837969,79.54425533)
\curveto(196.53837969,78.99904961)(196.09589389,78.55656381)(195.55068817,78.55656381)
\curveto(195.00548245,78.55656381)(194.56299665,78.99904961)(194.56299665,79.54425533)
\curveto(194.56299665,80.08946105)(195.00548245,80.53194685)(195.55068817,80.53194685)
\curveto(196.09589389,80.53194685)(196.53837969,80.08946105)(196.53837969,79.54425533)
\closepath
}
}
{
\newrgbcolor{curcolor}{0 0 0}
\pscustom[linewidth=0.24692288,linecolor=curcolor]
{
\newpath
\moveto(196.53837969,79.54425533)
\curveto(196.53837969,78.99904961)(196.09589389,78.55656381)(195.55068817,78.55656381)
\curveto(195.00548245,78.55656381)(194.56299665,78.99904961)(194.56299665,79.54425533)
\curveto(194.56299665,80.08946105)(195.00548245,80.53194685)(195.55068817,80.53194685)
\curveto(196.09589389,80.53194685)(196.53837969,80.08946105)(196.53837969,79.54425533)
\closepath
}
}
{
\newrgbcolor{curcolor}{0 0 0}
\pscustom[linestyle=none,fillstyle=solid,fillcolor=curcolor]
{
\newpath
\moveto(215.70066969,79.54425533)
\curveto(215.70066969,78.99904961)(215.25818389,78.55656381)(214.71297817,78.55656381)
\curveto(214.16777245,78.55656381)(213.72528665,78.99904961)(213.72528665,79.54425533)
\curveto(213.72528665,80.08946105)(214.16777245,80.53194685)(214.71297817,80.53194685)
\curveto(215.25818389,80.53194685)(215.70066969,80.08946105)(215.70066969,79.54425533)
\closepath
}
}
{
\newrgbcolor{curcolor}{0 0 0}
\pscustom[linewidth=0.24692288,linecolor=curcolor]
{
\newpath
\moveto(215.70066969,79.54425533)
\curveto(215.70066969,78.99904961)(215.25818389,78.55656381)(214.71297817,78.55656381)
\curveto(214.16777245,78.55656381)(213.72528665,78.99904961)(213.72528665,79.54425533)
\curveto(213.72528665,80.08946105)(214.16777245,80.53194685)(214.71297817,80.53194685)
\curveto(215.25818389,80.53194685)(215.70066969,80.08946105)(215.70066969,79.54425533)
\closepath
}
}
{
\newrgbcolor{curcolor}{0 0 0}
\pscustom[linewidth=0.98769152,linecolor=curcolor]
{
\newpath
\moveto(176.412122,75.59781533)
\lineto(195.574412,75.59781533)
\lineto(195.574412,75.59781533)
}
}
{
\newrgbcolor{curcolor}{0 0 0}
\pscustom[linestyle=none,fillstyle=solid,fillcolor=curcolor]
{
\newpath
\moveto(177.38005969,75.59781533)
\curveto(177.38005969,75.05260961)(176.93757389,74.61012381)(176.39236817,74.61012381)
\curveto(175.84716245,74.61012381)(175.40467665,75.05260961)(175.40467665,75.59781533)
\curveto(175.40467665,76.14302105)(175.84716245,76.58550685)(176.39236817,76.58550685)
\curveto(176.93757389,76.58550685)(177.38005969,76.14302105)(177.38005969,75.59781533)
\closepath
}
}
{
\newrgbcolor{curcolor}{0 0 0}
\pscustom[linewidth=0.24692288,linecolor=curcolor]
{
\newpath
\moveto(177.38005969,75.59781533)
\curveto(177.38005969,75.05260961)(176.93757389,74.61012381)(176.39236817,74.61012381)
\curveto(175.84716245,74.61012381)(175.40467665,75.05260961)(175.40467665,75.59781533)
\curveto(175.40467665,76.14302105)(175.84716245,76.58550685)(176.39236817,76.58550685)
\curveto(176.93757389,76.58550685)(177.38005969,76.14302105)(177.38005969,75.59781533)
\closepath
}
}
{
\newrgbcolor{curcolor}{0 0 0}
\pscustom[linestyle=none,fillstyle=solid,fillcolor=curcolor]
{
\newpath
\moveto(196.54234969,75.59781533)
\curveto(196.54234969,75.05260961)(196.09986389,74.61012381)(195.55465817,74.61012381)
\curveto(195.00945245,74.61012381)(194.56696665,75.05260961)(194.56696665,75.59781533)
\curveto(194.56696665,76.14302105)(195.00945245,76.58550685)(195.55465817,76.58550685)
\curveto(196.09986389,76.58550685)(196.54234969,76.14302105)(196.54234969,75.59781533)
\closepath
}
}
{
\newrgbcolor{curcolor}{0 0 0}
\pscustom[linewidth=0.24692288,linecolor=curcolor]
{
\newpath
\moveto(196.54234969,75.59781533)
\curveto(196.54234969,75.05260961)(196.09986389,74.61012381)(195.55465817,74.61012381)
\curveto(195.00945245,74.61012381)(194.56696665,75.05260961)(194.56696665,75.59781533)
\curveto(194.56696665,76.14302105)(195.00945245,76.58550685)(195.55465817,76.58550685)
\curveto(196.09986389,76.58550685)(196.54234969,76.14302105)(196.54234969,75.59781533)
\closepath
}
}
{
\newrgbcolor{curcolor}{0 0 0}
\pscustom[linewidth=0.98769152,linecolor=curcolor]
{
\newpath
\moveto(157.251412,71.68711533)
\lineto(176.413702,71.68711533)
\lineto(176.413702,71.68711533)
}
}
{
\newrgbcolor{curcolor}{0 0 0}
\pscustom[linestyle=none,fillstyle=solid,fillcolor=curcolor]
{
\newpath
\moveto(158.21934969,71.68711533)
\curveto(158.21934969,71.14190961)(157.77686389,70.69942381)(157.23165817,70.69942381)
\curveto(156.68645245,70.69942381)(156.24396665,71.14190961)(156.24396665,71.68711533)
\curveto(156.24396665,72.23232105)(156.68645245,72.67480685)(157.23165817,72.67480685)
\curveto(157.77686389,72.67480685)(158.21934969,72.23232105)(158.21934969,71.68711533)
\closepath
}
}
{
\newrgbcolor{curcolor}{0 0 0}
\pscustom[linewidth=0.24692288,linecolor=curcolor]
{
\newpath
\moveto(158.21934969,71.68711533)
\curveto(158.21934969,71.14190961)(157.77686389,70.69942381)(157.23165817,70.69942381)
\curveto(156.68645245,70.69942381)(156.24396665,71.14190961)(156.24396665,71.68711533)
\curveto(156.24396665,72.23232105)(156.68645245,72.67480685)(157.23165817,72.67480685)
\curveto(157.77686389,72.67480685)(158.21934969,72.23232105)(158.21934969,71.68711533)
\closepath
}
}
{
\newrgbcolor{curcolor}{0 0 0}
\pscustom[linestyle=none,fillstyle=solid,fillcolor=curcolor]
{
\newpath
\moveto(177.38163969,71.68711533)
\curveto(177.38163969,71.14190961)(176.93915389,70.69942381)(176.39394817,70.69942381)
\curveto(175.84874245,70.69942381)(175.40625665,71.14190961)(175.40625665,71.68711533)
\curveto(175.40625665,72.23232105)(175.84874245,72.67480685)(176.39394817,72.67480685)
\curveto(176.93915389,72.67480685)(177.38163969,72.23232105)(177.38163969,71.68711533)
\closepath
}
}
{
\newrgbcolor{curcolor}{0 0 0}
\pscustom[linewidth=0.24692288,linecolor=curcolor]
{
\newpath
\moveto(177.38163969,71.68711533)
\curveto(177.38163969,71.14190961)(176.93915389,70.69942381)(176.39394817,70.69942381)
\curveto(175.84874245,70.69942381)(175.40625665,71.14190961)(175.40625665,71.68711533)
\curveto(175.40625665,72.23232105)(175.84874245,72.67480685)(176.39394817,72.67480685)
\curveto(176.93915389,72.67480685)(177.38163969,72.23232105)(177.38163969,71.68711533)
\closepath
}
}
{
\newrgbcolor{curcolor}{0 0 0}
\pscustom[linewidth=1,linecolor=curcolor,linestyle=dashed,dash=1 4]
{
\newpath
\moveto(214.738792,83.79903533)
\lineto(269.738792,83.79903533)
\lineto(269.738792,83.79903533)
\lineto(269.738792,83.79903533)
}
}
{
\newrgbcolor{curcolor}{0 0 0}
\pscustom[linestyle=none,fillstyle=solid,fillcolor=curcolor]
{
\newpath
\moveto(215.718792,83.79903533)
\curveto(215.718792,83.24703533)(215.270792,82.79903533)(214.718792,82.79903533)
\curveto(214.166792,82.79903533)(213.718792,83.24703533)(213.718792,83.79903533)
\curveto(213.718792,84.35103533)(214.166792,84.79903533)(214.718792,84.79903533)
\curveto(215.270792,84.79903533)(215.718792,84.35103533)(215.718792,83.79903533)
\closepath
}
}
{
\newrgbcolor{curcolor}{0 0 0}
\pscustom[linewidth=0.25,linecolor=curcolor]
{
\newpath
\moveto(215.718792,83.79903533)
\curveto(215.718792,83.24703533)(215.270792,82.79903533)(214.718792,82.79903533)
\curveto(214.166792,82.79903533)(213.718792,83.24703533)(213.718792,83.79903533)
\curveto(213.718792,84.35103533)(214.166792,84.79903533)(214.718792,84.79903533)
\curveto(215.270792,84.79903533)(215.718792,84.35103533)(215.718792,83.79903533)
\closepath
}
}
{
\newrgbcolor{curcolor}{0 0 0}
\pscustom[linestyle=none,fillstyle=solid,fillcolor=curcolor]
{
\newpath
\moveto(270.718792,83.79903533)
\curveto(270.718792,83.24703533)(270.270792,82.79903533)(269.718792,82.79903533)
\curveto(269.166792,82.79903533)(268.718792,83.24703533)(268.718792,83.79903533)
\curveto(268.718792,84.35103533)(269.166792,84.79903533)(269.718792,84.79903533)
\curveto(270.270792,84.79903533)(270.718792,84.35103533)(270.718792,83.79903533)
\closepath
}
}
{
\newrgbcolor{curcolor}{0 0 0}
\pscustom[linewidth=0.25,linecolor=curcolor]
{
\newpath
\moveto(270.718792,83.79903533)
\curveto(270.718792,83.24703533)(270.270792,82.79903533)(269.718792,82.79903533)
\curveto(269.166792,82.79903533)(268.718792,83.24703533)(268.718792,83.79903533)
\curveto(268.718792,84.35103533)(269.166792,84.79903533)(269.718792,84.79903533)
\curveto(270.270792,84.79903533)(270.718792,84.35103533)(270.718792,83.79903533)
\closepath
}
}
{
\newrgbcolor{curcolor}{0 0 0}
\pscustom[linewidth=0.98769152,linecolor=curcolor]
{
\newpath
\moveto(195.391872,63.29425533)
\lineto(214.554162,63.29425533)
\lineto(214.554162,63.29425533)
}
}
{
\newrgbcolor{curcolor}{0 0 0}
\pscustom[linestyle=none,fillstyle=solid,fillcolor=curcolor]
{
\newpath
\moveto(196.35980969,63.29425533)
\curveto(196.35980969,62.74904961)(195.91732389,62.30656381)(195.37211817,62.30656381)
\curveto(194.82691245,62.30656381)(194.38442665,62.74904961)(194.38442665,63.29425533)
\curveto(194.38442665,63.83946105)(194.82691245,64.28194685)(195.37211817,64.28194685)
\curveto(195.91732389,64.28194685)(196.35980969,63.83946105)(196.35980969,63.29425533)
\closepath
}
}
{
\newrgbcolor{curcolor}{0 0 0}
\pscustom[linewidth=0.24692288,linecolor=curcolor]
{
\newpath
\moveto(196.35980969,63.29425533)
\curveto(196.35980969,62.74904961)(195.91732389,62.30656381)(195.37211817,62.30656381)
\curveto(194.82691245,62.30656381)(194.38442665,62.74904961)(194.38442665,63.29425533)
\curveto(194.38442665,63.83946105)(194.82691245,64.28194685)(195.37211817,64.28194685)
\curveto(195.91732389,64.28194685)(196.35980969,63.83946105)(196.35980969,63.29425533)
\closepath
}
}
{
\newrgbcolor{curcolor}{0 0 0}
\pscustom[linestyle=none,fillstyle=solid,fillcolor=curcolor]
{
\newpath
\moveto(215.52209969,63.29425533)
\curveto(215.52209969,62.74904961)(215.07961389,62.30656381)(214.53440817,62.30656381)
\curveto(213.98920245,62.30656381)(213.54671665,62.74904961)(213.54671665,63.29425533)
\curveto(213.54671665,63.83946105)(213.98920245,64.28194685)(214.53440817,64.28194685)
\curveto(215.07961389,64.28194685)(215.52209969,63.83946105)(215.52209969,63.29425533)
\closepath
}
}
{
\newrgbcolor{curcolor}{0 0 0}
\pscustom[linewidth=0.24692288,linecolor=curcolor]
{
\newpath
\moveto(215.52209969,63.29425533)
\curveto(215.52209969,62.74904961)(215.07961389,62.30656381)(214.53440817,62.30656381)
\curveto(213.98920245,62.30656381)(213.54671665,62.74904961)(213.54671665,63.29425533)
\curveto(213.54671665,63.83946105)(213.98920245,64.28194685)(214.53440817,64.28194685)
\curveto(215.07961389,64.28194685)(215.52209969,63.83946105)(215.52209969,63.29425533)
\closepath
}
}
{
\newrgbcolor{curcolor}{0 0 0}
\pscustom[linewidth=0.98769152,linecolor=curcolor]
{
\newpath
\moveto(176.233552,59.34781533)
\lineto(195.395842,59.34781533)
\lineto(195.395842,59.34781533)
}
}
{
\newrgbcolor{curcolor}{0 0 0}
\pscustom[linestyle=none,fillstyle=solid,fillcolor=curcolor]
{
\newpath
\moveto(177.20148969,59.34781533)
\curveto(177.20148969,58.80260961)(176.75900389,58.36012381)(176.21379817,58.36012381)
\curveto(175.66859245,58.36012381)(175.22610665,58.80260961)(175.22610665,59.34781533)
\curveto(175.22610665,59.89302105)(175.66859245,60.33550685)(176.21379817,60.33550685)
\curveto(176.75900389,60.33550685)(177.20148969,59.89302105)(177.20148969,59.34781533)
\closepath
}
}
{
\newrgbcolor{curcolor}{0 0 0}
\pscustom[linewidth=0.24692288,linecolor=curcolor]
{
\newpath
\moveto(177.20148969,59.34781533)
\curveto(177.20148969,58.80260961)(176.75900389,58.36012381)(176.21379817,58.36012381)
\curveto(175.66859245,58.36012381)(175.22610665,58.80260961)(175.22610665,59.34781533)
\curveto(175.22610665,59.89302105)(175.66859245,60.33550685)(176.21379817,60.33550685)
\curveto(176.75900389,60.33550685)(177.20148969,59.89302105)(177.20148969,59.34781533)
\closepath
}
}
{
\newrgbcolor{curcolor}{0 0 0}
\pscustom[linestyle=none,fillstyle=solid,fillcolor=curcolor]
{
\newpath
\moveto(196.36377969,59.34781533)
\curveto(196.36377969,58.80260961)(195.92129389,58.36012381)(195.37608817,58.36012381)
\curveto(194.83088245,58.36012381)(194.38839665,58.80260961)(194.38839665,59.34781533)
\curveto(194.38839665,59.89302105)(194.83088245,60.33550685)(195.37608817,60.33550685)
\curveto(195.92129389,60.33550685)(196.36377969,59.89302105)(196.36377969,59.34781533)
\closepath
}
}
{
\newrgbcolor{curcolor}{0 0 0}
\pscustom[linewidth=0.24692288,linecolor=curcolor]
{
\newpath
\moveto(196.36377969,59.34781533)
\curveto(196.36377969,58.80260961)(195.92129389,58.36012381)(195.37608817,58.36012381)
\curveto(194.83088245,58.36012381)(194.38839665,58.80260961)(194.38839665,59.34781533)
\curveto(194.38839665,59.89302105)(194.83088245,60.33550685)(195.37608817,60.33550685)
\curveto(195.92129389,60.33550685)(196.36377969,59.89302105)(196.36377969,59.34781533)
\closepath
}
}
{
\newrgbcolor{curcolor}{0 0 0}
\pscustom[linewidth=1,linecolor=curcolor,linestyle=dashed,dash=1 4]
{
\newpath
\moveto(214.560222,67.54903533)
\lineto(269.560222,67.54903533)
\lineto(269.560222,67.54903533)
\lineto(269.560222,67.54903533)
}
}
{
\newrgbcolor{curcolor}{0 0 0}
\pscustom[linestyle=none,fillstyle=solid,fillcolor=curcolor]
{
\newpath
\moveto(215.540222,67.54903533)
\curveto(215.540222,66.99703533)(215.092222,66.54903533)(214.540222,66.54903533)
\curveto(213.988222,66.54903533)(213.540222,66.99703533)(213.540222,67.54903533)
\curveto(213.540222,68.10103533)(213.988222,68.54903533)(214.540222,68.54903533)
\curveto(215.092222,68.54903533)(215.540222,68.10103533)(215.540222,67.54903533)
\closepath
}
}
{
\newrgbcolor{curcolor}{0 0 0}
\pscustom[linewidth=0.25,linecolor=curcolor]
{
\newpath
\moveto(215.540222,67.54903533)
\curveto(215.540222,66.99703533)(215.092222,66.54903533)(214.540222,66.54903533)
\curveto(213.988222,66.54903533)(213.540222,66.99703533)(213.540222,67.54903533)
\curveto(213.540222,68.10103533)(213.988222,68.54903533)(214.540222,68.54903533)
\curveto(215.092222,68.54903533)(215.540222,68.10103533)(215.540222,67.54903533)
\closepath
}
}
{
\newrgbcolor{curcolor}{0 0 0}
\pscustom[linestyle=none,fillstyle=solid,fillcolor=curcolor]
{
\newpath
\moveto(270.540222,67.54903533)
\curveto(270.540222,66.99703533)(270.092222,66.54903533)(269.540222,66.54903533)
\curveto(268.988222,66.54903533)(268.540222,66.99703533)(268.540222,67.54903533)
\curveto(268.540222,68.10103533)(268.988222,68.54903533)(269.540222,68.54903533)
\curveto(270.092222,68.54903533)(270.540222,68.10103533)(270.540222,67.54903533)
\closepath
}
}
{
\newrgbcolor{curcolor}{0 0 0}
\pscustom[linewidth=0.25,linecolor=curcolor]
{
\newpath
\moveto(270.540222,67.54903533)
\curveto(270.540222,66.99703533)(270.092222,66.54903533)(269.540222,66.54903533)
\curveto(268.988222,66.54903533)(268.540222,66.99703533)(268.540222,67.54903533)
\curveto(268.540222,68.10103533)(268.988222,68.54903533)(269.540222,68.54903533)
\curveto(270.092222,68.54903533)(270.540222,68.10103533)(270.540222,67.54903533)
\closepath
}
}
{
\newrgbcolor{curcolor}{0 0 0}
\pscustom[linewidth=0.98769152,linecolor=curcolor]
{
\newpath
\moveto(195.570442,50.61568533)
\lineto(214.732732,50.61568533)
\lineto(214.732732,50.61568533)
}
}
{
\newrgbcolor{curcolor}{0 0 0}
\pscustom[linestyle=none,fillstyle=solid,fillcolor=curcolor]
{
\newpath
\moveto(196.53837969,50.61568533)
\curveto(196.53837969,50.07047961)(196.09589389,49.62799381)(195.55068817,49.62799381)
\curveto(195.00548245,49.62799381)(194.56299665,50.07047961)(194.56299665,50.61568533)
\curveto(194.56299665,51.16089105)(195.00548245,51.60337685)(195.55068817,51.60337685)
\curveto(196.09589389,51.60337685)(196.53837969,51.16089105)(196.53837969,50.61568533)
\closepath
}
}
{
\newrgbcolor{curcolor}{0 0 0}
\pscustom[linewidth=0.24692288,linecolor=curcolor]
{
\newpath
\moveto(196.53837969,50.61568533)
\curveto(196.53837969,50.07047961)(196.09589389,49.62799381)(195.55068817,49.62799381)
\curveto(195.00548245,49.62799381)(194.56299665,50.07047961)(194.56299665,50.61568533)
\curveto(194.56299665,51.16089105)(195.00548245,51.60337685)(195.55068817,51.60337685)
\curveto(196.09589389,51.60337685)(196.53837969,51.16089105)(196.53837969,50.61568533)
\closepath
}
}
{
\newrgbcolor{curcolor}{0 0 0}
\pscustom[linestyle=none,fillstyle=solid,fillcolor=curcolor]
{
\newpath
\moveto(215.70066969,50.61568533)
\curveto(215.70066969,50.07047961)(215.25818389,49.62799381)(214.71297817,49.62799381)
\curveto(214.16777245,49.62799381)(213.72528665,50.07047961)(213.72528665,50.61568533)
\curveto(213.72528665,51.16089105)(214.16777245,51.60337685)(214.71297817,51.60337685)
\curveto(215.25818389,51.60337685)(215.70066969,51.16089105)(215.70066969,50.61568533)
\closepath
}
}
{
\newrgbcolor{curcolor}{0 0 0}
\pscustom[linewidth=0.24692288,linecolor=curcolor]
{
\newpath
\moveto(215.70066969,50.61568533)
\curveto(215.70066969,50.07047961)(215.25818389,49.62799381)(214.71297817,49.62799381)
\curveto(214.16777245,49.62799381)(213.72528665,50.07047961)(213.72528665,50.61568533)
\curveto(213.72528665,51.16089105)(214.16777245,51.60337685)(214.71297817,51.60337685)
\curveto(215.25818389,51.60337685)(215.70066969,51.16089105)(215.70066969,50.61568533)
\closepath
}
}
{
\newrgbcolor{curcolor}{0 0 0}
\pscustom[linewidth=1,linecolor=curcolor,linestyle=dashed,dash=1 4]
{
\newpath
\moveto(214.738792,54.87046533)
\lineto(269.738792,54.87046533)
\lineto(269.738792,54.87046533)
\lineto(269.738792,54.87046533)
}
}
{
\newrgbcolor{curcolor}{0 0 0}
\pscustom[linestyle=none,fillstyle=solid,fillcolor=curcolor]
{
\newpath
\moveto(215.718792,54.87046533)
\curveto(215.718792,54.31846533)(215.270792,53.87046533)(214.718792,53.87046533)
\curveto(214.166792,53.87046533)(213.718792,54.31846533)(213.718792,54.87046533)
\curveto(213.718792,55.42246533)(214.166792,55.87046533)(214.718792,55.87046533)
\curveto(215.270792,55.87046533)(215.718792,55.42246533)(215.718792,54.87046533)
\closepath
}
}
{
\newrgbcolor{curcolor}{0 0 0}
\pscustom[linewidth=0.25,linecolor=curcolor]
{
\newpath
\moveto(215.718792,54.87046533)
\curveto(215.718792,54.31846533)(215.270792,53.87046533)(214.718792,53.87046533)
\curveto(214.166792,53.87046533)(213.718792,54.31846533)(213.718792,54.87046533)
\curveto(213.718792,55.42246533)(214.166792,55.87046533)(214.718792,55.87046533)
\curveto(215.270792,55.87046533)(215.718792,55.42246533)(215.718792,54.87046533)
\closepath
}
}
{
\newrgbcolor{curcolor}{0 0 0}
\pscustom[linestyle=none,fillstyle=solid,fillcolor=curcolor]
{
\newpath
\moveto(270.718792,54.87046533)
\curveto(270.718792,54.31846533)(270.270792,53.87046533)(269.718792,53.87046533)
\curveto(269.166792,53.87046533)(268.718792,54.31846533)(268.718792,54.87046533)
\curveto(268.718792,55.42246533)(269.166792,55.87046533)(269.718792,55.87046533)
\curveto(270.270792,55.87046533)(270.718792,55.42246533)(270.718792,54.87046533)
\closepath
}
}
{
\newrgbcolor{curcolor}{0 0 0}
\pscustom[linewidth=0.25,linecolor=curcolor]
{
\newpath
\moveto(270.718792,54.87046533)
\curveto(270.718792,54.31846533)(270.270792,53.87046533)(269.718792,53.87046533)
\curveto(269.166792,53.87046533)(268.718792,54.31846533)(268.718792,54.87046533)
\curveto(268.718792,55.42246533)(269.166792,55.87046533)(269.718792,55.87046533)
\curveto(270.270792,55.87046533)(270.718792,55.42246533)(270.718792,54.87046533)
\closepath
}
}
{
\newrgbcolor{curcolor}{0 1 0}
\pscustom[linewidth=1,linecolor=curcolor]
{
\newpath
\moveto(262.142862,186.42856533)
\lineto(210.772222,186.42856533)
\lineto(211.158332,180.62738533)
\lineto(191.872622,180.62738533)
\lineto(192.258732,175.74325533)
\lineto(173.214292,175.35714533)
\lineto(173.214292,171.78571533)
\lineto(153.214292,171.78571533)
\lineto(153.214292,167.49999533)
\lineto(133.928572,167.49999533)
\lineto(133.928572,164.99999533)
\lineto(114.285712,164.99999533)
\lineto(114.285712,160.35714533)
\lineto(96.785712,160.35714533)
\lineto(96.785712,156.78571533)
\lineto(80.178572,156.87501533)
}
}
{
\newrgbcolor{curcolor}{0 1 0}
\pscustom[linestyle=none,fillstyle=solid,fillcolor=curcolor]
{
\newpath
\moveto(261.162862,186.42856533)
\curveto(261.162862,186.98056533)(261.610862,187.42856533)(262.162862,187.42856533)
\curveto(262.714862,187.42856533)(263.162862,186.98056533)(263.162862,186.42856533)
\curveto(263.162862,185.87656533)(262.714862,185.42856533)(262.162862,185.42856533)
\curveto(261.610862,185.42856533)(261.162862,185.87656533)(261.162862,186.42856533)
\closepath
}
}
{
\newrgbcolor{curcolor}{0 1 0}
\pscustom[linewidth=0.25,linecolor=curcolor]
{
\newpath
\moveto(261.162862,186.42856533)
\curveto(261.162862,186.98056533)(261.610862,187.42856533)(262.162862,187.42856533)
\curveto(262.714862,187.42856533)(263.162862,186.98056533)(263.162862,186.42856533)
\curveto(263.162862,185.87656533)(262.714862,185.42856533)(262.162862,185.42856533)
\curveto(261.610862,185.42856533)(261.162862,185.87656533)(261.162862,186.42856533)
\closepath
}
}
{
\newrgbcolor{curcolor}{0 1 0}
\pscustom[linestyle=none,fillstyle=solid,fillcolor=curcolor]
{
\newpath
\moveto(79.19858617,156.88028491)
\curveto(79.20155434,157.43227693)(79.65195682,157.8778615)(80.20394884,157.87489333)
\curveto(80.75594086,157.87192515)(81.20152543,157.42152268)(81.19855725,156.86953066)
\curveto(81.19558908,156.31753864)(80.7451866,155.87195407)(80.19319458,155.87492224)
\curveto(79.64120256,155.87789042)(79.19561799,156.32829289)(79.19858617,156.88028491)
\closepath
}
}
{
\newrgbcolor{curcolor}{0 1 0}
\pscustom[linewidth=0.25,linecolor=curcolor]
{
\newpath
\moveto(79.19858617,156.88028491)
\curveto(79.20155434,157.43227693)(79.65195682,157.8778615)(80.20394884,157.87489333)
\curveto(80.75594086,157.87192515)(81.20152543,157.42152268)(81.19855725,156.86953066)
\curveto(81.19558908,156.31753864)(80.7451866,155.87195407)(80.19319458,155.87492224)
\curveto(79.64120256,155.87789042)(79.19561799,156.32829289)(79.19858617,156.88028491)
\closepath
}
}
{
\newrgbcolor{curcolor}{0 1 0}
\pscustom[linewidth=0.1986801,linecolor=curcolor]
{
\newpath
\moveto(61.071429,189.28571533)
\lineto(60.803571,189.28571533)
}
}
{
\newrgbcolor{curcolor}{0 1 0}
\pscustom[linestyle=none,fillstyle=solid,fillcolor=curcolor]
{
\newpath
\moveto(60.292603,189.28571533)
\curveto(60.292603,189.724401)(60.64863774,190.08043574)(61.08732341,190.08043574)
\curveto(61.52600908,190.08043574)(61.88204382,189.724401)(61.88204382,189.28571533)
\curveto(61.88204382,188.84702966)(61.52600908,188.49099492)(61.08732341,188.49099492)
\curveto(60.64863774,188.49099492)(60.292603,188.84702966)(60.292603,189.28571533)
\closepath
}
}
{
\newrgbcolor{curcolor}{0 1 0}
\pscustom[linewidth=0.1986801,linecolor=curcolor]
{
\newpath
\moveto(60.292603,189.28571533)
\curveto(60.292603,189.724401)(60.64863774,190.08043574)(61.08732341,190.08043574)
\curveto(61.52600908,190.08043574)(61.88204382,189.724401)(61.88204382,189.28571533)
\curveto(61.88204382,188.84702966)(61.52600908,188.49099492)(61.08732341,188.49099492)
\curveto(60.64863774,188.49099492)(60.292603,188.84702966)(60.292603,189.28571533)
\closepath
}
}
{
\newrgbcolor{curcolor}{0 1 0}
\pscustom[linewidth=1,linecolor=curcolor]
{
\newpath
\moveto(229.642862,154.99999533)
\lineto(212.142862,154.99999533)
\lineto(212.142862,151.07142533)
\lineto(191.785712,151.07142533)
\lineto(191.785712,146.07142533)
\lineto(172.500002,146.07142533)
\lineto(172.500002,143.21428533)
\lineto(153.214292,143.21428533)
\lineto(153.214292,138.21428533)
\lineto(133.928572,138.21428533)
\lineto(133.928572,134.64285533)
\lineto(115.000002,134.64285533)
\lineto(115.000002,131.42856533)
\lineto(99.285712,131.42856533)
}
}
{
\newrgbcolor{curcolor}{0 1 0}
\pscustom[linestyle=none,fillstyle=solid,fillcolor=curcolor]
{
\newpath
\moveto(228.662862,154.99999533)
\curveto(228.662862,155.55199533)(229.110862,155.99999533)(229.662862,155.99999533)
\curveto(230.214862,155.99999533)(230.662862,155.55199533)(230.662862,154.99999533)
\curveto(230.662862,154.44799533)(230.214862,153.99999533)(229.662862,153.99999533)
\curveto(229.110862,153.99999533)(228.662862,154.44799533)(228.662862,154.99999533)
\closepath
}
}
{
\newrgbcolor{curcolor}{0 1 0}
\pscustom[linewidth=0.25,linecolor=curcolor]
{
\newpath
\moveto(228.662862,154.99999533)
\curveto(228.662862,155.55199533)(229.110862,155.99999533)(229.662862,155.99999533)
\curveto(230.214862,155.99999533)(230.662862,155.55199533)(230.662862,154.99999533)
\curveto(230.662862,154.44799533)(230.214862,153.99999533)(229.662862,153.99999533)
\curveto(229.110862,153.99999533)(228.662862,154.44799533)(228.662862,154.99999533)
\closepath
}
}
{
\newrgbcolor{curcolor}{0 1 0}
\pscustom[linestyle=none,fillstyle=solid,fillcolor=curcolor]
{
\newpath
\moveto(98.305712,131.42856533)
\curveto(98.305712,131.98056533)(98.753712,132.42856533)(99.305712,132.42856533)
\curveto(99.857712,132.42856533)(100.305712,131.98056533)(100.305712,131.42856533)
\curveto(100.305712,130.87656533)(99.857712,130.42856533)(99.305712,130.42856533)
\curveto(98.753712,130.42856533)(98.305712,130.87656533)(98.305712,131.42856533)
\closepath
}
}
{
\newrgbcolor{curcolor}{0 1 0}
\pscustom[linewidth=0.25,linecolor=curcolor]
{
\newpath
\moveto(98.305712,131.42856533)
\curveto(98.305712,131.98056533)(98.753712,132.42856533)(99.305712,132.42856533)
\curveto(99.857712,132.42856533)(100.305712,131.98056533)(100.305712,131.42856533)
\curveto(100.305712,130.87656533)(99.857712,130.42856533)(99.305712,130.42856533)
\curveto(98.753712,130.42856533)(98.305712,130.87656533)(98.305712,131.42856533)
\closepath
}
}
{
\newrgbcolor{curcolor}{0 1 0}
\pscustom[linewidth=1,linecolor=curcolor]
{
\newpath
\moveto(242.500002,128.92856533)
\lineto(211.428572,128.92856533)
\lineto(211.428572,124.64285533)
\lineto(192.500002,124.64285533)
\lineto(192.500002,120.71428533)
\lineto(173.571432,120.71428533)
\lineto(173.571432,116.78571533)
\lineto(153.571432,116.78571533)
\lineto(153.571432,112.85714533)
\lineto(134.285712,112.85714533)
\lineto(134.285712,109.28571533)
\lineto(118.214292,109.28571533)
\lineto(118.214292,109.28571533)
}
}
{
\newrgbcolor{curcolor}{0 1 0}
\pscustom[linestyle=none,fillstyle=solid,fillcolor=curcolor]
{
\newpath
\moveto(241.520002,128.92856533)
\curveto(241.520002,129.48056533)(241.968002,129.92856533)(242.520002,129.92856533)
\curveto(243.072002,129.92856533)(243.520002,129.48056533)(243.520002,128.92856533)
\curveto(243.520002,128.37656533)(243.072002,127.92856533)(242.520002,127.92856533)
\curveto(241.968002,127.92856533)(241.520002,128.37656533)(241.520002,128.92856533)
\closepath
}
}
{
\newrgbcolor{curcolor}{0 1 0}
\pscustom[linewidth=0.25,linecolor=curcolor]
{
\newpath
\moveto(241.520002,128.92856533)
\curveto(241.520002,129.48056533)(241.968002,129.92856533)(242.520002,129.92856533)
\curveto(243.072002,129.92856533)(243.520002,129.48056533)(243.520002,128.92856533)
\curveto(243.520002,128.37656533)(243.072002,127.92856533)(242.520002,127.92856533)
\curveto(241.968002,127.92856533)(241.520002,128.37656533)(241.520002,128.92856533)
\closepath
}
}
{
\newrgbcolor{curcolor}{0 1 0}
\pscustom[linestyle=none,fillstyle=solid,fillcolor=curcolor]
{
\newpath
\moveto(119.194292,109.28571533)
\curveto(119.194292,108.73371533)(118.746292,108.28571533)(118.194292,108.28571533)
\curveto(117.642292,108.28571533)(117.194292,108.73371533)(117.194292,109.28571533)
\curveto(117.194292,109.83771533)(117.642292,110.28571533)(118.194292,110.28571533)
\curveto(118.746292,110.28571533)(119.194292,109.83771533)(119.194292,109.28571533)
\closepath
}
}
{
\newrgbcolor{curcolor}{0 1 0}
\pscustom[linewidth=0.25,linecolor=curcolor]
{
\newpath
\moveto(119.194292,109.28571533)
\curveto(119.194292,108.73371533)(118.746292,108.28571533)(118.194292,108.28571533)
\curveto(117.642292,108.28571533)(117.194292,108.73371533)(117.194292,109.28571533)
\curveto(117.194292,109.83771533)(117.642292,110.28571533)(118.194292,110.28571533)
\curveto(118.746292,110.28571533)(119.194292,109.83771533)(119.194292,109.28571533)
\closepath
}
}
{
\newrgbcolor{curcolor}{0 1 0}
\pscustom[linewidth=1,linecolor=curcolor]
{
\newpath
\moveto(221.658792,106.07142533)
\lineto(211.785712,106.07142533)
\lineto(211.785712,101.07142533)
\lineto(192.142862,101.07142533)
\lineto(192.142862,97.85713533)
\lineto(173.214292,97.85713533)
\lineto(173.214292,94.28571533)
\lineto(153.928572,94.28571533)
\lineto(153.928572,90.71428533)
\lineto(137.500002,90.71428533)
\lineto(137.500002,90.35713533)
}
}
{
\newrgbcolor{curcolor}{0 1 0}
\pscustom[linestyle=none,fillstyle=solid,fillcolor=curcolor]
{
\newpath
\moveto(220.678792,106.07142533)
\curveto(220.678792,106.62342533)(221.126792,107.07142533)(221.678792,107.07142533)
\curveto(222.230792,107.07142533)(222.678792,106.62342533)(222.678792,106.07142533)
\curveto(222.678792,105.51942533)(222.230792,105.07142533)(221.678792,105.07142533)
\curveto(221.126792,105.07142533)(220.678792,105.51942533)(220.678792,106.07142533)
\closepath
}
}
{
\newrgbcolor{curcolor}{0 1 0}
\pscustom[linewidth=0.25,linecolor=curcolor]
{
\newpath
\moveto(220.678792,106.07142533)
\curveto(220.678792,106.62342533)(221.126792,107.07142533)(221.678792,107.07142533)
\curveto(222.230792,107.07142533)(222.678792,106.62342533)(222.678792,106.07142533)
\curveto(222.678792,105.51942533)(222.230792,105.07142533)(221.678792,105.07142533)
\curveto(221.126792,105.07142533)(220.678792,105.51942533)(220.678792,106.07142533)
\closepath
}
}
{
\newrgbcolor{curcolor}{0 1 0}
\pscustom[linestyle=none,fillstyle=solid,fillcolor=curcolor]
{
\newpath
\moveto(137.500002,89.37713533)
\curveto(136.948002,89.37713533)(136.500002,89.82513533)(136.500002,90.37713533)
\curveto(136.500002,90.92913533)(136.948002,91.37713533)(137.500002,91.37713533)
\curveto(138.052002,91.37713533)(138.500002,90.92913533)(138.500002,90.37713533)
\curveto(138.500002,89.82513533)(138.052002,89.37713533)(137.500002,89.37713533)
\closepath
}
}
{
\newrgbcolor{curcolor}{0 1 0}
\pscustom[linewidth=0.25,linecolor=curcolor]
{
\newpath
\moveto(137.500002,89.37713533)
\curveto(136.948002,89.37713533)(136.500002,89.82513533)(136.500002,90.37713533)
\curveto(136.500002,90.92913533)(136.948002,91.37713533)(137.500002,91.37713533)
\curveto(138.052002,91.37713533)(138.500002,90.92913533)(138.500002,90.37713533)
\curveto(138.500002,89.82513533)(138.052002,89.37713533)(137.500002,89.37713533)
\closepath
}
}
{
\newrgbcolor{curcolor}{0 1 0}
\pscustom[linewidth=1,linecolor=curcolor]
{
\newpath
\moveto(258.928572,86.78571533)
\lineto(211.428572,86.78571533)
\lineto(211.428572,82.14285533)
\lineto(191.785712,82.14285533)
\lineto(191.785712,77.85713533)
\lineto(172.857142,77.85713533)
\lineto(172.857142,73.57142533)
\lineto(156.785712,73.57142533)
\lineto(156.785712,73.92856533)
}
}
{
\newrgbcolor{curcolor}{0 1 0}
\pscustom[linestyle=none,fillstyle=solid,fillcolor=curcolor]
{
\newpath
\moveto(257.948572,86.78571533)
\curveto(257.948572,87.33771533)(258.396572,87.78571533)(258.948572,87.78571533)
\curveto(259.500572,87.78571533)(259.948572,87.33771533)(259.948572,86.78571533)
\curveto(259.948572,86.23371533)(259.500572,85.78571533)(258.948572,85.78571533)
\curveto(258.396572,85.78571533)(257.948572,86.23371533)(257.948572,86.78571533)
\closepath
}
}
{
\newrgbcolor{curcolor}{0 1 0}
\pscustom[linewidth=0.25,linecolor=curcolor]
{
\newpath
\moveto(257.948572,86.78571533)
\curveto(257.948572,87.33771533)(258.396572,87.78571533)(258.948572,87.78571533)
\curveto(259.500572,87.78571533)(259.948572,87.33771533)(259.948572,86.78571533)
\curveto(259.948572,86.23371533)(259.500572,85.78571533)(258.948572,85.78571533)
\curveto(258.396572,85.78571533)(257.948572,86.23371533)(257.948572,86.78571533)
\closepath
}
}
{
\newrgbcolor{curcolor}{0 1 0}
\pscustom[linestyle=none,fillstyle=solid,fillcolor=curcolor]
{
\newpath
\moveto(156.785712,74.90856533)
\curveto(157.337712,74.90856533)(157.785712,74.46056533)(157.785712,73.90856533)
\curveto(157.785712,73.35656533)(157.337712,72.90856533)(156.785712,72.90856533)
\curveto(156.233712,72.90856533)(155.785712,73.35656533)(155.785712,73.90856533)
\curveto(155.785712,74.46056533)(156.233712,74.90856533)(156.785712,74.90856533)
\closepath
}
}
{
\newrgbcolor{curcolor}{0 1 0}
\pscustom[linewidth=0.25,linecolor=curcolor]
{
\newpath
\moveto(156.785712,74.90856533)
\curveto(157.337712,74.90856533)(157.785712,74.46056533)(157.785712,73.90856533)
\curveto(157.785712,73.35656533)(157.337712,72.90856533)(156.785712,72.90856533)
\curveto(156.233712,72.90856533)(155.785712,73.35656533)(155.785712,73.90856533)
\curveto(155.785712,74.46056533)(156.233712,74.90856533)(156.785712,74.90856533)
\closepath
}
}
{
\newrgbcolor{curcolor}{0 0 0}
\pscustom[linestyle=none,fillstyle=solid,fillcolor=curcolor]
{
\newpath
\moveto(225.0880883,162.98946273)
\curveto(225.4280883,163.12279606)(225.7580883,163.28946273)(226.0780883,163.48946273)
\curveto(226.3980883,163.69612939)(226.69475497,163.95612939)(226.9680883,164.26946273)
\lineto(227.5480883,164.26946273)
\lineto(227.5480883,158.77946273)
\lineto(228.7180883,158.77946273)
\lineto(228.7180883,158.07946273)
\lineto(225.3880883,158.07946273)
\lineto(225.3880883,158.77946273)
\lineto(226.7280883,158.77946273)
\lineto(226.7280883,163.11946273)
\curveto(226.65475497,163.05279606)(226.56475497,162.98279606)(226.4580883,162.90946273)
\curveto(226.3580883,162.84279606)(226.24475497,162.77612939)(226.1180883,162.70946273)
\curveto(225.9980883,162.64279606)(225.87142164,162.57946273)(225.7380883,162.51946273)
\curveto(225.60475497,162.45946273)(225.47475497,162.40946273)(225.3480883,162.36946273)
\lineto(225.0880883,162.98946273)
\closepath
}
}
{
\newrgbcolor{curcolor}{0 0 0}
\pscustom[linestyle=none,fillstyle=solid,fillcolor=curcolor]
{
\newpath
\moveto(231.1980883,161.87946273)
\curveto(232.09142164,161.84612939)(232.74142164,161.64946273)(233.1480883,161.28946273)
\curveto(233.55475497,160.92946273)(233.7580883,160.44612939)(233.7580883,159.83946273)
\curveto(233.7580883,159.56612939)(233.71475497,159.31279606)(233.6280883,159.07946273)
\curveto(233.54142164,158.84612939)(233.40475497,158.64612939)(233.2180883,158.47946273)
\curveto(233.0380883,158.31279606)(232.80475497,158.18279606)(232.5180883,158.08946273)
\curveto(232.2380883,157.99612939)(231.90475497,157.94946273)(231.5180883,157.94946273)
\curveto(231.3580883,157.94946273)(231.1980883,157.96279606)(231.0380883,157.98946273)
\curveto(230.88475497,158.00946273)(230.7380883,158.03612939)(230.5980883,158.06946273)
\curveto(230.4580883,158.10279606)(230.33475497,158.13612939)(230.2280883,158.16946273)
\curveto(230.12142164,158.20946273)(230.04475497,158.24279606)(229.9980883,158.26946273)
\lineto(230.1580883,158.97946273)
\curveto(230.26475497,158.92612939)(230.4280883,158.86279606)(230.6480883,158.78946273)
\curveto(230.87475497,158.71612939)(231.1580883,158.67946273)(231.4980883,158.67946273)
\curveto(231.76475497,158.67946273)(231.9880883,158.70946273)(232.1680883,158.76946273)
\curveto(232.3480883,158.82946273)(232.49142164,158.90946273)(232.5980883,159.00946273)
\curveto(232.71142164,159.11612939)(232.79142164,159.23612939)(232.8380883,159.36946273)
\curveto(232.89142164,159.50279606)(232.9180883,159.64279606)(232.9180883,159.78946273)
\curveto(232.9180883,160.01612939)(232.8780883,160.21612939)(232.7980883,160.38946273)
\curveto(232.72475497,160.56946273)(232.59142164,160.71946273)(232.3980883,160.83946273)
\curveto(232.21142164,160.95946273)(231.95142164,161.04946273)(231.6180883,161.10946273)
\curveto(231.29142164,161.17612939)(230.8780883,161.20946273)(230.3780883,161.20946273)
\curveto(230.4180883,161.50279606)(230.4480883,161.77612939)(230.4680883,162.02946273)
\curveto(230.49475497,162.28946273)(230.51475497,162.53946273)(230.5280883,162.77946273)
\curveto(230.5480883,163.02612939)(230.56142164,163.26946273)(230.5680883,163.50946273)
\curveto(230.58142164,163.74946273)(230.59475497,164.00279606)(230.6080883,164.26946273)
\lineto(233.5880883,164.26946273)
\lineto(233.5880883,163.56946273)
\lineto(231.3280883,163.56946273)
\curveto(231.32142164,163.47612939)(231.31142164,163.35279606)(231.2980883,163.19946273)
\curveto(231.29142164,163.05279606)(231.28142164,162.89612939)(231.2680883,162.72946273)
\curveto(231.25475497,162.56279606)(231.24142164,162.40279606)(231.2280883,162.24946273)
\curveto(231.21475497,162.09612939)(231.20475497,161.97279606)(231.1980883,161.87946273)
\closepath
}
}
{
\newrgbcolor{curcolor}{0 0 0}
\pscustom[linestyle=none,fillstyle=solid,fillcolor=curcolor]
{
\newpath
\moveto(241.64311455,136.47941024)
\curveto(241.64311455,136.2660769)(241.59978122,136.05941024)(241.51311455,135.85941024)
\curveto(241.43311455,135.65941024)(241.32311455,135.46274357)(241.18311455,135.26941024)
\curveto(241.04978122,135.0760769)(240.89644788,134.8860769)(240.72311455,134.69941024)
\curveto(240.54978122,134.51274357)(240.37311455,134.32941024)(240.19311455,134.14941024)
\curveto(240.09311455,134.04941024)(239.97644788,133.92941024)(239.84311455,133.78941024)
\curveto(239.70978122,133.64941024)(239.58311455,133.5060769)(239.46311455,133.35941024)
\curveto(239.34311455,133.21274357)(239.24311455,133.06941024)(239.16311455,132.92941024)
\curveto(239.08311455,132.78941024)(239.04311455,132.66941024)(239.04311455,132.56941024)
\lineto(241.89311455,132.56941024)
\lineto(241.89311455,131.86941024)
\lineto(238.14311455,131.86941024)
\curveto(238.13644788,131.90274357)(238.13311455,131.9360769)(238.13311455,131.96941024)
\lineto(238.13311455,132.07941024)
\curveto(238.13311455,132.35941024)(238.17978122,132.61941024)(238.27311455,132.85941024)
\curveto(238.36644788,133.09941024)(238.48644788,133.3260769)(238.63311455,133.53941024)
\curveto(238.77978122,133.75274357)(238.94311455,133.95274357)(239.12311455,134.13941024)
\curveto(239.30978122,134.33274357)(239.49311455,134.51941024)(239.67311455,134.69941024)
\curveto(239.81978122,134.8460769)(239.95978122,134.98941024)(240.09311455,135.12941024)
\curveto(240.23311455,135.26941024)(240.35311455,135.40941024)(240.45311455,135.54941024)
\curveto(240.55978122,135.68941024)(240.64311455,135.83274357)(240.70311455,135.97941024)
\curveto(240.76978122,136.13274357)(240.80311455,136.28941024)(240.80311455,136.44941024)
\curveto(240.80311455,136.62941024)(240.77311455,136.78274357)(240.71311455,136.90941024)
\curveto(240.65978122,137.0360769)(240.58311455,137.13941024)(240.48311455,137.21941024)
\curveto(240.38978122,137.3060769)(240.27978122,137.36941024)(240.15311455,137.40941024)
\curveto(240.02644788,137.44941024)(239.88978122,137.46941024)(239.74311455,137.46941024)
\curveto(239.56978122,137.46941024)(239.40978122,137.4460769)(239.26311455,137.39941024)
\curveto(239.12311455,137.35274357)(238.99644788,137.2960769)(238.88311455,137.22941024)
\curveto(238.77644788,137.16274357)(238.68311455,137.0960769)(238.60311455,137.02941024)
\curveto(238.52311455,136.96941024)(238.46311455,136.91941024)(238.42311455,136.87941024)
\lineto(238.01311455,137.45941024)
\curveto(238.06644788,137.51941024)(238.14644788,137.59274357)(238.25311455,137.67941024)
\curveto(238.35978122,137.7660769)(238.48644788,137.8460769)(238.63311455,137.91941024)
\curveto(238.78644788,137.99941024)(238.95644788,138.0660769)(239.14311455,138.11941024)
\curveto(239.32978122,138.17274357)(239.52978122,138.19941024)(239.74311455,138.19941024)
\curveto(240.38978122,138.19941024)(240.86644788,138.04941024)(241.17311455,137.74941024)
\curveto(241.48644788,137.4560769)(241.64311455,137.03274357)(241.64311455,136.47941024)
\closepath
}
}
{
\newrgbcolor{curcolor}{0 0 0}
\pscustom[linestyle=none,fillstyle=solid,fillcolor=curcolor]
{
\newpath
\moveto(243.19311455,136.77941024)
\curveto(243.53311455,136.91274357)(243.86311455,137.07941024)(244.18311455,137.27941024)
\curveto(244.50311455,137.4860769)(244.79978122,137.7460769)(245.07311455,138.05941024)
\lineto(245.65311455,138.05941024)
\lineto(245.65311455,132.56941024)
\lineto(246.82311455,132.56941024)
\lineto(246.82311455,131.86941024)
\lineto(243.49311455,131.86941024)
\lineto(243.49311455,132.56941024)
\lineto(244.83311455,132.56941024)
\lineto(244.83311455,136.90941024)
\curveto(244.75978122,136.84274357)(244.66978122,136.77274357)(244.56311455,136.69941024)
\curveto(244.46311455,136.63274357)(244.34978122,136.5660769)(244.22311455,136.49941024)
\curveto(244.10311455,136.43274357)(243.97644788,136.36941024)(243.84311455,136.30941024)
\curveto(243.70978122,136.24941024)(243.57978122,136.19941024)(243.45311455,136.15941024)
\lineto(243.19311455,136.77941024)
\closepath
}
}
{
\newrgbcolor{curcolor}{0 0 0}
\pscustom[linestyle=none,fillstyle=solid,fillcolor=curcolor]
{
\newpath
\moveto(217.44348076,113.84557234)
\curveto(217.78348076,113.97890568)(218.11348076,114.14557234)(218.43348076,114.34557234)
\curveto(218.75348076,114.55223901)(219.05014743,114.81223901)(219.32348076,115.12557234)
\lineto(219.90348076,115.12557234)
\lineto(219.90348076,109.63557234)
\lineto(221.07348076,109.63557234)
\lineto(221.07348076,108.93557234)
\lineto(217.74348076,108.93557234)
\lineto(217.74348076,109.63557234)
\lineto(219.08348076,109.63557234)
\lineto(219.08348076,113.97557234)
\curveto(219.01014743,113.90890568)(218.92014743,113.83890568)(218.81348076,113.76557234)
\curveto(218.71348076,113.69890568)(218.60014743,113.63223901)(218.47348076,113.56557234)
\curveto(218.35348076,113.49890568)(218.22681409,113.43557234)(218.09348076,113.37557234)
\curveto(217.96014743,113.31557234)(217.83014743,113.26557234)(217.70348076,113.22557234)
\lineto(217.44348076,113.84557234)
\closepath
}
}
{
\newrgbcolor{curcolor}{0 0 0}
\pscustom[linestyle=none,fillstyle=solid,fillcolor=curcolor]
{
\newpath
\moveto(223.81348076,109.53557234)
\curveto(224.34014743,109.53557234)(224.71348076,109.63890568)(224.93348076,109.84557234)
\curveto(225.16014743,110.05890568)(225.27348076,110.34223901)(225.27348076,110.69557234)
\curveto(225.27348076,110.92223901)(225.22681409,111.11223901)(225.13348076,111.26557234)
\curveto(225.04014743,111.41890568)(224.91681409,111.54223901)(224.76348076,111.63557234)
\curveto(224.61014743,111.72890568)(224.43348076,111.79557234)(224.23348076,111.83557234)
\curveto(224.03348076,111.87557234)(223.82348076,111.89557234)(223.60348076,111.89557234)
\lineto(223.39348076,111.89557234)
\lineto(223.39348076,112.56557234)
\lineto(223.68348076,112.56557234)
\curveto(223.83014743,112.56557234)(223.98014743,112.57890568)(224.13348076,112.60557234)
\curveto(224.29348076,112.63890568)(224.43681409,112.69223901)(224.56348076,112.76557234)
\curveto(224.69681409,112.84557234)(224.80348076,112.95223901)(224.88348076,113.08557234)
\curveto(224.96348076,113.21890568)(225.00348076,113.38890568)(225.00348076,113.59557234)
\curveto(225.00348076,113.93557234)(224.89681409,114.17557234)(224.68348076,114.31557234)
\curveto(224.47681409,114.46223901)(224.23348076,114.53557234)(223.95348076,114.53557234)
\curveto(223.66681409,114.53557234)(223.42348076,114.49223901)(223.22348076,114.40557234)
\curveto(223.02348076,114.32557234)(222.85681409,114.24223901)(222.72348076,114.15557234)
\lineto(222.40348076,114.78557234)
\curveto(222.54348076,114.88557234)(222.75348076,114.98890568)(223.03348076,115.09557234)
\curveto(223.32014743,115.20890568)(223.63681409,115.26557234)(223.98348076,115.26557234)
\curveto(224.31014743,115.26557234)(224.59014743,115.22557234)(224.82348076,115.14557234)
\curveto(225.05681409,115.06557234)(225.24681409,114.95223901)(225.39348076,114.80557234)
\curveto(225.54681409,114.65890568)(225.66014743,114.48557234)(225.73348076,114.28557234)
\curveto(225.80681409,114.09223901)(225.84348076,113.87890568)(225.84348076,113.64557234)
\curveto(225.84348076,113.31890568)(225.75681409,113.04223901)(225.58348076,112.81557234)
\curveto(225.41681409,112.58890568)(225.20014743,112.41557234)(224.93348076,112.29557234)
\curveto(225.25348076,112.20223901)(225.53014743,112.01890568)(225.76348076,111.74557234)
\curveto(225.99681409,111.47890568)(226.11348076,111.12223901)(226.11348076,110.67557234)
\curveto(226.11348076,110.40890568)(226.06681409,110.15890568)(225.97348076,109.92557234)
\curveto(225.88681409,109.69890568)(225.75014743,109.50223901)(225.56348076,109.33557234)
\curveto(225.38348076,109.16890568)(225.14681409,109.03890568)(224.85348076,108.94557234)
\curveto(224.56681409,108.85223901)(224.22348076,108.80557234)(223.82348076,108.80557234)
\curveto(223.67014743,108.80557234)(223.51014743,108.81890568)(223.34348076,108.84557234)
\curveto(223.18348076,108.86557234)(223.03348076,108.89557234)(222.89348076,108.93557234)
\curveto(222.75348076,108.96890568)(222.62681409,109.00223901)(222.51348076,109.03557234)
\curveto(222.40681409,109.07557234)(222.33014743,109.10557234)(222.28348076,109.12557234)
\lineto(222.44348076,109.83557234)
\curveto(222.55014743,109.78223901)(222.72014743,109.71890568)(222.95348076,109.64557234)
\curveto(223.18681409,109.57223901)(223.47348076,109.53557234)(223.81348076,109.53557234)
\closepath
}
}
{
\newrgbcolor{curcolor}{0 0 0}
\pscustom[linestyle=none,fillstyle=solid,fillcolor=curcolor]
{
\newpath
\moveto(256.21744194,89.87804061)
\curveto(256.74410861,89.87804061)(257.11744194,89.98137394)(257.33744194,90.18804061)
\curveto(257.56410861,90.40137394)(257.67744194,90.68470727)(257.67744194,91.03804061)
\curveto(257.67744194,91.26470727)(257.63077527,91.45470727)(257.53744194,91.60804061)
\curveto(257.44410861,91.76137394)(257.32077527,91.88470727)(257.16744194,91.97804061)
\curveto(257.01410861,92.07137394)(256.83744194,92.13804061)(256.63744194,92.17804061)
\curveto(256.43744194,92.21804061)(256.22744194,92.23804061)(256.00744194,92.23804061)
\lineto(255.79744194,92.23804061)
\lineto(255.79744194,92.90804061)
\lineto(256.08744194,92.90804061)
\curveto(256.23410861,92.90804061)(256.38410861,92.92137394)(256.53744194,92.94804061)
\curveto(256.69744194,92.98137394)(256.84077527,93.03470727)(256.96744194,93.10804061)
\curveto(257.10077527,93.18804061)(257.20744194,93.29470727)(257.28744194,93.42804061)
\curveto(257.36744194,93.56137394)(257.40744194,93.73137394)(257.40744194,93.93804061)
\curveto(257.40744194,94.27804061)(257.30077527,94.51804061)(257.08744194,94.65804061)
\curveto(256.88077527,94.80470727)(256.63744194,94.87804061)(256.35744194,94.87804061)
\curveto(256.07077527,94.87804061)(255.82744194,94.83470727)(255.62744194,94.74804061)
\curveto(255.42744194,94.66804061)(255.26077527,94.58470727)(255.12744194,94.49804061)
\lineto(254.80744194,95.12804061)
\curveto(254.94744194,95.22804061)(255.15744194,95.33137394)(255.43744194,95.43804061)
\curveto(255.72410861,95.55137394)(256.04077527,95.60804061)(256.38744194,95.60804061)
\curveto(256.71410861,95.60804061)(256.99410861,95.56804061)(257.22744194,95.48804061)
\curveto(257.46077527,95.40804061)(257.65077527,95.29470727)(257.79744194,95.14804061)
\curveto(257.95077527,95.00137394)(258.06410861,94.82804061)(258.13744194,94.62804061)
\curveto(258.21077527,94.43470727)(258.24744194,94.22137394)(258.24744194,93.98804061)
\curveto(258.24744194,93.66137394)(258.16077527,93.38470727)(257.98744194,93.15804061)
\curveto(257.82077527,92.93137394)(257.60410861,92.75804061)(257.33744194,92.63804061)
\curveto(257.65744194,92.54470727)(257.93410861,92.36137394)(258.16744194,92.08804061)
\curveto(258.40077527,91.82137394)(258.51744194,91.46470727)(258.51744194,91.01804061)
\curveto(258.51744194,90.75137394)(258.47077527,90.50137394)(258.37744194,90.26804061)
\curveto(258.29077527,90.04137394)(258.15410861,89.84470727)(257.96744194,89.67804061)
\curveto(257.78744194,89.51137394)(257.55077527,89.38137394)(257.25744194,89.28804061)
\curveto(256.97077527,89.19470727)(256.62744194,89.14804061)(256.22744194,89.14804061)
\curveto(256.07410861,89.14804061)(255.91410861,89.16137394)(255.74744194,89.18804061)
\curveto(255.58744194,89.20804061)(255.43744194,89.23804061)(255.29744194,89.27804061)
\curveto(255.15744194,89.31137394)(255.03077527,89.34470727)(254.91744194,89.37804061)
\curveto(254.81077527,89.41804061)(254.73410861,89.44804061)(254.68744194,89.46804061)
\lineto(254.84744194,90.17804061)
\curveto(254.95410861,90.12470727)(255.12410861,90.06137394)(255.35744194,89.98804061)
\curveto(255.59077527,89.91470727)(255.87744194,89.87804061)(256.21744194,89.87804061)
\closepath
}
}
{
\newrgbcolor{curcolor}{0 0 0}
\pscustom[linestyle=none,fillstyle=solid,fillcolor=curcolor]
{
\newpath
\moveto(263.58744194,90.88804061)
\curveto(263.58744194,90.36804061)(263.42077527,89.94804061)(263.08744194,89.62804061)
\curveto(262.76077527,89.30804061)(262.26077527,89.14804061)(261.58744194,89.14804061)
\curveto(261.20077527,89.14804061)(260.88077527,89.19804061)(260.62744194,89.29804061)
\curveto(260.37410861,89.40470727)(260.17077527,89.53804061)(260.01744194,89.69804061)
\curveto(259.87077527,89.86470727)(259.76410861,90.04804061)(259.69744194,90.24804061)
\curveto(259.63744194,90.44804061)(259.60744194,90.64470727)(259.60744194,90.83804061)
\curveto(259.60744194,91.19137394)(259.70410861,91.50470727)(259.89744194,91.77804061)
\curveto(260.09077527,92.05137394)(260.32077527,92.27137394)(260.58744194,92.43804061)
\curveto(260.02077527,92.75804061)(259.73744194,93.24804061)(259.73744194,93.90804061)
\curveto(259.73744194,94.13470727)(259.78077527,94.35137394)(259.86744194,94.55804061)
\curveto(259.95410861,94.76470727)(260.07744194,94.94470727)(260.23744194,95.09804061)
\curveto(260.39744194,95.25137394)(260.59077527,95.37470727)(260.81744194,95.46804061)
\curveto(261.05077527,95.56137394)(261.31077527,95.60804061)(261.59744194,95.60804061)
\curveto(261.93077527,95.60804061)(262.21410861,95.55804061)(262.44744194,95.45804061)
\curveto(262.68744194,95.35804061)(262.88077527,95.22804061)(263.02744194,95.06804061)
\curveto(263.17410861,94.91470727)(263.28077527,94.74137394)(263.34744194,94.54804061)
\curveto(263.41410861,94.36137394)(263.44744194,94.17804061)(263.44744194,93.99804061)
\curveto(263.44744194,93.64470727)(263.35744194,93.33804061)(263.17744194,93.07804061)
\curveto(262.99744194,92.82470727)(262.79077527,92.62137394)(262.55744194,92.46804061)
\curveto(263.24410861,92.14137394)(263.58744194,91.61470727)(263.58744194,90.88804061)
\closepath
\moveto(260.40744194,90.82804061)
\curveto(260.40744194,90.72137394)(260.42744194,90.60804061)(260.46744194,90.48804061)
\curveto(260.50744194,90.37470727)(260.57410861,90.26804061)(260.66744194,90.16804061)
\curveto(260.76077527,90.06804061)(260.88410861,89.98470727)(261.03744194,89.91804061)
\curveto(261.19077527,89.85804061)(261.37744194,89.82804061)(261.59744194,89.82804061)
\curveto(261.80410861,89.82804061)(261.98077527,89.85470727)(262.12744194,89.90804061)
\curveto(262.28077527,89.96804061)(262.40410861,90.04470727)(262.49744194,90.13804061)
\curveto(262.59744194,90.23804061)(262.67077527,90.34804061)(262.71744194,90.46804061)
\curveto(262.76410861,90.58804061)(262.78744194,90.70804061)(262.78744194,90.82804061)
\curveto(262.78744194,91.20804061)(262.65077527,91.49804061)(262.37744194,91.69804061)
\curveto(262.10410861,91.90470727)(261.72744194,92.06137394)(261.24744194,92.16804061)
\curveto(260.98077527,92.02137394)(260.77410861,91.83804061)(260.62744194,91.61804061)
\curveto(260.48077527,91.39804061)(260.40744194,91.13470727)(260.40744194,90.82804061)
\closepath
\moveto(262.64744194,93.99804061)
\curveto(262.64744194,94.08470727)(262.62744194,94.18137394)(262.58744194,94.28804061)
\curveto(262.54744194,94.40137394)(262.48410861,94.50137394)(262.39744194,94.58804061)
\curveto(262.31744194,94.68137394)(262.21077527,94.75804061)(262.07744194,94.81804061)
\curveto(261.94410861,94.88470727)(261.78410861,94.91804061)(261.59744194,94.91804061)
\curveto(261.40410861,94.91804061)(261.24077527,94.88804061)(261.10744194,94.82804061)
\curveto(260.98077527,94.76804061)(260.87410861,94.69137394)(260.78744194,94.59804061)
\curveto(260.70077527,94.51137394)(260.63744194,94.41137394)(260.59744194,94.29804061)
\curveto(260.56410861,94.19137394)(260.54744194,94.08470727)(260.54744194,93.97804061)
\curveto(260.54744194,93.70470727)(260.64410861,93.44804061)(260.83744194,93.20804061)
\curveto(261.03744194,92.97470727)(261.36410861,92.80470727)(261.81744194,92.69804061)
\curveto(262.07077527,92.84470727)(262.27077527,93.01804061)(262.41744194,93.21804061)
\curveto(262.57077527,93.41804061)(262.64744194,93.67804061)(262.64744194,93.99804061)
\closepath
}
}
{
\newrgbcolor{curcolor}{0 0 0}
\pscustom[linestyle=none,fillstyle=solid,fillcolor=curcolor]
{
\newpath
\moveto(259.04951347,193.73412276)
\curveto(259.94284681,193.70078943)(260.59284681,193.50412276)(260.99951347,193.14412276)
\curveto(261.40618014,192.78412276)(261.60951347,192.30078943)(261.60951347,191.69412276)
\curveto(261.60951347,191.42078943)(261.56618014,191.16745609)(261.47951347,190.93412276)
\curveto(261.39284681,190.70078943)(261.25618014,190.50078943)(261.06951347,190.33412276)
\curveto(260.88951347,190.16745609)(260.65618014,190.03745609)(260.36951347,189.94412276)
\curveto(260.08951347,189.85078943)(259.75618014,189.80412276)(259.36951347,189.80412276)
\curveto(259.20951347,189.80412276)(259.04951347,189.81745609)(258.88951347,189.84412276)
\curveto(258.73618014,189.86412276)(258.58951347,189.89078943)(258.44951347,189.92412276)
\curveto(258.30951347,189.95745609)(258.18618014,189.99078943)(258.07951347,190.02412276)
\curveto(257.97284681,190.06412276)(257.89618014,190.09745609)(257.84951347,190.12412276)
\lineto(258.00951347,190.83412276)
\curveto(258.11618014,190.78078943)(258.27951347,190.71745609)(258.49951347,190.64412276)
\curveto(258.72618014,190.57078943)(259.00951347,190.53412276)(259.34951347,190.53412276)
\curveto(259.61618014,190.53412276)(259.83951347,190.56412276)(260.01951347,190.62412276)
\curveto(260.19951347,190.68412276)(260.34284681,190.76412276)(260.44951347,190.86412276)
\curveto(260.56284681,190.97078943)(260.64284681,191.09078943)(260.68951347,191.22412276)
\curveto(260.74284681,191.35745609)(260.76951347,191.49745609)(260.76951347,191.64412276)
\curveto(260.76951347,191.87078943)(260.72951347,192.07078943)(260.64951347,192.24412276)
\curveto(260.57618014,192.42412276)(260.44284681,192.57412276)(260.24951347,192.69412276)
\curveto(260.06284681,192.81412276)(259.80284681,192.90412276)(259.46951347,192.96412276)
\curveto(259.14284681,193.03078943)(258.72951347,193.06412276)(258.22951347,193.06412276)
\curveto(258.26951347,193.35745609)(258.29951347,193.63078943)(258.31951347,193.88412276)
\curveto(258.34618014,194.14412276)(258.36618014,194.39412276)(258.37951347,194.63412276)
\curveto(258.39951347,194.88078943)(258.41284681,195.12412276)(258.41951347,195.36412276)
\curveto(258.43284681,195.60412276)(258.44618014,195.85745609)(258.45951347,196.12412276)
\lineto(261.43951347,196.12412276)
\lineto(261.43951347,195.42412276)
\lineto(259.17951347,195.42412276)
\curveto(259.17284681,195.33078943)(259.16284681,195.20745609)(259.14951347,195.05412276)
\curveto(259.14284681,194.90745609)(259.13284681,194.75078943)(259.11951347,194.58412276)
\curveto(259.10618014,194.41745609)(259.09284681,194.25745609)(259.07951347,194.10412276)
\curveto(259.06618014,193.95078943)(259.05618014,193.82745609)(259.04951347,193.73412276)
\closepath
}
}
{
\newrgbcolor{curcolor}{0 0 0}
\pscustom[linestyle=none,fillstyle=solid,fillcolor=curcolor]
{
\newpath
\moveto(262.93951347,194.84412276)
\curveto(263.27951347,194.97745609)(263.60951347,195.14412276)(263.92951347,195.34412276)
\curveto(264.24951347,195.55078943)(264.54618014,195.81078943)(264.81951347,196.12412276)
\lineto(265.39951347,196.12412276)
\lineto(265.39951347,190.63412276)
\lineto(266.56951347,190.63412276)
\lineto(266.56951347,189.93412276)
\lineto(263.23951347,189.93412276)
\lineto(263.23951347,190.63412276)
\lineto(264.57951347,190.63412276)
\lineto(264.57951347,194.97412276)
\curveto(264.50618014,194.90745609)(264.41618014,194.83745609)(264.30951347,194.76412276)
\curveto(264.20951347,194.69745609)(264.09618014,194.63078943)(263.96951347,194.56412276)
\curveto(263.84951347,194.49745609)(263.72284681,194.43412276)(263.58951347,194.37412276)
\curveto(263.45618014,194.31412276)(263.32618014,194.26412276)(263.19951347,194.22412276)
\lineto(262.93951347,194.84412276)
\closepath
}
}
{
\newrgbcolor{curcolor}{0 1 0}
\pscustom[linewidth=1,linecolor=curcolor]
{
\newpath
\moveto(224.696712,70.71257533)
\lineto(211.318662,70.71257533)
\lineto(211.318662,65.79818533)
\lineto(192.207152,65.79818533)
\lineto(192.207152,61.97588533)
\lineto(175.825862,61.97588533)
\lineto(175.825862,62.52192533)
}
}
{
\newrgbcolor{curcolor}{0 1 0}
\pscustom[linestyle=none,fillstyle=solid,fillcolor=curcolor]
{
\newpath
\moveto(223.716712,70.71257533)
\curveto(223.716712,71.26457533)(224.164712,71.71257533)(224.716712,71.71257533)
\curveto(225.268712,71.71257533)(225.716712,71.26457533)(225.716712,70.71257533)
\curveto(225.716712,70.16057533)(225.268712,69.71257533)(224.716712,69.71257533)
\curveto(224.164712,69.71257533)(223.716712,70.16057533)(223.716712,70.71257533)
\closepath
}
}
{
\newrgbcolor{curcolor}{0 1 0}
\pscustom[linewidth=0.25,linecolor=curcolor]
{
\newpath
\moveto(223.716712,70.71257533)
\curveto(223.716712,71.26457533)(224.164712,71.71257533)(224.716712,71.71257533)
\curveto(225.268712,71.71257533)(225.716712,71.26457533)(225.716712,70.71257533)
\curveto(225.716712,70.16057533)(225.268712,69.71257533)(224.716712,69.71257533)
\curveto(224.164712,69.71257533)(223.716712,70.16057533)(223.716712,70.71257533)
\closepath
}
}
{
\newrgbcolor{curcolor}{0 1 0}
\pscustom[linestyle=none,fillstyle=solid,fillcolor=curcolor]
{
\newpath
\moveto(175.825862,63.50192533)
\curveto(176.377862,63.50192533)(176.825862,63.05392533)(176.825862,62.50192533)
\curveto(176.825862,61.94992533)(176.377862,61.50192533)(175.825862,61.50192533)
\curveto(175.273862,61.50192533)(174.825862,61.94992533)(174.825862,62.50192533)
\curveto(174.825862,63.05392533)(175.273862,63.50192533)(175.825862,63.50192533)
\closepath
}
}
{
\newrgbcolor{curcolor}{0 1 0}
\pscustom[linewidth=0.25,linecolor=curcolor]
{
\newpath
\moveto(175.825862,63.50192533)
\curveto(176.377862,63.50192533)(176.825862,63.05392533)(176.825862,62.50192533)
\curveto(176.825862,61.94992533)(176.377862,61.50192533)(175.825862,61.50192533)
\curveto(175.273862,61.50192533)(174.825862,61.94992533)(174.825862,62.50192533)
\curveto(174.825862,63.05392533)(175.273862,63.50192533)(175.825862,63.50192533)
\closepath
}
}
{
\newrgbcolor{curcolor}{0 0 0}
\pscustom[linestyle=none,fillstyle=solid,fillcolor=curcolor]
{
\newpath
\moveto(220.71974114,78.07976424)
\curveto(221.05974114,78.21309757)(221.38974114,78.37976424)(221.70974114,78.57976424)
\curveto(222.02974114,78.78643091)(222.3264078,79.04643091)(222.59974114,79.35976424)
\lineto(223.17974114,79.35976424)
\lineto(223.17974114,73.86976424)
\lineto(224.34974114,73.86976424)
\lineto(224.34974114,73.16976424)
\lineto(221.01974114,73.16976424)
\lineto(221.01974114,73.86976424)
\lineto(222.35974114,73.86976424)
\lineto(222.35974114,78.20976424)
\curveto(222.2864078,78.14309757)(222.1964078,78.07309757)(222.08974114,77.99976424)
\curveto(221.98974114,77.93309757)(221.8764078,77.86643091)(221.74974114,77.79976424)
\curveto(221.62974114,77.73309757)(221.50307447,77.66976424)(221.36974114,77.60976424)
\curveto(221.2364078,77.54976424)(221.1064078,77.49976424)(220.97974114,77.45976424)
\lineto(220.71974114,78.07976424)
\closepath
}
}
{
\newrgbcolor{curcolor}{0 0 0}
\pscustom[linestyle=none,fillstyle=solid,fillcolor=curcolor]
{
\newpath
\moveto(225.32974114,75.30976424)
\curveto(225.44307447,75.57643091)(225.5964078,75.88309757)(225.78974114,76.22976424)
\curveto(225.98974114,76.57643091)(226.21307447,76.93309757)(226.45974114,77.29976424)
\curveto(226.7064078,77.67309757)(226.96974114,78.03643091)(227.24974114,78.38976424)
\curveto(227.52974114,78.74976424)(227.81307447,79.07309757)(228.09974114,79.35976424)
\lineto(228.89974114,79.35976424)
\lineto(228.89974114,75.42976424)
\lineto(229.62974114,75.42976424)
\lineto(229.62974114,74.74976424)
\lineto(228.89974114,74.74976424)
\lineto(228.89974114,73.16976424)
\lineto(228.09974114,73.16976424)
\lineto(228.09974114,74.74976424)
\lineto(225.32974114,74.74976424)
\lineto(225.32974114,75.30976424)
\closepath
\moveto(228.09974114,78.37976424)
\curveto(227.91974114,78.18643091)(227.7364078,77.97309757)(227.54974114,77.73976424)
\curveto(227.36974114,77.50643091)(227.19307447,77.25976424)(227.01974114,76.99976424)
\curveto(226.8464078,76.74643091)(226.68307447,76.48643091)(226.52974114,76.21976424)
\curveto(226.38307447,75.95309757)(226.24974114,75.68976424)(226.12974114,75.42976424)
\lineto(228.09974114,75.42976424)
\lineto(228.09974114,78.37976424)
\closepath
}
}
{
\newrgbcolor{curcolor}{0 0 0}
\pscustom[linestyle=none,fillstyle=solid,fillcolor=curcolor]
{
\newpath
\moveto(248.84585991,60.39171248)
\curveto(249.37252658,60.39171248)(249.74585991,60.49504581)(249.96585991,60.70171248)
\curveto(250.19252658,60.91504581)(250.30585991,61.19837915)(250.30585991,61.55171248)
\curveto(250.30585991,61.77837915)(250.25919324,61.96837915)(250.16585991,62.12171248)
\curveto(250.07252658,62.27504581)(249.94919324,62.39837915)(249.79585991,62.49171248)
\curveto(249.64252658,62.58504581)(249.46585991,62.65171248)(249.26585991,62.69171248)
\curveto(249.06585991,62.73171248)(248.85585991,62.75171248)(248.63585991,62.75171248)
\lineto(248.42585991,62.75171248)
\lineto(248.42585991,63.42171248)
\lineto(248.71585991,63.42171248)
\curveto(248.86252658,63.42171248)(249.01252658,63.43504581)(249.16585991,63.46171248)
\curveto(249.32585991,63.49504581)(249.46919324,63.54837915)(249.59585991,63.62171248)
\curveto(249.72919324,63.70171248)(249.83585991,63.80837915)(249.91585991,63.94171248)
\curveto(249.99585991,64.07504581)(250.03585991,64.24504581)(250.03585991,64.45171248)
\curveto(250.03585991,64.79171248)(249.92919324,65.03171248)(249.71585991,65.17171248)
\curveto(249.50919324,65.31837915)(249.26585991,65.39171248)(248.98585991,65.39171248)
\curveto(248.69919324,65.39171248)(248.45585991,65.34837915)(248.25585991,65.26171248)
\curveto(248.05585991,65.18171248)(247.88919324,65.09837915)(247.75585991,65.01171248)
\lineto(247.43585991,65.64171248)
\curveto(247.57585991,65.74171248)(247.78585991,65.84504581)(248.06585991,65.95171248)
\curveto(248.35252658,66.06504581)(248.66919324,66.12171248)(249.01585991,66.12171248)
\curveto(249.34252658,66.12171248)(249.62252658,66.08171248)(249.85585991,66.00171248)
\curveto(250.08919324,65.92171248)(250.27919324,65.80837915)(250.42585991,65.66171248)
\curveto(250.57919324,65.51504581)(250.69252658,65.34171248)(250.76585991,65.14171248)
\curveto(250.83919324,64.94837915)(250.87585991,64.73504581)(250.87585991,64.50171248)
\curveto(250.87585991,64.17504581)(250.78919324,63.89837915)(250.61585991,63.67171248)
\curveto(250.44919324,63.44504581)(250.23252658,63.27171248)(249.96585991,63.15171248)
\curveto(250.28585991,63.05837915)(250.56252658,62.87504581)(250.79585991,62.60171248)
\curveto(251.02919324,62.33504581)(251.14585991,61.97837915)(251.14585991,61.53171248)
\curveto(251.14585991,61.26504581)(251.09919324,61.01504581)(251.00585991,60.78171248)
\curveto(250.91919324,60.55504581)(250.78252658,60.35837915)(250.59585991,60.19171248)
\curveto(250.41585991,60.02504581)(250.17919324,59.89504581)(249.88585991,59.80171248)
\curveto(249.59919324,59.70837915)(249.25585991,59.66171248)(248.85585991,59.66171248)
\curveto(248.70252658,59.66171248)(248.54252658,59.67504581)(248.37585991,59.70171248)
\curveto(248.21585991,59.72171248)(248.06585991,59.75171248)(247.92585991,59.79171248)
\curveto(247.78585991,59.82504581)(247.65919324,59.85837915)(247.54585991,59.89171248)
\curveto(247.43919324,59.93171248)(247.36252658,59.96171248)(247.31585991,59.98171248)
\lineto(247.47585991,60.69171248)
\curveto(247.58252658,60.63837915)(247.75252658,60.57504581)(247.98585991,60.50171248)
\curveto(248.21919324,60.42837915)(248.50585991,60.39171248)(248.84585991,60.39171248)
\closepath
}
}
{
\newrgbcolor{curcolor}{0 0 0}
\pscustom[linestyle=none,fillstyle=solid,fillcolor=curcolor]
{
\newpath
\moveto(253.58585991,63.59171248)
\curveto(254.47919324,63.55837915)(255.12919324,63.36171248)(255.53585991,63.00171248)
\curveto(255.94252658,62.64171248)(256.14585991,62.15837915)(256.14585991,61.55171248)
\curveto(256.14585991,61.27837915)(256.10252658,61.02504581)(256.01585991,60.79171248)
\curveto(255.92919324,60.55837915)(255.79252658,60.35837915)(255.60585991,60.19171248)
\curveto(255.42585991,60.02504581)(255.19252658,59.89504581)(254.90585991,59.80171248)
\curveto(254.62585991,59.70837915)(254.29252658,59.66171248)(253.90585991,59.66171248)
\curveto(253.74585991,59.66171248)(253.58585991,59.67504581)(253.42585991,59.70171248)
\curveto(253.27252658,59.72171248)(253.12585991,59.74837915)(252.98585991,59.78171248)
\curveto(252.84585991,59.81504581)(252.72252658,59.84837915)(252.61585991,59.88171248)
\curveto(252.50919324,59.92171248)(252.43252658,59.95504581)(252.38585991,59.98171248)
\lineto(252.54585991,60.69171248)
\curveto(252.65252658,60.63837915)(252.81585991,60.57504581)(253.03585991,60.50171248)
\curveto(253.26252658,60.42837915)(253.54585991,60.39171248)(253.88585991,60.39171248)
\curveto(254.15252658,60.39171248)(254.37585991,60.42171248)(254.55585991,60.48171248)
\curveto(254.73585991,60.54171248)(254.87919324,60.62171248)(254.98585991,60.72171248)
\curveto(255.09919324,60.82837915)(255.17919324,60.94837915)(255.22585991,61.08171248)
\curveto(255.27919324,61.21504581)(255.30585991,61.35504581)(255.30585991,61.50171248)
\curveto(255.30585991,61.72837915)(255.26585991,61.92837915)(255.18585991,62.10171248)
\curveto(255.11252658,62.28171248)(254.97919324,62.43171248)(254.78585991,62.55171248)
\curveto(254.59919324,62.67171248)(254.33919324,62.76171248)(254.00585991,62.82171248)
\curveto(253.67919324,62.88837915)(253.26585991,62.92171248)(252.76585991,62.92171248)
\curveto(252.80585991,63.21504581)(252.83585991,63.48837915)(252.85585991,63.74171248)
\curveto(252.88252658,64.00171248)(252.90252658,64.25171248)(252.91585991,64.49171248)
\curveto(252.93585991,64.73837915)(252.94919324,64.98171248)(252.95585991,65.22171248)
\curveto(252.96919324,65.46171248)(252.98252658,65.71504581)(252.99585991,65.98171248)
\lineto(255.97585991,65.98171248)
\lineto(255.97585991,65.28171248)
\lineto(253.71585991,65.28171248)
\curveto(253.70919324,65.18837915)(253.69919324,65.06504581)(253.68585991,64.91171248)
\curveto(253.67919324,64.76504581)(253.66919324,64.60837915)(253.65585991,64.44171248)
\curveto(253.64252658,64.27504581)(253.62919324,64.11504581)(253.61585991,63.96171248)
\curveto(253.60252658,63.80837915)(253.59252658,63.68504581)(253.58585991,63.59171248)
\closepath
}
}
{
\newrgbcolor{curcolor}{0 0 0}
\pscustom[linestyle=none,fillstyle=solid,fillcolor=curcolor]
{
\newpath
\moveto(217.71650627,52.41574752)
\curveto(218.05650627,52.54908085)(218.38650627,52.71574752)(218.70650627,52.91574752)
\curveto(219.02650627,53.12241418)(219.32317294,53.38241418)(219.59650627,53.69574752)
\lineto(220.17650627,53.69574752)
\lineto(220.17650627,48.20574752)
\lineto(221.34650627,48.20574752)
\lineto(221.34650627,47.50574752)
\lineto(218.01650627,47.50574752)
\lineto(218.01650627,48.20574752)
\lineto(219.35650627,48.20574752)
\lineto(219.35650627,52.54574752)
\curveto(219.28317294,52.47908085)(219.19317294,52.40908085)(219.08650627,52.33574752)
\curveto(218.98650627,52.26908085)(218.87317294,52.20241418)(218.74650627,52.13574752)
\curveto(218.62650627,52.06908085)(218.49983961,52.00574752)(218.36650627,51.94574752)
\curveto(218.23317294,51.88574752)(218.10317294,51.83574752)(217.97650627,51.79574752)
\lineto(217.71650627,52.41574752)
\closepath
}
}
{
\newrgbcolor{curcolor}{0 0 0}
\pscustom[linestyle=none,fillstyle=solid,fillcolor=curcolor]
{
\newpath
\moveto(224.08650627,48.10574752)
\curveto(224.61317294,48.10574752)(224.98650627,48.20908085)(225.20650627,48.41574752)
\curveto(225.43317294,48.62908085)(225.54650627,48.91241418)(225.54650627,49.26574752)
\curveto(225.54650627,49.49241418)(225.49983961,49.68241418)(225.40650627,49.83574752)
\curveto(225.31317294,49.98908085)(225.18983961,50.11241418)(225.03650627,50.20574752)
\curveto(224.88317294,50.29908085)(224.70650627,50.36574752)(224.50650627,50.40574752)
\curveto(224.30650627,50.44574752)(224.09650627,50.46574752)(223.87650627,50.46574752)
\lineto(223.66650627,50.46574752)
\lineto(223.66650627,51.13574752)
\lineto(223.95650627,51.13574752)
\curveto(224.10317294,51.13574752)(224.25317294,51.14908085)(224.40650627,51.17574752)
\curveto(224.56650627,51.20908085)(224.70983961,51.26241418)(224.83650627,51.33574752)
\curveto(224.96983961,51.41574752)(225.07650627,51.52241418)(225.15650627,51.65574752)
\curveto(225.23650627,51.78908085)(225.27650627,51.95908085)(225.27650627,52.16574752)
\curveto(225.27650627,52.50574752)(225.16983961,52.74574752)(224.95650627,52.88574752)
\curveto(224.74983961,53.03241418)(224.50650627,53.10574752)(224.22650627,53.10574752)
\curveto(223.93983961,53.10574752)(223.69650627,53.06241418)(223.49650627,52.97574752)
\curveto(223.29650627,52.89574752)(223.12983961,52.81241418)(222.99650627,52.72574752)
\lineto(222.67650627,53.35574752)
\curveto(222.81650627,53.45574752)(223.02650627,53.55908085)(223.30650627,53.66574752)
\curveto(223.59317294,53.77908085)(223.90983961,53.83574752)(224.25650627,53.83574752)
\curveto(224.58317294,53.83574752)(224.86317294,53.79574752)(225.09650627,53.71574752)
\curveto(225.32983961,53.63574752)(225.51983961,53.52241418)(225.66650627,53.37574752)
\curveto(225.81983961,53.22908085)(225.93317294,53.05574752)(226.00650627,52.85574752)
\curveto(226.07983961,52.66241418)(226.11650627,52.44908085)(226.11650627,52.21574752)
\curveto(226.11650627,51.88908085)(226.02983961,51.61241418)(225.85650627,51.38574752)
\curveto(225.68983961,51.15908085)(225.47317294,50.98574752)(225.20650627,50.86574752)
\curveto(225.52650627,50.77241418)(225.80317294,50.58908085)(226.03650627,50.31574752)
\curveto(226.26983961,50.04908085)(226.38650627,49.69241418)(226.38650627,49.24574752)
\curveto(226.38650627,48.97908085)(226.33983961,48.72908085)(226.24650627,48.49574752)
\curveto(226.15983961,48.26908085)(226.02317294,48.07241418)(225.83650627,47.90574752)
\curveto(225.65650627,47.73908085)(225.41983961,47.60908085)(225.12650627,47.51574752)
\curveto(224.83983961,47.42241418)(224.49650627,47.37574752)(224.09650627,47.37574752)
\curveto(223.94317294,47.37574752)(223.78317294,47.38908085)(223.61650627,47.41574752)
\curveto(223.45650627,47.43574752)(223.30650627,47.46574752)(223.16650627,47.50574752)
\curveto(223.02650627,47.53908085)(222.89983961,47.57241418)(222.78650627,47.60574752)
\curveto(222.67983961,47.64574752)(222.60317294,47.67574752)(222.55650627,47.69574752)
\lineto(222.71650627,48.40574752)
\curveto(222.82317294,48.35241418)(222.99317294,48.28908085)(223.22650627,48.21574752)
\curveto(223.45983961,48.14241418)(223.74650627,48.10574752)(224.08650627,48.10574752)
\closepath
}
}
{
\newrgbcolor{curcolor}{0 1 0}
\pscustom[linewidth=1,linecolor=curcolor]
{
\newpath
\moveto(251.725842,57.60753533)
\lineto(211.591682,57.60753533)
\lineto(211.591682,53.23919533)
\lineto(194.937372,53.23919533)
\lineto(194.937372,53.78523533)
}
}
{
\newrgbcolor{curcolor}{0 1 0}
\pscustom[linestyle=none,fillstyle=solid,fillcolor=curcolor]
{
\newpath
\moveto(250.745842,57.60753533)
\curveto(250.745842,58.15953533)(251.193842,58.60753533)(251.745842,58.60753533)
\curveto(252.297842,58.60753533)(252.745842,58.15953533)(252.745842,57.60753533)
\curveto(252.745842,57.05553533)(252.297842,56.60753533)(251.745842,56.60753533)
\curveto(251.193842,56.60753533)(250.745842,57.05553533)(250.745842,57.60753533)
\closepath
}
}
{
\newrgbcolor{curcolor}{0 1 0}
\pscustom[linewidth=0.25,linecolor=curcolor]
{
\newpath
\moveto(250.745842,57.60753533)
\curveto(250.745842,58.15953533)(251.193842,58.60753533)(251.745842,58.60753533)
\curveto(252.297842,58.60753533)(252.745842,58.15953533)(252.745842,57.60753533)
\curveto(252.745842,57.05553533)(252.297842,56.60753533)(251.745842,56.60753533)
\curveto(251.193842,56.60753533)(250.745842,57.05553533)(250.745842,57.60753533)
\closepath
}
}
{
\newrgbcolor{curcolor}{0 1 0}
\pscustom[linestyle=none,fillstyle=solid,fillcolor=curcolor]
{
\newpath
\moveto(194.937372,54.76523533)
\curveto(195.489372,54.76523533)(195.937372,54.31723533)(195.937372,53.76523533)
\curveto(195.937372,53.21323533)(195.489372,52.76523533)(194.937372,52.76523533)
\curveto(194.385372,52.76523533)(193.937372,53.21323533)(193.937372,53.76523533)
\curveto(193.937372,54.31723533)(194.385372,54.76523533)(194.937372,54.76523533)
\closepath
}
}
{
\newrgbcolor{curcolor}{0 1 0}
\pscustom[linewidth=0.25,linecolor=curcolor]
{
\newpath
\moveto(194.937372,54.76523533)
\curveto(195.489372,54.76523533)(195.937372,54.31723533)(195.937372,53.76523533)
\curveto(195.937372,53.21323533)(195.489372,52.76523533)(194.937372,52.76523533)
\curveto(194.385372,52.76523533)(193.937372,53.21323533)(193.937372,53.76523533)
\curveto(193.937372,54.31723533)(194.385372,54.76523533)(194.937372,54.76523533)
\closepath
}
}
{
\newrgbcolor{curcolor}{0 1 0}
\pscustom[linewidth=1,linecolor=curcolor]
{
\newpath
\moveto(221.966502,45.04854533)
\lineto(214.321902,45.04854533)
\lineto(214.321902,45.04854533)
\lineto(214.321902,45.04854533)
}
}
{
\newrgbcolor{curcolor}{0 1 0}
\pscustom[linestyle=none,fillstyle=solid,fillcolor=curcolor]
{
\newpath
\moveto(220.986502,45.04854533)
\curveto(220.986502,45.60054533)(221.434502,46.04854533)(221.986502,46.04854533)
\curveto(222.538502,46.04854533)(222.986502,45.60054533)(222.986502,45.04854533)
\curveto(222.986502,44.49654533)(222.538502,44.04854533)(221.986502,44.04854533)
\curveto(221.434502,44.04854533)(220.986502,44.49654533)(220.986502,45.04854533)
\closepath
}
}
{
\newrgbcolor{curcolor}{0 1 0}
\pscustom[linewidth=0.25,linecolor=curcolor]
{
\newpath
\moveto(220.986502,45.04854533)
\curveto(220.986502,45.60054533)(221.434502,46.04854533)(221.986502,46.04854533)
\curveto(222.538502,46.04854533)(222.986502,45.60054533)(222.986502,45.04854533)
\curveto(222.986502,44.49654533)(222.538502,44.04854533)(221.986502,44.04854533)
\curveto(221.434502,44.04854533)(220.986502,44.49654533)(220.986502,45.04854533)
\closepath
}
}
{
\newrgbcolor{curcolor}{0 1 0}
\pscustom[linestyle=none,fillstyle=solid,fillcolor=curcolor]
{
\newpath
\moveto(215.301902,45.04854533)
\curveto(215.301902,44.49654533)(214.853902,44.04854533)(214.301902,44.04854533)
\curveto(213.749902,44.04854533)(213.301902,44.49654533)(213.301902,45.04854533)
\curveto(213.301902,45.60054533)(213.749902,46.04854533)(214.301902,46.04854533)
\curveto(214.853902,46.04854533)(215.301902,45.60054533)(215.301902,45.04854533)
\closepath
}
}
{
\newrgbcolor{curcolor}{0 1 0}
\pscustom[linewidth=0.25,linecolor=curcolor]
{
\newpath
\moveto(215.301902,45.04854533)
\curveto(215.301902,44.49654533)(214.853902,44.04854533)(214.301902,44.04854533)
\curveto(213.749902,44.04854533)(213.301902,44.49654533)(213.301902,45.04854533)
\curveto(213.301902,45.60054533)(213.749902,46.04854533)(214.301902,46.04854533)
\curveto(214.853902,46.04854533)(215.301902,45.60054533)(215.301902,45.04854533)
\closepath
}
}
{
\newrgbcolor{curcolor}{0 0 0}
\pscustom[linestyle=none,fillstyle=solid,fillcolor=curcolor]
{
\newpath
\moveto(62.73325152,8.90903606)
\curveto(63.16323347,9.07617939)(63.5805689,9.28510854)(63.9852578,9.53582353)
\curveto(64.3899467,9.79489568)(64.76512703,10.12082516)(65.11079879,10.51361197)
\lineto(65.84429742,10.51361197)
\lineto(65.84429742,3.63148563)
\lineto(67.3239412,3.63148563)
\lineto(67.3239412,2.75398318)
\lineto(63.11264736,2.75398318)
\lineto(63.11264736,3.63148563)
\lineto(64.80728212,3.63148563)
\lineto(64.80728212,9.0720008)
\curveto(64.71454092,8.98842914)(64.60072216,8.9006789)(64.46582586,8.80875007)
\curveto(64.33936058,8.72517841)(64.19603327,8.64160675)(64.03584391,8.55803508)
\curveto(63.88408557,8.47446342)(63.72389622,8.39507034)(63.55527584,8.31985585)
\curveto(63.38665547,8.24464135)(63.22225061,8.18196261)(63.06206125,8.13181961)
\lineto(62.73325152,8.90903606)
\closepath
}
}
{
\newrgbcolor{curcolor}{0 0 0}
\pscustom[linestyle=none,fillstyle=solid,fillcolor=curcolor]
{
\newpath
\moveto(76.78354322,7.5175679)
\curveto(77.91329973,7.47578206)(78.73532405,7.22924566)(79.24961619,6.77795869)
\curveto(79.76390833,6.32667172)(80.0210544,5.72077717)(80.0210544,4.96027505)
\curveto(80.0210544,4.61763123)(79.96625278,4.30005892)(79.85664954,4.0075581)
\curveto(79.74704629,3.71505729)(79.57421041,3.4643423)(79.33814189,3.25541315)
\curveto(79.11050438,3.04648399)(78.81541873,2.88351925)(78.45288492,2.76651893)
\curveto(78.09878214,2.6495186)(77.6772312,2.59101844)(77.18823212,2.59101844)
\curveto(76.98588767,2.59101844)(76.78354322,2.60773277)(76.58119877,2.64116143)
\curveto(76.38728534,2.66623293)(76.20180293,2.6996616)(76.02475154,2.74144743)
\curveto(75.84770015,2.78323326)(75.6917263,2.82501909)(75.55683,2.86680492)
\curveto(75.4219337,2.91694792)(75.32497699,2.95873375)(75.26595986,2.99216241)
\lineto(75.4683043,3.88220061)
\curveto(75.6032006,3.81534328)(75.80976056,3.7359502)(76.08798418,3.64402138)
\curveto(76.37463881,3.55209255)(76.73295711,3.50612813)(77.16293906,3.50612813)
\curveto(77.50017981,3.50612813)(77.78261894,3.54373538)(78.01025644,3.61894988)
\curveto(78.23789395,3.69416437)(78.41916085,3.79445037)(78.55405715,3.91980786)
\curveto(78.69738446,4.05352252)(78.79855669,4.20395151)(78.85757382,4.37109483)
\curveto(78.92502197,4.53823816)(78.95874604,4.71373865)(78.95874604,4.8975963)
\curveto(78.95874604,5.18173995)(78.90815993,5.43245494)(78.80698771,5.64974126)
\curveto(78.7142465,5.87538474)(78.54562613,6.06342098)(78.30112659,6.21384997)
\curveto(78.06505806,6.36427896)(77.73624833,6.47710071)(77.3146974,6.5523152)
\curveto(76.90157748,6.63588686)(76.37885432,6.6776727)(75.74652792,6.6776727)
\curveto(75.79711403,7.04538801)(75.83505362,7.38803182)(75.86034667,7.70560413)
\curveto(75.89407075,8.03153362)(75.9193638,8.34492735)(75.93622584,8.64578533)
\curveto(75.9615189,8.95500048)(75.97838094,9.26003704)(75.98681195,9.56089502)
\curveto(76.00367399,9.86175301)(76.02053603,10.17932532)(76.03739807,10.51361197)
\lineto(79.80606342,10.51361197)
\lineto(79.80606342,9.63610952)
\lineto(76.94794809,9.63610952)
\curveto(76.93951707,9.51910919)(76.92687054,9.36450162)(76.9100085,9.1722868)
\curveto(76.90157748,8.98842914)(76.88893095,8.79203574)(76.87206892,8.58310658)
\curveto(76.85520688,8.37417743)(76.83834484,8.17360544)(76.8214828,7.98139062)
\curveto(76.80462077,7.7891758)(76.79197424,7.63456822)(76.78354322,7.5175679)
\closepath
}
}
{
\newrgbcolor{curcolor}{0 0 0}
\pscustom[linestyle=none,fillstyle=solid,fillcolor=curcolor]
{
\newpath
\moveto(81.70304359,8.90903606)
\curveto(82.13302554,9.07617939)(82.55036097,9.28510854)(82.95504986,9.53582353)
\curveto(83.35973876,9.79489568)(83.73491909,10.12082516)(84.08059086,10.51361197)
\lineto(84.81408949,10.51361197)
\lineto(84.81408949,3.63148563)
\lineto(86.29373327,3.63148563)
\lineto(86.29373327,2.75398318)
\lineto(82.08243943,2.75398318)
\lineto(82.08243943,3.63148563)
\lineto(83.77707419,3.63148563)
\lineto(83.77707419,9.0720008)
\curveto(83.68433298,8.98842914)(83.57051423,8.9006789)(83.43561793,8.80875007)
\curveto(83.30915265,8.72517841)(83.16582533,8.64160675)(83.00563598,8.55803508)
\curveto(82.85387764,8.47446342)(82.69368828,8.39507034)(82.52506791,8.31985585)
\curveto(82.35644754,8.24464135)(82.19204267,8.18196261)(82.03185332,8.13181961)
\lineto(81.70304359,8.90903606)
\closepath
}
}
{
\newrgbcolor{curcolor}{0 0 0}
\pscustom[linestyle=none,fillstyle=solid,fillcolor=curcolor]
{
\newpath
\moveto(94.34956972,8.90903606)
\curveto(94.77955167,9.07617939)(95.19688709,9.28510854)(95.60157599,9.53582353)
\curveto(96.00626489,9.79489568)(96.38144522,10.12082516)(96.72711699,10.51361197)
\lineto(97.46061561,10.51361197)
\lineto(97.46061561,3.63148563)
\lineto(98.9402594,3.63148563)
\lineto(98.9402594,2.75398318)
\lineto(94.72896556,2.75398318)
\lineto(94.72896556,3.63148563)
\lineto(96.42360032,3.63148563)
\lineto(96.42360032,9.0720008)
\curveto(96.33085911,8.98842914)(96.21704036,8.9006789)(96.08214406,8.80875007)
\curveto(95.95567878,8.72517841)(95.81235146,8.64160675)(95.6521621,8.55803508)
\curveto(95.50040377,8.47446342)(95.34021441,8.39507034)(95.17159404,8.31985585)
\curveto(95.00297366,8.24464135)(94.8385688,8.18196261)(94.67837945,8.13181961)
\lineto(94.34956972,8.90903606)
\closepath
}
}
{
\newrgbcolor{curcolor}{0 0 0}
\pscustom[linestyle=none,fillstyle=solid,fillcolor=curcolor]
{
\newpath
\moveto(100.1796201,5.43663352)
\curveto(100.32294742,5.77092017)(100.51686085,6.15534981)(100.76136039,6.58992245)
\curveto(101.01429095,7.02449509)(101.29673008,7.47160348)(101.60867777,7.93124762)
\curveto(101.92062546,8.39924893)(102.2536507,8.85471448)(102.60775349,9.29764429)
\curveto(102.96185627,9.74893126)(103.32017457,10.15425382)(103.68270837,10.51361197)
\lineto(104.69443061,10.51361197)
\lineto(104.69443061,5.58706251)
\lineto(105.61762716,5.58706251)
\lineto(105.61762716,4.73463156)
\lineto(104.69443061,4.73463156)
\lineto(104.69443061,2.75398318)
\lineto(103.68270837,2.75398318)
\lineto(103.68270837,4.73463156)
\lineto(100.1796201,4.73463156)
\lineto(100.1796201,5.43663352)
\closepath
\moveto(103.68270837,9.28510854)
\curveto(103.45507087,9.04275072)(103.22321785,8.7753214)(102.98714933,8.48282059)
\curveto(102.75951182,8.19031977)(102.53608983,7.88110462)(102.31688334,7.55517514)
\curveto(102.09767686,7.23760283)(101.8911169,6.91167335)(101.69720347,6.5773867)
\curveto(101.51172106,6.24310005)(101.34310068,5.91299199)(101.19134235,5.58706251)
\lineto(103.68270837,5.58706251)
\lineto(103.68270837,9.28510854)
\closepath
}
}
{
\newrgbcolor{curcolor}{0 0 0}
\pscustom[linestyle=none,fillstyle=solid,fillcolor=curcolor]
{
\newpath
\moveto(117.68241683,8.53296359)
\curveto(117.68241683,8.26553427)(117.62761521,8.00646212)(117.51801197,7.75574713)
\curveto(117.41683974,7.50503215)(117.27772793,7.25849574)(117.10067654,7.01613793)
\curveto(116.93205617,6.77378011)(116.73814274,6.53560087)(116.51893625,6.30160022)
\curveto(116.29972976,6.06759957)(116.07630777,5.8377775)(115.84867026,5.61213401)
\curveto(115.72220498,5.48677652)(115.57466216,5.33634753)(115.40604178,5.16084704)
\curveto(115.23742141,4.98534655)(115.07723205,4.80566747)(114.92547372,4.62180982)
\curveto(114.77371538,4.43795216)(114.6472501,4.25827309)(114.54607788,4.0827726)
\curveto(114.44490565,3.90727211)(114.39431954,3.75684312)(114.39431954,3.63148563)
\lineto(117.99858003,3.63148563)
\lineto(117.99858003,2.75398318)
\lineto(113.25613202,2.75398318)
\curveto(113.247701,2.79576901)(113.24348549,2.83755484)(113.24348549,2.87934067)
\lineto(113.24348549,3.01723391)
\curveto(113.24348549,3.36823489)(113.30250262,3.69416437)(113.42053688,3.99502235)
\curveto(113.53857114,4.29588034)(113.69032948,4.58002399)(113.87581189,4.8474533)
\curveto(114.0612943,5.11488262)(114.26785426,5.36559761)(114.49549176,5.59959826)
\curveto(114.73156029,5.84195608)(114.9634133,6.07595673)(115.19105081,6.30160022)
\curveto(115.37653322,6.48545787)(115.55358461,6.66513695)(115.72220498,6.84063744)
\curveto(115.89925638,7.01613793)(116.05101471,7.19163842)(116.17747999,7.3671389)
\curveto(116.31237629,7.54263939)(116.41776403,7.72231847)(116.49364319,7.90617612)
\curveto(116.57795338,8.09839094)(116.62010848,8.29478435)(116.62010848,8.49535634)
\curveto(116.62010848,8.72099982)(116.58216889,8.91321465)(116.50628972,9.0720008)
\curveto(116.43884157,9.23078696)(116.34188486,9.36032304)(116.21541958,9.46060903)
\curveto(116.09738532,9.56925219)(115.95827351,9.64864527)(115.79808415,9.69878827)
\curveto(115.6378948,9.74893126)(115.46505891,9.77400276)(115.2795765,9.77400276)
\curveto(115.06037002,9.77400276)(114.85802557,9.74475268)(114.67254316,9.68625252)
\curveto(114.49549176,9.62775235)(114.33530241,9.55671644)(114.19197509,9.47314478)
\curveto(114.05707879,9.38957312)(113.93904453,9.30600146)(113.83787231,9.22242979)
\curveto(113.73670008,9.1472153)(113.66082091,9.08453655)(113.6102348,9.03439356)
\lineto(113.09172715,9.76146701)
\curveto(113.1591753,9.83668151)(113.26034753,9.92861034)(113.39524382,10.0372535)
\curveto(113.53014012,10.14589666)(113.69032948,10.24618265)(113.87581189,10.33811148)
\curveto(114.06972532,10.43839747)(114.2847163,10.52196913)(114.52078482,10.58882646)
\curveto(114.75685334,10.65568379)(115.0097839,10.68911246)(115.2795765,10.68911246)
\curveto(116.09738532,10.68911246)(116.70020315,10.50107622)(117.08803001,10.12500374)
\curveto(117.48428789,9.75728843)(117.68241683,9.22660838)(117.68241683,8.53296359)
\closepath
}
}
{
\newrgbcolor{curcolor}{0 0 0}
\pscustom[linestyle=none,fillstyle=solid,fillcolor=curcolor]
{
\newpath
\moveto(122.58927067,6.77795869)
\curveto(122.58927067,6.56067237)(122.52182252,6.37263613)(122.38692622,6.21384997)
\curveto(122.26046094,6.05506382)(122.09184057,5.97567074)(121.8810651,5.97567074)
\curveto(121.66185861,5.97567074)(121.48480722,6.05506382)(121.34991092,6.21384997)
\curveto(121.21501462,6.37263613)(121.14756647,6.56067237)(121.14756647,6.77795869)
\curveto(121.14756647,6.99524501)(121.21501462,7.18745983)(121.34991092,7.35460316)
\curveto(121.48480722,7.52174648)(121.66185861,7.60531814)(121.8810651,7.60531814)
\curveto(122.09184057,7.60531814)(122.26046094,7.52174648)(122.38692622,7.35460316)
\curveto(122.52182252,7.18745983)(122.58927067,6.99524501)(122.58927067,6.77795869)
\closepath
\moveto(119.27588032,6.64006545)
\curveto(119.27588032,7.94378337)(119.49930232,8.94246473)(119.94614631,9.63610952)
\curveto(120.40142132,10.33811148)(121.03796323,10.68911246)(121.85577204,10.68911246)
\curveto(122.68201188,10.68911246)(123.31855379,10.33811148)(123.76539778,9.63610952)
\curveto(124.21224177,8.94246473)(124.43566376,7.94378337)(124.43566376,6.64006545)
\curveto(124.43566376,5.33634753)(124.21224177,4.33348758)(123.76539778,3.63148563)
\curveto(123.31855379,2.93784083)(122.68201188,2.59101844)(121.85577204,2.59101844)
\curveto(121.03796323,2.59101844)(120.40142132,2.93784083)(119.94614631,3.63148563)
\curveto(119.49930232,4.33348758)(119.27588032,5.33634753)(119.27588032,6.64006545)
\closepath
\moveto(123.37335541,6.64006545)
\curveto(123.37335541,7.06628092)(123.34806235,7.4674249)(123.29747624,7.84349738)
\curveto(123.24689013,8.22792702)(123.16257994,8.56221367)(123.04454568,8.84635732)
\curveto(122.92651142,9.13050097)(122.77053757,9.35614445)(122.57662414,9.52328778)
\curveto(122.38271071,9.6904311)(122.14242668,9.77400276)(121.85577204,9.77400276)
\curveto(121.56911741,9.77400276)(121.32883338,9.6904311)(121.13491995,9.52328778)
\curveto(120.94100652,9.35614445)(120.78503267,9.13050097)(120.66699841,8.84635732)
\curveto(120.54896415,8.56221367)(120.46465396,8.22792702)(120.41406785,7.84349738)
\curveto(120.36348173,7.4674249)(120.33818868,7.06628092)(120.33818868,6.64006545)
\curveto(120.33818868,6.21384997)(120.36348173,5.80852741)(120.41406785,5.42409777)
\curveto(120.46465396,5.04802529)(120.54896415,4.71791723)(120.66699841,4.43377358)
\curveto(120.78503267,4.14962993)(120.94100652,3.92398644)(121.13491995,3.75684312)
\curveto(121.32883338,3.5896998)(121.56911741,3.50612813)(121.85577204,3.50612813)
\curveto(122.14242668,3.50612813)(122.38271071,3.5896998)(122.57662414,3.75684312)
\curveto(122.77053757,3.92398644)(122.92651142,4.14962993)(123.04454568,4.43377358)
\curveto(123.16257994,4.71791723)(123.24689013,5.04802529)(123.29747624,5.42409777)
\curveto(123.34806235,5.80852741)(123.37335541,6.21384997)(123.37335541,6.64006545)
\closepath
}
}
{
\newrgbcolor{curcolor}{0 0 0}
\pscustom[linestyle=none,fillstyle=solid,fillcolor=curcolor]
{
\newpath
\moveto(132.28915193,8.90903606)
\curveto(132.71913389,9.07617939)(133.13646931,9.28510854)(133.54115821,9.53582353)
\curveto(133.94584711,9.79489568)(134.32102744,10.12082516)(134.66669921,10.51361197)
\lineto(135.40019783,10.51361197)
\lineto(135.40019783,3.63148563)
\lineto(136.87984161,3.63148563)
\lineto(136.87984161,2.75398318)
\lineto(132.66854777,2.75398318)
\lineto(132.66854777,3.63148563)
\lineto(134.36318253,3.63148563)
\lineto(134.36318253,9.0720008)
\curveto(134.27044133,8.98842914)(134.15662257,8.9006789)(134.02172627,8.80875007)
\curveto(133.89526099,8.72517841)(133.75193368,8.64160675)(133.59174432,8.55803508)
\curveto(133.43998598,8.47446342)(133.27979663,8.39507034)(133.11117626,8.31985585)
\curveto(132.94255588,8.24464135)(132.77815102,8.18196261)(132.61796166,8.13181961)
\lineto(132.28915193,8.90903606)
\closepath
}
}
{
\newrgbcolor{curcolor}{0 0 0}
\pscustom[linestyle=none,fillstyle=solid,fillcolor=curcolor]
{
\newpath
\moveto(141.55905795,6.77795869)
\curveto(141.55905795,6.56067237)(141.4916098,6.37263613)(141.3567135,6.21384997)
\curveto(141.23024822,6.05506382)(141.06162784,5.97567074)(140.85085238,5.97567074)
\curveto(140.63164589,5.97567074)(140.4545945,6.05506382)(140.3196982,6.21384997)
\curveto(140.1848019,6.37263613)(140.11735375,6.56067237)(140.11735375,6.77795869)
\curveto(140.11735375,6.99524501)(140.1848019,7.18745983)(140.3196982,7.35460316)
\curveto(140.4545945,7.52174648)(140.63164589,7.60531814)(140.85085238,7.60531814)
\curveto(141.06162784,7.60531814)(141.23024822,7.52174648)(141.3567135,7.35460316)
\curveto(141.4916098,7.18745983)(141.55905795,6.99524501)(141.55905795,6.77795869)
\closepath
\moveto(138.2456676,6.64006545)
\curveto(138.2456676,7.94378337)(138.46908959,8.94246473)(138.91593359,9.63610952)
\curveto(139.3712086,10.33811148)(140.00775051,10.68911246)(140.82555932,10.68911246)
\curveto(141.65179915,10.68911246)(142.28834106,10.33811148)(142.73518506,9.63610952)
\curveto(143.18202905,8.94246473)(143.40545104,7.94378337)(143.40545104,6.64006545)
\curveto(143.40545104,5.33634753)(143.18202905,4.33348758)(142.73518506,3.63148563)
\curveto(142.28834106,2.93784083)(141.65179915,2.59101844)(140.82555932,2.59101844)
\curveto(140.00775051,2.59101844)(139.3712086,2.93784083)(138.91593359,3.63148563)
\curveto(138.46908959,4.33348758)(138.2456676,5.33634753)(138.2456676,6.64006545)
\closepath
\moveto(142.34314269,6.64006545)
\curveto(142.34314269,7.06628092)(142.31784963,7.4674249)(142.26726352,7.84349738)
\curveto(142.21667741,8.22792702)(142.13236722,8.56221367)(142.01433296,8.84635732)
\curveto(141.89629869,9.13050097)(141.74032485,9.35614445)(141.54641142,9.52328778)
\curveto(141.35249799,9.6904311)(141.11221396,9.77400276)(140.82555932,9.77400276)
\curveto(140.53890468,9.77400276)(140.29862065,9.6904311)(140.10470722,9.52328778)
\curveto(139.91079379,9.35614445)(139.75481995,9.13050097)(139.63678568,8.84635732)
\curveto(139.51875142,8.56221367)(139.43444124,8.22792702)(139.38385512,7.84349738)
\curveto(139.33326901,7.4674249)(139.30797596,7.06628092)(139.30797596,6.64006545)
\curveto(139.30797596,6.21384997)(139.33326901,5.80852741)(139.38385512,5.42409777)
\curveto(139.43444124,5.04802529)(139.51875142,4.71791723)(139.63678568,4.43377358)
\curveto(139.75481995,4.14962993)(139.91079379,3.92398644)(140.10470722,3.75684312)
\curveto(140.29862065,3.5896998)(140.53890468,3.50612813)(140.82555932,3.50612813)
\curveto(141.11221396,3.50612813)(141.35249799,3.5896998)(141.54641142,3.75684312)
\curveto(141.74032485,3.92398644)(141.89629869,4.14962993)(142.01433296,4.43377358)
\curveto(142.13236722,4.71791723)(142.21667741,5.04802529)(142.26726352,5.42409777)
\curveto(142.31784963,5.80852741)(142.34314269,6.21384997)(142.34314269,6.64006545)
\closepath
}
}
{
\newrgbcolor{curcolor}{0 0 0}
\pscustom[linestyle=none,fillstyle=solid,fillcolor=curcolor]
{
\newpath
\moveto(152.99152121,3.50612813)
\curveto(153.65757169,3.50612813)(154.12970874,3.63566421)(154.40793236,3.89473636)
\curveto(154.69458699,4.16216568)(154.83791431,4.51734524)(154.83791431,4.96027505)
\curveto(154.83791431,5.2444187)(154.77889718,5.48259793)(154.66086292,5.67481276)
\curveto(154.54282865,5.86702758)(154.38685481,6.02163515)(154.19294138,6.13863548)
\curveto(153.99902795,6.2556358)(153.77560595,6.33920747)(153.52267539,6.38935046)
\curveto(153.26974483,6.43949346)(153.00416774,6.46456496)(152.72594413,6.46456496)
\lineto(152.46036704,6.46456496)
\lineto(152.46036704,7.30446016)
\lineto(152.82711635,7.30446016)
\curveto(153.01259876,7.30446016)(153.20229668,7.32117449)(153.39621011,7.35460316)
\curveto(153.59855456,7.39638899)(153.77982146,7.46324632)(153.94001082,7.55517514)
\curveto(154.10863119,7.65546114)(154.24352749,7.7891758)(154.34469972,7.95631912)
\curveto(154.44587194,8.12346244)(154.49645805,8.33657018)(154.49645805,8.59564233)
\curveto(154.49645805,9.02185781)(154.36156175,9.32271579)(154.09176915,9.49821628)
\curveto(153.83040757,9.68207393)(153.52267539,9.77400276)(153.16857261,9.77400276)
\curveto(152.8060388,9.77400276)(152.49830662,9.71968118)(152.24537606,9.61103802)
\curveto(151.9924455,9.51075203)(151.78167003,9.40628745)(151.61304966,9.29764429)
\lineto(151.20836076,10.08739649)
\curveto(151.38541215,10.21275399)(151.65098924,10.34229006)(152.00509203,10.47600472)
\curveto(152.36762583,10.61807655)(152.76809922,10.68911246)(153.20651219,10.68911246)
\curveto(153.61963211,10.68911246)(153.97373489,10.63896946)(154.26882055,10.53868347)
\curveto(154.5639062,10.43839747)(154.80419023,10.29632565)(154.98967265,10.11246799)
\curveto(155.18358608,9.92861034)(155.32691339,9.71132402)(155.4196546,9.46060903)
\curveto(155.5123958,9.21825121)(155.55876641,8.95082189)(155.55876641,8.65832108)
\curveto(155.55876641,8.24881994)(155.44916316,7.90199754)(155.22995668,7.61785389)
\curveto(155.01918121,7.33371024)(154.7451731,7.11642392)(154.40793236,6.96599493)
\curveto(154.81262125,6.8489946)(155.16250853,6.61917253)(155.45759418,6.27652872)
\curveto(155.75267984,5.94224207)(155.90022266,5.49513368)(155.90022266,4.93520355)
\curveto(155.90022266,4.6009169)(155.84120553,4.28752317)(155.72317127,3.99502235)
\curveto(155.61356803,3.7108787)(155.44073215,3.4643423)(155.20466362,3.25541315)
\curveto(154.97702612,3.04648399)(154.67772495,2.88351925)(154.30676013,2.76651893)
\curveto(153.94422633,2.6495186)(153.51002886,2.59101844)(153.00416774,2.59101844)
\curveto(152.81025431,2.59101844)(152.60790986,2.60773277)(152.3971344,2.64116143)
\curveto(152.19478995,2.66623293)(152.00509203,2.70384018)(151.82804063,2.75398318)
\curveto(151.65098924,2.79576901)(151.49079989,2.83755484)(151.34747257,2.87934067)
\curveto(151.21257627,2.92948367)(151.11561955,2.96709092)(151.05660242,2.99216241)
\lineto(151.25894687,3.88220061)
\curveto(151.39384317,3.81534328)(151.60883415,3.7359502)(151.9039198,3.64402138)
\curveto(152.19900546,3.55209255)(152.56153926,3.50612813)(152.99152121,3.50612813)
\closepath
}
}
{
\newrgbcolor{curcolor}{0 0 0}
\pscustom[linestyle=none,fillstyle=solid,fillcolor=curcolor]
{
\newpath
\moveto(158.5939341,2.75398318)
\curveto(158.63608919,3.34734198)(158.74147692,3.97412944)(158.9100973,4.63434557)
\curveto(159.08714869,5.30291886)(159.29792416,5.94642066)(159.5424237,6.56485095)
\curveto(159.78692324,7.19163842)(160.05671584,7.76828288)(160.35180149,8.29478435)
\curveto(160.64688715,8.82964298)(160.93775729,9.26839421)(161.22441193,9.61103802)
\lineto(157.43045352,9.61103802)
\lineto(157.43045352,10.51361197)
\lineto(162.38789251,10.51361197)
\lineto(162.38789251,9.64864527)
\curveto(162.13496195,9.35614445)(161.86095384,8.96335764)(161.56586819,8.47028484)
\curveto(161.27078253,7.97721204)(160.98834341,7.42146048)(160.71855081,6.80303019)
\curveto(160.45718923,6.19295706)(160.22955172,5.53691951)(160.03563829,4.83491755)
\curveto(159.84172486,4.14127276)(159.71947509,3.44762797)(159.66888898,2.75398318)
\lineto(158.5939341,2.75398318)
\closepath
}
}
{
\newrgbcolor{curcolor}{0 0 0}
\pscustom[linestyle=none,fillstyle=solid,fillcolor=curcolor]
{
\newpath
\moveto(181.25651522,7.40474615)
\curveto(181.25651522,5.85867041)(180.87711938,4.69284573)(180.1183277,3.90727211)
\curveto(179.35953602,3.13005566)(178.21713298,2.73726885)(176.6911186,2.72891168)
\lineto(176.65317902,3.60641413)
\curveto(177.59745311,3.60641413)(178.35624479,3.79027178)(178.92955406,4.15798709)
\curveto(179.51129435,4.53405957)(179.89490571,5.17756137)(180.08038812,6.08849248)
\curveto(179.87804367,5.99656365)(179.65462167,5.92134916)(179.41012213,5.86284899)
\curveto(179.16562259,5.812706)(178.90847652,5.7876345)(178.63868392,5.7876345)
\curveto(178.19183993,5.7876345)(177.8166596,5.85031324)(177.51314292,5.97567074)
\curveto(177.20962625,6.1093854)(176.96512671,6.28488589)(176.7796443,6.50217221)
\curveto(176.59416189,6.72781569)(176.45926559,6.98270926)(176.3749554,7.26685291)
\curveto(176.29064521,7.55099656)(176.24849012,7.85185454)(176.24849012,8.16942686)
\curveto(176.24849012,8.45357051)(176.29486072,8.74189274)(176.38760193,9.03439356)
\curveto(176.48034313,9.33525154)(176.62367045,9.60685944)(176.81758388,9.84921726)
\curveto(177.01149731,10.09157508)(177.26021236,10.29214706)(177.56372904,10.45093322)
\curveto(177.86724571,10.60971938)(178.22977951,10.68911246)(178.65133045,10.68911246)
\curveto(179.51129435,10.68911246)(180.16048279,10.39661164)(180.59889577,9.81161001)
\curveto(181.03730874,9.22660838)(181.25651522,8.42432043)(181.25651522,7.40474615)
\closepath
\moveto(178.75250267,6.64006545)
\curveto(179.02229527,6.64006545)(179.27101032,6.66513695)(179.49864783,6.71527994)
\curveto(179.73471635,6.76542294)(179.96235385,6.83645885)(180.18156034,6.92838768)
\curveto(180.18999136,7.01195934)(180.19420687,7.09135242)(180.19420687,7.16656692)
\lineto(180.19420687,7.40474615)
\curveto(180.19420687,7.73067563)(180.16891381,8.03989078)(180.1183277,8.3323916)
\curveto(180.07617261,8.62489241)(179.99607793,8.87978598)(179.87804367,9.0970723)
\curveto(179.76844042,9.31435862)(179.61246658,9.48568053)(179.41012213,9.61103802)
\curveto(179.2162087,9.74475268)(178.96749365,9.81161001)(178.66397698,9.81161001)
\curveto(178.41104641,9.81161001)(178.20027095,9.75728843)(178.03165057,9.64864527)
\curveto(177.8630302,9.54835928)(177.72391839,9.4188232)(177.61431515,9.26003704)
\curveto(177.50471191,9.10960805)(177.42461723,8.93828614)(177.37403112,8.74607132)
\curveto(177.33187602,8.5538565)(177.31079848,8.36999885)(177.31079848,8.19449836)
\curveto(177.31079848,7.68471122)(177.42461723,7.29610299)(177.65225473,7.02867367)
\curveto(177.88832326,6.76960152)(178.25507257,6.64006545)(178.75250267,6.64006545)
\closepath
}
}
{
\newrgbcolor{curcolor}{0 0 0}
\pscustom[linestyle=none,fillstyle=solid,fillcolor=curcolor]
{
\newpath
\moveto(190.93110343,3.50612813)
\curveto(191.59715391,3.50612813)(192.06929096,3.63566421)(192.34751457,3.89473636)
\curveto(192.63416921,4.16216568)(192.77749653,4.51734524)(192.77749653,4.96027505)
\curveto(192.77749653,5.2444187)(192.71847939,5.48259793)(192.60044513,5.67481276)
\curveto(192.48241087,5.86702758)(192.32643703,6.02163515)(192.1325236,6.13863548)
\curveto(191.93861017,6.2556358)(191.71518817,6.33920747)(191.46225761,6.38935046)
\curveto(191.20932705,6.43949346)(190.94374996,6.46456496)(190.66552634,6.46456496)
\lineto(190.39994925,6.46456496)
\lineto(190.39994925,7.30446016)
\lineto(190.76669857,7.30446016)
\curveto(190.95218098,7.30446016)(191.1418789,7.32117449)(191.33579233,7.35460316)
\curveto(191.53813678,7.39638899)(191.71940368,7.46324632)(191.87959303,7.55517514)
\curveto(192.04821341,7.65546114)(192.18310971,7.7891758)(192.28428193,7.95631912)
\curveto(192.38545416,8.12346244)(192.43604027,8.33657018)(192.43604027,8.59564233)
\curveto(192.43604027,9.02185781)(192.30114397,9.32271579)(192.03135137,9.49821628)
\curveto(191.76998979,9.68207393)(191.46225761,9.77400276)(191.10815482,9.77400276)
\curveto(190.74562102,9.77400276)(190.43788884,9.71968118)(190.18495828,9.61103802)
\curveto(189.93202772,9.51075203)(189.72125225,9.40628745)(189.55263187,9.29764429)
\lineto(189.14794298,10.08739649)
\curveto(189.32499437,10.21275399)(189.59057146,10.34229006)(189.94467424,10.47600472)
\curveto(190.30720805,10.61807655)(190.70768144,10.68911246)(191.14609441,10.68911246)
\curveto(191.55921432,10.68911246)(191.91331711,10.63896946)(192.20840276,10.53868347)
\curveto(192.50348842,10.43839747)(192.74377245,10.29632565)(192.92925486,10.11246799)
\curveto(193.12316829,9.92861034)(193.26649561,9.71132402)(193.35923682,9.46060903)
\curveto(193.45197802,9.21825121)(193.49834862,8.95082189)(193.49834862,8.65832108)
\curveto(193.49834862,8.24881994)(193.38874538,7.90199754)(193.1695389,7.61785389)
\curveto(192.95876343,7.33371024)(192.68475532,7.11642392)(192.34751457,6.96599493)
\curveto(192.75220347,6.8489946)(193.10209075,6.61917253)(193.3971764,6.27652872)
\curveto(193.69226205,5.94224207)(193.83980488,5.49513368)(193.83980488,4.93520355)
\curveto(193.83980488,4.6009169)(193.78078775,4.28752317)(193.66275349,3.99502235)
\curveto(193.55315025,3.7108787)(193.38031436,3.4643423)(193.14424584,3.25541315)
\curveto(192.91660833,3.04648399)(192.61730717,2.88351925)(192.24634235,2.76651893)
\curveto(191.88380854,2.6495186)(191.44961108,2.59101844)(190.94374996,2.59101844)
\curveto(190.74983653,2.59101844)(190.54749208,2.60773277)(190.33671661,2.64116143)
\curveto(190.13437216,2.66623293)(189.94467424,2.70384018)(189.76762285,2.75398318)
\curveto(189.59057146,2.79576901)(189.4303821,2.83755484)(189.28705479,2.87934067)
\curveto(189.15215849,2.92948367)(189.05520177,2.96709092)(188.99618464,2.99216241)
\lineto(189.19852909,3.88220061)
\curveto(189.33342539,3.81534328)(189.54841637,3.7359502)(189.84350202,3.64402138)
\curveto(190.13858767,3.55209255)(190.50112148,3.50612813)(190.93110343,3.50612813)
\closepath
}
}
{
\newrgbcolor{curcolor}{0 0 0}
\pscustom[linestyle=none,fillstyle=solid,fillcolor=curcolor]
{
\newpath
\moveto(197.25437607,3.50612813)
\curveto(197.92042655,3.50612813)(198.3925636,3.63566421)(198.67078722,3.89473636)
\curveto(198.95744185,4.16216568)(199.10076917,4.51734524)(199.10076917,4.96027505)
\curveto(199.10076917,5.2444187)(199.04175204,5.48259793)(198.92371778,5.67481276)
\curveto(198.80568351,5.86702758)(198.64970967,6.02163515)(198.45579624,6.13863548)
\curveto(198.26188281,6.2556358)(198.03846081,6.33920747)(197.78553025,6.38935046)
\curveto(197.53259969,6.43949346)(197.2670226,6.46456496)(196.98879899,6.46456496)
\lineto(196.7232219,6.46456496)
\lineto(196.7232219,7.30446016)
\lineto(197.08997121,7.30446016)
\curveto(197.27545362,7.30446016)(197.46515154,7.32117449)(197.65906497,7.35460316)
\curveto(197.86140942,7.39638899)(198.04267632,7.46324632)(198.20286568,7.55517514)
\curveto(198.37148605,7.65546114)(198.50638235,7.7891758)(198.60755457,7.95631912)
\curveto(198.7087268,8.12346244)(198.75931291,8.33657018)(198.75931291,8.59564233)
\curveto(198.75931291,9.02185781)(198.62441661,9.32271579)(198.35462401,9.49821628)
\curveto(198.09326243,9.68207393)(197.78553025,9.77400276)(197.43142747,9.77400276)
\curveto(197.06889366,9.77400276)(196.76116148,9.71968118)(196.50823092,9.61103802)
\curveto(196.25530036,9.51075203)(196.04452489,9.40628745)(195.87590452,9.29764429)
\lineto(195.47121562,10.08739649)
\curveto(195.64826701,10.21275399)(195.9138441,10.34229006)(196.26794689,10.47600472)
\curveto(196.63048069,10.61807655)(197.03095408,10.68911246)(197.46936705,10.68911246)
\curveto(197.88248697,10.68911246)(198.23658975,10.63896946)(198.53167541,10.53868347)
\curveto(198.82676106,10.43839747)(199.06704509,10.29632565)(199.25252751,10.11246799)
\curveto(199.44644094,9.92861034)(199.58976825,9.71132402)(199.68250946,9.46060903)
\curveto(199.77525066,9.21825121)(199.82162127,8.95082189)(199.82162127,8.65832108)
\curveto(199.82162127,8.24881994)(199.71201802,7.90199754)(199.49281154,7.61785389)
\curveto(199.28203607,7.33371024)(199.00802796,7.11642392)(198.67078722,6.96599493)
\curveto(199.07547611,6.8489946)(199.42536339,6.61917253)(199.72044904,6.27652872)
\curveto(200.0155347,5.94224207)(200.16307752,5.49513368)(200.16307752,4.93520355)
\curveto(200.16307752,4.6009169)(200.10406039,4.28752317)(199.98602613,3.99502235)
\curveto(199.87642289,3.7108787)(199.70358701,3.4643423)(199.46751848,3.25541315)
\curveto(199.23988098,3.04648399)(198.94057981,2.88351925)(198.56961499,2.76651893)
\curveto(198.20708119,2.6495186)(197.77288372,2.59101844)(197.2670226,2.59101844)
\curveto(197.07310917,2.59101844)(196.87076472,2.60773277)(196.65998926,2.64116143)
\curveto(196.45764481,2.66623293)(196.26794689,2.70384018)(196.09089549,2.75398318)
\curveto(195.9138441,2.79576901)(195.75365475,2.83755484)(195.61032743,2.87934067)
\curveto(195.47543113,2.92948367)(195.37847441,2.96709092)(195.31945728,2.99216241)
\lineto(195.52180173,3.88220061)
\curveto(195.65669803,3.81534328)(195.87168901,3.7359502)(196.16677466,3.64402138)
\curveto(196.46186032,3.55209255)(196.82439412,3.50612813)(197.25437607,3.50612813)
\closepath
}
}
{
\newrgbcolor{curcolor}{0 0 0}
\pscustom[linestyle=none,fillstyle=solid,fillcolor=curcolor]
{
\newpath
\moveto(214.22601192,5.86284899)
\curveto(214.22601192,6.61499395)(214.32718414,7.27938866)(214.52952859,7.85603313)
\curveto(214.74030406,8.44103476)(215.03538971,8.92992898)(215.41478556,9.32271579)
\curveto(215.80261242,9.7155026)(216.27053395,10.01636058)(216.81855017,10.22528974)
\curveto(217.36656638,10.43421889)(217.98624626,10.54286205)(218.67758979,10.55121922)
\lineto(218.76611549,9.67371677)
\curveto(218.3192715,9.6653596)(217.91036709,9.6152166)(217.53940227,9.52328778)
\curveto(217.17686846,9.43971612)(216.85227424,9.29764429)(216.56561961,9.0970723)
\curveto(216.27896497,8.90485748)(216.03868094,8.6541425)(215.84476751,8.34492735)
\curveto(215.65085408,8.0357122)(215.50331125,7.65546114)(215.40213903,7.20417416)
\curveto(215.60448348,7.29610299)(215.82368996,7.37131749)(216.05975849,7.42981765)
\curveto(216.30425803,7.48831781)(216.5614041,7.5175679)(216.8311967,7.5175679)
\curveto(217.26960967,7.5175679)(217.64057449,7.45071057)(217.94409117,7.31699591)
\curveto(218.25603886,7.18328125)(218.50475391,7.00360218)(218.69023632,6.77795869)
\curveto(218.87571873,6.56067237)(219.01061503,6.3057788)(219.09492522,6.01327798)
\curveto(219.1792354,5.72077717)(219.2213905,5.41991919)(219.2213905,5.11070404)
\curveto(219.2213905,4.82656039)(219.17501989,4.53405957)(219.08227869,4.23320159)
\curveto(218.98953748,3.94070077)(218.84621017,3.66909287)(218.65229674,3.41837789)
\curveto(218.45838331,3.17602007)(218.20966825,2.97544808)(217.90615158,2.81666192)
\curveto(217.60263491,2.66623293)(217.2401011,2.59101844)(216.81855017,2.59101844)
\curveto(215.95015524,2.59101844)(215.3009668,2.87934067)(214.87098485,3.45598514)
\curveto(214.4410029,4.0326296)(214.22601192,4.83491755)(214.22601192,5.86284899)
\closepath
\moveto(216.71737794,6.66513695)
\curveto(216.44758535,6.66513695)(216.19887029,6.64006545)(215.97123279,6.58992245)
\curveto(215.7520263,6.53977945)(215.52860431,6.46456496)(215.3009668,6.36427896)
\curveto(215.29253579,6.2807073)(215.28832028,6.19713564)(215.28832028,6.11356398)
\lineto(215.28832028,5.86284899)
\curveto(215.28832028,5.53691951)(215.30939782,5.22770436)(215.35155292,4.93520355)
\curveto(215.40213903,4.6510599)(215.48223371,4.39616633)(215.59183695,4.17052284)
\curveto(215.70987121,3.95323652)(215.86584506,3.77773603)(216.05975849,3.64402138)
\curveto(216.25367192,3.51866388)(216.50238697,3.45598514)(216.80590364,3.45598514)
\curveto(217.0588342,3.45598514)(217.26960967,3.50612813)(217.43823004,3.60641413)
\curveto(217.60685042,3.71505729)(217.74596223,3.84877195)(217.85556547,4.0075581)
\curveto(217.96516871,4.16634426)(218.04104788,4.34184475)(218.08320297,4.53405957)
\curveto(218.13378909,4.73463156)(218.15908214,4.9226678)(218.15908214,5.09816829)
\curveto(218.15908214,5.60795543)(218.04104788,5.99656365)(217.80497936,6.26399297)
\curveto(217.57734185,6.53142229)(217.21480805,6.66513695)(216.71737794,6.66513695)
\closepath
}
}
{
\newrgbcolor{curcolor}{0 0 0}
\pscustom[linewidth=0.48927385,linecolor=curcolor]
{
\newpath
\moveto(52.4990589,13.03547626)
\lineto(222.46796149,13.03547626)
\lineto(222.46796149,0.24465232)
\lineto(52.4990589,0.24465232)
\closepath
}
}
{
\newrgbcolor{curcolor}{0 0 0}
\pscustom[linewidth=0.51800942,linecolor=curcolor]
{
\newpath
\moveto(71.388227,0.29461833)
\lineto(71.388227,12.89401333)
}
}
{
\newrgbcolor{curcolor}{0 0 0}
\pscustom[linewidth=0.5154072,linecolor=curcolor]
{
\newpath
\moveto(90.26845,0.38892933)
\lineto(90.26845,12.86205533)
}
}
{
\newrgbcolor{curcolor}{0 0 0}
\pscustom[linewidth=0.51930571,linecolor=curcolor]
{
\newpath
\moveto(109.10774,0.38892833)
\lineto(109.10774,13.05145733)
}
}
{
\newrgbcolor{curcolor}{0 0 0}
\pscustom[linewidth=0.52573889,linecolor=curcolor]
{
\newpath
\moveto(128.03631,0.21035733)
\lineto(128.03631,13.18855933)
}
}
{
\newrgbcolor{curcolor}{0 0 0}
\pscustom[linewidth=0.51670998,linecolor=curcolor]
{
\newpath
\moveto(146.87559,0.38892833)
\lineto(146.87559,12.92518833)
}
}
{
\newrgbcolor{curcolor}{0 0 0}
\pscustom[linewidth=0.52059871,linecolor=curcolor]
{
\newpath
\moveto(165.6256,0.38892933)
\lineto(165.6256,13.11459333)
}
}
{
\newrgbcolor{curcolor}{0 0 0}
\pscustom[linewidth=0.51670998,linecolor=curcolor]
{
\newpath
\moveto(184.55417,0.29964333)
\lineto(184.55417,12.83590333)
}
}
{
\newrgbcolor{curcolor}{0 0 0}
\pscustom[linewidth=0.51670998,linecolor=curcolor]
{
\newpath
\moveto(203.66131,0.38892933)
\lineto(203.66131,12.92518933)
}
}
{
\newrgbcolor{curcolor}{0 0 0}
\pscustom[linestyle=none,fillstyle=solid,fillcolor=curcolor]
{
\newpath
\moveto(4.73422218,7.79842477)
\curveto(4.73422218,7.31996613)(4.60831201,6.89606856)(4.35649168,6.52673207)
\curveto(4.10467134,6.16578959)(3.76471389,5.88459022)(3.33661932,5.68313395)
\curveto(3.46252949,5.48167768)(3.60103067,5.25084237)(3.75212287,4.99062803)
\curveto(3.91160909,4.73880769)(4.0710953,4.46600233)(4.23058151,4.17221194)
\curveto(4.39006772,3.88681556)(4.54535693,3.58882816)(4.69644913,3.27824975)
\curveto(4.85593534,2.97606534)(5.00283054,2.67807795)(5.13713472,2.38428755)
\lineto(4.02912524,2.38428755)
\curveto(3.77730491,2.98865636)(3.50030254,3.56784313)(3.19811814,4.12184787)
\curveto(2.90432775,4.67585261)(2.61893137,5.1249322)(2.341929,5.46908666)
\curveto(2.29156493,5.46069265)(2.21601883,5.45649565)(2.11529069,5.45649565)
\lineto(1.92642544,5.45649565)
\lineto(1.04505427,5.45649565)
\lineto(1.04505427,2.38428755)
\lineto(-0.00000012,2.38428755)
\lineto(-0.00000012,10.06480778)
\curveto(0.12591004,10.09838383)(0.26860823,10.12776287)(0.42809445,10.1529449)
\curveto(0.59597467,10.17812693)(0.76385489,10.19491496)(0.93173512,10.20330897)
\curveto(1.10800935,10.22009699)(1.28008658,10.23268801)(1.44796681,10.24108202)
\curveto(1.61584703,10.24947603)(1.76693923,10.25367304)(1.90124341,10.25367304)
\curveto(2.84137266,10.25367304)(3.5464696,10.04801976)(4.01653423,9.63671321)
\curveto(4.49499286,9.22540667)(4.73422218,8.61264385)(4.73422218,7.79842477)
\closepath
\moveto(1.98938053,9.35971084)
\curveto(1.79631827,9.35971084)(1.60745302,9.35551384)(1.42278477,9.34711983)
\curveto(1.24651054,9.34711983)(1.12060037,9.33872582)(1.04505427,9.32193779)
\lineto(1.04505427,6.31268479)
\lineto(1.71237816,6.31268479)
\curveto(2.01456256,6.31268479)(2.28317092,6.33366981)(2.51820323,6.37563987)
\curveto(2.76162956,6.41760993)(2.96728283,6.49315603)(3.13516305,6.60227817)
\curveto(3.31143729,6.71979433)(3.44574147,6.87508354)(3.53807559,7.06814579)
\curveto(3.63040971,7.26960206)(3.67657677,7.52981641)(3.67657677,7.84878883)
\curveto(3.67657677,8.15097324)(3.63040971,8.39859656)(3.53807559,8.59165882)
\curveto(3.44574147,8.78472108)(3.3240283,8.93581328)(3.1729361,9.04493543)
\curveto(3.0218439,9.16245158)(2.84137266,9.24219469)(2.63152238,9.28416474)
\curveto(2.43006611,9.33452881)(2.21601883,9.35971084)(1.98938053,9.35971084)
\closepath
}
}
{
\newrgbcolor{curcolor}{0 0 0}
\pscustom[linestyle=none,fillstyle=solid,fillcolor=curcolor]
{
\newpath
\moveto(8.86407568,8.36502052)
\curveto(9.65311273,8.36502052)(10.26167854,8.11739719)(10.68977311,7.62215053)
\curveto(11.11786768,7.13529788)(11.33191497,6.39242789)(11.33191497,5.39354056)
\lineto(11.33191497,5.02840108)
\lineto(7.1516974,5.02840108)
\curveto(7.19366746,4.42403227)(7.39092672,3.96236166)(7.74347519,3.64338923)
\curveto(8.10441767,3.33281082)(8.60805834,3.17752161)(9.2543972,3.17752161)
\curveto(9.62373369,3.17752161)(9.93850911,3.20690065)(10.19872346,3.26565873)
\curveto(10.45893781,3.32441681)(10.65619707,3.38737189)(10.79050125,3.45452398)
\lineto(10.92900243,2.57315281)
\curveto(10.80309227,2.50600072)(10.57645396,2.43465162)(10.24908753,2.35910552)
\curveto(9.92172109,2.28355942)(9.5523846,2.24578637)(9.14107805,2.24578637)
\curveto(8.63743738,2.24578637)(8.19255479,2.32133247)(7.80643027,2.47242467)
\curveto(7.42869977,2.63191088)(7.11392435,2.84595817)(6.86210402,3.11456653)
\curveto(6.61028368,3.38317489)(6.42141843,3.70214731)(6.29550826,4.0714838)
\curveto(6.16959809,4.4492143)(6.10664301,4.85632385)(6.10664301,5.29281243)
\curveto(6.10664301,5.81324112)(6.18638612,6.26651773)(6.34587233,6.65264224)
\curveto(6.50535854,7.03876675)(6.71520882,7.35773918)(6.97542317,7.60955951)
\curveto(7.23563751,7.86137985)(7.52942791,8.0502451)(7.85679434,8.17615527)
\curveto(8.18416078,8.30206544)(8.51992122,8.36502052)(8.86407568,8.36502052)
\closepath
\moveto(10.27426956,5.8719992)
\curveto(10.27426956,6.36724586)(10.14416239,6.75756738)(9.88394804,7.04296376)
\curveto(9.62373369,7.33675415)(9.27957924,7.48364935)(8.85148467,7.48364935)
\curveto(8.60805834,7.48364935)(8.38561705,7.43748228)(8.18416078,7.34514816)
\curveto(7.99109852,7.25281404)(7.8232183,7.13110088)(7.68052011,6.98000868)
\curveto(7.53782192,6.82891647)(7.42450277,6.65683925)(7.34056265,6.46377699)
\curveto(7.25662254,6.27071473)(7.20206147,6.07345547)(7.17687944,5.8719992)
\lineto(10.27426956,5.8719992)
\closepath
}
}
{
\newrgbcolor{curcolor}{0 0 0}
\pscustom[linestyle=none,fillstyle=solid,fillcolor=curcolor]
{
\newpath
\moveto(16.31795761,3.8448455)
\curveto(16.31795761,4.05469578)(16.22982049,4.22677301)(16.05354626,4.36107719)
\curveto(15.88566603,4.49538137)(15.67161875,4.61289752)(15.4114044,4.71362566)
\curveto(15.15958407,4.81435379)(14.8825817,4.91088492)(14.5803973,5.00321904)
\curveto(14.27821289,5.10394718)(13.99701352,5.22566034)(13.73679917,5.36835853)
\curveto(13.48497884,5.51105672)(13.27093155,5.68733095)(13.09465732,5.89718123)
\curveto(12.92677709,6.10703151)(12.84283698,6.37983688)(12.84283698,6.71559732)
\curveto(12.84283698,7.18566195)(13.03170223,7.57598347)(13.40943274,7.88656188)
\curveto(13.79555725,8.20553431)(14.39572905,8.36502052)(15.20994813,8.36502052)
\curveto(15.52892056,8.36502052)(15.856287,8.33983849)(16.19204744,8.28947442)
\curveto(16.5362019,8.24750436)(16.82999229,8.18874629)(17.07341862,8.11320018)
\lineto(16.88455336,7.18146494)
\curveto(16.81740128,7.21504099)(16.72506715,7.24861703)(16.607551,7.28219308)
\curveto(16.49003484,7.32416313)(16.35573066,7.35773918)(16.20463846,7.38292121)
\curveto(16.05354626,7.41649726)(15.88986304,7.44167929)(15.71358881,7.45846731)
\curveto(15.54570858,7.47525534)(15.38202536,7.48364935)(15.22253915,7.48364935)
\curveto(14.31598594,7.48364935)(13.86270934,7.23602602)(13.86270934,6.74077936)
\curveto(13.86270934,6.56450512)(13.94664945,6.41341292)(14.11452968,6.28750275)
\curveto(14.29080391,6.1699866)(14.5090482,6.06086445)(14.76926255,5.96013632)
\curveto(15.02947689,5.85940818)(15.31067627,5.75448304)(15.61286067,5.6453609)
\curveto(15.91504507,5.54463276)(16.19624445,5.4187226)(16.45645879,5.26763039)
\curveto(16.71667314,5.11653819)(16.93072043,4.93186995)(17.09860065,4.71362566)
\curveto(17.27487488,4.50377538)(17.363012,4.23516702)(17.363012,3.90780058)
\curveto(17.363012,3.37897788)(17.15735873,2.96767133)(16.74605218,2.67388094)
\curveto(16.33474563,2.38848456)(15.68420977,2.24578637)(14.79444458,2.24578637)
\curveto(14.39153204,2.24578637)(14.02219555,2.27936241)(13.68643511,2.3465145)
\curveto(13.35067466,2.41366659)(13.03170223,2.51439473)(12.72951783,2.64869891)
\lineto(12.9309741,3.59302516)
\curveto(13.22476449,3.45872099)(13.5311459,3.34540183)(13.85011832,3.25306771)
\curveto(14.17748476,3.1691276)(14.52583622,3.12715754)(14.89517272,3.12715754)
\curveto(15.84369598,3.12715754)(16.31795761,3.36638686)(16.31795761,3.8448455)
\closepath
}
}
{
\newrgbcolor{curcolor}{0 0 0}
\pscustom[linestyle=none,fillstyle=solid,fillcolor=curcolor]
{
\newpath
\moveto(23.59556435,2.54797077)
\curveto(23.36892605,2.48921269)(23.06674165,2.42625761)(22.68901114,2.35910552)
\curveto(22.31967465,2.29195343)(21.88318607,2.25837739)(21.3795454,2.25837739)
\curveto(20.93466281,2.25837739)(20.56532631,2.32133247)(20.27153592,2.44724264)
\curveto(19.97774553,2.58154682)(19.73851621,2.76621506)(19.55384797,3.00124738)
\curveto(19.36917972,3.2446737)(19.23907255,3.53007008)(19.16352645,3.85743652)
\curveto(19.08798035,4.18480295)(19.0502073,4.54574543)(19.0502073,4.94026396)
\lineto(19.0502073,8.22651934)
\lineto(20.08267067,8.22651934)
\lineto(20.08267067,5.16690226)
\curveto(20.08267067,4.4450173)(20.18759581,3.93298262)(20.39744609,3.63079821)
\curveto(20.61569038,3.32861381)(20.97663286,3.17752161)(21.48027353,3.17752161)
\curveto(21.58939568,3.17752161)(21.69851782,3.18171862)(21.80763997,3.19011263)
\curveto(21.92515612,3.19850664)(22.03427827,3.20690065)(22.1350064,3.21529466)
\curveto(22.23573454,3.23208268)(22.32387166,3.2446737)(22.39941776,3.25306771)
\curveto(22.47496386,3.26985573)(22.52532792,3.28244675)(22.55050996,3.29084076)
\lineto(22.55050996,8.22651934)
\lineto(23.59556435,8.22651934)
\lineto(23.59556435,2.54797077)
\closepath
}
}
{
\newrgbcolor{curcolor}{0 0 0}
\pscustom[linestyle=none,fillstyle=solid,fillcolor=curcolor]
{
\newpath
\moveto(28.74529116,2.24578637)
\curveto(28.37595467,2.24578637)(28.06957326,2.29615044)(27.82614694,2.39687857)
\curveto(27.58272061,2.49760671)(27.38965836,2.64869891)(27.24696017,2.85015518)
\curveto(27.10426198,3.05161144)(27.00353384,3.29923477)(26.94477576,3.59302516)
\curveto(26.88601769,3.89520957)(26.85663865,4.24356103)(26.85663865,4.63807956)
\lineto(26.85663865,10.16553592)
\lineto(25.1694424,10.16553592)
\lineto(25.1694424,11.04690709)
\lineto(27.88910202,11.04690709)
\lineto(27.88910202,4.63807956)
\curveto(27.88910202,4.34428917)(27.90589004,4.10505985)(27.93946609,3.9203916)
\curveto(27.98143614,3.73572335)(28.04019422,3.58463115)(28.11574032,3.467115)
\curveto(28.19968044,3.35799285)(28.30040857,3.28244675)(28.41792473,3.24047669)
\curveto(28.53544088,3.19850664)(28.67394207,3.17752161)(28.83342828,3.17752161)
\curveto(29.0768546,3.17752161)(29.30349291,3.20690065)(29.51334318,3.26565873)
\curveto(29.72319346,3.32441681)(29.88687668,3.38737189)(30.00439284,3.45452398)
\lineto(30.15548504,2.57315281)
\curveto(30.10512097,2.54797077)(30.03377188,2.51439473)(29.94143775,2.47242467)
\curveto(29.84910363,2.43884863)(29.73998149,2.40527258)(29.61407132,2.37169654)
\curveto(29.48816115,2.33812049)(29.34965997,2.30874145)(29.19856777,2.28355942)
\curveto(29.05586958,2.25837739)(28.90477737,2.24578637)(28.74529116,2.24578637)
\closepath
}
}
{
\newrgbcolor{curcolor}{0 0 0}
\pscustom[linestyle=none,fillstyle=solid,fillcolor=curcolor]
{
\newpath
\moveto(33.84465391,8.22651934)
\lineto(36.31249319,8.22651934)
\lineto(36.31249319,7.35773918)
\lineto(33.84465391,7.35773918)
\lineto(33.84465391,4.63807956)
\curveto(33.84465391,4.34428917)(33.86563893,4.10505985)(33.90760899,3.9203916)
\curveto(33.94957905,3.73572335)(34.01673114,3.58463115)(34.10906526,3.467115)
\curveto(34.20979339,3.35799285)(34.33570356,3.28244675)(34.48679576,3.24047669)
\curveto(34.63788796,3.19850664)(34.82255621,3.17752161)(35.0408005,3.17752161)
\curveto(35.3429849,3.17752161)(35.58641123,3.20270364)(35.77107947,3.25306771)
\curveto(35.95574772,3.30343178)(36.13202195,3.37058387)(36.29990218,3.45452398)
\lineto(36.45099438,2.57315281)
\curveto(36.33347822,2.52278874)(36.14461297,2.45563665)(35.88439862,2.37169654)
\curveto(35.63257829,2.28775643)(35.31780287,2.24578637)(34.94007237,2.24578637)
\curveto(34.50358378,2.24578637)(34.14683831,2.29615044)(33.86983594,2.39687857)
\curveto(33.59283357,2.49760671)(33.37458928,2.64869891)(33.21510307,2.85015518)
\curveto(33.06401087,3.05161144)(32.95908573,3.29923477)(32.90032765,3.59302516)
\curveto(32.84156957,3.89520957)(32.81219053,4.24356103)(32.81219053,4.63807956)
\lineto(32.81219053,7.35773918)
\lineto(31.57827089,7.35773918)
\lineto(31.57827089,8.22651934)
\lineto(32.81219053,8.22651934)
\lineto(32.81219053,9.86335152)
\lineto(33.84465391,10.03962575)
\lineto(33.84465391,8.22651934)
\closepath
}
}
{
\newrgbcolor{curcolor}{0 0 0}
\pscustom[linestyle=none,fillstyle=solid,fillcolor=curcolor]
{
\newpath
\moveto(41.16003179,3.19011263)
\curveto(41.16003179,2.93829229)(41.07609168,2.715851)(40.90821145,2.52278874)
\curveto(40.74033123,2.32972648)(40.51788993,2.23319535)(40.24088756,2.23319535)
\curveto(39.95549118,2.23319535)(39.72885288,2.32972648)(39.56097266,2.52278874)
\curveto(39.39309244,2.715851)(39.30915232,2.93829229)(39.30915232,3.19011263)
\curveto(39.30915232,3.45032697)(39.39309244,3.67696528)(39.56097266,3.87002753)
\curveto(39.72885288,4.06308979)(39.95549118,4.15962092)(40.24088756,4.15962092)
\curveto(40.51788993,4.15962092)(40.74033123,4.06308979)(40.90821145,3.87002753)
\curveto(41.07609168,3.67696528)(41.16003179,3.45032697)(41.16003179,3.19011263)
\closepath
\moveto(41.16003179,7.25701104)
\curveto(41.16003179,7.00519071)(41.07609168,6.78274941)(40.90821145,6.58968716)
\curveto(40.74033123,6.3966249)(40.51788993,6.30009377)(40.24088756,6.30009377)
\curveto(39.95549118,6.30009377)(39.72885288,6.3966249)(39.56097266,6.58968716)
\curveto(39.39309244,6.78274941)(39.30915232,7.00519071)(39.30915232,7.25701104)
\curveto(39.30915232,7.51722539)(39.39309244,7.74386369)(39.56097266,7.93692595)
\curveto(39.72885288,8.12998821)(39.95549118,8.22651934)(40.24088756,8.22651934)
\curveto(40.51788993,8.22651934)(40.74033123,8.12998821)(40.90821145,7.93692595)
\curveto(41.07609168,7.74386369)(41.16003179,7.51722539)(41.16003179,7.25701104)
\closepath
}
}
\end{pspicture}
}
\caption{Conditional swap algorithm in action: In this figure a 9 out of 52 shuffle has been completed to ilustrade how the algorithm works. Each bit in the $seeds$ indicate if a gate should be swapped. Since the size of $seeds$ is so big I have tried to ilustrate which wire each value is located at before moved in a layer resulting in 1, 51, 14, 20, 10, 37, 9, 33, 6.}
\end{figure}


%%%%%%%%%%%%%%%%%%%%%%% COMPARISONS %%%%%%%%%%%%%%%%%%%%%%%%
%%%%%%%%%%%%%%%%%%%%%%%%%%%%%%%%%%%%%%%%%%%%%%%%%%%%%%%%%%%%
\section{Algorithm comparison}
\todo{describe circuit}
First of all it is important to understand that we cannot just plug the algorithm in to the MPC protocol. Since we use a MPC protocol that uses garbled circuits for the evaluation of the shuffle algorithm we need to provide a circuit that represent this algorithm. Therefor it is essential to understand how such circuits works and how to construct them. These circuits consist of different gates types. Special for the MPC protocol used here shuch type of circuit are all constructed of gates which has two input and one output wire. This results in 16 different gate types which can be used to construct a circuit. These 16 different gate types comes from the result of having two different values $0$ and $1$ for each of the wires. This gives $4$ different ways to combine the two values which results in $2^4$ different combinations. Of these 16 different gates are some of the well known $AND$, $OR$ and $XOR$. All the different gate types can be seen in figure.

\begin{figure}
\label{truth_table}
\centering
\scalebox{.5}{
\begin{tabular}{c c || c c c c c c c c c c c c c c c c}
$l$ & $r$ & $0$ & NOR & $\neg x$ AND $y$ & $\neg x$ & $x$ AND $\neg y$ & $\neg y$ & XOR & NAND & AND & NXOR & $y$ & If $x$ Then $y$ & $x$ & If $y$ Then $x$ & OR & $1$    \\   
\hline
0 & 0 & 0 & 1 & 0 & 1 & 0 & 1 & 0 & 1 & 0 & 1 & 0 & 1 & 0 & 1 & 0 & 1    \\
0 & 1 & 0 & 0 & 1 & 1 & 0 & 0 & 1 & 1 & 0 & 0 & 1 & 1 & 0 & 0 & 1 & 1    \\
1 & 0 & 0 & 0 & 0 & 0 & 1 & 1 & 1 & 1 & 0 & 0 & 0 & 0 & 1 & 1 & 1 & 1    \\
1 & 1 & 0 & 0 & 0 & 0 & 0 & 0 & 0 & 0 & 1 & 1 & 1 & 1 & 1 & 1 & 1 & 1
\end{tabular}
}
\caption{A table of the 16 different gate types that can be used in a circuit of the type used in duplo}
\end{figure}

Since circuits is a static representation of a given algorithm it grow fast in size and become rather complex. Therefor to make the construction of the circuits for the shuffle algorithms easier I have used a compiler called $Frigate$\footnote{I used version 2 which is linked to in appendix \ref{appendix1}. This version is optimized to construct circuits for the duplo protocol.} to help with the circuit generation. The compiler takes a high-level program of a $C$ like structure and translates it to a circuit of the correct format for the MPC protocol chosen to use. This compiler gives the possibility to easy implement the shuffle algorithm and generate a complex circuit that represent the right algorithm.

\bigskip
In the next section I will try to compare the to different algorithms and their circuit representation. The first to look for in the two algorithms is the amount of loops. In the $Fisher\text{-}Yates$ algorithm there is one for-loop and which makes a direct swap resulting in $n$ total swaps. Where as in the $Conditional~swap$ algorithm we have both an outer for-loop and an inner for-loop. Which at creates $\frac{n^2}{2}$ calls to the conditional swap function. At first sight this seems to be many more calls to swap than for the $Fisher\text{-}Yates$ algorithm. But as mentioned earlier circuits are a static representation of an algorithm and therefore the two algorithms do not differ much from each other when comparing their circuit representation. First of all since circuits is a static representation of the algorithm the $Fisher\text{-}Yates$ algorithm can not do a swap based on an input to the algorithm unless all possible swaps are represented in the circuit. This resulting in the need for conditional swaps in the algorithm. Therefore in each of the rounds of the for-loop it should be possible for a value at a given index to go to all the indexes there or higher. This resulting in $n-1$ conditional swaps in the first round. $n-2$ conditional swaps in the second round and so forth. Since both algorithms now need to use conditional swaps they are easy to compare. We know that this gives exactly the same amount of conditional swaps for both algorithms, which is $(n-1)!$ if we do not take the optimizations into account. The biggest difference is how the two algorithms swap in the rounds of the outer for-loop. Here the $Fisher\text{-}Yates$ algorithm has conditional swaps from index $i$ to all the indexes $i'$ where $i<i'$. Where the $Conditional~swap$ algorithm has swaps from $j$ to $j'= j+1$ where $j$ is running from $i$ defined by the forloop to $n-1$.

Another different aspect of the algorithms is to compare the input. Both algorithms takes a $deck$ and $seeds$ as input. Where the $deck$ is the representation of the cards which is the same for both algorithms. But the input $seeds$ differs a lot. This is because of how the two algorithms handle the swaps. Therefore this is the interesting part to look at. First of all it is important to remember what the goal is for this thesis. It is to create a secure distributed poker game. Therefore we use a MPC protocol that help us overcome the security part. We can therefore expect that one of the players will try to cheat. As the protocol takes $seeds$ from both parties on how to shuffle the $deck$ we need to handle this in some way. The easy way is just to $XOR$ these $seeds$ together to get a new seed to use in the shuffle algorithm. This is completely fine in the $Conditional~swap$ algorithm since it uses one bit of randomness at a time for each swap gate. We know that the $XOR$ of to random bits yields a equally random bit. But for the $Fisher\text{-}Yates$ it makes the algorithm insecure since the $XOR$ of the two $seeds$ can result in a new seed that do not forfill the requirements to the inputs. This will give a bias that result in a higher probability of getting low cards. This is because the algorithm relies on the random $r$ values of the seed to be in given intervals. Therefore when $XOR$ the $seeds$ from the two parties it can not be guarantee that the random $r$ values for the shuffle is inside the correct interval. The easy solution to this will be to take the modulo reduction of $r$ to fit inside the desired interval. Here I will come with an example to show the problems by doing so. In our case we need to represent $52$ cards. These can be represented in base 2 using 6 bits since $\lceil{log_2(52)}\rceil=6$. But as we know $2^6=64$. This implies that we have $11$ possible values for our $r$ in the first round after the $seeds$ have been $XOR$'ed together. This result in the first $11$ values from $0$ to $10$ to have the probability $\frac{2}{64}$ while the last $31$ values from $11$ to $51$ have probability $\frac{1}{64}$ giving us a bias. 
Instead of using $XOR$ to construct the new $r$ values for the shuffle algorithm the $seeds$ will be added $6$ bits at a time. This resulting in a uniform propability on the $r$ values. Here it is important to notice that while $6$ bits is enough to hold the 52 card values it is not sufficient to hold the sum of such two values.

\bigskip

At last a short comment on other shuffle algorithmic possibilities. There do exist other types of sorting networks that could be used. In \citeA{psi} they also uses a algorithm known as the $Bitonic$ algorithm refering to the way the network is constructed. Such a network is constructed of what is known as $half-cleansers$ known from sorting networks. These $half-cleansers$ are constructed such that the input has one peak, $i_1\leq \dots \leq p \geq \dots \geq i_n$. Then the output is half sorted such that the highest values are in one of the two halfs. This creates a circuit of size $O(n\cdot log(n))$ which is better then what the $Conditional~swap$ and $Fisher\text{-}Yates$ algorithms can aquire since it creates a circuit of size $O(n^2)$. But as argued earlier the $Conditional~swap$ and $Fisher\text{-}Yates$ algorithms are easily optimized to the setting studied in this thesis.

\todo{Bitonic algorithm, optimization? can maybe be done by flipping such a sorting network and removing half cleansers}

As a result of this we get that for some card games it can be an idea to check if one algorithm can outperform another. In out case we need only $20$ out of $52$ cards. We now know that a $Bitonic$ algorithm would produce a smaller output but since the two algorithm studied here are easy to optimize such that they produce a relative small circuit. This implies that there will ba a cross over at some point where it is better to shuffle a complete deck than only parts of it even if only a part is needed. This could be in a setting when playing with more the two players for a game of poker.